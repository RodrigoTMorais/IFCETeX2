Apresentação concisa dos pontos relevantes do trabalho. Deve ser informativo, apresentando finalidades, metodologia, resultados e conclusões; composto de uma sequência de frases concisas, afirmativas e não de enumeração de tópicos. Deve-se usar o verbo na voz ativa e na terceira pessoa do singular, contendo de 150 a 500 palavras. Deve-se evitar símbolos que não sejam de uso corrente e fórmulas, equações, diagramas etc. que não sejam absolutamente necessários. Após o texto do resumo, recomenda-se que sejam inseridas de 3 a 5 palavras-chave. Xxxxxxx xxxxxxxx xxxxxx xxxxxx xxxxxxxx xxxxxxxxxxxxxxxxxxx. Xxxxxxxxxxx xxxxxxxx xxxxxxx xxxxxxx xxxxxxxx xxxxxxxx xxxxxx xxxxxxxx xxxxxxx xxxxxxxxxxxxxx  xxxxxxxx xxxxxxxxxxxxx xxxxxxxxxxx  xxx xxxxxxxxxxxxxx  xxxxxxxx xxxxxxxxxxxxx xxxxxxxxxxx  xxx xxxxxxxxxxxxxx  xxxxxxxx xxxxxxxxxxxxx xxxxxxxxxxx  xxx xxxxxxxx xxxxxx xxxxxx xxxxxxxx xxxxxxxxxxxxxxxxxxx. Xxxxxxxxxxx xxxxxxxx xxxxxxx xxxxxxx xxxxxxxx xxxxxxxx xxxxxx xxxxxxxx xxxxxx xxxxxx xxxxxxxx xxxxxxxxxxxxxxxxxxx. Xxxxxxxxxxx xxxxxxxx xxxxxxx xxxxxxx xxxxxxxx xxxxxxxx xxxxxx. Xxxxxxxx xxxxxx xxxxxx xxxxxxxx xxxxxxxxxxxxxxxxxxx. Xxxxxxxxxxx xxxxxxxx xxxxxxx xxxxxxx xxxxxxxx xxxxxxxx xxxxxx. Xxxxxxxx xxxxxx xxxxxx xxxxxxxx xxxxxxxxxxxxxxxxxxx. Xxxxxxxxxxx xxxxxxxx xxxxxxx xxxxxxx xxxxxxxx xxxxxxxx xxxxxx.

% Separe as palavras-chave por ponto e vírgula
\palavraschave{Palavra 1; Palavra 2; Palavra 3; Palavra 4; Palavra 5.}