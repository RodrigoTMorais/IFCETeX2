\input{preambulo}
%arquivo de template para os PUDS

%definição das variaveis das seções
\newcommand{\disciplina}{\def \disciplina}
\newcommand{\imprimirdisciplina}{\disciplina}

\newcommand{\codigo}{\def \codigo}
\newcommand{\imprimircodigo}{\codigo}

\newcommand{\cargaHorariaTotal}{\def \cargaHorariaTotal}
\newcommand{\imprimircargaHorariaTotal}{\cargaHorariaTotal}

\newcommand{\cargaHorariaPratica}{\def \cargaHorariaPratica}
\newcommand{\imprimircargaHorariaPratica}{\cargaHorariaPratica}

\newcommand{\cargaHorariaTeorica}{\def \cargaHorariaTeorica}
\newcommand{\imprimircargaHorariaTeorica}{\cargaHorariaTeorica}

\newcommand{\creditos}{\def \creditos}
\newcommand{\imprimircreditos}{\creditos}

\newcommand{\codigoPrerequisitos}{\def \codigoPrerequisitos}
\newcommand{\imprimircodigoPrerequisitos}{\codigoPrerequisitos}

\newcommand{\semestre}{\def \semestre}
\newcommand{\imprimirsemestre}{\semestre}

\newcommand{\nivel}{\def \nivel}
\newcommand{\imprimirnivel}{\nivel}

\newcommand{\codigoEquivalencias}{\def \codigoEquivalencias}
\newcommand{\imprimircodigoEquivalencias}{\codigoEquivalencias}

\newcommand{\ementa}{\def \ementa}
\newcommand{\imprimirementa}{\ementa}

\newcommand{\objetivo}{\def \objetivo}
\newcommand{\imprimirobjetivo}{\objetivo}

\newcommand{\programa}{\def \programa}
\newcommand{\imprimirprograma}{\programa}

\newcommand{\metodologiaEnsino}{\def \metodologiaEnsino}
\newcommand{\imprimirmetodologiaEnsino}{\metodologiaEnsino}

\newcommand{\recursos}{\def \recursos}
\newcommand{\imprimirrecursos}{\recursos}

\newcommand{\avaliacao}{\def \avaliacao}
\newcommand{\imprimiravaliacao}{\avaliacao}

\newcommand{\bibliografiaBasica}{\def \bibliografiaBasica}
\newcommand{\imprimirbibliografiaBasica}{\bibliografiaBasica}

\newcommand{\bibliografiaComplementar}{\def \bibliografiaComplementar}
\newcommand{\imprimirbibliografiaComplementar}{\bibliografiaComplementar}

\newcommand{\versao}{\def \versao}
\newcommand{\imprimirversao}{\versao}


%comando de impressão da estrutura
\newcommand{\imprimirPUD}{
%Cabeçalho do PUD
\begin{Spacing}{1}

\noindent \begin{minipage}{2.5cm}%
\includegraphics[scale=0.12]{logo-ifce}
\end{minipage}
\hspace{0.3cm}
\begin{minipage}{13cm}%
\centering INSTITUTO FEDERAL DE EDUCAÇÃO, CIÊNCIA E TECNOLOGIA DO CEARÁ- IFCE\\
CAMPUS JUAZEIRO DO NORTE\\
CURSO SUPERIOR EM AUTOMAÇÃO INDUSTRIAL\\
PROGRAMA DE UNIDADE DIDÁTICA – PUD\\
\end{minipage}%
\end{Spacing}

\begin{longtable}{|p{14cm}|}
%primeiro cabeçalho
\hline
\rowcolor{lightgray}
\multicolumn{1}{p{14cm}}{\textbf{Disciplina: \imprimirdisciplina}}\\
\hline
\endfirsthead

%cabeçalho
\hline
continuação PUD \imprimirdisciplina\\
\hline
\endhead

\hline
continua...\\
\hline
\endfoot

\hline
\rowcolor{lightgray}

\begin{tabular}{p{5.5 cm}| l}
coordenação & departamento pedagogico\\[16 ex]
\end{tabular}\\

\hline

\endlastfoot

%elementos
\textbf{Código:} \imprimircodigo\\


\textbf{Carga Horária } Teórica: \imprimircargaHorariaTeorica, Prática \imprimircargaHorariaPratica, Total: \imprimircargaHorariaTotal\\


\textbf{Número de créditos:} \imprimircreditos\\


\textbf{Código pré-requisitos:} \imprimircodigoPrerequisitos\\


\textbf{Semestre:} \imprimirsemestre\\


\textbf{Nível:} \imprimirnivel\\
\hline

\rowcolor{lightgray}
\multicolumn{1}{|p{14cm}|}{\textbf{Ementa}}\\
\hline
\multicolumn{1}{|p{14cm}|}{\imprimirementa}\\
\hline

\rowcolor{lightgray}
\multicolumn{1}{|p{14cm}|}{\textbf{Objetivo}}\\
\hline
\imprimirobjetivo\\
\hline

\rowcolor{lightgray}
\multicolumn{1}{|p{14cm}|}{\textbf{Programa}}\\
\hline
\imprimirprograma\\
\hline

\rowcolor{lightgray}
\multicolumn{1}{|p{14cm}|}{\textbf{Metodologia de ensino}}\\
\hline
\imprimirmetodologiaEnsino\\
\hline

\rowcolor{lightgray}
\multicolumn{1}{|p{14cm}|}{\textbf{Recursos}}\\
\hline
\imprimirrecursos\\
\hline

\rowcolor{lightgray}
\multicolumn{1}{|p{14cm}|}{\textbf{Avaliação}}\\
\hline
\imprimiravaliacao\\
\hline

\rowcolor{lightgray}
\multicolumn{1}{|p{14cm}|}{\textbf{Bibliografia básica}}\\
\hline
\imprimirbibliografiaBasica\\
\hline

\rowcolor{lightgray}
\multicolumn{1}{|p{14cm}|}{\textbf{Bibliografia complementar}}\\
\hline
\imprimirbibliografiaComplementar\\
\hline
\end{longtable}
\pagebreak
}
\begin{document}

\disciplina{Engenharia assistida por computador}
\codigo{AUT2439}
\cargaHorariaTotal{80}
\cargaHorariaPratica{60}
\cargaHorariaTeorica{20}
\creditos{4}
\codigoPrerequisitos{AUT2406, AUT2423}
\semestre{7º}
\nivel{Superior}

\ementa{
Ensinar a utilização de software de engenharia assistida por computado para simulação de comportamentos mecânicos. Utilização de métodos de análise de elementos finitos em projetos de automação, Projeto de métodos de fabricação.
}

\objetivo{
• Conhecer um software de engenharia auxiliada por computador;\\
• Desenhar e realizar montagens virtuais de máquinas;\\
• Integrar projetos elétricos, mecânicos e eletrônicos gerando documentação de montagem;\\
• Operação de máquinas de comando numérico;\\
• Criação de artefatos de manufatura CNC.\\

}

\programa{
• INTRODUÇÃO À DISCIPLINA: Apresentação dos professores e estudantes,\\ Apresentação do plano de curso, Metodologia do ensino, aprendizagem e avaliação, A disciplina no currículo e integração com outras disciplinas na formação do profissional, aplicação do CAE na cadeia produtiva moderna.\\
• INTRODUÇÃO A UM SOFTWARE CAE: finalidade, aplicações, fluxo de trabalho.\\
• REVISÃO DE CAD; Desenho de sólidos básicos, recursos de padrão, cotas, furação.\\
• DIVISÃO E DERIVAÇÃO DE DESENHOS: Divisão de desenho conceitual em partes, derivação de desenhos, árvore derivativa.\\
• MONTAGEM VIRTUAL: Introdução de componentes, restrições e posicionamentos, restrições mecânicas, adição de componentes padronizados.\\
• ASSISTENTES DE PROJETO: uso de assistentes de estruturas, engrenagens, polias, molas e rolamentos.\\
• ANÁLISES: Geração de análise mecânica, estrutural e de esforços.\\
• ASSISTENTE DE MANUFATURA: Configuração de ferramentas, posicionamento, processos de manufatura, geração de programa CNC;\\
• CRIAÇÃO DE DESENHOS TÉCNICOS: Geração de vistas explodidas, conjuntos e sub conjuntos, listas de materiais.\\
}

\metodologiaEnsino{
Aulas expositivas;
Aulas práticas;
Pratica de laboratório;
projetos;
Vídeo Aulas;
}

\recursos{
Computadores com software apropriado;\\
sistema de projeção;\\
Material para pratica de construção (filamento, MDF, parafuso, porca, componentes eletrônicos);\\
laboratório de prototipagem;\\
}

\avaliacao{
Avaliações práticas;
projetos;
construção de artefatos;
avaliação continuada por desempenho em aulas;
}

\bibliografiaBasica{
• MANFÉ, Giovanni. Desenho técnico mecânico: curso completo para as escolas técnicas e ciclo básico das faculdades de engenharia: v. 1. São Paulo: Hemus, 2004. 277 p. ISBN 85-289-0007-X.\\

• DEHMLOW, Martin; KIEL, E. Desenho mecânico - v.1. São Paulo: EPU : EDUSP, 1974. 48p.\\

• JONES, Franklin D. Manual técnico para desenhistas e projetistas de máquinas v.1. 14. ed. São Paulo: Hemus, 1975. 418 p. (1).\\
}

\bibliografiaComplementar{
• DEHMLOW, Martin; KIEL, E. Desenho mecânico - v.2. São Paulo: EPU : EDUSP, 1974. 48p.\\

• JONES, Franklin D. Manual técnico para desenhistas e projetistas de máquinas v.2. 14. ed. São Paulo: Hemus, 1975. v. 2 . 421 p. (2).\\

• ALVES FILHO, Avelino. Elementos finitos: a base da tecnologia CAE. 5. ed. São Paulo, SP: Érica, 2007. 292 p. ISBN 9788571947412.\\

• ALVES FILHO, Avelino. Elementos finitos: a base da tecnologia CAE/Análise dinâmica. 2. ed. São Paulo, SP: Érica, 2009. 301 p. ISBN 9788536500508.\\

• FISH, Jacob; BELYTSCHKO, Ted. Um primeiro curso em elementos finitos. KOURY, Ricardo Nicolau Nassar (Trad.), MACHADO, Luiz (Trad.). Rio de Janeiro, RJ: LTC, 2009. 241 p.
}


\imprimirPUD

\end{document}