\input{preambulo}
%arquivo de template para os PUDS

%definição das variaveis das seções
\newcommand{\disciplina}{\def \disciplina}
\newcommand{\imprimirdisciplina}{\disciplina}

\newcommand{\codigo}{\def \codigo}
\newcommand{\imprimircodigo}{\codigo}

\newcommand{\cargaHorariaTotal}{\def \cargaHorariaTotal}
\newcommand{\imprimircargaHorariaTotal}{\cargaHorariaTotal}

\newcommand{\cargaHorariaPratica}{\def \cargaHorariaPratica}
\newcommand{\imprimircargaHorariaPratica}{\cargaHorariaPratica}

\newcommand{\cargaHorariaTeorica}{\def \cargaHorariaTeorica}
\newcommand{\imprimircargaHorariaTeorica}{\cargaHorariaTeorica}

\newcommand{\creditos}{\def \creditos}
\newcommand{\imprimircreditos}{\creditos}

\newcommand{\codigoPrerequisitos}{\def \codigoPrerequisitos}
\newcommand{\imprimircodigoPrerequisitos}{\codigoPrerequisitos}

\newcommand{\semestre}{\def \semestre}
\newcommand{\imprimirsemestre}{\semestre}

\newcommand{\nivel}{\def \nivel}
\newcommand{\imprimirnivel}{\nivel}

\newcommand{\codigoEquivalencias}{\def \codigoEquivalencias}
\newcommand{\imprimircodigoEquivalencias}{\codigoEquivalencias}

\newcommand{\ementa}{\def \ementa}
\newcommand{\imprimirementa}{\ementa}

\newcommand{\objetivo}{\def \objetivo}
\newcommand{\imprimirobjetivo}{\objetivo}

\newcommand{\programa}{\def \programa}
\newcommand{\imprimirprograma}{\programa}

\newcommand{\metodologiaEnsino}{\def \metodologiaEnsino}
\newcommand{\imprimirmetodologiaEnsino}{\metodologiaEnsino}

\newcommand{\recursos}{\def \recursos}
\newcommand{\imprimirrecursos}{\recursos}

\newcommand{\avaliacao}{\def \avaliacao}
\newcommand{\imprimiravaliacao}{\avaliacao}

\newcommand{\bibliografiaBasica}{\def \bibliografiaBasica}
\newcommand{\imprimirbibliografiaBasica}{\bibliografiaBasica}

\newcommand{\bibliografiaComplementar}{\def \bibliografiaComplementar}
\newcommand{\imprimirbibliografiaComplementar}{\bibliografiaComplementar}

\newcommand{\versao}{\def \versao}
\newcommand{\imprimirversao}{\versao}


%comando de impressão da estrutura
\newcommand{\imprimirPUD}{
%Cabeçalho do PUD
\begin{Spacing}{1}

\noindent \begin{minipage}{2.5cm}%
\includegraphics[scale=0.12]{logo-ifce}
\end{minipage}
\hspace{0.3cm}
\begin{minipage}{13cm}%
\centering INSTITUTO FEDERAL DE EDUCAÇÃO, CIÊNCIA E TECNOLOGIA DO CEARÁ- IFCE\\
CAMPUS JUAZEIRO DO NORTE\\
CURSO SUPERIOR EM AUTOMAÇÃO INDUSTRIAL\\
PROGRAMA DE UNIDADE DIDÁTICA – PUD\\
\end{minipage}%
\end{Spacing}

\begin{longtable}{|p{14cm}|}
%primeiro cabeçalho
\hline
\rowcolor{lightgray}
\multicolumn{1}{p{14cm}}{\textbf{Disciplina: \imprimirdisciplina}}\\
\hline
\endfirsthead

%cabeçalho
\hline
continuação PUD \imprimirdisciplina\\
\hline
\endhead

\hline
continua...\\
\hline
\endfoot

\hline
\rowcolor{lightgray}

\begin{tabular}{p{5.5 cm}| l}
coordenação & departamento pedagogico\\[16 ex]
\end{tabular}\\

\hline

\endlastfoot

%elementos
\textbf{Código:} \imprimircodigo\\


\textbf{Carga Horária } Teórica: \imprimircargaHorariaTeorica, Prática \imprimircargaHorariaPratica, Total: \imprimircargaHorariaTotal\\


\textbf{Número de créditos:} \imprimircreditos\\


\textbf{Código pré-requisitos:} \imprimircodigoPrerequisitos\\


\textbf{Semestre:} \imprimirsemestre\\


\textbf{Nível:} \imprimirnivel\\
\hline

\rowcolor{lightgray}
\multicolumn{1}{|p{14cm}|}{\textbf{Ementa}}\\
\hline
\multicolumn{1}{|p{14cm}|}{\imprimirementa}\\
\hline

\rowcolor{lightgray}
\multicolumn{1}{|p{14cm}|}{\textbf{Objetivo}}\\
\hline
\imprimirobjetivo\\
\hline

\rowcolor{lightgray}
\multicolumn{1}{|p{14cm}|}{\textbf{Programa}}\\
\hline
\imprimirprograma\\
\hline

\rowcolor{lightgray}
\multicolumn{1}{|p{14cm}|}{\textbf{Metodologia de ensino}}\\
\hline
\imprimirmetodologiaEnsino\\
\hline

\rowcolor{lightgray}
\multicolumn{1}{|p{14cm}|}{\textbf{Recursos}}\\
\hline
\imprimirrecursos\\
\hline

\rowcolor{lightgray}
\multicolumn{1}{|p{14cm}|}{\textbf{Avaliação}}\\
\hline
\imprimiravaliacao\\
\hline

\rowcolor{lightgray}
\multicolumn{1}{|p{14cm}|}{\textbf{Bibliografia básica}}\\
\hline
\imprimirbibliografiaBasica\\
\hline

\rowcolor{lightgray}
\multicolumn{1}{|p{14cm}|}{\textbf{Bibliografia complementar}}\\
\hline
\imprimirbibliografiaComplementar\\
\hline
\end{longtable}
\pagebreak
}
\begin{document}

\disciplina{Processos de fabricação}
\codigo{AUT2429}
\cargaHorariaTotal{120}
\cargaHorariaPratica{80}
\cargaHorariaTeorica{40}
\creditos{6}
\codigoPrerequisitos{AUT2426}
\semestre{6º}
\nivel{Superior}

\ementa{
Conhecer os métodos e os processos de produção mecânica. Conhecer as características dos instrumentos, máquinas, equipamentos e instalações e suas aplicações. Avaliar a influencia do processo e do produto no meio ambiente.
}

\objetivo{
• Conhecer os métodos e os processos de produção mecânica;\\
• Conhecer as características dos instrumentos, máquinas, equipamentos e instalações e suas aplicações; \\
• Avaliar a influencia do processo e do produto no meio ambiente.\\
}

\programa{
• Processo de Conformação dos Metais\\
• Laminação\\
• Trefilação\\
• Forjamento\\
• Estampagem\\
• Processo de Soldagem\\
• Solda oxiacetilênica\\
• Solda elétrica com eletrodo revestido\\
• TIG\\
• MIG/MAG\\
• Arco voltaico submerso\\
• Processo de Usinagem\\
• Características\\
• Equipamentos\\
• Ferramentas\\
• Aspectos de segurança dos processos de usinagem: furação, torneamento, aplainamento, mandrilhamento, retificação, brochamento, fabricação de engrenagens.\\
• Definição e cálculos dos dados de corte em usinagem: velocidade, rotação e avanço de corte, tempo de corte.\\
• Materiais para ferramentas de corte: aços rápidos, metal duro, cerâmica e diamante.\\
• Fluidos de corte, geometria de corte das ferramentas, dispositivos e acessórios de fixação.\\
}

\metodologiaEnsino{
Aulas expositivas;\\
Lista de exercícios.\\
Livros contidos na bibliografia; \\
Quadro e pincel.\\
Data-show;\\
Práticas de laboratório.\\
}

\recursos{
Livros contidos na bibliografia;\\
Quadro e pincel.\\
Data-show;\\
Laboratório de mecânica Industrial;\\
}

\avaliacao{
Avaliação escrita\\
Listas de exercícios;\\
Atividades práticas\\
Poderão ser inseridas outras avaliações durante o semestre.\\
}

\bibliografiaBasica{
• MARQUES, Paulo V; MODENESI,   Paulo J; BRANCARENSE,   Alexandre Q.   Soldagem:
fundamentos e tecnologia. UFMG, 2009.\\

• WEISS, Almiro. Soldagem. Curitiba: Editora do Livro Técnico, 2010.\\

• SCOTTI, Américo; PONOMAREU, Vladimir. Soldagem MIG/MAG: melhor entendimento, melhor desempenho. São Paulo: Artliber, 2008.\\
}

\bibliografiaComplementar{
• DINIZ, Anselmo E; MARCONDES, Francisco C; COPPINI, Nivaldo L. Tecnologia da usinagem dos materiais. São Paulo: Artliber, 2010.\\

• CETLIN, Paulo R; HELMAN, Horácio. Fundamentos da Conformação: Mecânica dos Metais.São Paulo: Artliber, 2010.\\

• CHIAVERINI, Vicente. Tecnologia mecância II: processo de fabricação e tratamento. São Paulo: Makon Books do Brasil, 1986.\\

• MACHADO, Álisson et al. Teoria da usinagem dos materiais. São Paulo: Blucher, 2009. \\

• SANTOS, Sandro C; SALES, Wisley F. Aspectos tribológicos da usinagem dos mateirias. São Paulo:Artliber, 2007.\\
}


\imprimirPUD

\end{document}