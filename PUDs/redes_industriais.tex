\input{preambulo}
%arquivo de template para os PUDS

%definição das variaveis das seções
\newcommand{\disciplina}{\def \disciplina}
\newcommand{\imprimirdisciplina}{\disciplina}

\newcommand{\codigo}{\def \codigo}
\newcommand{\imprimircodigo}{\codigo}

\newcommand{\cargaHorariaTotal}{\def \cargaHorariaTotal}
\newcommand{\imprimircargaHorariaTotal}{\cargaHorariaTotal}

\newcommand{\cargaHorariaPratica}{\def \cargaHorariaPratica}
\newcommand{\imprimircargaHorariaPratica}{\cargaHorariaPratica}

\newcommand{\cargaHorariaTeorica}{\def \cargaHorariaTeorica}
\newcommand{\imprimircargaHorariaTeorica}{\cargaHorariaTeorica}

\newcommand{\creditos}{\def \creditos}
\newcommand{\imprimircreditos}{\creditos}

\newcommand{\codigoPrerequisitos}{\def \codigoPrerequisitos}
\newcommand{\imprimircodigoPrerequisitos}{\codigoPrerequisitos}

\newcommand{\semestre}{\def \semestre}
\newcommand{\imprimirsemestre}{\semestre}

\newcommand{\nivel}{\def \nivel}
\newcommand{\imprimirnivel}{\nivel}

\newcommand{\codigoEquivalencias}{\def \codigoEquivalencias}
\newcommand{\imprimircodigoEquivalencias}{\codigoEquivalencias}

\newcommand{\ementa}{\def \ementa}
\newcommand{\imprimirementa}{\ementa}

\newcommand{\objetivo}{\def \objetivo}
\newcommand{\imprimirobjetivo}{\objetivo}

\newcommand{\programa}{\def \programa}
\newcommand{\imprimirprograma}{\programa}

\newcommand{\metodologiaEnsino}{\def \metodologiaEnsino}
\newcommand{\imprimirmetodologiaEnsino}{\metodologiaEnsino}

\newcommand{\recursos}{\def \recursos}
\newcommand{\imprimirrecursos}{\recursos}

\newcommand{\avaliacao}{\def \avaliacao}
\newcommand{\imprimiravaliacao}{\avaliacao}

\newcommand{\bibliografiaBasica}{\def \bibliografiaBasica}
\newcommand{\imprimirbibliografiaBasica}{\bibliografiaBasica}

\newcommand{\bibliografiaComplementar}{\def \bibliografiaComplementar}
\newcommand{\imprimirbibliografiaComplementar}{\bibliografiaComplementar}

\newcommand{\versao}{\def \versao}
\newcommand{\imprimirversao}{\versao}


%comando de impressão da estrutura
\newcommand{\imprimirPUD}{
%Cabeçalho do PUD
\begin{Spacing}{1}

\noindent \begin{minipage}{2.5cm}%
\includegraphics[scale=0.12]{logo-ifce}
\end{minipage}
\hspace{0.3cm}
\begin{minipage}{13cm}%
\centering INSTITUTO FEDERAL DE EDUCAÇÃO, CIÊNCIA E TECNOLOGIA DO CEARÁ- IFCE\\
CAMPUS JUAZEIRO DO NORTE\\
CURSO SUPERIOR EM AUTOMAÇÃO INDUSTRIAL\\
PROGRAMA DE UNIDADE DIDÁTICA – PUD\\
\end{minipage}%
\end{Spacing}

\begin{longtable}{|p{14cm}|}
%primeiro cabeçalho
\hline
\rowcolor{lightgray}
\multicolumn{1}{p{14cm}}{\textbf{Disciplina: \imprimirdisciplina}}\\
\hline
\endfirsthead

%cabeçalho
\hline
continuação PUD \imprimirdisciplina\\
\hline
\endhead

\hline
continua...\\
\hline
\endfoot

\hline
\rowcolor{lightgray}

\begin{tabular}{p{5.5 cm}| l}
coordenação & departamento pedagogico\\[16 ex]
\end{tabular}\\

\hline

\endlastfoot

%elementos
\textbf{Código:} \imprimircodigo\\


\textbf{Carga Horária } Teórica: \imprimircargaHorariaTeorica, Prática \imprimircargaHorariaPratica, Total: \imprimircargaHorariaTotal\\


\textbf{Número de créditos:} \imprimircreditos\\


\textbf{Código pré-requisitos:} \imprimircodigoPrerequisitos\\


\textbf{Semestre:} \imprimirsemestre\\


\textbf{Nível:} \imprimirnivel\\
\hline

\rowcolor{lightgray}
\multicolumn{1}{|p{14cm}|}{\textbf{Ementa}}\\
\hline
\multicolumn{1}{|p{14cm}|}{\imprimirementa}\\
\hline

\rowcolor{lightgray}
\multicolumn{1}{|p{14cm}|}{\textbf{Objetivo}}\\
\hline
\imprimirobjetivo\\
\hline

\rowcolor{lightgray}
\multicolumn{1}{|p{14cm}|}{\textbf{Programa}}\\
\hline
\imprimirprograma\\
\hline

\rowcolor{lightgray}
\multicolumn{1}{|p{14cm}|}{\textbf{Metodologia de ensino}}\\
\hline
\imprimirmetodologiaEnsino\\
\hline

\rowcolor{lightgray}
\multicolumn{1}{|p{14cm}|}{\textbf{Recursos}}\\
\hline
\imprimirrecursos\\
\hline

\rowcolor{lightgray}
\multicolumn{1}{|p{14cm}|}{\textbf{Avaliação}}\\
\hline
\imprimiravaliacao\\
\hline

\rowcolor{lightgray}
\multicolumn{1}{|p{14cm}|}{\textbf{Bibliografia básica}}\\
\hline
\imprimirbibliografiaBasica\\
\hline

\rowcolor{lightgray}
\multicolumn{1}{|p{14cm}|}{\textbf{Bibliografia complementar}}\\
\hline
\imprimirbibliografiaComplementar\\
\hline
\end{longtable}
\pagebreak
}
\begin{document}

\disciplina{Redes industriais}
\codigo{AUT2437}
\cargaHorariaTotal{80}
\cargaHorariaPratica{40}
\cargaHorariaTeorica{40}
\creditos{4}
\codigoPrerequisitos{AUT2436}
\semestre{7º}
\nivel{Superior}

\ementa{
Principais Redes Industriais; Protocolo Elétrico 485; As cinco linguagens de programação para CLPs normalizadas pela IEC; Modelagem de processos com GRAFCET; Sistemas Supervisores.
}

\objetivo{
• Conhecer as principais Redes Industriais;\\
• Familiarizar-se com uso das principais linguagens de programação para CLP; \\
• Conhecer técnicas de modelagem de processos;\\
• Elaborar aplicações com CLPs para automação de processos; Integrar CLPs a sistemas de supervisão.\\
}

\programa{
• Conceitos e definições de SDCD As Redes Industriais\\
• Rede Modbus Rede Profbus Redes Fielbus\\
• O protocolo HART\\

• O protocolo CANOpen\\

• Redes DiviceNet, ControlNet, Ethernet/IP Protocolo OPC\\
• As linguagens definidas pela Norma IEC 61131-3 Linguagem Ladder (LD)\\
• Lista de Instruções (IL) Texto Estruturado (ST)\\
• Diagrama de Bloco de Funções (FBD)\\

• Diagrama de Funções Sequenciais – SFC ou GRAFCET Ambientes de Programação.\\
• Modelagem, programação e simulação. \\
• Gravação programas no Twido e TPW-03\\
• Desenvolvimento de projetos com as bancadas de teste Integrando o CLP a sistemas supervisórios\\
}

\metodologiaEnsino{
Aulas expositivas; \\
Aulas em campo;\\
Aulas práticas em laboratórios; \\
Seminários;\\
Listas de exercícios.\\
}

\recursos{
Quadro;\\
Datashow;\\
Laboratório de Sistemas Industriais.\\
}

\avaliacao{
Análise e correção dos projetos de automação; \\
Provas escritas;\\
Práticas individuais e em grupo no laboratório; \\
Seminários;\\
Apresentação de relatório;\\
}

\bibliografiaBasica{
• LUGLI, Alexandre Baratella; SANTOS, Max Mauro Dias. Redes industriais para automação industrial: AS- I, PROFIBUS e PROFINET. São Paulo: Érica, 2012.\\

• LUGLI, Alexandre Baratella; SANTOS, Max Mauro Dias. Sistemas fieldbus para automação industrial: devicenet, CANopen, SDS e Ethernet. São Paulo: Érica, 2009.\\
• FRANCHI e VALTER. Controladores lógicos programáveis: sistemas discretos. São Paulo: ÉRICA, 2010.
}

\bibliografiaComplementar{
• LOPEZ, Ricardo Aldabó. Sistemas de redes para controle e automação. Rio de Janeiro: Book Express, 2000.\\

• MORAES, Cícero Couto de; CASTRUCCI, Plínio de Lauro. Engenharia de automação industrial. Rio de Janeiro: LTC, 2007.\\

• DERFLER JR, Frank J. Guia de conectividade. Rio de Janeiro: Campus, 1995.\\

• TORRES, Gabriel. Redes de Computadores. Rio de Janeiro: Editora Novaterra. 2010. \\
• KUROSE, James F; ROSS, Keith W. Redes de Computador e a internet: Uma Abordagem top-down. São Paulo: Editora Pearson. 6a edição, 2013.\\

• MORIMOTO, Carlos Eduardo. Redes, Guia Prático. Porto Alegre: Editora Sul Editores. 2009.\\
}


\imprimirPUD

\end{document}