\input{preambulo}
%arquivo de template para os PUDS

%definição das variaveis das seções
\newcommand{\disciplina}{\def \disciplina}
\newcommand{\imprimirdisciplina}{\disciplina}

\newcommand{\codigo}{\def \codigo}
\newcommand{\imprimircodigo}{\codigo}

\newcommand{\cargaHorariaTotal}{\def \cargaHorariaTotal}
\newcommand{\imprimircargaHorariaTotal}{\cargaHorariaTotal}

\newcommand{\cargaHorariaPratica}{\def \cargaHorariaPratica}
\newcommand{\imprimircargaHorariaPratica}{\cargaHorariaPratica}

\newcommand{\cargaHorariaTeorica}{\def \cargaHorariaTeorica}
\newcommand{\imprimircargaHorariaTeorica}{\cargaHorariaTeorica}

\newcommand{\creditos}{\def \creditos}
\newcommand{\imprimircreditos}{\creditos}

\newcommand{\codigoPrerequisitos}{\def \codigoPrerequisitos}
\newcommand{\imprimircodigoPrerequisitos}{\codigoPrerequisitos}

\newcommand{\semestre}{\def \semestre}
\newcommand{\imprimirsemestre}{\semestre}

\newcommand{\nivel}{\def \nivel}
\newcommand{\imprimirnivel}{\nivel}

\newcommand{\codigoEquivalencias}{\def \codigoEquivalencias}
\newcommand{\imprimircodigoEquivalencias}{\codigoEquivalencias}

\newcommand{\ementa}{\def \ementa}
\newcommand{\imprimirementa}{\ementa}

\newcommand{\objetivo}{\def \objetivo}
\newcommand{\imprimirobjetivo}{\objetivo}

\newcommand{\programa}{\def \programa}
\newcommand{\imprimirprograma}{\programa}

\newcommand{\metodologiaEnsino}{\def \metodologiaEnsino}
\newcommand{\imprimirmetodologiaEnsino}{\metodologiaEnsino}

\newcommand{\recursos}{\def \recursos}
\newcommand{\imprimirrecursos}{\recursos}

\newcommand{\avaliacao}{\def \avaliacao}
\newcommand{\imprimiravaliacao}{\avaliacao}

\newcommand{\bibliografiaBasica}{\def \bibliografiaBasica}
\newcommand{\imprimirbibliografiaBasica}{\bibliografiaBasica}

\newcommand{\bibliografiaComplementar}{\def \bibliografiaComplementar}
\newcommand{\imprimirbibliografiaComplementar}{\bibliografiaComplementar}

\newcommand{\versao}{\def \versao}
\newcommand{\imprimirversao}{\versao}


%comando de impressão da estrutura
\newcommand{\imprimirPUD}{
%Cabeçalho do PUD
\begin{Spacing}{1}

\noindent \begin{minipage}{2.5cm}%
\includegraphics[scale=0.12]{logo-ifce}
\end{minipage}
\hspace{0.3cm}
\begin{minipage}{13cm}%
\centering INSTITUTO FEDERAL DE EDUCAÇÃO, CIÊNCIA E TECNOLOGIA DO CEARÁ- IFCE\\
CAMPUS JUAZEIRO DO NORTE\\
CURSO SUPERIOR EM AUTOMAÇÃO INDUSTRIAL\\
PROGRAMA DE UNIDADE DIDÁTICA – PUD\\
\end{minipage}%
\end{Spacing}

\begin{longtable}{|p{14cm}|}
%primeiro cabeçalho
\hline
\rowcolor{lightgray}
\multicolumn{1}{p{14cm}}{\textbf{Disciplina: \imprimirdisciplina}}\\
\hline
\endfirsthead

%cabeçalho
\hline
continuação PUD \imprimirdisciplina\\
\hline
\endhead

\hline
continua...\\
\hline
\endfoot

\hline
\rowcolor{lightgray}

\begin{tabular}{p{5.5 cm}| l}
coordenação & departamento pedagogico\\[16 ex]
\end{tabular}\\

\hline

\endlastfoot

%elementos
\textbf{Código:} \imprimircodigo\\


\textbf{Carga Horária } Teórica: \imprimircargaHorariaTeorica, Prática \imprimircargaHorariaPratica, Total: \imprimircargaHorariaTotal\\


\textbf{Número de créditos:} \imprimircreditos\\


\textbf{Código pré-requisitos:} \imprimircodigoPrerequisitos\\


\textbf{Semestre:} \imprimirsemestre\\


\textbf{Nível:} \imprimirnivel\\
\hline

\rowcolor{lightgray}
\multicolumn{1}{|p{14cm}|}{\textbf{Ementa}}\\
\hline
\multicolumn{1}{|p{14cm}|}{\imprimirementa}\\
\hline

\rowcolor{lightgray}
\multicolumn{1}{|p{14cm}|}{\textbf{Objetivo}}\\
\hline
\imprimirobjetivo\\
\hline

\rowcolor{lightgray}
\multicolumn{1}{|p{14cm}|}{\textbf{Programa}}\\
\hline
\imprimirprograma\\
\hline

\rowcolor{lightgray}
\multicolumn{1}{|p{14cm}|}{\textbf{Metodologia de ensino}}\\
\hline
\imprimirmetodologiaEnsino\\
\hline

\rowcolor{lightgray}
\multicolumn{1}{|p{14cm}|}{\textbf{Recursos}}\\
\hline
\imprimirrecursos\\
\hline

\rowcolor{lightgray}
\multicolumn{1}{|p{14cm}|}{\textbf{Avaliação}}\\
\hline
\imprimiravaliacao\\
\hline

\rowcolor{lightgray}
\multicolumn{1}{|p{14cm}|}{\textbf{Bibliografia básica}}\\
\hline
\imprimirbibliografiaBasica\\
\hline

\rowcolor{lightgray}
\multicolumn{1}{|p{14cm}|}{\textbf{Bibliografia complementar}}\\
\hline
\imprimirbibliografiaComplementar\\
\hline
\end{longtable}
\pagebreak
}
\begin{document}


\disciplina{Eletricidade 1}
\codigo{AUT2401}
\cargaHorariaTotal{120h}
\cargaHorariaPratica{80h}
\cargaHorariaTeorica{40h}
\creditos{6}
\codigoPrerequisitos{-}
\semestre{1º}
\nivel{Superior}
\ementa{Princípios da eletrostática e as leis básicas da eletrodinâmica. Conhecer as Principais formas de ondas que modelam as grandezas elétricas. Definir os Efeitos resistivo, capacitivos e indutivos em análise de circuitos. Circuitos elétricos de corrente contínua.}

\objetivo{
• Realizar conexões série e paralela de fontes de tensão e resistores elétricos.\\
• Calcular resistências de condutores elétricos.\\
• Realizar as operações de análise de circuitos, aplicando as relações tensão corrente e potência, primeira e segunda lei de Ohm, lei das tensões e das correntes de Kirchhoff, equações do divisor de tensão e divisor de corrente.\\
• Aplicar os teoremas da superposição, Thévenin, Norton, Millman, Compensação e Máxima Transferência de energia em análise de circuitos lineares de corrente contínua.\\
}

\programa{
• Definições e notações\\
• Unidades múltiplas e submúltiplas do SI Carga elétrica (Q) Campo e potencial elétrico\\
• Fontes de diferença de potencial elétrico Corrente resistência e condutividade elétrica Conexão série\\
• Conexão paralela Notação de ddp\\
• Notação de corrente elétrica O circuito elétrico\\
• Relações entre tensão corrente e potência elétrica Primeira lei de Ohm\\
• Potência elétrica Trabalho e Energia\\
• Fontes de ddp – modelo real Estudo da resistência elétrica\\
• Resistência linear e resistência não linear característica tensão corrente\\
• Resistência de condutores elétricos\\
• Segunda lei de Ohm e resistividade elétrica\\
• Medida de fios e cabos condutores\\
• Coeficiente de temperatura de resistência Análise do circuito série,
• Cálculo da resistência equivalente / LTK – lei das tensões de Kirchhofff\\ • Divisor de tensão / equação do divisor de tensão\\
• Análise do circuito paralelo\\
• Cálculo da resistência equivalente/ LCK- lei das correntes de Kirchhoff  Divisor de corrente / equação do divisor de corrente\\
• Análise de circuitos série-paralelo com uma fonte de tensão Cálculo da resistência equivalente vista pela fonte\\
• Cálculo da corrente total, correntes e tensões nos braços do circuito\\ • Análise de circuitos série-paralelo com mais de uma fonte de tensão Teorema da superposição\\
• Aplicação do teorema na análise dos circuitos. Análise dos circuitos ponte,
• Teorema de Thévenin\\
• Aplicação do teorema de Thévenin na análise dos circuitos Circuito básico da ponte de Wheatstone\\
• Circuito básico da ponte de Kelvin Teoremas de Norton e Millman Conceito de fonte de corrente\\
• Aplicação do teorema de Norton em análise de circuitos cc\\
• Aplicação do teorema de Millman em análise de circuitos CC Teoremas da máxima transferência de energia\\
• Aplicação em análise de circuitos CC\\
}

\metodologiaEnsino{
Aulas expositivas.\\
Leitura e pesquisa.\\
Resolução de lista de exercícios.\\
}

\recursos{
Livros contidos na bibliografia.\\
Quadro e pincel.\\
Data-show.\\
Lista de exercícios.\\
}

\avaliacao{
Avaliação escrita.\\
Avaliação de exercícios resolvidos.\\
Poderão ser inseridas outras avaliações durante o semestre.\\
}

\bibliografiaBasica{
• BOYLESTAD, Robert L. Introdução à análise de circuitos. São Paulo: Pearson, 2004.\\
• BOYLESTAD, Robert L. Introdução à análise de circuitos 1. 12 ed. São Paulo: Pearson, 2012.\\
• BURIAN Jr., Yaro; Lyra, CAVALCANTI, Ana Cristina. Circuitos elétricos 2. São Paulo: Pearson, 2006. Notas 1 e 2: disponíveis na Biblioteca Virtual Universitária - Link: http://bvu.ifce.edu.br/login.php\\

}

\bibliografiaComplementar{
• CAPUANO, Francisco Gabriel. Laboratório de eletricidade e eletrônica. São Paulo: Érica, 2010.\\
• MENDONÇA, Roberlam Gonçalves; SILVA, Rui Vagner Rodrigues. Eletricidade básica. Curitiba:Livro Técnico, 2010.\\
• CLOSSE, Charles M. Circuitos lineares. Rio de Janeiro: LTC, 1975.\\
• GUSOW, Milton. Eletricidade básica. São Paulo: McGraw-Hill, 1997.\\
• WOLSKI, Belmiro. Eletricidade básica. Curitiba: Editora do Livro Técnico, 2010.\\
}

\imprimirPUD

\end{document}