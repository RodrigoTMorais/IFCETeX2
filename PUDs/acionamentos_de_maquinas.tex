\input{preambulo}
%arquivo de template para os PUDS

%definição das variaveis das seções
\newcommand{\disciplina}{\def \disciplina}
\newcommand{\imprimirdisciplina}{\disciplina}

\newcommand{\codigo}{\def \codigo}
\newcommand{\imprimircodigo}{\codigo}

\newcommand{\cargaHorariaTotal}{\def \cargaHorariaTotal}
\newcommand{\imprimircargaHorariaTotal}{\cargaHorariaTotal}

\newcommand{\cargaHorariaPratica}{\def \cargaHorariaPratica}
\newcommand{\imprimircargaHorariaPratica}{\cargaHorariaPratica}

\newcommand{\cargaHorariaTeorica}{\def \cargaHorariaTeorica}
\newcommand{\imprimircargaHorariaTeorica}{\cargaHorariaTeorica}

\newcommand{\creditos}{\def \creditos}
\newcommand{\imprimircreditos}{\creditos}

\newcommand{\codigoPrerequisitos}{\def \codigoPrerequisitos}
\newcommand{\imprimircodigoPrerequisitos}{\codigoPrerequisitos}

\newcommand{\semestre}{\def \semestre}
\newcommand{\imprimirsemestre}{\semestre}

\newcommand{\nivel}{\def \nivel}
\newcommand{\imprimirnivel}{\nivel}

\newcommand{\codigoEquivalencias}{\def \codigoEquivalencias}
\newcommand{\imprimircodigoEquivalencias}{\codigoEquivalencias}

\newcommand{\ementa}{\def \ementa}
\newcommand{\imprimirementa}{\ementa}

\newcommand{\objetivo}{\def \objetivo}
\newcommand{\imprimirobjetivo}{\objetivo}

\newcommand{\programa}{\def \programa}
\newcommand{\imprimirprograma}{\programa}

\newcommand{\metodologiaEnsino}{\def \metodologiaEnsino}
\newcommand{\imprimirmetodologiaEnsino}{\metodologiaEnsino}

\newcommand{\recursos}{\def \recursos}
\newcommand{\imprimirrecursos}{\recursos}

\newcommand{\avaliacao}{\def \avaliacao}
\newcommand{\imprimiravaliacao}{\avaliacao}

\newcommand{\bibliografiaBasica}{\def \bibliografiaBasica}
\newcommand{\imprimirbibliografiaBasica}{\bibliografiaBasica}

\newcommand{\bibliografiaComplementar}{\def \bibliografiaComplementar}
\newcommand{\imprimirbibliografiaComplementar}{\bibliografiaComplementar}

\newcommand{\versao}{\def \versao}
\newcommand{\imprimirversao}{\versao}


%comando de impressão da estrutura
\newcommand{\imprimirPUD}{
%Cabeçalho do PUD
\begin{Spacing}{1}

\noindent \begin{minipage}{2.5cm}%
\includegraphics[scale=0.12]{logo-ifce}
\end{minipage}
\hspace{0.3cm}
\begin{minipage}{13cm}%
\centering INSTITUTO FEDERAL DE EDUCAÇÃO, CIÊNCIA E TECNOLOGIA DO CEARÁ- IFCE\\
CAMPUS JUAZEIRO DO NORTE\\
CURSO SUPERIOR EM AUTOMAÇÃO INDUSTRIAL\\
PROGRAMA DE UNIDADE DIDÁTICA – PUD\\
\end{minipage}%
\end{Spacing}

\begin{longtable}{|p{14cm}|}
%primeiro cabeçalho
\hline
\rowcolor{lightgray}
\multicolumn{1}{p{14cm}}{\textbf{Disciplina: \imprimirdisciplina}}\\
\hline
\endfirsthead

%cabeçalho
\hline
continuação PUD \imprimirdisciplina\\
\hline
\endhead

\hline
continua...\\
\hline
\endfoot

\hline
\rowcolor{lightgray}

\begin{tabular}{p{5.5 cm}| l}
coordenação & departamento pedagogico\\[16 ex]
\end{tabular}\\

\hline

\endlastfoot

%elementos
\textbf{Código:} \imprimircodigo\\


\textbf{Carga Horária } Teórica: \imprimircargaHorariaTeorica, Prática \imprimircargaHorariaPratica, Total: \imprimircargaHorariaTotal\\


\textbf{Número de créditos:} \imprimircreditos\\


\textbf{Código pré-requisitos:} \imprimircodigoPrerequisitos\\


\textbf{Semestre:} \imprimirsemestre\\


\textbf{Nível:} \imprimirnivel\\
\hline

\rowcolor{lightgray}
\multicolumn{1}{|p{14cm}|}{\textbf{Ementa}}\\
\hline
\multicolumn{1}{|p{14cm}|}{\imprimirementa}\\
\hline

\rowcolor{lightgray}
\multicolumn{1}{|p{14cm}|}{\textbf{Objetivo}}\\
\hline
\imprimirobjetivo\\
\hline

\rowcolor{lightgray}
\multicolumn{1}{|p{14cm}|}{\textbf{Programa}}\\
\hline
\imprimirprograma\\
\hline

\rowcolor{lightgray}
\multicolumn{1}{|p{14cm}|}{\textbf{Metodologia de ensino}}\\
\hline
\imprimirmetodologiaEnsino\\
\hline

\rowcolor{lightgray}
\multicolumn{1}{|p{14cm}|}{\textbf{Recursos}}\\
\hline
\imprimirrecursos\\
\hline

\rowcolor{lightgray}
\multicolumn{1}{|p{14cm}|}{\textbf{Avaliação}}\\
\hline
\imprimiravaliacao\\
\hline

\rowcolor{lightgray}
\multicolumn{1}{|p{14cm}|}{\textbf{Bibliografia básica}}\\
\hline
\imprimirbibliografiaBasica\\
\hline

\rowcolor{lightgray}
\multicolumn{1}{|p{14cm}|}{\textbf{Bibliografia complementar}}\\
\hline
\imprimirbibliografiaComplementar\\
\hline
\end{longtable}
\pagebreak
}
\begin{document}

\disciplina{Acionamentos de Máquinas}
\codigo{AUT2431}
\cargaHorariaTotal{80}
\cargaHorariaPratica{20}
\cargaHorariaTeorica{30}
\creditos{4}
\codigoPrerequisitos{AUT2425}
\semestre{6º}
\nivel{Superior}

\ementa{
Componentes: tiristores (triac, diac, SCR, Mosfet, GTO, IGBT); esquemas na área de eletrônica de potência; conversores de tensão CC/CC não isolados; conversores de tensão CC/CA (Inversores); tipos de controle de velocidade de motor C.A. e C.C.
}

\objetivo{
• Projetar conversores de tensão C.C./C.C. não isolados utilizando software dedicado Simular circuitos utilizando o P-Spice versão estudante\\
• Simular conversores C.C./C.C. e conversores C.C./C.A.\\
}

\programa{
• Dispositivos de Potência Tiristores\\
• Triac Diac SCR\\
• Transistor Bipolar de Potência MOSFET de Potência\\
• GTO – Gate Turn Off IGBT\\
• Circuitos para Disparos de Tiristores Tipos de Disparos\\
• Transformadores de Pulso Acopladores Ópticos Circuitos Integrados\\
• Software de Simulação PSPICE (Versão Estudante) Principio de Funcionamento\\
• Desenho dos Esquemas Elétricos\\
• Configuração dos parâmetros de Simulação Interpretação dos Dados de Simulação.\\
• Conversores Estáticos\\
• Conversores C.C./C.C não Isolados Elevador de Tensão - Boost Abaixador de Tensão - Buck Conversores C.C./C.A. (Inversores) Push-Pull\\
• Meia Ponte Monofásica\\
• Ponte Inversora Monofásica Ponte Inversora Trifásica Inversor com transformador\\
• Técnicas de Modulação Controle PWM\\
• Modulação em Frequência Variação de TON e T\\
• Controle de Velocidade do Motor C.A. Considerações Básicas sobre Motor de Indução\\
• Formas de Controle de Velocidade do Motor de Indução\\
• Cuidados na Utilização de Conversores para Acionamento de Motores de Indução Tipos de Frenagem do Motor de Indução\\
• Aplicações para o Controle de Velocidade de Motores de Indução Controle de  Velocidade do Motor C.C.\\
• Considerações Básicas sobre o Motor C.C. Equações Básicas do Motor C.C.Independente Considerações sobre o Controle de Velocidade Formas de Controle de Velocidade do Motor C.C.\\
• Controle de Velocidade através da Tensão de Campo ou Excitação Controle de Velocidade através da Tensão de Armadura\\
• Controle Misto de Velocidade Tipos de Parada do Motor C.C. Parada por Inércia\\
• Parada por Frenagem\\
• Frenagem Resistiva Frenagem Regenerativa\\
• Quadrantes de Operação da Máquina C.C. Acionamento em 1 Quadrante Acionamento em 2 Quadrantes\\
• Acionamento em 4 Quadrantes\\
}

\metodologiaEnsino{
Aulas expositivas;\\
Aulas práticas em laboratório; \\
Lista de exercícios;\\
Simulação computacional utilizando software dedicado.\\
}

\recursos{
Livros contidos na bibliografia;\\ 
Quadro e pincel.\\
Data-show\\
}

\avaliacao{
Avaliação escrita;\\
Práticas individuais e em grupo no laboratório;\\
Relatório de prática;\\
Listas de exercícios;\\
Poderão ser inseridas outras avaliações durante o semestre.\\
}

\bibliografiaBasica{
• BARBI, Ivo; MARTINS, Denizar Cruz. Eletrônica de potência: conversores CC-CC básicos não isolados. Florianópolis: Edição do Autor, 2000.\\

• BARBI, Ivo. Conversores CC-CC isolados em alta frequência com comutação suave. Florianópolis. Edição dos Autores, 1999.\\

• FRANCHI, C. M. Inversores de Frequência: teoria e aplicação. São Paulo: Érica. 2009.
}

\bibliografiaComplementar{
• BIM, Edson. Máquinas elétricas e acionamento. Rio de Janeiro: Elsevier, 2009.\\

• FRANCHI, C. M. Acionamentos elétricos. São Paulo: Érica. 2011.\\

• Ahmed, Ashfaq. Eletrônica de potência19. São Paulo: Pearson: 2000.\\

• LANDER, Cyril W. Eletrônica industrial: teoria e aplicações. 2 ed. São Paulo: Makron Books, 1996. BARBI, Ivo. Eletrônica de potência. Florianópolis: Edição do Autor, 2002.\\

• FITZGERALD, A. E; KINGSLEY Jr, Charles. Máquinas elétricas. Porto Alegre: Bookman, 2006. KOSOW, Irving L. Máquinas elétricas e transformadores. São Paulo: Globo, 2005.\\
}


\imprimirPUD

\end{document}