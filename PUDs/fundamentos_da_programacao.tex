\input{preambulo}
%arquivo de template para os PUDS

%definição das variaveis das seções
\newcommand{\disciplina}{\def \disciplina}
\newcommand{\imprimirdisciplina}{\disciplina}

\newcommand{\codigo}{\def \codigo}
\newcommand{\imprimircodigo}{\codigo}

\newcommand{\cargaHorariaTotal}{\def \cargaHorariaTotal}
\newcommand{\imprimircargaHorariaTotal}{\cargaHorariaTotal}

\newcommand{\cargaHorariaPratica}{\def \cargaHorariaPratica}
\newcommand{\imprimircargaHorariaPratica}{\cargaHorariaPratica}

\newcommand{\cargaHorariaTeorica}{\def \cargaHorariaTeorica}
\newcommand{\imprimircargaHorariaTeorica}{\cargaHorariaTeorica}

\newcommand{\creditos}{\def \creditos}
\newcommand{\imprimircreditos}{\creditos}

\newcommand{\codigoPrerequisitos}{\def \codigoPrerequisitos}
\newcommand{\imprimircodigoPrerequisitos}{\codigoPrerequisitos}

\newcommand{\semestre}{\def \semestre}
\newcommand{\imprimirsemestre}{\semestre}

\newcommand{\nivel}{\def \nivel}
\newcommand{\imprimirnivel}{\nivel}

\newcommand{\codigoEquivalencias}{\def \codigoEquivalencias}
\newcommand{\imprimircodigoEquivalencias}{\codigoEquivalencias}

\newcommand{\ementa}{\def \ementa}
\newcommand{\imprimirementa}{\ementa}

\newcommand{\objetivo}{\def \objetivo}
\newcommand{\imprimirobjetivo}{\objetivo}

\newcommand{\programa}{\def \programa}
\newcommand{\imprimirprograma}{\programa}

\newcommand{\metodologiaEnsino}{\def \metodologiaEnsino}
\newcommand{\imprimirmetodologiaEnsino}{\metodologiaEnsino}

\newcommand{\recursos}{\def \recursos}
\newcommand{\imprimirrecursos}{\recursos}

\newcommand{\avaliacao}{\def \avaliacao}
\newcommand{\imprimiravaliacao}{\avaliacao}

\newcommand{\bibliografiaBasica}{\def \bibliografiaBasica}
\newcommand{\imprimirbibliografiaBasica}{\bibliografiaBasica}

\newcommand{\bibliografiaComplementar}{\def \bibliografiaComplementar}
\newcommand{\imprimirbibliografiaComplementar}{\bibliografiaComplementar}

\newcommand{\versao}{\def \versao}
\newcommand{\imprimirversao}{\versao}


%comando de impressão da estrutura
\newcommand{\imprimirPUD}{
%Cabeçalho do PUD
\begin{Spacing}{1}

\noindent \begin{minipage}{2.5cm}%
\includegraphics[scale=0.12]{logo-ifce}
\end{minipage}
\hspace{0.3cm}
\begin{minipage}{13cm}%
\centering INSTITUTO FEDERAL DE EDUCAÇÃO, CIÊNCIA E TECNOLOGIA DO CEARÁ- IFCE\\
CAMPUS JUAZEIRO DO NORTE\\
CURSO SUPERIOR EM AUTOMAÇÃO INDUSTRIAL\\
PROGRAMA DE UNIDADE DIDÁTICA – PUD\\
\end{minipage}%
\end{Spacing}

\begin{longtable}{|p{14cm}|}
%primeiro cabeçalho
\hline
\rowcolor{lightgray}
\multicolumn{1}{p{14cm}}{\textbf{Disciplina: \imprimirdisciplina}}\\
\hline
\endfirsthead

%cabeçalho
\hline
continuação PUD \imprimirdisciplina\\
\hline
\endhead

\hline
continua...\\
\hline
\endfoot

\hline
\rowcolor{lightgray}

\begin{tabular}{p{5.5 cm}| l}
coordenação & departamento pedagogico\\[16 ex]
\end{tabular}\\

\hline

\endlastfoot

%elementos
\textbf{Código:} \imprimircodigo\\


\textbf{Carga Horária } Teórica: \imprimircargaHorariaTeorica, Prática \imprimircargaHorariaPratica, Total: \imprimircargaHorariaTotal\\


\textbf{Número de créditos:} \imprimircreditos\\


\textbf{Código pré-requisitos:} \imprimircodigoPrerequisitos\\


\textbf{Semestre:} \imprimirsemestre\\


\textbf{Nível:} \imprimirnivel\\
\hline

\rowcolor{lightgray}
\multicolumn{1}{|p{14cm}|}{\textbf{Ementa}}\\
\hline
\multicolumn{1}{|p{14cm}|}{\imprimirementa}\\
\hline

\rowcolor{lightgray}
\multicolumn{1}{|p{14cm}|}{\textbf{Objetivo}}\\
\hline
\imprimirobjetivo\\
\hline

\rowcolor{lightgray}
\multicolumn{1}{|p{14cm}|}{\textbf{Programa}}\\
\hline
\imprimirprograma\\
\hline

\rowcolor{lightgray}
\multicolumn{1}{|p{14cm}|}{\textbf{Metodologia de ensino}}\\
\hline
\imprimirmetodologiaEnsino\\
\hline

\rowcolor{lightgray}
\multicolumn{1}{|p{14cm}|}{\textbf{Recursos}}\\
\hline
\imprimirrecursos\\
\hline

\rowcolor{lightgray}
\multicolumn{1}{|p{14cm}|}{\textbf{Avaliação}}\\
\hline
\imprimiravaliacao\\
\hline

\rowcolor{lightgray}
\multicolumn{1}{|p{14cm}|}{\textbf{Bibliografia básica}}\\
\hline
\imprimirbibliografiaBasica\\
\hline

\rowcolor{lightgray}
\multicolumn{1}{|p{14cm}|}{\textbf{Bibliografia complementar}}\\
\hline
\imprimirbibliografiaComplementar\\
\hline
\end{longtable}
\pagebreak
}
\begin{document}

\disciplina{Fundamentos da programacao}
\codigo{AUT2404}
\cargaHorariaTotal{80}
\cargaHorariaPratica{50}
\cargaHorariaTeorica{30}
\creditos{4}
\codigoPrerequisitos{-}
\semestre{1º}
\nivel{superior}

\ementa{
Técnicas para construção de fluxogramas. Aplicar Técnicas para construção de algoritmos estruturados. Estruturas de dados, decisão e repetição em Portugol. Aplicar Modularização para construção de programas.
}

\objetivo{
• Conhecer técnicas de lógica de programação.\\
• Desenvolver algoritmos em linguagem Portugol , utilizando matrizes, registros, sub-rotinas e funções.\\
}

\programa{
• Introdução a programação abordagem algorítmica (Portugol)\\
• Algoritmos não computacionais\\
• Formas de apresentação\\
• Fluxograma\\
• Diagrama Estruturado\\
• Portugol\\
• Tipos de dados\\
• Variáveis, Constantes e Expressões\\
• Nomes de variáveis\\
• Declaração e atribuição de variáveis e constantes\\
• Operadores Aritméticos e Lógicos\\
• Expressões Aritméticas e Lógicas\\
• Comandos de Entrada e Saída\\
• Estruturas de Decisão\\
• Construção SE-ENTÃO\\
• SE Aninhados\\
• Construção ESCOLHA-CASO\\
• Estruturas de Repetição\\
• Laços de Repetição com teste no início ( ENQUANTO)\\
• Laços de Repetição com teste no final ( REPITA-ATÉ)\\
• Laços de Repetição com variável de controle (PARA)\\
• Laços Aninhados\\
• Estrutura de Dados\\
• Vetores\\
• Matrizes c. Registros\\
• Modularização\\
• Conceitos Básicos de Sub-rotinas e Funções\\
}

\metodologiaEnsino{
Aulas expositivas.\\
Leitura e pesquisa\\
Aulas práticas em laboratório de informática.\\
Resolução de exercícios utilizando software apropriado.\\
}

\recursos{
Utilização de Laboratório de Informática\\
Livros contidos na bibliografia\\
Quadro e pincel\\
Data-show\\
Lista de exercícios\\
}

\avaliacao{
Avaliação escrita.\\
Resolução individual ou em grupo de algoritmos no software apropriado.\\
Avaliação de exercícios resolvidos.\\
Poderão ser inseridas outras avaliações durante o semestre.\\
}

\bibliografiaBasica{
• FORBELLONE, André Luiz Villar; EBERSPACHER, Henri Frederico. Lógica de Programação6: a construção de algoritmos e estruturas de dados. 3 ed. São Paulo: Pearson, 2005.\\
• MANZANO, José Augusto N. G.; OLIVEIRA, Jayr Figueiredo. Algoritmos: lógica para desenvolvimento de programação de computadores. São Paulo: Érica, 2012.\\
• LOPES, Anita; GARCIA, Guto. Introdução à programação: 500 algoritmos resolvidos.Rio de Janeiro: Campus, 2002.\\
• SALMON, Wesley C. Lógica. Rio de Janeiro: Livros Técnicos e Científicos, 2002.\\
}

\bibliografiaComplementar{
• AGUILAR, Luis Joyanes. Fundamentos de programação: algoritmos, estruturas de dados e objetos. São Paulo: McGraw-Hill, 2008.\\
• CARBONI, Irenice de Fátima. Lógica de programação. São Paulo: Píoneira Thomson Learning, 2003.\\
• GUIMARÃES, Ângelo de Moura; LAGES, Newton Alberto de Castilho. Algoritmos e estruturas de dados. Rio de Janeiro: LTC, 1994.\\
• SEBESTA, Robert W. Conceitos de linguagens de programação. Porto Alegre: Bookman, 2003.\\
}


\imprimirPUD

\end{document}