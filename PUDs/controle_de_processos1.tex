\input{preambulo}
%arquivo de template para os PUDS

%definição das variaveis das seções
\newcommand{\disciplina}{\def \disciplina}
\newcommand{\imprimirdisciplina}{\disciplina}

\newcommand{\codigo}{\def \codigo}
\newcommand{\imprimircodigo}{\codigo}

\newcommand{\cargaHorariaTotal}{\def \cargaHorariaTotal}
\newcommand{\imprimircargaHorariaTotal}{\cargaHorariaTotal}

\newcommand{\cargaHorariaPratica}{\def \cargaHorariaPratica}
\newcommand{\imprimircargaHorariaPratica}{\cargaHorariaPratica}

\newcommand{\cargaHorariaTeorica}{\def \cargaHorariaTeorica}
\newcommand{\imprimircargaHorariaTeorica}{\cargaHorariaTeorica}

\newcommand{\creditos}{\def \creditos}
\newcommand{\imprimircreditos}{\creditos}

\newcommand{\codigoPrerequisitos}{\def \codigoPrerequisitos}
\newcommand{\imprimircodigoPrerequisitos}{\codigoPrerequisitos}

\newcommand{\semestre}{\def \semestre}
\newcommand{\imprimirsemestre}{\semestre}

\newcommand{\nivel}{\def \nivel}
\newcommand{\imprimirnivel}{\nivel}

\newcommand{\codigoEquivalencias}{\def \codigoEquivalencias}
\newcommand{\imprimircodigoEquivalencias}{\codigoEquivalencias}

\newcommand{\ementa}{\def \ementa}
\newcommand{\imprimirementa}{\ementa}

\newcommand{\objetivo}{\def \objetivo}
\newcommand{\imprimirobjetivo}{\objetivo}

\newcommand{\programa}{\def \programa}
\newcommand{\imprimirprograma}{\programa}

\newcommand{\metodologiaEnsino}{\def \metodologiaEnsino}
\newcommand{\imprimirmetodologiaEnsino}{\metodologiaEnsino}

\newcommand{\recursos}{\def \recursos}
\newcommand{\imprimirrecursos}{\recursos}

\newcommand{\avaliacao}{\def \avaliacao}
\newcommand{\imprimiravaliacao}{\avaliacao}

\newcommand{\bibliografiaBasica}{\def \bibliografiaBasica}
\newcommand{\imprimirbibliografiaBasica}{\bibliografiaBasica}

\newcommand{\bibliografiaComplementar}{\def \bibliografiaComplementar}
\newcommand{\imprimirbibliografiaComplementar}{\bibliografiaComplementar}

\newcommand{\versao}{\def \versao}
\newcommand{\imprimirversao}{\versao}


%comando de impressão da estrutura
\newcommand{\imprimirPUD}{
%Cabeçalho do PUD
\begin{Spacing}{1}

\noindent \begin{minipage}{2.5cm}%
\includegraphics[scale=0.12]{logo-ifce}
\end{minipage}
\hspace{0.3cm}
\begin{minipage}{13cm}%
\centering INSTITUTO FEDERAL DE EDUCAÇÃO, CIÊNCIA E TECNOLOGIA DO CEARÁ- IFCE\\
CAMPUS JUAZEIRO DO NORTE\\
CURSO SUPERIOR EM AUTOMAÇÃO INDUSTRIAL\\
PROGRAMA DE UNIDADE DIDÁTICA – PUD\\
\end{minipage}%
\end{Spacing}

\begin{longtable}{|p{14cm}|}
%primeiro cabeçalho
\hline
\rowcolor{lightgray}
\multicolumn{1}{p{14cm}}{\textbf{Disciplina: \imprimirdisciplina}}\\
\hline
\endfirsthead

%cabeçalho
\hline
continuação PUD \imprimirdisciplina\\
\hline
\endhead

\hline
continua...\\
\hline
\endfoot

\hline
\rowcolor{lightgray}

\begin{tabular}{p{5.5 cm}| l}
coordenação & departamento pedagogico\\[16 ex]
\end{tabular}\\

\hline

\endlastfoot

%elementos
\textbf{Código:} \imprimircodigo\\


\textbf{Carga Horária } Teórica: \imprimircargaHorariaTeorica, Prática \imprimircargaHorariaPratica, Total: \imprimircargaHorariaTotal\\


\textbf{Número de créditos:} \imprimircreditos\\


\textbf{Código pré-requisitos:} \imprimircodigoPrerequisitos\\


\textbf{Semestre:} \imprimirsemestre\\


\textbf{Nível:} \imprimirnivel\\
\hline

\rowcolor{lightgray}
\multicolumn{1}{|p{14cm}|}{\textbf{Ementa}}\\
\hline
\multicolumn{1}{|p{14cm}|}{\imprimirementa}\\
\hline

\rowcolor{lightgray}
\multicolumn{1}{|p{14cm}|}{\textbf{Objetivo}}\\
\hline
\imprimirobjetivo\\
\hline

\rowcolor{lightgray}
\multicolumn{1}{|p{14cm}|}{\textbf{Programa}}\\
\hline
\imprimirprograma\\
\hline

\rowcolor{lightgray}
\multicolumn{1}{|p{14cm}|}{\textbf{Metodologia de ensino}}\\
\hline
\imprimirmetodologiaEnsino\\
\hline

\rowcolor{lightgray}
\multicolumn{1}{|p{14cm}|}{\textbf{Recursos}}\\
\hline
\imprimirrecursos\\
\hline

\rowcolor{lightgray}
\multicolumn{1}{|p{14cm}|}{\textbf{Avaliação}}\\
\hline
\imprimiravaliacao\\
\hline

\rowcolor{lightgray}
\multicolumn{1}{|p{14cm}|}{\textbf{Bibliografia básica}}\\
\hline
\imprimirbibliografiaBasica\\
\hline

\rowcolor{lightgray}
\multicolumn{1}{|p{14cm}|}{\textbf{Bibliografia complementar}}\\
\hline
\imprimirbibliografiaComplementar\\
\hline
\end{longtable}
\pagebreak
}
\begin{document}

\disciplina{Controle de processos 1}
\codigo{AUT2426}
\cargaHorariaTotal{80}
\cargaHorariaPratica{20}
\cargaHorariaTeorica{60}
\creditos{4}
\codigoPrerequisitos{AUT2407}
\semestre{5º}
\nivel{Superior}

\ementa{
Introdução aos sistemas de controle. Tipos de sistemas de controle. Controle em malha aberta, controle em malha fechada, modelagem matemática de sistemas dinâmicos, transformada de Laplace, transformada inversa de Laplace pelo método da expansão em frações parciais, solução de equações diferenciais lineares e invariantes no tempo, função de transferência.
}

\objetivo{
• Apresentar uma introdução aos sistemas de controle;\\

• Conhecer aplicações da transformada de Laplace de funções de tempo, frequentemente utilizadas em engenharia de controle;\\

• Abordar a modelagem matemática de sistemas dinâmicos (em especial de sistemas elétricos, eletrônicos e sistemas mecânicos);\\

• Utilizar ferramentas de modelagem na resolução de equações diferenciais lineares invariantes no tempo Determinar a função de transferência de sistemas de equações diferenciais lineares invariantes no tempo.\\
}

\programa{
• O Controle Industrial Histórico Terminologia\\
• Tipos de sistemas de controle Sistemas de controle realimentados Servossistemas\\
• Sistemas reguladores automáticos Sistemas de controle de processos Sistemas de controle em malha fechada Sistemas de controle em malha aberta\\
• Sistemas de controle em malha fechada X malha aberta Sistemas de controle lineares X não-lineares\\
• Exemplos de sistemas de controle A Transformada de Laplace\\
• Revisão de variáveis e funções complexas Teorema de Euler\\
• Definição\\

• Existência da transformada de Laplace\\

• Exemplos de transformadas de funções importantes Tabela de transformadas de Laplace\\
• Propriedades da transformada de Laplace Teoremas a cerca da transformada de Laplace A transformação inversa de Laplace\\
• Método da expansão em frações parciais para a determinação das transformadas inversas de Laplace Expansão em frações parciais quando a transformada envolve apenas pólos distintos\\
• Aplicações da transformada e transformada inversa de Laplace na resolução de circuitos elétricos \\
• Resolução genérica de circuito RC\\
• Resolução genérica de circuito RL\\
• Resolução de circuitos em regime senoidal\\

• Resolução de equações diferenciais lineares invariantes no tempo Função de transferência\\
• Definição Comentários\\
• Aplicação em sistemas físicos\\
}

\metodologiaEnsino{
Aulas expositivas;\\
Lista de exercícios;\\
Simulação computacional utilizando software dedicado.\\
}

\recursos{
Livros contidos na bibliografia;\\
Quadro e pincel.\\
Data-show\\
}

\avaliacao{
Avaliação escrita;\\
Práticas individuais e em grupo no laboratório;\\
Listas de exercícios;\\
Poderão ser inseridas outras avaliações durante o semestre.\\
}

\bibliografiaBasica{
• OGATA, Katsuhiko. Engenharia de controle moderno. Rio de Janeiro: Prentice Hall do Brasil, 2010.\\

• NISE, Norman S. Engenharia de sistemas de controle. Rio de Janeiro: LTC. 2018.\\

• DORF, Richard C.; BISCHOP, Robert H. Sistemas de controle modernos. Rio de Janeiro: LTC, 2018.\\
}

\bibliografiaComplementar{
• SILVEIRA, Paulo R. da; SANTOS, Winderson E. Automação e controle discreto. São Paulo: Érica, 2007.\\

• SPIEGEL, Murray R. Transformadas de ornas: 263   problemas resolvidos,	614 problemas propostos.  São Paulo: Makon Books, 1971.\\

• CARVALHO, J. L. Martins de. Sistemas de controle automático. Rio de Janeiro: LTC, 2000.\\

• PHILLIPS, Charles L.; HARBOR, Royce D. Sistemas de controle e realimentação. São Paulo: Makron Books, 1996.\\

• ALBERTOS Perez, P.; Sala, Antonio. Multivariable Control Systems: An 
Engineering Approach. Springer, 2004. \\
}


\imprimirPUD

\end{document}