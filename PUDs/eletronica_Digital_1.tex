\input{preambulo}
%arquivo de template para os PUDS

%definição das variaveis das seções
\newcommand{\disciplina}{\def \disciplina}
\newcommand{\imprimirdisciplina}{\disciplina}

\newcommand{\codigo}{\def \codigo}
\newcommand{\imprimircodigo}{\codigo}

\newcommand{\cargaHorariaTotal}{\def \cargaHorariaTotal}
\newcommand{\imprimircargaHorariaTotal}{\cargaHorariaTotal}

\newcommand{\cargaHorariaPratica}{\def \cargaHorariaPratica}
\newcommand{\imprimircargaHorariaPratica}{\cargaHorariaPratica}

\newcommand{\cargaHorariaTeorica}{\def \cargaHorariaTeorica}
\newcommand{\imprimircargaHorariaTeorica}{\cargaHorariaTeorica}

\newcommand{\creditos}{\def \creditos}
\newcommand{\imprimircreditos}{\creditos}

\newcommand{\codigoPrerequisitos}{\def \codigoPrerequisitos}
\newcommand{\imprimircodigoPrerequisitos}{\codigoPrerequisitos}

\newcommand{\semestre}{\def \semestre}
\newcommand{\imprimirsemestre}{\semestre}

\newcommand{\nivel}{\def \nivel}
\newcommand{\imprimirnivel}{\nivel}

\newcommand{\codigoEquivalencias}{\def \codigoEquivalencias}
\newcommand{\imprimircodigoEquivalencias}{\codigoEquivalencias}

\newcommand{\ementa}{\def \ementa}
\newcommand{\imprimirementa}{\ementa}

\newcommand{\objetivo}{\def \objetivo}
\newcommand{\imprimirobjetivo}{\objetivo}

\newcommand{\programa}{\def \programa}
\newcommand{\imprimirprograma}{\programa}

\newcommand{\metodologiaEnsino}{\def \metodologiaEnsino}
\newcommand{\imprimirmetodologiaEnsino}{\metodologiaEnsino}

\newcommand{\recursos}{\def \recursos}
\newcommand{\imprimirrecursos}{\recursos}

\newcommand{\avaliacao}{\def \avaliacao}
\newcommand{\imprimiravaliacao}{\avaliacao}

\newcommand{\bibliografiaBasica}{\def \bibliografiaBasica}
\newcommand{\imprimirbibliografiaBasica}{\bibliografiaBasica}

\newcommand{\bibliografiaComplementar}{\def \bibliografiaComplementar}
\newcommand{\imprimirbibliografiaComplementar}{\bibliografiaComplementar}

\newcommand{\versao}{\def \versao}
\newcommand{\imprimirversao}{\versao}


%comando de impressão da estrutura
\newcommand{\imprimirPUD}{
%Cabeçalho do PUD
\begin{Spacing}{1}

\noindent \begin{minipage}{2.5cm}%
\includegraphics[scale=0.12]{logo-ifce}
\end{minipage}
\hspace{0.3cm}
\begin{minipage}{13cm}%
\centering INSTITUTO FEDERAL DE EDUCAÇÃO, CIÊNCIA E TECNOLOGIA DO CEARÁ- IFCE\\
CAMPUS JUAZEIRO DO NORTE\\
CURSO SUPERIOR EM AUTOMAÇÃO INDUSTRIAL\\
PROGRAMA DE UNIDADE DIDÁTICA – PUD\\
\end{minipage}%
\end{Spacing}

\begin{longtable}{|p{14cm}|}
%primeiro cabeçalho
\hline
\rowcolor{lightgray}
\multicolumn{1}{p{14cm}}{\textbf{Disciplina: \imprimirdisciplina}}\\
\hline
\endfirsthead

%cabeçalho
\hline
continuação PUD \imprimirdisciplina\\
\hline
\endhead

\hline
continua...\\
\hline
\endfoot

\hline
\rowcolor{lightgray}

\begin{tabular}{p{5.5 cm}| l}
coordenação & departamento pedagogico\\[16 ex]
\end{tabular}\\

\hline

\endlastfoot

%elementos
\textbf{Código:} \imprimircodigo\\


\textbf{Carga Horária } Teórica: \imprimircargaHorariaTeorica, Prática \imprimircargaHorariaPratica, Total: \imprimircargaHorariaTotal\\


\textbf{Número de créditos:} \imprimircreditos\\


\textbf{Código pré-requisitos:} \imprimircodigoPrerequisitos\\


\textbf{Semestre:} \imprimirsemestre\\


\textbf{Nível:} \imprimirnivel\\
\hline

\rowcolor{lightgray}
\multicolumn{1}{|p{14cm}|}{\textbf{Ementa}}\\
\hline
\multicolumn{1}{|p{14cm}|}{\imprimirementa}\\
\hline

\rowcolor{lightgray}
\multicolumn{1}{|p{14cm}|}{\textbf{Objetivo}}\\
\hline
\imprimirobjetivo\\
\hline

\rowcolor{lightgray}
\multicolumn{1}{|p{14cm}|}{\textbf{Programa}}\\
\hline
\imprimirprograma\\
\hline

\rowcolor{lightgray}
\multicolumn{1}{|p{14cm}|}{\textbf{Metodologia de ensino}}\\
\hline
\imprimirmetodologiaEnsino\\
\hline

\rowcolor{lightgray}
\multicolumn{1}{|p{14cm}|}{\textbf{Recursos}}\\
\hline
\imprimirrecursos\\
\hline

\rowcolor{lightgray}
\multicolumn{1}{|p{14cm}|}{\textbf{Avaliação}}\\
\hline
\imprimiravaliacao\\
\hline

\rowcolor{lightgray}
\multicolumn{1}{|p{14cm}|}{\textbf{Bibliografia básica}}\\
\hline
\imprimirbibliografiaBasica\\
\hline

\rowcolor{lightgray}
\multicolumn{1}{|p{14cm}|}{\textbf{Bibliografia complementar}}\\
\hline
\imprimirbibliografiaComplementar\\
\hline
\end{longtable}
\pagebreak
}
\begin{document}

\disciplina{Eletrônica digital 1}
\codigo{AUT2403}
\cargaHorariaTotal{80}
\cargaHorariaPratica{20}
\cargaHorariaTeorica{60}
\creditos{4}
\codigoPrerequisitos{-}
\semestre{1º}
\nivel{superior}
\ementa{
Métodos de conversão de um sistema de numeração (decimal, binário, octal e
hexadecimal) e suas operações (soma, subtração, multiplicação). Representação de Números decimais usando o código BCD. Compreender o propósito dos Códigos alfanuméricos, como o código ASCII. Compreender as Operações e funções lógicas básicas (AND, OR e NOT) e suas funções derivadas. Avaliar o Potencial da álgebra de Booleana (teoremas, propriedades e postulados) e mapa de Karnaugh na simplificação de circuito lógicos complexos. Conhecer as Características básicas de CI’s digitais TTL e CMOS. Analisar o Funcionamento de circuitos lógicos combinacionais. Compreender os Circuitos somadores e subtratores. e Projetar Projeto de circuitos lógicos simples.
}

\objetivo{
• Realizar conversões numéricas das bases decimal, octal, hexadecimal e binário para seuequivalente em qualquer outro sistema de numeração.\\
• Realizar as operações aritméticas nas bases decimal, hexadecimal, octal e binário.\\
• Desenhar e interpretar os símbolos de portas lógicas do padrão IEEE/ANSI.\\
• Implementar circuitos lógicos usando as portas básicas AND, OR e NOT.\\
• Executar os passos necessários para obter a forma mais simplificada de uma expressão lógica.\\
• Interpretar os estudos de casos na análise de defeitos de circuitos combinacionais.\\
• Usar somadores completos no projeto de somadores binários paralelos.\\
• Implementar circuitos lógicos combinacionais.\\
}

\programa{
• Códigos binários Sistemas de numeração\\
• Sistema ponderado, bases 10, 2, 8 e 16 Conversão entre bases Aritmética binária\\
• Adição binária Subtração binária Multiplicação binária Complemento de dez Complemento de dois\\
• Álgebra de Boole\\
• Variáveis lógicas Tabelas da verdade\\
• Funções de uma variável Funções de duas variáveis lógicas\\
• Funções lógicas básicas (OR, AND e NOT) Funções lógicas derivadas Portas lógicas\\
• Propriedades da Álgebra de Boole Teoremas de Morgan\\
• Diagramas de Venn\\
• Levantamento de expressões lógicas Síntese de Circuitos Lógicos\\
• Tabelas da verdade e soma de produtos\\
• Realização de expressões lógicas com portas AND, NAND, OR e NOT Análise de Circuitos Lógicos\\
• Circuitos integrados digitais\\
• Características da família CMOS Características da família TTL; Minimização de expressões lógicas: Mapas de Karnaugh\\
• Circuitos Somadores: Soma em complemento de 2; Soma em complemento de 1; Meio-Somadores Somadores Completos; Codificadores e Decodificadores: Conversores de códigos; Decodificador\\
BCD-7 segmentos;
}

\metodologiaEnsino{
Aulas expositivas.\\
Leitura e pesquisa.\\
Aulas práticas em laboratório de Informática - Simuladores.\\
Aulas práticas em laboratório – Sistemas Digitais.\\
Resolução de lista de exercícios.\\
Desenvolvimento de projetos: software e hardware.\\
}

\recursos{
Livros contidos na bibliografia.\\
Artigos.\\
Quadro e pincel.\\
Data-show.\\
Laboratório.\\
Computadores, dispositivos, equipamentos e softwares.\\
Lista de exercícios.\\
}

\avaliacao{
Avaliação escrita.\\
Práticas individuais e em grupo no laboratório.\\
Relatório de prática.\\
Avaliação de exercícios resolvidos.\\
Poderão ser inseridas outras avaliações durante o semestre.\\
}

\bibliografiaBasica{
• LOURENÇO, A. C.; CRUZ, E. C. A.; FERREIRA, S. R. e CHOURI, S. Jr. Circuitos
digitais: estude e use. São Paulo: Érica: 2007.\\
• TEIXEIRA, Hugo Tanzarella; TAVARES, Marley Fagundes; PEREIRA, Rodrigo Vinícius
Mendonça. Sistemas digitais. Londrina : Editora e Distribuidora Educacional S.A. 2017.\\
• WIDMER, Neal S.; MOSS, Gregory L.; TOCCI, Ronald J. Sistemas Digitais: princípios e aplicações. 12a. Edição. São Paulo: Pearson Education do Brasil, 2018.\\
}

\bibliografiaComplementar{
• IDOETA, Ivan Valeije; CAPUANO, Francisco Gabriel. Elementos de eletrônica digital. 39. ed. rev.atual. São Paulo: Érica, 2007.\\
• LEACH, Donald P. Eletrônica digital no laboratório. São Paulo: Makron Books,1993.\\
• MALVINO, A. P.; LEACH, D. P. Eletrônica digital: princípios e aplicações. McGraw- Hill, 1988.\\
• MENDONÇA, A.; ZELENOVSKY, R. Eletronica digital: curso prático e exercicios. 2a Edição. Rio de Janeiro: MZ EDITORA. 2007.\\
}

\imprimirPUD

\end{document}
