\input{preambulo}
%arquivo de template para os PUDS

%definição das variaveis das seções
\newcommand{\disciplina}{\def \disciplina}
\newcommand{\imprimirdisciplina}{\disciplina}

\newcommand{\codigo}{\def \codigo}
\newcommand{\imprimircodigo}{\codigo}

\newcommand{\cargaHorariaTotal}{\def \cargaHorariaTotal}
\newcommand{\imprimircargaHorariaTotal}{\cargaHorariaTotal}

\newcommand{\cargaHorariaPratica}{\def \cargaHorariaPratica}
\newcommand{\imprimircargaHorariaPratica}{\cargaHorariaPratica}

\newcommand{\cargaHorariaTeorica}{\def \cargaHorariaTeorica}
\newcommand{\imprimircargaHorariaTeorica}{\cargaHorariaTeorica}

\newcommand{\creditos}{\def \creditos}
\newcommand{\imprimircreditos}{\creditos}

\newcommand{\codigoPrerequisitos}{\def \codigoPrerequisitos}
\newcommand{\imprimircodigoPrerequisitos}{\codigoPrerequisitos}

\newcommand{\semestre}{\def \semestre}
\newcommand{\imprimirsemestre}{\semestre}

\newcommand{\nivel}{\def \nivel}
\newcommand{\imprimirnivel}{\nivel}

\newcommand{\codigoEquivalencias}{\def \codigoEquivalencias}
\newcommand{\imprimircodigoEquivalencias}{\codigoEquivalencias}

\newcommand{\ementa}{\def \ementa}
\newcommand{\imprimirementa}{\ementa}

\newcommand{\objetivo}{\def \objetivo}
\newcommand{\imprimirobjetivo}{\objetivo}

\newcommand{\programa}{\def \programa}
\newcommand{\imprimirprograma}{\programa}

\newcommand{\metodologiaEnsino}{\def \metodologiaEnsino}
\newcommand{\imprimirmetodologiaEnsino}{\metodologiaEnsino}

\newcommand{\recursos}{\def \recursos}
\newcommand{\imprimirrecursos}{\recursos}

\newcommand{\avaliacao}{\def \avaliacao}
\newcommand{\imprimiravaliacao}{\avaliacao}

\newcommand{\bibliografiaBasica}{\def \bibliografiaBasica}
\newcommand{\imprimirbibliografiaBasica}{\bibliografiaBasica}

\newcommand{\bibliografiaComplementar}{\def \bibliografiaComplementar}
\newcommand{\imprimirbibliografiaComplementar}{\bibliografiaComplementar}

\newcommand{\versao}{\def \versao}
\newcommand{\imprimirversao}{\versao}


%comando de impressão da estrutura
\newcommand{\imprimirPUD}{
%Cabeçalho do PUD
\begin{Spacing}{1}

\noindent \begin{minipage}{2.5cm}%
\includegraphics[scale=0.12]{logo-ifce}
\end{minipage}
\hspace{0.3cm}
\begin{minipage}{13cm}%
\centering INSTITUTO FEDERAL DE EDUCAÇÃO, CIÊNCIA E TECNOLOGIA DO CEARÁ- IFCE\\
CAMPUS JUAZEIRO DO NORTE\\
CURSO SUPERIOR EM AUTOMAÇÃO INDUSTRIAL\\
PROGRAMA DE UNIDADE DIDÁTICA – PUD\\
\end{minipage}%
\end{Spacing}

\begin{longtable}{|p{14cm}|}
%primeiro cabeçalho
\hline
\rowcolor{lightgray}
\multicolumn{1}{p{14cm}}{\textbf{Disciplina: \imprimirdisciplina}}\\
\hline
\endfirsthead

%cabeçalho
\hline
continuação PUD \imprimirdisciplina\\
\hline
\endhead

\hline
continua...\\
\hline
\endfoot

\hline
\rowcolor{lightgray}

\begin{tabular}{p{5.5 cm}| l}
coordenação & departamento pedagogico\\[16 ex]
\end{tabular}\\

\hline

\endlastfoot

%elementos
\textbf{Código:} \imprimircodigo\\


\textbf{Carga Horária } Teórica: \imprimircargaHorariaTeorica, Prática \imprimircargaHorariaPratica, Total: \imprimircargaHorariaTotal\\


\textbf{Número de créditos:} \imprimircreditos\\


\textbf{Código pré-requisitos:} \imprimircodigoPrerequisitos\\


\textbf{Semestre:} \imprimirsemestre\\


\textbf{Nível:} \imprimirnivel\\
\hline

\rowcolor{lightgray}
\multicolumn{1}{|p{14cm}|}{\textbf{Ementa}}\\
\hline
\multicolumn{1}{|p{14cm}|}{\imprimirementa}\\
\hline

\rowcolor{lightgray}
\multicolumn{1}{|p{14cm}|}{\textbf{Objetivo}}\\
\hline
\imprimirobjetivo\\
\hline

\rowcolor{lightgray}
\multicolumn{1}{|p{14cm}|}{\textbf{Programa}}\\
\hline
\imprimirprograma\\
\hline

\rowcolor{lightgray}
\multicolumn{1}{|p{14cm}|}{\textbf{Metodologia de ensino}}\\
\hline
\imprimirmetodologiaEnsino\\
\hline

\rowcolor{lightgray}
\multicolumn{1}{|p{14cm}|}{\textbf{Recursos}}\\
\hline
\imprimirrecursos\\
\hline

\rowcolor{lightgray}
\multicolumn{1}{|p{14cm}|}{\textbf{Avaliação}}\\
\hline
\imprimiravaliacao\\
\hline

\rowcolor{lightgray}
\multicolumn{1}{|p{14cm}|}{\textbf{Bibliografia básica}}\\
\hline
\imprimirbibliografiaBasica\\
\hline

\rowcolor{lightgray}
\multicolumn{1}{|p{14cm}|}{\textbf{Bibliografia complementar}}\\
\hline
\imprimirbibliografiaComplementar\\
\hline
\end{longtable}
\pagebreak
}
\begin{document}

\disciplina{Trabalho de conclusão de curso}
\codigo{AUT2441}
\cargaHorariaTotal{40}
\cargaHorariaPratica{0}
\cargaHorariaTeorica{40}
\creditos{2}
\codigoPrerequisitos{AUT2435}
\semestre{7º}
\nivel{Superior}

\ementa{
Desenvolvimento de Trabalho de Conclusão de Curso (nas modalidades previstas no PPC do Curso), considerando as orientações e sugestões das normas da Associação Brasileira de Normas Técnicas (ABNT), bem como o manual de normalização de trabalhos acadêmicos do IFCE e documentos internos ao IFCE, campus Juazeiro do Norte. Defesa pública e/ou apresentação do Trabalho de Conclusão de Curso.
}

\objetivo{
• Propiciar condições para que os alunos possam desenvolver seu Trabalho de Conclusão de Curso, considerando os princípios técnico-metodológicos do trabalho científico, e defendê-lo e/ou apresentá-lo publicamente \\
• Conhecer as etapas principais do processo de pesquisa científica; \\
• Identificar um problema e definir um objeto de estudo específico e relevante;\\
• Fazer planejamento de atividades de pesquisa; \\
• Conhecer a estrutura de: projeto de pesquisa; artigo científico; relatório técnico e/ou científico; \\
• Dominar os padrões de textualidade do texto científico, habilitando o aluno a redigir um trabalho científico (projetos, relatórios, artigos científicos, monografias e/ou teses) com organização, unidade, clareza e concisão;\\
• Construir o relatório de pesquisa científica; \\
• Aprender técnicas de apresentação de trabalho em público\\
}

\programa{
• NORMAS, SUGESTÕES E ORIENTAÇÕES PARA ELABORAÇÃO DO TCC \\
• Conceituação, definição e modalidades do Trabalho de Conclusão de Curso;\\
• Normas técnicas da ABNT; 1.3 Coleta e tabulação dos dados. 1.4 Análise dos dados: quantitativos e qualitativos.\\
• Apresentar os vários formatos de documentos científicos que poderão resultar no TCC, conforme PPC do curso.\\
• Pesquisa Bibliográfica: periódicos da CAPES – Acesso Cafe; Web of Science;\\ Revistas abertas da grande área Engenharia IV\\
• Oficina de escrita de documento científico: artigo científico, relatório técnico, monografia, etc\\
• Oficina de escrita de documento científico: Título, Resumo, Palavras-Chave\\
• Oficina de escrita de documento científico: Introdução, Referencial Teórico, Metodologia\\
• Oficina de escrita de documento científico: Discussões de Resultados e Conclusões\\
• Leitura e Discussão de Caminhos para escrita de artigo científico\\
• Apresentação de templates de TCC nos formatos: .docx,  .tex, dentre outros.\\ 

• DEFESA, CORREÇÃO E DEPÓSITO DO TCC – MONOGRAFIA, RELATÓRIO TÉCNICO \\
• Orientação para apresentação de relatórios de pesquisa científica.\\ 
• Defesa pública do Trabalho de Conclusão de Curso. \\
• Encaminhamento das correções do Trabalho de Conclusão de Curso, a partir das sugestões e/ou modificações apresentadas pela banca avaliadora.\\ 
• Entrega das cópias do Trabalho de Conclusão de Curso\\
• ARTIGOS PUBLICADOS EM PERIÓDICOS E CONGRESSOS \\
• Orientações para entrega do trabalho\\
}

\metodologiaEnsino{

}

\recursos{

}

\avaliacao{

}

\bibliografiaBasica{
• 
}

\bibliografiaComplementar{
• 
}


\imprimirPUD

\end{document}