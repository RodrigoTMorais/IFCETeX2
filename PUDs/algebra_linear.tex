\input{preambulo}
%arquivo de template para os PUDS

%definição das variaveis das seções
\newcommand{\disciplina}{\def \disciplina}
\newcommand{\imprimirdisciplina}{\disciplina}

\newcommand{\codigo}{\def \codigo}
\newcommand{\imprimircodigo}{\codigo}

\newcommand{\cargaHorariaTotal}{\def \cargaHorariaTotal}
\newcommand{\imprimircargaHorariaTotal}{\cargaHorariaTotal}

\newcommand{\cargaHorariaPratica}{\def \cargaHorariaPratica}
\newcommand{\imprimircargaHorariaPratica}{\cargaHorariaPratica}

\newcommand{\cargaHorariaTeorica}{\def \cargaHorariaTeorica}
\newcommand{\imprimircargaHorariaTeorica}{\cargaHorariaTeorica}

\newcommand{\creditos}{\def \creditos}
\newcommand{\imprimircreditos}{\creditos}

\newcommand{\codigoPrerequisitos}{\def \codigoPrerequisitos}
\newcommand{\imprimircodigoPrerequisitos}{\codigoPrerequisitos}

\newcommand{\semestre}{\def \semestre}
\newcommand{\imprimirsemestre}{\semestre}

\newcommand{\nivel}{\def \nivel}
\newcommand{\imprimirnivel}{\nivel}

\newcommand{\codigoEquivalencias}{\def \codigoEquivalencias}
\newcommand{\imprimircodigoEquivalencias}{\codigoEquivalencias}

\newcommand{\ementa}{\def \ementa}
\newcommand{\imprimirementa}{\ementa}

\newcommand{\objetivo}{\def \objetivo}
\newcommand{\imprimirobjetivo}{\objetivo}

\newcommand{\programa}{\def \programa}
\newcommand{\imprimirprograma}{\programa}

\newcommand{\metodologiaEnsino}{\def \metodologiaEnsino}
\newcommand{\imprimirmetodologiaEnsino}{\metodologiaEnsino}

\newcommand{\recursos}{\def \recursos}
\newcommand{\imprimirrecursos}{\recursos}

\newcommand{\avaliacao}{\def \avaliacao}
\newcommand{\imprimiravaliacao}{\avaliacao}

\newcommand{\bibliografiaBasica}{\def \bibliografiaBasica}
\newcommand{\imprimirbibliografiaBasica}{\bibliografiaBasica}

\newcommand{\bibliografiaComplementar}{\def \bibliografiaComplementar}
\newcommand{\imprimirbibliografiaComplementar}{\bibliografiaComplementar}

\newcommand{\versao}{\def \versao}
\newcommand{\imprimirversao}{\versao}


%comando de impressão da estrutura
\newcommand{\imprimirPUD}{
%Cabeçalho do PUD
\begin{Spacing}{1}

\noindent \begin{minipage}{2.5cm}%
\includegraphics[scale=0.12]{logo-ifce}
\end{minipage}
\hspace{0.3cm}
\begin{minipage}{13cm}%
\centering INSTITUTO FEDERAL DE EDUCAÇÃO, CIÊNCIA E TECNOLOGIA DO CEARÁ- IFCE\\
CAMPUS JUAZEIRO DO NORTE\\
CURSO SUPERIOR EM AUTOMAÇÃO INDUSTRIAL\\
PROGRAMA DE UNIDADE DIDÁTICA – PUD\\
\end{minipage}%
\end{Spacing}

\begin{longtable}{|p{14cm}|}
%primeiro cabeçalho
\hline
\rowcolor{lightgray}
\multicolumn{1}{p{14cm}}{\textbf{Disciplina: \imprimirdisciplina}}\\
\hline
\endfirsthead

%cabeçalho
\hline
continuação PUD \imprimirdisciplina\\
\hline
\endhead

\hline
continua...\\
\hline
\endfoot

\hline
\rowcolor{lightgray}

\begin{tabular}{p{5.5 cm}| l}
coordenação & departamento pedagogico\\[16 ex]
\end{tabular}\\

\hline

\endlastfoot

%elementos
\textbf{Código:} \imprimircodigo\\


\textbf{Carga Horária } Teórica: \imprimircargaHorariaTeorica, Prática \imprimircargaHorariaPratica, Total: \imprimircargaHorariaTotal\\


\textbf{Número de créditos:} \imprimircreditos\\


\textbf{Código pré-requisitos:} \imprimircodigoPrerequisitos\\


\textbf{Semestre:} \imprimirsemestre\\


\textbf{Nível:} \imprimirnivel\\
\hline

\rowcolor{lightgray}
\multicolumn{1}{|p{14cm}|}{\textbf{Ementa}}\\
\hline
\multicolumn{1}{|p{14cm}|}{\imprimirementa}\\
\hline

\rowcolor{lightgray}
\multicolumn{1}{|p{14cm}|}{\textbf{Objetivo}}\\
\hline
\imprimirobjetivo\\
\hline

\rowcolor{lightgray}
\multicolumn{1}{|p{14cm}|}{\textbf{Programa}}\\
\hline
\imprimirprograma\\
\hline

\rowcolor{lightgray}
\multicolumn{1}{|p{14cm}|}{\textbf{Metodologia de ensino}}\\
\hline
\imprimirmetodologiaEnsino\\
\hline

\rowcolor{lightgray}
\multicolumn{1}{|p{14cm}|}{\textbf{Recursos}}\\
\hline
\imprimirrecursos\\
\hline

\rowcolor{lightgray}
\multicolumn{1}{|p{14cm}|}{\textbf{Avaliação}}\\
\hline
\imprimiravaliacao\\
\hline

\rowcolor{lightgray}
\multicolumn{1}{|p{14cm}|}{\textbf{Bibliografia básica}}\\
\hline
\imprimirbibliografiaBasica\\
\hline

\rowcolor{lightgray}
\multicolumn{1}{|p{14cm}|}{\textbf{Bibliografia complementar}}\\
\hline
\imprimirbibliografiaComplementar\\
\hline
\end{longtable}
\pagebreak
}
\begin{document}

\disciplina{Algebra linear}
\codigo{AUT2448}
\cargaHorariaTotal{40}
\cargaHorariaPratica{0}
\cargaHorariaTeorica{40}
\creditos{2}
\codigoPrerequisitos{-}
\semestre{opcional}
\nivel{Superior}

\ementa{
Equações diferenciais de 1a ordem. Propriedades gerais das equações. Equações diferenciais lineares de 2a ordem com coeficientes constantes. 
Equações diferenciais lineares de 2a ordem com coeficientes variáveis. Transformada de Laplace. Matemática física e classificação de EDPs.
}

\objetivo{
• Modelar, resolver e interpretar as soluções de fenômenos regidos por EDOs (equações diferenciais ordinárias).
}

\programa{
• Equações diferenciais de 1a ordem\\
• Modelos Simples; Equações separáveis; Equações lineares de primeira ordem;\\ 
• Equações exatas;\\
• aplicações.\\
• Propriedades gerais das equações\\
• Aspectos geométricos, teoremas de existência de soluções, unicidade e dependência contínua.\\
• Equações diferenciais lineares de 2a ordem com coeficientes constantes\\
• Soluções explícitas das equações homogêneas; método de variação de parâmetros e método de coeficientes a determinar; aplicações\\
• Equações diferenciais lineares de 2a ordem com coeficientes variáveis\\
• Resolução de equações utilizando séries de potências; método de Frobenius; aplicações.\\
• Transformada de Laplace\\
• Condições de Existência, Propriedades, Resolução de equações diferenciais lineares e de sistemas de equações diferenciais lineares; aplicações.\\
• Física-matemática e classificação de EDPs.\\
• Aplicações.\\
}

\metodologiaEnsino{
Aulas expositivas.\\
Leitura e pesquisa.\\
Vídeo-Aulas.\\
Resolução de exercícios utilizando software apropriado.\\
}

\recursos{
Laboratório de Informática ou Computador Pessoal.\\
Livros contidos na bibliografia.\\
Caderno.\\
Quadro e pincel.\\
Data-show.\\
Lista de exercícios.\\
}

\avaliacao{
Avaliação escrita.\\
Resolução individual ou em grupo.\\
Avaliação de exercícios resolvidos.\\
Poderão ser inseridas outras avaliações durante o semestre.\\
}

\bibliografiaBasica{
• BOYCE, William E.; DIPRIMA, Richard C. Equações diferenciais elementares e
problemas de valores de contorno. 7.ed. Rio de Janeiro, RJ: LTC, 2002.\\

• NAGLE, R. Kent; SAFF, Edward B.; SNIDER, Arthur David. Equações diferenciais. 8. ed. São Paulo, SP: Pearson Education do Brasil, 2012.\\

• ZILL, Dennis G; CULLEN, Michael R. Equações diferenciais. São Paulo: Makron Books, 2013.v.1.\\
}

\bibliografiaComplementar{
• BARBOSA, Celso Antônio Silva. Cálculo diferencial e integral. Fortaleza, CE: Livro Técnico, 2004. v.2.\\

• BRAGA, Carmen Lys Ribeiro. Notas de física-matemática: equações diferenciais, funções de Green e distribuições. São Paulo, SP: Livraria da Física, 2006.\\

• BOULOS, Paulo; CAMARGO, Ivan de. Geometria analítica: um tratamento vetorial. São Paulo, SP: MacGraw-Hill, 1987.\\

• BRONSON, Richard; COSTA, Gabriel B. Equações diferenciais. 3.ed. Porto Alegre, RS: Bookman, 2008.\\
}


\imprimirPUD

\end{document}