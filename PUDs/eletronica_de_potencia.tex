\input{preambulo}
%arquivo de template para os PUDS

%definição das variaveis das seções
\newcommand{\disciplina}{\def \disciplina}
\newcommand{\imprimirdisciplina}{\disciplina}

\newcommand{\codigo}{\def \codigo}
\newcommand{\imprimircodigo}{\codigo}

\newcommand{\cargaHorariaTotal}{\def \cargaHorariaTotal}
\newcommand{\imprimircargaHorariaTotal}{\cargaHorariaTotal}

\newcommand{\cargaHorariaPratica}{\def \cargaHorariaPratica}
\newcommand{\imprimircargaHorariaPratica}{\cargaHorariaPratica}

\newcommand{\cargaHorariaTeorica}{\def \cargaHorariaTeorica}
\newcommand{\imprimircargaHorariaTeorica}{\cargaHorariaTeorica}

\newcommand{\creditos}{\def \creditos}
\newcommand{\imprimircreditos}{\creditos}

\newcommand{\codigoPrerequisitos}{\def \codigoPrerequisitos}
\newcommand{\imprimircodigoPrerequisitos}{\codigoPrerequisitos}

\newcommand{\semestre}{\def \semestre}
\newcommand{\imprimirsemestre}{\semestre}

\newcommand{\nivel}{\def \nivel}
\newcommand{\imprimirnivel}{\nivel}

\newcommand{\codigoEquivalencias}{\def \codigoEquivalencias}
\newcommand{\imprimircodigoEquivalencias}{\codigoEquivalencias}

\newcommand{\ementa}{\def \ementa}
\newcommand{\imprimirementa}{\ementa}

\newcommand{\objetivo}{\def \objetivo}
\newcommand{\imprimirobjetivo}{\objetivo}

\newcommand{\programa}{\def \programa}
\newcommand{\imprimirprograma}{\programa}

\newcommand{\metodologiaEnsino}{\def \metodologiaEnsino}
\newcommand{\imprimirmetodologiaEnsino}{\metodologiaEnsino}

\newcommand{\recursos}{\def \recursos}
\newcommand{\imprimirrecursos}{\recursos}

\newcommand{\avaliacao}{\def \avaliacao}
\newcommand{\imprimiravaliacao}{\avaliacao}

\newcommand{\bibliografiaBasica}{\def \bibliografiaBasica}
\newcommand{\imprimirbibliografiaBasica}{\bibliografiaBasica}

\newcommand{\bibliografiaComplementar}{\def \bibliografiaComplementar}
\newcommand{\imprimirbibliografiaComplementar}{\bibliografiaComplementar}

\newcommand{\versao}{\def \versao}
\newcommand{\imprimirversao}{\versao}


%comando de impressão da estrutura
\newcommand{\imprimirPUD}{
%Cabeçalho do PUD
\begin{Spacing}{1}

\noindent \begin{minipage}{2.5cm}%
\includegraphics[scale=0.12]{logo-ifce}
\end{minipage}
\hspace{0.3cm}
\begin{minipage}{13cm}%
\centering INSTITUTO FEDERAL DE EDUCAÇÃO, CIÊNCIA E TECNOLOGIA DO CEARÁ- IFCE\\
CAMPUS JUAZEIRO DO NORTE\\
CURSO SUPERIOR EM AUTOMAÇÃO INDUSTRIAL\\
PROGRAMA DE UNIDADE DIDÁTICA – PUD\\
\end{minipage}%
\end{Spacing}

\begin{longtable}{|p{14cm}|}
%primeiro cabeçalho
\hline
\rowcolor{lightgray}
\multicolumn{1}{p{14cm}}{\textbf{Disciplina: \imprimirdisciplina}}\\
\hline
\endfirsthead

%cabeçalho
\hline
continuação PUD \imprimirdisciplina\\
\hline
\endhead

\hline
continua...\\
\hline
\endfoot

\hline
\rowcolor{lightgray}

\begin{tabular}{p{5.5 cm}| l}
coordenação & departamento pedagogico\\[16 ex]
\end{tabular}\\

\hline

\endlastfoot

%elementos
\textbf{Código:} \imprimircodigo\\


\textbf{Carga Horária } Teórica: \imprimircargaHorariaTeorica, Prática \imprimircargaHorariaPratica, Total: \imprimircargaHorariaTotal\\


\textbf{Número de créditos:} \imprimircreditos\\


\textbf{Código pré-requisitos:} \imprimircodigoPrerequisitos\\


\textbf{Semestre:} \imprimirsemestre\\


\textbf{Nível:} \imprimirnivel\\
\hline

\rowcolor{lightgray}
\multicolumn{1}{|p{14cm}|}{\textbf{Ementa}}\\
\hline
\multicolumn{1}{|p{14cm}|}{\imprimirementa}\\
\hline

\rowcolor{lightgray}
\multicolumn{1}{|p{14cm}|}{\textbf{Objetivo}}\\
\hline
\imprimirobjetivo\\
\hline

\rowcolor{lightgray}
\multicolumn{1}{|p{14cm}|}{\textbf{Programa}}\\
\hline
\imprimirprograma\\
\hline

\rowcolor{lightgray}
\multicolumn{1}{|p{14cm}|}{\textbf{Metodologia de ensino}}\\
\hline
\imprimirmetodologiaEnsino\\
\hline

\rowcolor{lightgray}
\multicolumn{1}{|p{14cm}|}{\textbf{Recursos}}\\
\hline
\imprimirrecursos\\
\hline

\rowcolor{lightgray}
\multicolumn{1}{|p{14cm}|}{\textbf{Avaliação}}\\
\hline
\imprimiravaliacao\\
\hline

\rowcolor{lightgray}
\multicolumn{1}{|p{14cm}|}{\textbf{Bibliografia básica}}\\
\hline
\imprimirbibliografiaBasica\\
\hline

\rowcolor{lightgray}
\multicolumn{1}{|p{14cm}|}{\textbf{Bibliografia complementar}}\\
\hline
\imprimirbibliografiaComplementar\\
\hline
\end{longtable}
\pagebreak
}
\begin{document}

\disciplina{Eletrônica de potência}
\codigo{AUT2420}
\cargaHorariaTotal{80}
\cargaHorariaPratica{20}
\cargaHorariaTeorica{60}
\creditos{4}
\codigoPrerequisitos{AUT2416, AUT2411}
\semestre{4º}
\nivel{Superior}

\ementa{
Dispositivos semicondutores de potência; Software de Simulação dedicado; Conversores CA-CA: Circuitos Retificadores; Conversores CC-CC; Conversores CC-CA: Inversores.
}

\objetivo{
• Conhecer o princípio de funcionamento dos semicondutores de potência\\
• Conversores CA-CC e suas topologias\\
• Conversores CC-CC e suas topologias\\
• Conversores CC-CA e suas topologias\\
• Simular circuitos dos conversores CA-CC, CC-CC e CC-CA utilizando software dedicado.\\
}

\programa{
• Software de simulação dedicado.\\
• Desenho dos esquemas elétricos.\\
• Configuração dos parâmetros de simulação.\\
• Interpretação dos dados de simulação.\\

• Dispositivos semicondutores de potência.\\
• Diodo de potência.\\
• Tiristores (SCR, DIAC, TRIAC e GTO).\\
• MOSFET e IGBT.\\
• Simulação dos dispositivos semicondutores.\\

• Conversores CA-CC: Circuitos retificadores.\\
• Retificadores monofásicos controlados e não controlados.\\
• Retificadores trifásicos controlados e não controlados.\\
• Simulação dos circuitos retificadores monofásicos e trifásicos.\\

• Conversores CC-CC: Reguladores chaveados não isolados.\\
• Conversor CC-CC Buck.\\
• Conversor CC-CC Boost.\\
• Simulação de conversores CC-CC Buck e Boost.\\
• Projeto e implementação de um conversor CC-CC Buck ou Boost.\\

• Conversores CC-CA: Inversores.\\
• Inversor monofásico de meia ponte (half bridge).\\
• Inversor monofásico de ponte completa (full bridge).\\
• Inversor monofásico de ponte completa (full bridge) com modulação PWM e Filtro de saída.\\
• Simulação dos inversores monofásicos.\\
• Projeto e implementação de um inversor monofásico half bridge ou full bridge.\\

}

\metodologiaEnsino{
Aulas expositivas de caráter informativo com questionamentos críticos sobre os assuntos abordados em sala com os estudantes.\\
Aulas práticas em laboratório (Lab. de medidas elétricas e Eletricidade e Lab. de Informática).\\
Aulas para esclarecimento de dúvidas.\\
Simulação computacional utilizando software dedicado licenciado para o IFCE ou nas versões lite, gratuita ou trial.\\
Projetos para implementação de circuitos.\\
Visita técnica.\\
}

\recursos{
Livros contidos na bibliografia.\\
Pesquisa em artigos científicos e livros não contidos na bibliografia.\\
Quadro; pincel e datashow.\\
Laboratório específico.\\
}

\avaliacao{
Avaliação de aprendizagem escrita (conforme o R.O.D.).\\
Práticas individuais ou em grupo em laboratório.\\
Relatório de prática.\\
Listas de exercícios.\\
Poderão ser inseridas outras avaliações durante o semestre letivo.\\
}

\bibliografiaBasica{
• ALMEIDA, José Luiz A. Dispositivos semicondutores: Tiristores: controle de potência em CC e CA. 12 ed. São Paulo: Editora Érica, 2011.\\

• RASHID, Muhammad H. Eletrônica de potência: Dispositivos, circuitos e aplicações. 4 ed. São Paulo: Pearson Education do Brasil, 2014. Disponível em: < http://bvu.ifce.edu.br/ > Acesso em 15 jun. 2017.\\

• AHMED, Ashfaq. Eletrônica de potência. São Paulo: Pearson Prentice Hall, 2000.\\
}

\bibliografiaComplementar{
• BARBI, Ivo. Eletrônica de Potência. 4ª Ed. Florianópolis: Edição do Autor, 2002.\\

    • FIGINI, Gianfranco. Eletrônica industrial: circuitos e aplicações. São Paulo: Hemus S.A., 2002.\\
    
    • LANDER, Cyril W. Eletrônica Industrial: teoria e aplicações. 2ª Ed. São Paulo: Makron Books, 1996.\\
    
    • HART, Daniel W. Eletrônica de Potência: análise e projetos de circuitos. Porto Alegre: AMGH, 2012.\\
    
    • ARRABAÇA, Devair A.; GIMENEZ, Salvador P. Eletrônica de Potência: conversores de energia (CA/CC): teoria, prática e simulação. 1ª Ed. Editora Érica LTDA. São Paulo – SP, 2014.\\
    
    • ARRABAÇA, Devair A.; GIMENEZ, Salvador P. Conversores de Energia Elétrica CC/CC para Aplicações em Eletrônica de Potência. 1ª Ed. Editora Érica LTDA. São Paulo – SP, 2013.\\
    
    • MOHAN, Ned. Eletrônica de Potência: curso introdutório. 1ª Ed. – Rio de Janeiro: LTC, 2014\\
}


\imprimirPUD

\end{document}