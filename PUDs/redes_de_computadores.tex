\input{preambulo}
%arquivo de template para os PUDS

%definição das variaveis das seções
\newcommand{\disciplina}{\def \disciplina}
\newcommand{\imprimirdisciplina}{\disciplina}

\newcommand{\codigo}{\def \codigo}
\newcommand{\imprimircodigo}{\codigo}

\newcommand{\cargaHorariaTotal}{\def \cargaHorariaTotal}
\newcommand{\imprimircargaHorariaTotal}{\cargaHorariaTotal}

\newcommand{\cargaHorariaPratica}{\def \cargaHorariaPratica}
\newcommand{\imprimircargaHorariaPratica}{\cargaHorariaPratica}

\newcommand{\cargaHorariaTeorica}{\def \cargaHorariaTeorica}
\newcommand{\imprimircargaHorariaTeorica}{\cargaHorariaTeorica}

\newcommand{\creditos}{\def \creditos}
\newcommand{\imprimircreditos}{\creditos}

\newcommand{\codigoPrerequisitos}{\def \codigoPrerequisitos}
\newcommand{\imprimircodigoPrerequisitos}{\codigoPrerequisitos}

\newcommand{\semestre}{\def \semestre}
\newcommand{\imprimirsemestre}{\semestre}

\newcommand{\nivel}{\def \nivel}
\newcommand{\imprimirnivel}{\nivel}

\newcommand{\codigoEquivalencias}{\def \codigoEquivalencias}
\newcommand{\imprimircodigoEquivalencias}{\codigoEquivalencias}

\newcommand{\ementa}{\def \ementa}
\newcommand{\imprimirementa}{\ementa}

\newcommand{\objetivo}{\def \objetivo}
\newcommand{\imprimirobjetivo}{\objetivo}

\newcommand{\programa}{\def \programa}
\newcommand{\imprimirprograma}{\programa}

\newcommand{\metodologiaEnsino}{\def \metodologiaEnsino}
\newcommand{\imprimirmetodologiaEnsino}{\metodologiaEnsino}

\newcommand{\recursos}{\def \recursos}
\newcommand{\imprimirrecursos}{\recursos}

\newcommand{\avaliacao}{\def \avaliacao}
\newcommand{\imprimiravaliacao}{\avaliacao}

\newcommand{\bibliografiaBasica}{\def \bibliografiaBasica}
\newcommand{\imprimirbibliografiaBasica}{\bibliografiaBasica}

\newcommand{\bibliografiaComplementar}{\def \bibliografiaComplementar}
\newcommand{\imprimirbibliografiaComplementar}{\bibliografiaComplementar}

\newcommand{\versao}{\def \versao}
\newcommand{\imprimirversao}{\versao}


%comando de impressão da estrutura
\newcommand{\imprimirPUD}{
%Cabeçalho do PUD
\begin{Spacing}{1}

\noindent \begin{minipage}{2.5cm}%
\includegraphics[scale=0.12]{logo-ifce}
\end{minipage}
\hspace{0.3cm}
\begin{minipage}{13cm}%
\centering INSTITUTO FEDERAL DE EDUCAÇÃO, CIÊNCIA E TECNOLOGIA DO CEARÁ- IFCE\\
CAMPUS JUAZEIRO DO NORTE\\
CURSO SUPERIOR EM AUTOMAÇÃO INDUSTRIAL\\
PROGRAMA DE UNIDADE DIDÁTICA – PUD\\
\end{minipage}%
\end{Spacing}

\begin{longtable}{|p{14cm}|}
%primeiro cabeçalho
\hline
\rowcolor{lightgray}
\multicolumn{1}{p{14cm}}{\textbf{Disciplina: \imprimirdisciplina}}\\
\hline
\endfirsthead

%cabeçalho
\hline
continuação PUD \imprimirdisciplina\\
\hline
\endhead

\hline
continua...\\
\hline
\endfoot

\hline
\rowcolor{lightgray}

\begin{tabular}{p{5.5 cm}| l}
coordenação & departamento pedagogico\\[16 ex]
\end{tabular}\\

\hline

\endlastfoot

%elementos
\textbf{Código:} \imprimircodigo\\


\textbf{Carga Horária } Teórica: \imprimircargaHorariaTeorica, Prática \imprimircargaHorariaPratica, Total: \imprimircargaHorariaTotal\\


\textbf{Número de créditos:} \imprimircreditos\\


\textbf{Código pré-requisitos:} \imprimircodigoPrerequisitos\\


\textbf{Semestre:} \imprimirsemestre\\


\textbf{Nível:} \imprimirnivel\\
\hline

\rowcolor{lightgray}
\multicolumn{1}{|p{14cm}|}{\textbf{Ementa}}\\
\hline
\multicolumn{1}{|p{14cm}|}{\imprimirementa}\\
\hline

\rowcolor{lightgray}
\multicolumn{1}{|p{14cm}|}{\textbf{Objetivo}}\\
\hline
\imprimirobjetivo\\
\hline

\rowcolor{lightgray}
\multicolumn{1}{|p{14cm}|}{\textbf{Programa}}\\
\hline
\imprimirprograma\\
\hline

\rowcolor{lightgray}
\multicolumn{1}{|p{14cm}|}{\textbf{Metodologia de ensino}}\\
\hline
\imprimirmetodologiaEnsino\\
\hline

\rowcolor{lightgray}
\multicolumn{1}{|p{14cm}|}{\textbf{Recursos}}\\
\hline
\imprimirrecursos\\
\hline

\rowcolor{lightgray}
\multicolumn{1}{|p{14cm}|}{\textbf{Avaliação}}\\
\hline
\imprimiravaliacao\\
\hline

\rowcolor{lightgray}
\multicolumn{1}{|p{14cm}|}{\textbf{Bibliografia básica}}\\
\hline
\imprimirbibliografiaBasica\\
\hline

\rowcolor{lightgray}
\multicolumn{1}{|p{14cm}|}{\textbf{Bibliografia complementar}}\\
\hline
\imprimirbibliografiaComplementar\\
\hline
\end{longtable}
\pagebreak
}
\begin{document}

\disciplina{Redes de computadores}
\codigo{AUT2433}
\cargaHorariaTotal{80}
\cargaHorariaPratica{20}
\cargaHorariaTeorica{60}
\creditos{4}
\codigoPrerequisitos{-}
\semestre{2º}
\nivel{Superior}

\ementa{
Conceitos de redes de computadores. Princípios de telecomunicações. Modelos e arquiteturas de redes. Modelo OSI, redes locais, redes de longa distância. Protocolos, arquitetura TCP/IP, aplicações TCP/IP e montagem de redes.
}

\objetivo{
• Apreender os conceitos básicos de redes de computadores.\\
• Conhecer Identificar os componentes de uma rede de computadores.\\
• Introduzir Diferenciar os conceitos de rede local e de longa distância.\\
• Conhecer os modelos de arquitetura de rede.\\
• Discutir o modelo OSI.\\\
• Apresentar a arquitetura TCP/IP e detalhar os principais protocolos e aplicações.\\
• Aprender a Confeccionar cabos para redes ethernet.\\
• Aprender a Configurar máquinas para participar de uma rede.\\
• Projetar e montar uma rede local.\\
• Interconectar redes locais.\\
• Entender o funcionamento dos serviços de internet básico.\\
• Entender os conceitos de rede sem fio.\\
• Montar uma rede sem fio.\\
}

\programa{
• Introdução às redes de computadores Protocolos\\
• Modelo OSI\\
• Padrão IEEE 802\\
• TCP/IP\\
• Fundamentos\\
• Endereçamento IP\\
• ARP, RARP, IP, ICMP\\
• UDP, TCP\\
• DNS, FTP, SMTP, HTTP\\
• Práticas – Simulador – Laboratório de Informática (Packet Tracer)\\
• Outros Protocolos\\
• IPX/SPX 5.2 X.25\\
• Frame Relay\\
• ATM – Seminário 6. Redes sem Fio\\
• Cabeamento (Coaxial, Par Trançado e Fibra Ótica) 8. Arquiteturas de redes locais\\
• Ethernet\\
• Token Ring\\
• Equipamentos de Redes\\
• Segurança de Redes\\
• Laboratório\\
• Montagem – Redes\\
}

\metodologiaEnsino{
Aulas expositivas.\\
Aulas práticas em laboratório de Informática – Simulador Packet Tracer.\\
Aulas práticas em laboratório – Sistemas Digitais (Redes).\\
Resolução de lista de exercícios.\\
Realização de seminários.\\
Leitura e pesquisa.\\
}

\recursos{
Livros contidos na bibliografia.\\
Artigos.\\
Quadro e pincel.\\
Software (simulador), Computadores, conectores, placas de redes, alicates de crimpar e cabos.\\
Data- show.\\
Lista de exercícios.\\
}

\avaliacao{
Avaliação escrita.\\
Práticas individuais e em grupo no laboratório.\\
Apresentação de Seminários.\\
Produção de artigo.\\
Avaliação de exercícios realizados.\\
Implementação de Algoritmos.\\
Poderão ser inseridas outras avaliações durante o semestre.\\
}

\bibliografiaBasica{
• KUROSE, J. F.; ROSS, K. W. Redes de computadores e a internet: uma abordagem top-down. São Paulo: Pearson Addison Wesley, 2006.\\
• KUROSE, J. F.; ROSS, K. W. Redes de computadores e a internet: – uma
abordagem top-down. 6 ed. São Paulo: Pearson Addison Wesley, 2013.\\
• TANEMBAUM, A. S. Redes de computadores. Rio de Janeiro: Campus, 2003.\\
}

\bibliografiaComplementar{
• TANENBAUM, Andrew S.; WETHERAL, David. Redes de computadores. 5 ed.
São Paulo: Pearson, 2011.\\
• TORRES, G. Redes de computadores: curso completo. Rio de Janeiro: Axcel Books do Brasil, 2003.\\
• OLSEN, Diogo Roberto; LAURIANO, Marcos Aurélio Pchek. Redes de computadores. Curitiba: Livro Técnico, 2010.\\
• PAQUET, Catherine; TEARE, Diane. Construindo redes cisco escaláveis. São Paulo:
Pearson, 2003.\\
• TORRES, G. Redes de computadores: versão revisada e atualizada. 2a ed. Rio de
Janeiro: Clube do Hardware, 2019.\\
}


\imprimirPUD

\end{document}