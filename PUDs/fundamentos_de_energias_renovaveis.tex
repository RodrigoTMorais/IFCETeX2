\input{preambulo}
%arquivo de template para os PUDS

%definição das variaveis das seções
\newcommand{\disciplina}{\def \disciplina}
\newcommand{\imprimirdisciplina}{\disciplina}

\newcommand{\codigo}{\def \codigo}
\newcommand{\imprimircodigo}{\codigo}

\newcommand{\cargaHorariaTotal}{\def \cargaHorariaTotal}
\newcommand{\imprimircargaHorariaTotal}{\cargaHorariaTotal}

\newcommand{\cargaHorariaPratica}{\def \cargaHorariaPratica}
\newcommand{\imprimircargaHorariaPratica}{\cargaHorariaPratica}

\newcommand{\cargaHorariaTeorica}{\def \cargaHorariaTeorica}
\newcommand{\imprimircargaHorariaTeorica}{\cargaHorariaTeorica}

\newcommand{\creditos}{\def \creditos}
\newcommand{\imprimircreditos}{\creditos}

\newcommand{\codigoPrerequisitos}{\def \codigoPrerequisitos}
\newcommand{\imprimircodigoPrerequisitos}{\codigoPrerequisitos}

\newcommand{\semestre}{\def \semestre}
\newcommand{\imprimirsemestre}{\semestre}

\newcommand{\nivel}{\def \nivel}
\newcommand{\imprimirnivel}{\nivel}

\newcommand{\codigoEquivalencias}{\def \codigoEquivalencias}
\newcommand{\imprimircodigoEquivalencias}{\codigoEquivalencias}

\newcommand{\ementa}{\def \ementa}
\newcommand{\imprimirementa}{\ementa}

\newcommand{\objetivo}{\def \objetivo}
\newcommand{\imprimirobjetivo}{\objetivo}

\newcommand{\programa}{\def \programa}
\newcommand{\imprimirprograma}{\programa}

\newcommand{\metodologiaEnsino}{\def \metodologiaEnsino}
\newcommand{\imprimirmetodologiaEnsino}{\metodologiaEnsino}

\newcommand{\recursos}{\def \recursos}
\newcommand{\imprimirrecursos}{\recursos}

\newcommand{\avaliacao}{\def \avaliacao}
\newcommand{\imprimiravaliacao}{\avaliacao}

\newcommand{\bibliografiaBasica}{\def \bibliografiaBasica}
\newcommand{\imprimirbibliografiaBasica}{\bibliografiaBasica}

\newcommand{\bibliografiaComplementar}{\def \bibliografiaComplementar}
\newcommand{\imprimirbibliografiaComplementar}{\bibliografiaComplementar}

\newcommand{\versao}{\def \versao}
\newcommand{\imprimirversao}{\versao}


%comando de impressão da estrutura
\newcommand{\imprimirPUD}{
%Cabeçalho do PUD
\begin{Spacing}{1}

\noindent \begin{minipage}{2.5cm}%
\includegraphics[scale=0.12]{logo-ifce}
\end{minipage}
\hspace{0.3cm}
\begin{minipage}{13cm}%
\centering INSTITUTO FEDERAL DE EDUCAÇÃO, CIÊNCIA E TECNOLOGIA DO CEARÁ- IFCE\\
CAMPUS JUAZEIRO DO NORTE\\
CURSO SUPERIOR EM AUTOMAÇÃO INDUSTRIAL\\
PROGRAMA DE UNIDADE DIDÁTICA – PUD\\
\end{minipage}%
\end{Spacing}

\begin{longtable}{|p{14cm}|}
%primeiro cabeçalho
\hline
\rowcolor{lightgray}
\multicolumn{1}{p{14cm}}{\textbf{Disciplina: \imprimirdisciplina}}\\
\hline
\endfirsthead

%cabeçalho
\hline
continuação PUD \imprimirdisciplina\\
\hline
\endhead

\hline
continua...\\
\hline
\endfoot

\hline
\rowcolor{lightgray}

\begin{tabular}{p{5.5 cm}| l}
coordenação & departamento pedagogico\\[16 ex]
\end{tabular}\\

\hline

\endlastfoot

%elementos
\textbf{Código:} \imprimircodigo\\


\textbf{Carga Horária } Teórica: \imprimircargaHorariaTeorica, Prática \imprimircargaHorariaPratica, Total: \imprimircargaHorariaTotal\\


\textbf{Número de créditos:} \imprimircreditos\\


\textbf{Código pré-requisitos:} \imprimircodigoPrerequisitos\\


\textbf{Semestre:} \imprimirsemestre\\


\textbf{Nível:} \imprimirnivel\\
\hline

\rowcolor{lightgray}
\multicolumn{1}{|p{14cm}|}{\textbf{Ementa}}\\
\hline
\multicolumn{1}{|p{14cm}|}{\imprimirementa}\\
\hline

\rowcolor{lightgray}
\multicolumn{1}{|p{14cm}|}{\textbf{Objetivo}}\\
\hline
\imprimirobjetivo\\
\hline

\rowcolor{lightgray}
\multicolumn{1}{|p{14cm}|}{\textbf{Programa}}\\
\hline
\imprimirprograma\\
\hline

\rowcolor{lightgray}
\multicolumn{1}{|p{14cm}|}{\textbf{Metodologia de ensino}}\\
\hline
\imprimirmetodologiaEnsino\\
\hline

\rowcolor{lightgray}
\multicolumn{1}{|p{14cm}|}{\textbf{Recursos}}\\
\hline
\imprimirrecursos\\
\hline

\rowcolor{lightgray}
\multicolumn{1}{|p{14cm}|}{\textbf{Avaliação}}\\
\hline
\imprimiravaliacao\\
\hline

\rowcolor{lightgray}
\multicolumn{1}{|p{14cm}|}{\textbf{Bibliografia básica}}\\
\hline
\imprimirbibliografiaBasica\\
\hline

\rowcolor{lightgray}
\multicolumn{1}{|p{14cm}|}{\textbf{Bibliografia complementar}}\\
\hline
\imprimirbibliografiaComplementar\\
\hline
\end{longtable}
\pagebreak
}
\begin{document}

\disciplina{Fundamentos de energias renováveis}
\codigo{AUT2449}
\cargaHorariaTotal{40}
\cargaHorariaPratica{0}
\cargaHorariaTeorica{40}
\creditos{2}
\codigoPrerequisitos{-}
\semestre{opcional}
\nivel{Superior}

\ementa{
Histórico da matriz energética brasileira. Classificação
das fontes de energia renováveis. Impacto ambiental causado pela utilização das
energias renováveis.
}

\objetivo{
• Entender as diversas aplicações das energias renováveis.
• Utilizar a legislação ambiental em favorecimento da diversificação da matriz energética.\\
• Orientar a aplicação das normas e preceitos da legislação ambiental no combate à poluição ambiental.\\
}

\programa{
• Introdução aos conceitos básicos sobre energias renováveis: A importância da energia;\\
• Tipos e fontes de energia; Produção de energia; Impactos ambientais; O efeito estufa;\
• Mecanismos de desenvolvimento limpo.\\
• Recursos energéticos alternativos disponíveis no território brasileiro: Energia solar; Energia eólica; Biomassa.\\
• Energia hidráulica: Definição de PCH; Centrais quanto à capacidade de regularização;\\
• Centrais quanto ao sistema de adução; Centrais quanto à potência instalada e quanto à queda de projeto; Componentes de uma PCH; Estudos necessários para implantação do empreendimento; Geradores hidrocinéticos.\\
• Energia do Hidrogênio: O hidrogênio; Células a combustível; Princípio de funcionamento da célula a combustível; Principais componentes de um sistema com célula a combustível;\\
• Tecnologias empregadas em células a combustível.\\
• Energia oceânica: Energia das marés; Energia das ondas; Energia das correntes marítimas; Principais aplicações.\\
• Sistemas Híbridos: Estratégias de operação; Vantagens e desvantagens; \\
• Características de sistemas isolados e interligados.\\
}

\metodologiaEnsino{
Aulas expositivas.\\
Leitura e pesquisa.\\
Vídeo-Aulas.\\
}

\recursos{
Livros contidos na bibliografia.\\
Caderno.\\
Quadro e pincel.\\
Data-show.\\
Lista de exercícios.\\
}

\avaliacao{
Avaliação escrita.\\
Resolução individual ou em grupo.\\
Avaliação de exercícios realizados.\\
Poderão ser inseridas outras avaliações durante o semestre.\\
}

\bibliografiaBasica{
• GOLDEMBERG, Jose; PALETTA, Francisco C. Energias Renováveis - Série Energia e
Sustentabilidade. São Paulo: Editora Blucher, 2012.\\

• ROSA, Aldo V. da. Processos de Energias Renováveis. 3a. ed. São Paulo: Editora Saraiva, 2015.\\

• VECCHIA, Rodnei. O Ambiente e as Energias Renováveis. São Paulo: Editora
Manole, 2010.\\
}

\bibliografiaComplementar{
• REIS, L. B.; CUNHA, E. C. N. Energia Elétrica e Sustentabilidade. 2a ed. São Paulo: Editora Manole, 2014.\\

• ROVERE, Emilio Lebre La. Energias Renováveis No Brasil: Desafios e Oportunidades. São Paulo: Editora Brasileira de Arte e Cultura, 2010.\\

• SOARES, Cláudia Alexandra Dias; SILVA, Suzana Tavares da. Direito das Energias
Renováveis. São Paulo: Editora Brasileira de Arte e Cultura, 2014.\\

• TOLMASQUIM, Mauricio Tiomno. Energia Renovável: Hidráulica, Biomassa, Eólica, Solar, Oceânica. Rio de Janeiro: Editora EPE, 2016.\\

• WALISIEWICZ, Marek. Energia Alternativa: solar, eólica, hidrelétrica e de biocombustíveis. São Paulo: Editora Publifolha, 2008.\\

• WOLFGANG, Palz. Energia Solar e Fontes Alternativas. 2a . ed. Curitiba: Editora Hemus, 2005.
}


\imprimirPUD

\end{document}