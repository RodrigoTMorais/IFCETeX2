\input{preambulo}
%arquivo de template para os PUDS

%definição das variaveis das seções
\newcommand{\disciplina}{\def \disciplina}
\newcommand{\imprimirdisciplina}{\disciplina}

\newcommand{\codigo}{\def \codigo}
\newcommand{\imprimircodigo}{\codigo}

\newcommand{\cargaHorariaTotal}{\def \cargaHorariaTotal}
\newcommand{\imprimircargaHorariaTotal}{\cargaHorariaTotal}

\newcommand{\cargaHorariaPratica}{\def \cargaHorariaPratica}
\newcommand{\imprimircargaHorariaPratica}{\cargaHorariaPratica}

\newcommand{\cargaHorariaTeorica}{\def \cargaHorariaTeorica}
\newcommand{\imprimircargaHorariaTeorica}{\cargaHorariaTeorica}

\newcommand{\creditos}{\def \creditos}
\newcommand{\imprimircreditos}{\creditos}

\newcommand{\codigoPrerequisitos}{\def \codigoPrerequisitos}
\newcommand{\imprimircodigoPrerequisitos}{\codigoPrerequisitos}

\newcommand{\semestre}{\def \semestre}
\newcommand{\imprimirsemestre}{\semestre}

\newcommand{\nivel}{\def \nivel}
\newcommand{\imprimirnivel}{\nivel}

\newcommand{\codigoEquivalencias}{\def \codigoEquivalencias}
\newcommand{\imprimircodigoEquivalencias}{\codigoEquivalencias}

\newcommand{\ementa}{\def \ementa}
\newcommand{\imprimirementa}{\ementa}

\newcommand{\objetivo}{\def \objetivo}
\newcommand{\imprimirobjetivo}{\objetivo}

\newcommand{\programa}{\def \programa}
\newcommand{\imprimirprograma}{\programa}

\newcommand{\metodologiaEnsino}{\def \metodologiaEnsino}
\newcommand{\imprimirmetodologiaEnsino}{\metodologiaEnsino}

\newcommand{\recursos}{\def \recursos}
\newcommand{\imprimirrecursos}{\recursos}

\newcommand{\avaliacao}{\def \avaliacao}
\newcommand{\imprimiravaliacao}{\avaliacao}

\newcommand{\bibliografiaBasica}{\def \bibliografiaBasica}
\newcommand{\imprimirbibliografiaBasica}{\bibliografiaBasica}

\newcommand{\bibliografiaComplementar}{\def \bibliografiaComplementar}
\newcommand{\imprimirbibliografiaComplementar}{\bibliografiaComplementar}

\newcommand{\versao}{\def \versao}
\newcommand{\imprimirversao}{\versao}


%comando de impressão da estrutura
\newcommand{\imprimirPUD}{
%Cabeçalho do PUD
\begin{Spacing}{1}

\noindent \begin{minipage}{2.5cm}%
\includegraphics[scale=0.12]{logo-ifce}
\end{minipage}
\hspace{0.3cm}
\begin{minipage}{13cm}%
\centering INSTITUTO FEDERAL DE EDUCAÇÃO, CIÊNCIA E TECNOLOGIA DO CEARÁ- IFCE\\
CAMPUS JUAZEIRO DO NORTE\\
CURSO SUPERIOR EM AUTOMAÇÃO INDUSTRIAL\\
PROGRAMA DE UNIDADE DIDÁTICA – PUD\\
\end{minipage}%
\end{Spacing}

\begin{longtable}{|p{14cm}|}
%primeiro cabeçalho
\hline
\rowcolor{lightgray}
\multicolumn{1}{p{14cm}}{\textbf{Disciplina: \imprimirdisciplina}}\\
\hline
\endfirsthead

%cabeçalho
\hline
continuação PUD \imprimirdisciplina\\
\hline
\endhead

\hline
continua...\\
\hline
\endfoot

\hline
\rowcolor{lightgray}

\begin{tabular}{p{5.5 cm}| l}
coordenação & departamento pedagogico\\[16 ex]
\end{tabular}\\

\hline

\endlastfoot

%elementos
\textbf{Código:} \imprimircodigo\\


\textbf{Carga Horária } Teórica: \imprimircargaHorariaTeorica, Prática \imprimircargaHorariaPratica, Total: \imprimircargaHorariaTotal\\


\textbf{Número de créditos:} \imprimircreditos\\


\textbf{Código pré-requisitos:} \imprimircodigoPrerequisitos\\


\textbf{Semestre:} \imprimirsemestre\\


\textbf{Nível:} \imprimirnivel\\
\hline

\rowcolor{lightgray}
\multicolumn{1}{|p{14cm}|}{\textbf{Ementa}}\\
\hline
\multicolumn{1}{|p{14cm}|}{\imprimirementa}\\
\hline

\rowcolor{lightgray}
\multicolumn{1}{|p{14cm}|}{\textbf{Objetivo}}\\
\hline
\imprimirobjetivo\\
\hline

\rowcolor{lightgray}
\multicolumn{1}{|p{14cm}|}{\textbf{Programa}}\\
\hline
\imprimirprograma\\
\hline

\rowcolor{lightgray}
\multicolumn{1}{|p{14cm}|}{\textbf{Metodologia de ensino}}\\
\hline
\imprimirmetodologiaEnsino\\
\hline

\rowcolor{lightgray}
\multicolumn{1}{|p{14cm}|}{\textbf{Recursos}}\\
\hline
\imprimirrecursos\\
\hline

\rowcolor{lightgray}
\multicolumn{1}{|p{14cm}|}{\textbf{Avaliação}}\\
\hline
\imprimiravaliacao\\
\hline

\rowcolor{lightgray}
\multicolumn{1}{|p{14cm}|}{\textbf{Bibliografia básica}}\\
\hline
\imprimirbibliografiaBasica\\
\hline

\rowcolor{lightgray}
\multicolumn{1}{|p{14cm}|}{\textbf{Bibliografia complementar}}\\
\hline
\imprimirbibliografiaComplementar\\
\hline
\end{longtable}
\pagebreak
}
\begin{document}

\disciplina{Física 2}
\codigo{AUT2424}
\cargaHorariaTotal{40}
\cargaHorariaPratica{0}
\cargaHorariaTeorica{40}
\creditos{4}
\codigoPrerequisitos{AUT2413}
\semestre{4º}
\nivel{Superior}

\ementa{
Conceitos de eletricidade e magnetismo.Lei de Ampére, Lei de Biot-Savart, Lei de Faraday e Lei de Lenz. Circuitos RL, RC e RLC ressonantes.
}

\objetivo{
• Calcular Força Magnética sobre condutores, toróides e bobinas.\\
• Calcular Torque sobre bobinas móveis.\\
• Calcular Tensões induzidas em bobinas.\\
• Elaborar circuitos ressonantes.\\
• Dimensionar resistores, capacitores e indutores em um circuito.\\
• Mostrar no osciloscópio as oscilações forçadas e amortecidas.\\
}

\programa{
• Corrente e resistência Corrente elétrica Densidade de corrente\\
• Resistência, resistividade e condutividade Lei de Ohm\\
• Transferências de energia em um circuito elétrico Supercondutividade\\
• Campo magnético O campo magnético\\
• Força magnética sobre uma carga em movimento Força de Lorentz\\
• Efeito Hall\\
• Força magnética sobre uma corrente elétrica\\
• Torque sobre uma espira percorrida por uma corrente Dipolo magnético\\
• Lei de Ampére Lei de Biot-Savart\\
• Aplicações da Lei de Biot-Savart Dois condutores paralelos\\
• A Lei de Ampére Solenóides e toróides Lei de Faraday\\
• As experiências de Faraday Lei da indução de Faraday Lei de Lenz FEM devida ao movimento\\
• Campo elétrico induzido Indutância Indutância\\
• Cálculo da Indutância Circuitos RL\\
• Energia armazenada em um campo magnético Densidade de energia\\
• Oscilações eletromagnéticas Estudo qualitativo do circuito LC Estudo quantitativo do circuito\\
• LC Oscilações amortecidas e forçadas ( Circuito RLC)\\
}

\metodologiaEnsino{
Aulas expositivas.\\
Aulas práticas em laboratório.\\
Resolução de lista de exercícios.\\
Leitura e pesquisa.\\
}

\recursos{
Livros contidos na bibliografia.\\
Quadro e pincel.\\
Data-show\\
Laboratório de física\\
Lista de exercícios.\\
}

\avaliacao{
Avaliação escrita.\\
Práticas individuais e em grupo no laboratório.\\
Avaliação de exercícios resolvidos.\\
}

\bibliografiaBasica{
• BRANISLAV, M. Notaros. Eletromagnetismo17. São Paulo: Pearson, 2012.\\

• GONÇALVES, Dalton. Física 3: eletricidade, eletromagnetismo e corrente alternada. Rio de Janeiro: Ao Livro Técnico, 1993.\\

• HALLIDAY, David; RESNICK, Robert; WALKER, Jearl. Fundamentos de Física 3. Rio de Janeiro: LTC, 2007.\\
}

\bibliografiaComplementar{
• HAYT JUNIOR, William Hart; BUCK, John A. Eletromagnetismo. Rio de Janeiro: LTC, 2003.\\

• SILVA, Claudio Elias da et al. Eletromagnetismo fundamentos e simulações 18. São Paulo: Pearson, 2014.\\

• WOLSKI, Belmiro. Eletromagnetismo. Curitiba: Base Editorial, 2010.\\

• NUSSENZVEIG, H. Moysés. Curso de física básica 3: eletromagnetismo. São Paulo: Edgard Blücher, 2007.\\

• TIPLER, Paul A. Física 3: para cientistas e engenheiros - eletricidade e magnetismo. Rio de Janeiro: Livro Técnico e Científicos, 1995. v.3.\\

• YOUNG, H. D; FREEDMAN, R.A. Física III: eletromagnetismo. São Paulo: Addison Wesley, 2008.\\
}


\imprimirPUD

\end{document}