\input{preambulo}
%arquivo de template para os PUDS

%definição das variaveis das seções
\newcommand{\disciplina}{\def \disciplina}
\newcommand{\imprimirdisciplina}{\disciplina}

\newcommand{\codigo}{\def \codigo}
\newcommand{\imprimircodigo}{\codigo}

\newcommand{\cargaHorariaTotal}{\def \cargaHorariaTotal}
\newcommand{\imprimircargaHorariaTotal}{\cargaHorariaTotal}

\newcommand{\cargaHorariaPratica}{\def \cargaHorariaPratica}
\newcommand{\imprimircargaHorariaPratica}{\cargaHorariaPratica}

\newcommand{\cargaHorariaTeorica}{\def \cargaHorariaTeorica}
\newcommand{\imprimircargaHorariaTeorica}{\cargaHorariaTeorica}

\newcommand{\creditos}{\def \creditos}
\newcommand{\imprimircreditos}{\creditos}

\newcommand{\codigoPrerequisitos}{\def \codigoPrerequisitos}
\newcommand{\imprimircodigoPrerequisitos}{\codigoPrerequisitos}

\newcommand{\semestre}{\def \semestre}
\newcommand{\imprimirsemestre}{\semestre}

\newcommand{\nivel}{\def \nivel}
\newcommand{\imprimirnivel}{\nivel}

\newcommand{\codigoEquivalencias}{\def \codigoEquivalencias}
\newcommand{\imprimircodigoEquivalencias}{\codigoEquivalencias}

\newcommand{\ementa}{\def \ementa}
\newcommand{\imprimirementa}{\ementa}

\newcommand{\objetivo}{\def \objetivo}
\newcommand{\imprimirobjetivo}{\objetivo}

\newcommand{\programa}{\def \programa}
\newcommand{\imprimirprograma}{\programa}

\newcommand{\metodologiaEnsino}{\def \metodologiaEnsino}
\newcommand{\imprimirmetodologiaEnsino}{\metodologiaEnsino}

\newcommand{\recursos}{\def \recursos}
\newcommand{\imprimirrecursos}{\recursos}

\newcommand{\avaliacao}{\def \avaliacao}
\newcommand{\imprimiravaliacao}{\avaliacao}

\newcommand{\bibliografiaBasica}{\def \bibliografiaBasica}
\newcommand{\imprimirbibliografiaBasica}{\bibliografiaBasica}

\newcommand{\bibliografiaComplementar}{\def \bibliografiaComplementar}
\newcommand{\imprimirbibliografiaComplementar}{\bibliografiaComplementar}

\newcommand{\versao}{\def \versao}
\newcommand{\imprimirversao}{\versao}


%comando de impressão da estrutura
\newcommand{\imprimirPUD}{
%Cabeçalho do PUD
\begin{Spacing}{1}

\noindent \begin{minipage}{2.5cm}%
\includegraphics[scale=0.12]{logo-ifce}
\end{minipage}
\hspace{0.3cm}
\begin{minipage}{13cm}%
\centering INSTITUTO FEDERAL DE EDUCAÇÃO, CIÊNCIA E TECNOLOGIA DO CEARÁ- IFCE\\
CAMPUS JUAZEIRO DO NORTE\\
CURSO SUPERIOR EM AUTOMAÇÃO INDUSTRIAL\\
PROGRAMA DE UNIDADE DIDÁTICA – PUD\\
\end{minipage}%
\end{Spacing}

\begin{longtable}{|p{14cm}|}
%primeiro cabeçalho
\hline
\rowcolor{lightgray}
\multicolumn{1}{p{14cm}}{\textbf{Disciplina: \imprimirdisciplina}}\\
\hline
\endfirsthead

%cabeçalho
\hline
continuação PUD \imprimirdisciplina\\
\hline
\endhead

\hline
continua...\\
\hline
\endfoot

\hline
\rowcolor{lightgray}

\begin{tabular}{p{5.5 cm}| l}
coordenação & departamento pedagogico\\[16 ex]
\end{tabular}\\

\hline

\endlastfoot

%elementos
\textbf{Código:} \imprimircodigo\\


\textbf{Carga Horária } Teórica: \imprimircargaHorariaTeorica, Prática \imprimircargaHorariaPratica, Total: \imprimircargaHorariaTotal\\


\textbf{Número de créditos:} \imprimircreditos\\


\textbf{Código pré-requisitos:} \imprimircodigoPrerequisitos\\


\textbf{Semestre:} \imprimirsemestre\\


\textbf{Nível:} \imprimirnivel\\
\hline

\rowcolor{lightgray}
\multicolumn{1}{|p{14cm}|}{\textbf{Ementa}}\\
\hline
\multicolumn{1}{|p{14cm}|}{\imprimirementa}\\
\hline

\rowcolor{lightgray}
\multicolumn{1}{|p{14cm}|}{\textbf{Objetivo}}\\
\hline
\imprimirobjetivo\\
\hline

\rowcolor{lightgray}
\multicolumn{1}{|p{14cm}|}{\textbf{Programa}}\\
\hline
\imprimirprograma\\
\hline

\rowcolor{lightgray}
\multicolumn{1}{|p{14cm}|}{\textbf{Metodologia de ensino}}\\
\hline
\imprimirmetodologiaEnsino\\
\hline

\rowcolor{lightgray}
\multicolumn{1}{|p{14cm}|}{\textbf{Recursos}}\\
\hline
\imprimirrecursos\\
\hline

\rowcolor{lightgray}
\multicolumn{1}{|p{14cm}|}{\textbf{Avaliação}}\\
\hline
\imprimiravaliacao\\
\hline

\rowcolor{lightgray}
\multicolumn{1}{|p{14cm}|}{\textbf{Bibliografia básica}}\\
\hline
\imprimirbibliografiaBasica\\
\hline

\rowcolor{lightgray}
\multicolumn{1}{|p{14cm}|}{\textbf{Bibliografia complementar}}\\
\hline
\imprimirbibliografiaComplementar\\
\hline
\end{longtable}
\pagebreak
}
\begin{document}

\disciplina{Eletrônica digital 2}
\codigo{AUT2409}
\cargaHorariaTotal{80}
\cargaHorariaPratica{20}
\cargaHorariaTeorica{60}
\creditos{4}
\codigoPrerequisitos{AUT2403}
\semestre{2º}
\nivel{Superior}

\ementa{
Flip-Flops. Registradores. Contadores. Memórias. Unidade lógica e aritmética.
}

\objetivo{
• Conhecer as diferenças entre circuitos combinacionais e circuitos sequenciais.\\
• Entender Identificar a diferença entre sistemas síncronos e assíncronos.\\
• Entender o funcionamento dos flip-flops.\\
• Projetar sistemas utilizando flip-flops.\\
• Reconhecer os diversos símbolos IEE/ANSI para flip-flops, contadores, registradores e somadores.\\
• Construir um flip-flop com portas NAND ou NOR e analisar seu funcionamento.\\
• Desenhar as formas de onda de saída de vários tipos de flip-flop em resposta a um conjunto de sinais de entrada.\\
• Implementar circuitos lógicos sequenciais utilizando flip-flops.\\
• Reconhecer e entender a operação de diversos tipos de registradores.\\
• Construir contadores crescentes e decrescentes.\\
• Implementar contadores síncrono com sequência de contagem arbitrária.\\
• Construir circuitos digitais utilizando contadores comerciais.\\
• Combinar CI’s de memória para formar módulos de memórias com capacidade e/ou tamanho de palavras maiores.\\
• Determinar a capacidade de um dispositivo de memória a partir de suas entradas e saídas.\\
• Usar um circuito integrado ULA para realizar várias operações lógicas e aritméticas sobre os dados de entrada.\\
}

\programa{
• Flip-Flops\\
• Flip-flop RS básico com portas lógicas Flip-flop RS com entrada de clock\\
• Sincronização com sinais de clock e Diagramas de tempo\\
• Flip-flop mestre-escravo: sensibilidade à transição do sinal de clock Flip-flop tipo D Flip-flop tipo JK Flip-flop tipo T\\
• Conversão de flip-flops e Flip-flops comerciais Registradores\\
• Construção de registradores Registradores de deslocamento Registradores comerciais Contadores\\
• Conceitos básicos\\
• Construção de contadores com flip-flops Classificação e Contadores comerciais
• Memórias\\
• Classificação\\
• Célula básica de memória Decodificação de endereços Memórias comerciais Unidade lógica e aritmética\\
• Operações básicas entre bits ULA comercial\\
}

\metodologiaEnsino{
Aulas expositivas.\\
Aulas práticas em laboratórios.\\
Apresentação de Seminários.\\
Resolução de listas de exercícios.\\
Leitura e pesquisa.\\
}

\recursos{
Livros contidos na bibliografia.\\
Manuais Técnicos.\\
Quadro e pincel.\\
Laboratório de eletrônica.\\
Data-show.\\
Transporte para aulas de campo.\\
Lista de exercícios.\\
}

\avaliacao{
Análise e correção dos projetos de automação.\\
Provas escritas.\\
Práticas individuais e em grupo no laboratório.\\
Realização de Seminários.\\
Apresentação de relatório.\\
Avaliação de exercícios resolvidos.\\
}

\bibliografiaBasica{
• IDOETA I. V.; CAPUANO F. G. Elementos de eletrônica digital. São Paulo: Érica, 2007.\\
• LOURENÇO, A. C.; CRUZ, E. C. A.; FERREIRA, S. R; CHOURI, S. Jr. Circuitos digitais. São Paulo: Érica, 2007. ( Coleção Estude e Use)\\
• TOCCI, R. J.; WIDMER, N. S.; MOSS, G. L. Sistemas digitais: princípios e aplicações. São Paulo: Pearson Prentice Hall, 2003.\\
}

\bibliografiaComplementar{
• SEDRA, Adel S.; SMITH, KENNETH, C. Microeletrônica. São Paulo: Pearson, 2000.\\
• CAPUANO, Francisco Gabriel. Exercícios de eletrônica digital: resolvidos e propostos. São Paulo: Érica, 1996.\\
• TOCCI, Ronaldo J.; LASKOWSKI, Lester P. Microprocessadores e Microcomputadores:
hardware e software. Rio de Janeiro: Prentice Hall, 1990.\\
• AGNER, Flávio Rech; REIS, André Inácio; RIBAS, Renato Perez. Fundamentos de circuitos digitais. Porto Alegre, RS: Bookman: Instituto de Informática da UFRGS, 2008. 166 p. (Livros Didáticos; v. 17).\\
• MALVINO, Albert Paul. .Eletronica Digital Vol. 2-principios e Aplic.– Vol 1 e 2 –Mc Graw Hill. 1ª Edição. São Paulo- 1988\\
}


\imprimirPUD

\end{document}