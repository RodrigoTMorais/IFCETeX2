\input{preambulo}
%arquivo de template para os PUDS

%definição das variaveis das seções
\newcommand{\disciplina}{\def \disciplina}
\newcommand{\imprimirdisciplina}{\disciplina}

\newcommand{\codigo}{\def \codigo}
\newcommand{\imprimircodigo}{\codigo}

\newcommand{\cargaHorariaTotal}{\def \cargaHorariaTotal}
\newcommand{\imprimircargaHorariaTotal}{\cargaHorariaTotal}

\newcommand{\cargaHorariaPratica}{\def \cargaHorariaPratica}
\newcommand{\imprimircargaHorariaPratica}{\cargaHorariaPratica}

\newcommand{\cargaHorariaTeorica}{\def \cargaHorariaTeorica}
\newcommand{\imprimircargaHorariaTeorica}{\cargaHorariaTeorica}

\newcommand{\creditos}{\def \creditos}
\newcommand{\imprimircreditos}{\creditos}

\newcommand{\codigoPrerequisitos}{\def \codigoPrerequisitos}
\newcommand{\imprimircodigoPrerequisitos}{\codigoPrerequisitos}

\newcommand{\semestre}{\def \semestre}
\newcommand{\imprimirsemestre}{\semestre}

\newcommand{\nivel}{\def \nivel}
\newcommand{\imprimirnivel}{\nivel}

\newcommand{\codigoEquivalencias}{\def \codigoEquivalencias}
\newcommand{\imprimircodigoEquivalencias}{\codigoEquivalencias}

\newcommand{\ementa}{\def \ementa}
\newcommand{\imprimirementa}{\ementa}

\newcommand{\objetivo}{\def \objetivo}
\newcommand{\imprimirobjetivo}{\objetivo}

\newcommand{\programa}{\def \programa}
\newcommand{\imprimirprograma}{\programa}

\newcommand{\metodologiaEnsino}{\def \metodologiaEnsino}
\newcommand{\imprimirmetodologiaEnsino}{\metodologiaEnsino}

\newcommand{\recursos}{\def \recursos}
\newcommand{\imprimirrecursos}{\recursos}

\newcommand{\avaliacao}{\def \avaliacao}
\newcommand{\imprimiravaliacao}{\avaliacao}

\newcommand{\bibliografiaBasica}{\def \bibliografiaBasica}
\newcommand{\imprimirbibliografiaBasica}{\bibliografiaBasica}

\newcommand{\bibliografiaComplementar}{\def \bibliografiaComplementar}
\newcommand{\imprimirbibliografiaComplementar}{\bibliografiaComplementar}

\newcommand{\versao}{\def \versao}
\newcommand{\imprimirversao}{\versao}


%comando de impressão da estrutura
\newcommand{\imprimirPUD}{
%Cabeçalho do PUD
\begin{Spacing}{1}

\noindent \begin{minipage}{2.5cm}%
\includegraphics[scale=0.12]{logo-ifce}
\end{minipage}
\hspace{0.3cm}
\begin{minipage}{13cm}%
\centering INSTITUTO FEDERAL DE EDUCAÇÃO, CIÊNCIA E TECNOLOGIA DO CEARÁ- IFCE\\
CAMPUS JUAZEIRO DO NORTE\\
CURSO SUPERIOR EM AUTOMAÇÃO INDUSTRIAL\\
PROGRAMA DE UNIDADE DIDÁTICA – PUD\\
\end{minipage}%
\end{Spacing}

\begin{longtable}{|p{14cm}|}
%primeiro cabeçalho
\hline
\rowcolor{lightgray}
\multicolumn{1}{p{14cm}}{\textbf{Disciplina: \imprimirdisciplina}}\\
\hline
\endfirsthead

%cabeçalho
\hline
continuação PUD \imprimirdisciplina\\
\hline
\endhead

\hline
continua...\\
\hline
\endfoot

\hline
\rowcolor{lightgray}

\begin{tabular}{p{5.5 cm}| l}
coordenação & departamento pedagogico\\[16 ex]
\end{tabular}\\

\hline

\endlastfoot

%elementos
\textbf{Código:} \imprimircodigo\\


\textbf{Carga Horária } Teórica: \imprimircargaHorariaTeorica, Prática \imprimircargaHorariaPratica, Total: \imprimircargaHorariaTotal\\


\textbf{Número de créditos:} \imprimircreditos\\


\textbf{Código pré-requisitos:} \imprimircodigoPrerequisitos\\


\textbf{Semestre:} \imprimirsemestre\\


\textbf{Nível:} \imprimirnivel\\
\hline

\rowcolor{lightgray}
\multicolumn{1}{|p{14cm}|}{\textbf{Ementa}}\\
\hline
\multicolumn{1}{|p{14cm}|}{\imprimirementa}\\
\hline

\rowcolor{lightgray}
\multicolumn{1}{|p{14cm}|}{\textbf{Objetivo}}\\
\hline
\imprimirobjetivo\\
\hline

\rowcolor{lightgray}
\multicolumn{1}{|p{14cm}|}{\textbf{Programa}}\\
\hline
\imprimirprograma\\
\hline

\rowcolor{lightgray}
\multicolumn{1}{|p{14cm}|}{\textbf{Metodologia de ensino}}\\
\hline
\imprimirmetodologiaEnsino\\
\hline

\rowcolor{lightgray}
\multicolumn{1}{|p{14cm}|}{\textbf{Recursos}}\\
\hline
\imprimirrecursos\\
\hline

\rowcolor{lightgray}
\multicolumn{1}{|p{14cm}|}{\textbf{Avaliação}}\\
\hline
\imprimiravaliacao\\
\hline

\rowcolor{lightgray}
\multicolumn{1}{|p{14cm}|}{\textbf{Bibliografia básica}}\\
\hline
\imprimirbibliografiaBasica\\
\hline

\rowcolor{lightgray}
\multicolumn{1}{|p{14cm}|}{\textbf{Bibliografia complementar}}\\
\hline
\imprimirbibliografiaComplementar\\
\hline
\end{longtable}
\pagebreak
}
\begin{document}

\disciplina{Projetos em eletrônica}
\codigo{AUT2419}
\cargaHorariaTotal{80}
\cargaHorariaPratica{40}
\cargaHorariaTeorica{40}
\creditos{4}
\codigoPrerequisitos{AUT2416}
\semestre{4º}
\nivel{Superior}

\ementa{
Características básicas de um amplificador ideal. Modos de operação de um amplificador operacional. Projetos de controle em malha aberta e em malha fechada com AOPs. Princípios de funcionamento de um temporizador utilizando o CI 555. Sistemas temporizados. Princípios de funcionamento de circuitos osciladores. Projetos com circuitos osciladores. operar Operacionamento com sensores e transdutores de tensão, princípios básicos de projetos e montagens de circuitos eletrônicos.
}

\objetivo{
• Identificar o diagrama de pinos do amplificador operacional 741 e LM 349.\\
• Projetar e implementar circuitos lineares básicos com o amplificador\\ Operacional, comparadores de tensão com o amplificador operacional, controladores ON-OFF com o amplificador operacional.\\
• Projetar circuitos transdutores de entrada com o amplificador operacional e sensores.\\
• Projetar e implementar circuitos temporizados com o CI 555.\\
• Acionar cargas com drives de correntes e atuadores a relé.\\
• Projetar e montar osciladores com 555 e transmissores FM.\\
• Confecionar placas de circuitos.\\
}

\programa{
• Amplificadores Operacionais – A. O.\\
• Características do AOP Ganho de Tensão Impedância de Entrada Impedância de Saída \\
• Resposta em Frequência (BW) Modos de Operação do AOP Sem realimentação – Malha aberta Com realimentação – Malha fechada Realimentação Positiva – Oscilador\\
• Realimentação Negativa – Amplificador Efeito da realimentação negativa em A.O.P\\
• Conceito de Curto Circuito Virtual e Terra Virtual Circuitos lineares Básicos com AOP O amplificador Inversor – Função de Transferência\\
• O amplificador Não Inversor – Função de Transferência O seguidor de tensão – BUFFER O\\
• Amplificador Somador Inversor\\
• O Amplificador Somador não Inversor O amplificador Diferencial ou subtrator\\
•  Amplificador de CA com AOP Aplicações não – Lineares com AOPs Comparadores Comparador Regenerativo ou Schmitt Trigger \\
• Osciladores Oscilador com ponte de Wien \\
• Temporizador 555 Monoestável Astável\\
• Acionamento de Carga com Relé ( Projeto) Acionamento de Carga com Sensores \\
• Soldagem e des soldagem de componentes
• Manufatura de placas de circuitos\\
}

\metodologiaEnsino{
Aulas expositivas.\\
Aulas práticas em laboratório.\\
Resolução de lista de exercícios.\\
Leitura e pesquisa.\\
Simulação computacional utilizando software dedicado.\\
}

\recursos{
Livros contidos na bibliografia.\\
Equipamentos instrumentais de laboratório.\\
Protobords, componentes disponíveis no laboratório, placas de circuitos impressos, etc.\\
Quadro e pincel.\\
Data-show.\\
Computador com software específico.\\
Lista de exercícios.\\
}

\avaliacao{
Avaliação de aprendizagem escrita.\\
Práticas individuais e em grupo no laboratório.\\
Relatório de prática.\\
Avaliação de exercícios resolvidos.\\
Poderão ser inseridas outras avaliações durante o semestre.\\
}

\bibliografiaBasica{
• BOYLESTAD, Robert L; NASHELSKY, Louis. Dispositivos eletrônicos e teoria de
circuitos. São Paulo: Pearson, 2012.\\

• PERTENCE JÚNIOR, Antônio. Eletrônica analógica: amplificadores operacionais e filtros ativos. Porto Alegra: Artmed, 2003.\\

• U.S Navy, Brureau of Naval Personnel. Training Publication Division. Curso completo de eletrônica. São Paulo: Hemus, s.d.\\
}

\bibliografiaComplementar{
• BOURGERON, R. 1300 Esquemas e circuitos eletrônicos. Curitiba: Hemus, 2002.\\

• CIPELLI, Antônio Marcos V et. Al. Teoria e desenvolvimento de projeto de circuitos eletrônicos. São Paulo: Érica, 2001.\\

• CRUZ, Eduardo Cesar Alves; CHOUERI JUNIOR, Salomão. Eletrônica aplicada. São
Paulo: Erica, 2010.\\

• SANDIGE, Richard S. Digital design essentials. Upper Saddle River: Prentice Hall, 2001. 670 p. (Prentice Hall Xilinx Design Series) ISBN 0201476894.\\

• WAGNER, Flávio Rech; RIBAS, Renato Perez; REIS, André Inácio. Fundamentos de circuitos digitais. Porto Alegre: UFRGS. Instituto de Informática: Sagra Luzzatto, 2006. 164 p. (Série Livros didáticos. n.17) ISBN 8524107030.\\
}


\imprimirPUD

\end{document}