\input{preambulo}
%arquivo de template para os PUDS

%definição das variaveis das seções
\newcommand{\disciplina}{\def \disciplina}
\newcommand{\imprimirdisciplina}{\disciplina}

\newcommand{\codigo}{\def \codigo}
\newcommand{\imprimircodigo}{\codigo}

\newcommand{\cargaHorariaTotal}{\def \cargaHorariaTotal}
\newcommand{\imprimircargaHorariaTotal}{\cargaHorariaTotal}

\newcommand{\cargaHorariaPratica}{\def \cargaHorariaPratica}
\newcommand{\imprimircargaHorariaPratica}{\cargaHorariaPratica}

\newcommand{\cargaHorariaTeorica}{\def \cargaHorariaTeorica}
\newcommand{\imprimircargaHorariaTeorica}{\cargaHorariaTeorica}

\newcommand{\creditos}{\def \creditos}
\newcommand{\imprimircreditos}{\creditos}

\newcommand{\codigoPrerequisitos}{\def \codigoPrerequisitos}
\newcommand{\imprimircodigoPrerequisitos}{\codigoPrerequisitos}

\newcommand{\semestre}{\def \semestre}
\newcommand{\imprimirsemestre}{\semestre}

\newcommand{\nivel}{\def \nivel}
\newcommand{\imprimirnivel}{\nivel}

\newcommand{\codigoEquivalencias}{\def \codigoEquivalencias}
\newcommand{\imprimircodigoEquivalencias}{\codigoEquivalencias}

\newcommand{\ementa}{\def \ementa}
\newcommand{\imprimirementa}{\ementa}

\newcommand{\objetivo}{\def \objetivo}
\newcommand{\imprimirobjetivo}{\objetivo}

\newcommand{\programa}{\def \programa}
\newcommand{\imprimirprograma}{\programa}

\newcommand{\metodologiaEnsino}{\def \metodologiaEnsino}
\newcommand{\imprimirmetodologiaEnsino}{\metodologiaEnsino}

\newcommand{\recursos}{\def \recursos}
\newcommand{\imprimirrecursos}{\recursos}

\newcommand{\avaliacao}{\def \avaliacao}
\newcommand{\imprimiravaliacao}{\avaliacao}

\newcommand{\bibliografiaBasica}{\def \bibliografiaBasica}
\newcommand{\imprimirbibliografiaBasica}{\bibliografiaBasica}

\newcommand{\bibliografiaComplementar}{\def \bibliografiaComplementar}
\newcommand{\imprimirbibliografiaComplementar}{\bibliografiaComplementar}

\newcommand{\versao}{\def \versao}
\newcommand{\imprimirversao}{\versao}


%comando de impressão da estrutura
\newcommand{\imprimirPUD}{
%Cabeçalho do PUD
\begin{Spacing}{1}

\noindent \begin{minipage}{2.5cm}%
\includegraphics[scale=0.12]{logo-ifce}
\end{minipage}
\hspace{0.3cm}
\begin{minipage}{13cm}%
\centering INSTITUTO FEDERAL DE EDUCAÇÃO, CIÊNCIA E TECNOLOGIA DO CEARÁ- IFCE\\
CAMPUS JUAZEIRO DO NORTE\\
CURSO SUPERIOR EM AUTOMAÇÃO INDUSTRIAL\\
PROGRAMA DE UNIDADE DIDÁTICA – PUD\\
\end{minipage}%
\end{Spacing}

\begin{longtable}{|p{14cm}|}
%primeiro cabeçalho
\hline
\rowcolor{lightgray}
\multicolumn{1}{p{14cm}}{\textbf{Disciplina: \imprimirdisciplina}}\\
\hline
\endfirsthead

%cabeçalho
\hline
continuação PUD \imprimirdisciplina\\
\hline
\endhead

\hline
continua...\\
\hline
\endfoot

\hline
\rowcolor{lightgray}

\begin{tabular}{p{5.5 cm}| l}
coordenação & departamento pedagogico\\[16 ex]
\end{tabular}\\

\hline

\endlastfoot

%elementos
\textbf{Código:} \imprimircodigo\\


\textbf{Carga Horária } Teórica: \imprimircargaHorariaTeorica, Prática \imprimircargaHorariaPratica, Total: \imprimircargaHorariaTotal\\


\textbf{Número de créditos:} \imprimircreditos\\


\textbf{Código pré-requisitos:} \imprimircodigoPrerequisitos\\


\textbf{Semestre:} \imprimirsemestre\\


\textbf{Nível:} \imprimirnivel\\
\hline

\rowcolor{lightgray}
\multicolumn{1}{|p{14cm}|}{\textbf{Ementa}}\\
\hline
\multicolumn{1}{|p{14cm}|}{\imprimirementa}\\
\hline

\rowcolor{lightgray}
\multicolumn{1}{|p{14cm}|}{\textbf{Objetivo}}\\
\hline
\imprimirobjetivo\\
\hline

\rowcolor{lightgray}
\multicolumn{1}{|p{14cm}|}{\textbf{Programa}}\\
\hline
\imprimirprograma\\
\hline

\rowcolor{lightgray}
\multicolumn{1}{|p{14cm}|}{\textbf{Metodologia de ensino}}\\
\hline
\imprimirmetodologiaEnsino\\
\hline

\rowcolor{lightgray}
\multicolumn{1}{|p{14cm}|}{\textbf{Recursos}}\\
\hline
\imprimirrecursos\\
\hline

\rowcolor{lightgray}
\multicolumn{1}{|p{14cm}|}{\textbf{Avaliação}}\\
\hline
\imprimiravaliacao\\
\hline

\rowcolor{lightgray}
\multicolumn{1}{|p{14cm}|}{\textbf{Bibliografia básica}}\\
\hline
\imprimirbibliografiaBasica\\
\hline

\rowcolor{lightgray}
\multicolumn{1}{|p{14cm}|}{\textbf{Bibliografia complementar}}\\
\hline
\imprimirbibliografiaComplementar\\
\hline
\end{longtable}
\pagebreak
}
\begin{document}

\disciplina{Eletrônica geral}
\codigo{AUT2416}
\cargaHorariaTotal{80}
\cargaHorariaPratica{40}
\cargaHorariaTeorica{40}
\creditos{4}
\codigoPrerequisitos{AUT2401}
\semestre{3º}
\nivel{Superior}

\ementa{
Princípios de funcionamento dos transformadores. Processos de retificação, filtragem e regulação de tensão. Tipos de retificadores usados na implementação de fontes de alimentação. Tipos de circuitos reguladores de tensão, de funcionamento dos transistores, dos circuitos de polarizações de transistores; princípios de funcionamento dos drives de corrente, de funcionamento dos pré- amplificadores, de funcionamento dos amplificadores; princípio de funcionamento do relé (atuador) e de funcionamento dos sensores – LDR – Reed- Switch – termistores.
}

\objetivo{
• Projetar e montar fontes de alimentação simples e simétricas.\\
• Utilizar transformadores de tensão Projetar e montar pré-amplificadores de tensão e amplificadores classe A.\\
• Polarizar diodos retificadores, Zener e LED’S.\\
• Identificar os tipos (NPN ou PNP) de transistor e seus terminais ( Coletor – Base – Emissor) com o multímetro e pelos manuais do fabricante.\\
• Polarizar Transistores como chaves digitais ou amplificadores de tensão.\\
• Acionar cargas com drives de corrente.\\
• Utilizar sensores em circuitos eletrônicos.\\
}

\programa{
• Física dos Semicondutores\\
• Junção PN Diodo Polarizações Curvas\\
• Circuitos a diodo Dobradores de tensão Ceifadores\
• Limitadores e Grampeadores Diodos especiais Zener LED\\
• Transformador Circuitos retificadores\\
• Retificador de Meia Onda Retificador de Onda Completa\\
• Retificador de Onda Completa em Center-tap Retificador de Onda Completa em Ponte\\
• Filtros a capacitor de entrada\\
• Regulador de tensão Regulador de tensão Positiva Regulador de tensão Negativa\\
• Fontes Reguladas\\
• Fontes Simétricas Reguladas\\
• Confecção de Placas de Circuitos Impressos Transistor Bipolar\\
• Tipos\\
• Curvas características e dados técnicos Retas de carga\\
• Regiões de operação\\
• Circuitos de polarização\\
• Transistor Como Fonte de Corrente Transistor como Chave eletrônica Fontes a transistores estabilizadas Relés\\
• LDR – Resistor dependente de Luz\\
• Termistores – Resistências variáveis com a temperatura Reed-Switch – Chaves Magnéticas\\
• Amplificadores a transistores bipolares Pré – Amplificadores\\
• Amplificadores classe A\\
}

\metodologiaEnsino{
Aulas expositivas.\\
Aulas práticas em laboratório.\\
Resolução de lista de exercícios.\\
Leitura e pesquisa.\\
}

\recursos{
Livros contidos na bibliografia.\\
Laboratório de eletrônica\\
Quadro e pincel.\\
Data-show.\\
Simulação computacional utilizando software dedicado.\\
LIsta de exercícios.\\
}

\avaliacao{
Avaliação de aprendizagem escrita.\\
Práticas individuais e em grupo no laboratório.\\
Relatório de prática.\\
Avaliação de exercícios resolvidos.\\
Poderão ser inseridas outras avaliações durante o semestre.\\
}

\bibliografiaBasica{
• BOYLESTAD, Robert; NASHELSKY, Louis. Dispositivos eletrônicos e teoria de circuitos. Rio de Janeiro: Prentice-Hall do Brasil, 2007.\\

• FREITAS, Marcos Antônio Arantes de. ; MENDONÇA, Roberlan ornasti de. Eletrônica
básica. Curitiba: Editora do Livro Técnico, 2010.\\

• MALVINO, Albert Paul . Eletrônica – Volume 1. São Paulo: Makron Books, 1997.\\
}

\bibliografiaComplementar{
• MALVINO, Albert Paul. Eletrônica – Volume 2. São Paulo: Makron books, 1997.\\

• CIPELLI, Antônio Marcos V et. Al. Teoria e desenvolvimento de projeto de circuitos eletrônicos. São Paulo: Érica, 2001.\\

• CRUZ, Eduardo Cesar Alves; CHOUERI JUNIOR, Salomão. Eletrônica aplicada. São Paulo: Erica, 2010.\\

• PAIXÃO, Renato Rodrigues. 850 Exercícios de eletrônica resolvidos e propostos. São Paulo: Érica, 1991.\\

• URBANETZ JUNIOR, Jair; MAIA, José da Silva. Eletrônica aplicada. Curitiba: Base Editorial, 2010.\\
}


\imprimirPUD

\end{document}