\input{preambulo}
%arquivo de template para os PUDS

%definição das variaveis das seções
\newcommand{\disciplina}{\def \disciplina}
\newcommand{\imprimirdisciplina}{\disciplina}

\newcommand{\codigo}{\def \codigo}
\newcommand{\imprimircodigo}{\codigo}

\newcommand{\cargaHorariaTotal}{\def \cargaHorariaTotal}
\newcommand{\imprimircargaHorariaTotal}{\cargaHorariaTotal}

\newcommand{\cargaHorariaPratica}{\def \cargaHorariaPratica}
\newcommand{\imprimircargaHorariaPratica}{\cargaHorariaPratica}

\newcommand{\cargaHorariaTeorica}{\def \cargaHorariaTeorica}
\newcommand{\imprimircargaHorariaTeorica}{\cargaHorariaTeorica}

\newcommand{\creditos}{\def \creditos}
\newcommand{\imprimircreditos}{\creditos}

\newcommand{\codigoPrerequisitos}{\def \codigoPrerequisitos}
\newcommand{\imprimircodigoPrerequisitos}{\codigoPrerequisitos}

\newcommand{\semestre}{\def \semestre}
\newcommand{\imprimirsemestre}{\semestre}

\newcommand{\nivel}{\def \nivel}
\newcommand{\imprimirnivel}{\nivel}

\newcommand{\codigoEquivalencias}{\def \codigoEquivalencias}
\newcommand{\imprimircodigoEquivalencias}{\codigoEquivalencias}

\newcommand{\ementa}{\def \ementa}
\newcommand{\imprimirementa}{\ementa}

\newcommand{\objetivo}{\def \objetivo}
\newcommand{\imprimirobjetivo}{\objetivo}

\newcommand{\programa}{\def \programa}
\newcommand{\imprimirprograma}{\programa}

\newcommand{\metodologiaEnsino}{\def \metodologiaEnsino}
\newcommand{\imprimirmetodologiaEnsino}{\metodologiaEnsino}

\newcommand{\recursos}{\def \recursos}
\newcommand{\imprimirrecursos}{\recursos}

\newcommand{\avaliacao}{\def \avaliacao}
\newcommand{\imprimiravaliacao}{\avaliacao}

\newcommand{\bibliografiaBasica}{\def \bibliografiaBasica}
\newcommand{\imprimirbibliografiaBasica}{\bibliografiaBasica}

\newcommand{\bibliografiaComplementar}{\def \bibliografiaComplementar}
\newcommand{\imprimirbibliografiaComplementar}{\bibliografiaComplementar}

\newcommand{\versao}{\def \versao}
\newcommand{\imprimirversao}{\versao}


%comando de impressão da estrutura
\newcommand{\imprimirPUD}{
%Cabeçalho do PUD
\begin{Spacing}{1}

\noindent \begin{minipage}{2.5cm}%
\includegraphics[scale=0.12]{logo-ifce}
\end{minipage}
\hspace{0.3cm}
\begin{minipage}{13cm}%
\centering INSTITUTO FEDERAL DE EDUCAÇÃO, CIÊNCIA E TECNOLOGIA DO CEARÁ- IFCE\\
CAMPUS JUAZEIRO DO NORTE\\
CURSO SUPERIOR EM AUTOMAÇÃO INDUSTRIAL\\
PROGRAMA DE UNIDADE DIDÁTICA – PUD\\
\end{minipage}%
\end{Spacing}

\begin{longtable}{|p{14cm}|}
%primeiro cabeçalho
\hline
\rowcolor{lightgray}
\multicolumn{1}{p{14cm}}{\textbf{Disciplina: \imprimirdisciplina}}\\
\hline
\endfirsthead

%cabeçalho
\hline
continuação PUD \imprimirdisciplina\\
\hline
\endhead

\hline
continua...\\
\hline
\endfoot

\hline
\rowcolor{lightgray}

\begin{tabular}{p{5.5 cm}| l}
coordenação & departamento pedagogico\\[16 ex]
\end{tabular}\\

\hline

\endlastfoot

%elementos
\textbf{Código:} \imprimircodigo\\


\textbf{Carga Horária } Teórica: \imprimircargaHorariaTeorica, Prática \imprimircargaHorariaPratica, Total: \imprimircargaHorariaTotal\\


\textbf{Número de créditos:} \imprimircreditos\\


\textbf{Código pré-requisitos:} \imprimircodigoPrerequisitos\\


\textbf{Semestre:} \imprimirsemestre\\


\textbf{Nível:} \imprimirnivel\\
\hline

\rowcolor{lightgray}
\multicolumn{1}{|p{14cm}|}{\textbf{Ementa}}\\
\hline
\multicolumn{1}{|p{14cm}|}{\imprimirementa}\\
\hline

\rowcolor{lightgray}
\multicolumn{1}{|p{14cm}|}{\textbf{Objetivo}}\\
\hline
\imprimirobjetivo\\
\hline

\rowcolor{lightgray}
\multicolumn{1}{|p{14cm}|}{\textbf{Programa}}\\
\hline
\imprimirprograma\\
\hline

\rowcolor{lightgray}
\multicolumn{1}{|p{14cm}|}{\textbf{Metodologia de ensino}}\\
\hline
\imprimirmetodologiaEnsino\\
\hline

\rowcolor{lightgray}
\multicolumn{1}{|p{14cm}|}{\textbf{Recursos}}\\
\hline
\imprimirrecursos\\
\hline

\rowcolor{lightgray}
\multicolumn{1}{|p{14cm}|}{\textbf{Avaliação}}\\
\hline
\imprimiravaliacao\\
\hline

\rowcolor{lightgray}
\multicolumn{1}{|p{14cm}|}{\textbf{Bibliografia básica}}\\
\hline
\imprimirbibliografiaBasica\\
\hline

\rowcolor{lightgray}
\multicolumn{1}{|p{14cm}|}{\textbf{Bibliografia complementar}}\\
\hline
\imprimirbibliografiaComplementar\\
\hline
\end{longtable}
\pagebreak
}
\begin{document}

\disciplina{Metodologia da pesquisa Científica}
\codigo{AUT2435}
\cargaHorariaTotal{40}
\cargaHorariaPratica{0}
\cargaHorariaTeorica{40}
\creditos{2}
\codigoPrerequisitos{-}
\semestre{1º}
\nivel{Superior}

\ementa{
Fundamentos do conhecimento científico aplicados à Automação Industrial. Estudo da metodologia científica para a compreensão da ciência como método e técnica de pesquisa. Investigação da produção do conhecimento de Automação Industrial no que diz respeito aos seus campos de intervenção profissional. Compreender a estrutura básica das formas do conhecimento humano em seus diferentes campos: o senso comum, o religioso, o filosófico e o científico. A organização do trabalho científico conforme as normas da ABNT. A estrutura de um projeto de pesquisa, aplicação prática do mesmo na coleta e análise dos dados.
}

\objetivo{
• Compreender os elementos constitutivos do trabalho acadêmico, técnico e
cientifico.\\
• Posicionar-se criticamente a respeito do papel da pesquisa científica nos diferentes âmbitos de atuação do profissional.\\
• Discutir e reconhecer a utilidade da pesquisa científica para o engrandecimento da sua área de atuação.\\
• Distinguir e reconhecer diferentes concepções e tendências metodológicas no âmbito da pesquisa científica.\\
• Possibilitar aos alunos as Apresentar condições para a elaboração de um projeto de pesquisa, resenha, artigos, relatórios de pesquisas e pesquisas bibliográficas de acordo com as normas da ABNT.\\
• Apresentar Conhecer as formas de apresentação e exposição do trabalho científico dentro da metodologia cientifica.\\
}

\programa{
• A organização dos estudos acadêmicos.\\
• Métodos de documentação/Fichamento.\\
• A leitura. Análise e interpretação de texto.\\
• A escrita acadêmica: Estilo e linguagem. \\
Definição de ciências e conhecimento científico\\
• Conhecimento Formas de Conhecimento:\\
• Senso comum; Teológico; Filosófico; Científico. Áreas da Ciência:\\
• Ciência Básica e Aplicada.\\
• Tipos de Análise Científica:\\
• Classificação das ciências e métodos científicos;\\
• A constituição dos primeiros fundamentos para o conhecimento científico:\\
• Positivismo;\\
• Estruturalismo;\\
• Materialismo Histórico-Dialético. Tipos, Métodos e Técnicas de Pesquisa;\\
• Definição de Método Científico: Indutivo; Dedutivo;\\
• Hipotético dedutivo;\\
• Dialético;\\
• A pesquisa:\\
• Processo de Pesquisa;\\
• Modalidades da pesquisa:\\
• Quanto aos paradigmas;\\
• Quanto à abordagem; \\
• Quanto ao nível. Delineamentos e Tipos de Pesquisa; \\
• A produção científica e seus passos Passos para elaboração de uma pesquisa científica: Delimitação do Tema; Formulação do Problema; Definição dos objetos de estudo; Estipulação do Objetivo.\\
• Levantamento\\
• Bibliográfico;\\
• Compilação dos trabalhos e obras sobre o tema; Fichamento; Levantamento das Limitações da Pesquisa;\\
• Construção das Hipóteses; Variáveis da pesquisa.\\
• Definição dos procedimentos e instrumentos a empregar na pesquisa A seleção da amostra; Técnica e instrumentos para coleta de dados: Entrevista, Questionário, Observação, documentos, formulário, teste. A programação da Pesquisa (cronograma).\\
• Desenvolvimento e Execução da Pesquisa Revisão da Literatura Coleta dos dados\\
• Análise e interpretação\\
• Tratamento dos dados Codificação dos Resultados.\\
• Normas da ABNT\\
• Elementos projetos de pesquisa: Introdução\\
• Problema de pesquisa\\
• Hipóteses\\
• Questões a investigar Objetivos:\\
• Objetivo geral Objetivos específicos Justificativa\\
• Revisão da literatura Metodologia Caracterização do estudo População e amostra\\
• Variáveis de estudo\\
• Instrumentos para coletas\\
• Procedimentos para coleta de dados. Questões éticas\\
• Cronograma Recursos Referências\\
• Como elaborar trabalhos científicos: artigos, resenha e pesquisa bibliográfica de acordo com as normas da ABNT;\\
• A dissertação, a tese, os relatórios\\
}

\metodologiaEnsino{
Aulas expositivas.\\
Aulas práticas em laboratório.\\
Aulas teóricas.\\
Leituras programadas.\\
Realização de Seminários.\\
}

\recursos{
Livros e artigos científicos.\\
Quadro branco e pincel.\\
Data-show.\\
Textos.\\
Projetor multimídia.\\
Qual laboratório (equipamentos e materiais)?\\
Computadores.\\
}

\avaliacao{
A avaliação do componente curricular ocorrerá em seus aspectos quantitativos e qualitativos e serão levados em consideração as seguintes atividades:\\
Participação do discente em atividades que exijam produção individual e em equipe.\\
Planejamento, organização, coerência de ideias e clareza na elaboração de trabalhos escritos ou destinados à demonstração do domínio dos conhecimentos técnico-pedagógicos e científicos adquiridos na disciplina.\\
Elaboração e apresentação de projeto de pesquisa.\\
Escrita e apresentação de texto científico (artigo científico).\\
}

\bibliografiaBasica{
• CERVO, Amado Luis; BERVIAN, Pedro Alcino; SILVA, Roberto da. Metodologia científica28 . São Paulo: Pearson, 2007.\\
• GIL, Antônio Carlos. Como elaborar projetos de pesquisa. São Paulo: Atlas, 2002.\\
• MARCONI, Marina de Andrade; LAKATOS, Eva Maria. Metodologia científica. São Paulo: Atlas, 2008.\\
• MOURA, Maria Lucia Seidl de; FERREIRA, Maria Cristina; PAINE, Patricia Ann. Manual de elaboração de projetos de pesquisa. Rio de Janeiro: EdUERJ, 1998.\\
• RUDIO, Fran Victor. Introdução ao projeto de pesquisa científica. Petrópolis: Vozes, 2004.\\
}

\bibliografiaComplementar{
• CARVALHO, Maria Cecília M de. Construindo o saber: metodologia científica: fundamentos e técnicas. Campinas,SP: Papiros, 2007.\\
• CASTRO, Claudio de Moura. A prática da pesquisa30. São Paulo: Pearson, 2006.\\
• COSTA, Sérgio Francisco. Método científico: os caminhos da investigação. São Paulo: Harbra, 2001.
• ECO, Humberto. Como se faz uma tese. São Paulo: Pespectiva, 2007.\\
• MAGALHÃES, Gildo. Introdução a metodologia de pesquisa29: caminhos da ciência e tecnologia. São Paulo: Ática, 2005.\\
}


\imprimirPUD

\end{document}