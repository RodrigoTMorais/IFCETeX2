\input{preambulo}
%arquivo de template para os PUDS

%definição das variaveis das seções
\newcommand{\disciplina}{\def \disciplina}
\newcommand{\imprimirdisciplina}{\disciplina}

\newcommand{\codigo}{\def \codigo}
\newcommand{\imprimircodigo}{\codigo}

\newcommand{\cargaHorariaTotal}{\def \cargaHorariaTotal}
\newcommand{\imprimircargaHorariaTotal}{\cargaHorariaTotal}

\newcommand{\cargaHorariaPratica}{\def \cargaHorariaPratica}
\newcommand{\imprimircargaHorariaPratica}{\cargaHorariaPratica}

\newcommand{\cargaHorariaTeorica}{\def \cargaHorariaTeorica}
\newcommand{\imprimircargaHorariaTeorica}{\cargaHorariaTeorica}

\newcommand{\creditos}{\def \creditos}
\newcommand{\imprimircreditos}{\creditos}

\newcommand{\codigoPrerequisitos}{\def \codigoPrerequisitos}
\newcommand{\imprimircodigoPrerequisitos}{\codigoPrerequisitos}

\newcommand{\semestre}{\def \semestre}
\newcommand{\imprimirsemestre}{\semestre}

\newcommand{\nivel}{\def \nivel}
\newcommand{\imprimirnivel}{\nivel}

\newcommand{\codigoEquivalencias}{\def \codigoEquivalencias}
\newcommand{\imprimircodigoEquivalencias}{\codigoEquivalencias}

\newcommand{\ementa}{\def \ementa}
\newcommand{\imprimirementa}{\ementa}

\newcommand{\objetivo}{\def \objetivo}
\newcommand{\imprimirobjetivo}{\objetivo}

\newcommand{\programa}{\def \programa}
\newcommand{\imprimirprograma}{\programa}

\newcommand{\metodologiaEnsino}{\def \metodologiaEnsino}
\newcommand{\imprimirmetodologiaEnsino}{\metodologiaEnsino}

\newcommand{\recursos}{\def \recursos}
\newcommand{\imprimirrecursos}{\recursos}

\newcommand{\avaliacao}{\def \avaliacao}
\newcommand{\imprimiravaliacao}{\avaliacao}

\newcommand{\bibliografiaBasica}{\def \bibliografiaBasica}
\newcommand{\imprimirbibliografiaBasica}{\bibliografiaBasica}

\newcommand{\bibliografiaComplementar}{\def \bibliografiaComplementar}
\newcommand{\imprimirbibliografiaComplementar}{\bibliografiaComplementar}

\newcommand{\versao}{\def \versao}
\newcommand{\imprimirversao}{\versao}


%comando de impressão da estrutura
\newcommand{\imprimirPUD}{
%Cabeçalho do PUD
\begin{Spacing}{1}

\noindent \begin{minipage}{2.5cm}%
\includegraphics[scale=0.12]{logo-ifce}
\end{minipage}
\hspace{0.3cm}
\begin{minipage}{13cm}%
\centering INSTITUTO FEDERAL DE EDUCAÇÃO, CIÊNCIA E TECNOLOGIA DO CEARÁ- IFCE\\
CAMPUS JUAZEIRO DO NORTE\\
CURSO SUPERIOR EM AUTOMAÇÃO INDUSTRIAL\\
PROGRAMA DE UNIDADE DIDÁTICA – PUD\\
\end{minipage}%
\end{Spacing}

\begin{longtable}{|p{14cm}|}
%primeiro cabeçalho
\hline
\rowcolor{lightgray}
\multicolumn{1}{p{14cm}}{\textbf{Disciplina: \imprimirdisciplina}}\\
\hline
\endfirsthead

%cabeçalho
\hline
continuação PUD \imprimirdisciplina\\
\hline
\endhead

\hline
continua...\\
\hline
\endfoot

\hline
\rowcolor{lightgray}

\begin{tabular}{p{5.5 cm}| l}
coordenação & departamento pedagogico\\[16 ex]
\end{tabular}\\

\hline

\endlastfoot

%elementos
\textbf{Código:} \imprimircodigo\\


\textbf{Carga Horária } Teórica: \imprimircargaHorariaTeorica, Prática \imprimircargaHorariaPratica, Total: \imprimircargaHorariaTotal\\


\textbf{Número de créditos:} \imprimircreditos\\


\textbf{Código pré-requisitos:} \imprimircodigoPrerequisitos\\


\textbf{Semestre:} \imprimirsemestre\\


\textbf{Nível:} \imprimirnivel\\
\hline

\rowcolor{lightgray}
\multicolumn{1}{|p{14cm}|}{\textbf{Ementa}}\\
\hline
\multicolumn{1}{|p{14cm}|}{\imprimirementa}\\
\hline

\rowcolor{lightgray}
\multicolumn{1}{|p{14cm}|}{\textbf{Objetivo}}\\
\hline
\imprimirobjetivo\\
\hline

\rowcolor{lightgray}
\multicolumn{1}{|p{14cm}|}{\textbf{Programa}}\\
\hline
\imprimirprograma\\
\hline

\rowcolor{lightgray}
\multicolumn{1}{|p{14cm}|}{\textbf{Metodologia de ensino}}\\
\hline
\imprimirmetodologiaEnsino\\
\hline

\rowcolor{lightgray}
\multicolumn{1}{|p{14cm}|}{\textbf{Recursos}}\\
\hline
\imprimirrecursos\\
\hline

\rowcolor{lightgray}
\multicolumn{1}{|p{14cm}|}{\textbf{Avaliação}}\\
\hline
\imprimiravaliacao\\
\hline

\rowcolor{lightgray}
\multicolumn{1}{|p{14cm}|}{\textbf{Bibliografia básica}}\\
\hline
\imprimirbibliografiaBasica\\
\hline

\rowcolor{lightgray}
\multicolumn{1}{|p{14cm}|}{\textbf{Bibliografia complementar}}\\
\hline
\imprimirbibliografiaComplementar\\
\hline
\end{longtable}
\pagebreak
}
\begin{document}

\disciplina{Robótica Industrial}
\codigo{AUT2447}
\cargaHorariaTotal{40}
\cargaHorariaPratica{20}
\cargaHorariaTeorica{20}
\creditos{2}
\codigoPrerequisitos{-}
\semestre{opcional}
\nivel{Superior}

\ementa{
Funcionamento de robôs industriais manipuladores, componentes e
acessórios, modos de programação e utilização.
}

\objetivo{
• Ao final do curso o aluno deverá ser capaz de reconhecer robôs manipuladores industriais, seus tipos e modos de funcionamento e ensino.\\
• Deverá ser capaz de integrar ao sistema produtivo e programar para realização de tarefas.
}

\programa{
• Histórico e origens da robótica\\
• Robótica industrial versus robótica móvel\\
• Graus de liberdade\\
• Topologia dos manipuladores robóticos\\
• Componentes e acessórios dos robôs\\
• Mecanismos usados em robôs industriais\\
• Notação de DENAVIT-HARTENBERG\\
• Modelagem cinemática\\
• Modelos matemáticos e computacionais\\
• Modos de ensino de robos\\
• Programação off-line de robos\\
• Dispositivos anexos e de segurança.\\
}

\metodologiaEnsino{
Aulas expositivas.\\
Aulas práticas em laboratório.\\
Vídeo-Aulas.\\
Leitura e pesquisa.\\
Resolução de exercícios utilizando software apropriado.\\
}

\recursos{
Quadro e pincel.\\
Data-show.\\
Lista de exercícios.\\
laboratório de informática\\
}

\avaliacao{
Avaliação escrita.\\
Resolução de atividades individual ou em grupo.\\
Avaliação de exercícios resolvidos.\\
Poderão ser inseridas outras avaliações durante o semestre.\\
}

\bibliografiaBasica{
• BEKEY, George A. Autonomous robots: from biological inspiration to implementation and control. Massachusetts (EUA): Massachusetts Institute of Technology - MIT, 2005.\\

• CRAIG, John J. Introduction to robotics: mechanics and control. 3.ed. São Paulo: Pearson Prentice Hall, 2005.\\

• CRAIG, JOHN J. Robótica. São Paulo: Pearson Education do Brasil, 2013.\\
}

\bibliografiaComplementar{
• MITTAL, R. K.; NAGRATH, I. J. Robotics and control. New Delhi: Tata McGraw- Hill, 2006.\\

• PAZOS, Fernando. Automação de sistemas \& robótica. Rio de Janeiro: Axcel Books, 2002.\\

• ROSÁRIO, João Maurício. Princípios de mecatrônica. São Paulo: Pearson Prentice Hall, 2006.\\

• SALANT, Michael A. Introdução à robótica. São Paulo: McGraw-Hill, 1990.\\

• MADRID, Marconi Kolm. Curso sobre robôs industriais. Fortaleza (CE): UFC, 1992.\\
}


\imprimirPUD

\end{document}