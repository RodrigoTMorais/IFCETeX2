\input{preambulo}
%arquivo de template para os PUDS

%definição das variaveis das seções
\newcommand{\disciplina}{\def \disciplina}
\newcommand{\imprimirdisciplina}{\disciplina}

\newcommand{\codigo}{\def \codigo}
\newcommand{\imprimircodigo}{\codigo}

\newcommand{\cargaHorariaTotal}{\def \cargaHorariaTotal}
\newcommand{\imprimircargaHorariaTotal}{\cargaHorariaTotal}

\newcommand{\cargaHorariaPratica}{\def \cargaHorariaPratica}
\newcommand{\imprimircargaHorariaPratica}{\cargaHorariaPratica}

\newcommand{\cargaHorariaTeorica}{\def \cargaHorariaTeorica}
\newcommand{\imprimircargaHorariaTeorica}{\cargaHorariaTeorica}

\newcommand{\creditos}{\def \creditos}
\newcommand{\imprimircreditos}{\creditos}

\newcommand{\codigoPrerequisitos}{\def \codigoPrerequisitos}
\newcommand{\imprimircodigoPrerequisitos}{\codigoPrerequisitos}

\newcommand{\semestre}{\def \semestre}
\newcommand{\imprimirsemestre}{\semestre}

\newcommand{\nivel}{\def \nivel}
\newcommand{\imprimirnivel}{\nivel}

\newcommand{\codigoEquivalencias}{\def \codigoEquivalencias}
\newcommand{\imprimircodigoEquivalencias}{\codigoEquivalencias}

\newcommand{\ementa}{\def \ementa}
\newcommand{\imprimirementa}{\ementa}

\newcommand{\objetivo}{\def \objetivo}
\newcommand{\imprimirobjetivo}{\objetivo}

\newcommand{\programa}{\def \programa}
\newcommand{\imprimirprograma}{\programa}

\newcommand{\metodologiaEnsino}{\def \metodologiaEnsino}
\newcommand{\imprimirmetodologiaEnsino}{\metodologiaEnsino}

\newcommand{\recursos}{\def \recursos}
\newcommand{\imprimirrecursos}{\recursos}

\newcommand{\avaliacao}{\def \avaliacao}
\newcommand{\imprimiravaliacao}{\avaliacao}

\newcommand{\bibliografiaBasica}{\def \bibliografiaBasica}
\newcommand{\imprimirbibliografiaBasica}{\bibliografiaBasica}

\newcommand{\bibliografiaComplementar}{\def \bibliografiaComplementar}
\newcommand{\imprimirbibliografiaComplementar}{\bibliografiaComplementar}

\newcommand{\versao}{\def \versao}
\newcommand{\imprimirversao}{\versao}


%comando de impressão da estrutura
\newcommand{\imprimirPUD}{
%Cabeçalho do PUD
\begin{Spacing}{1}

\noindent \begin{minipage}{2.5cm}%
\includegraphics[scale=0.12]{logo-ifce}
\end{minipage}
\hspace{0.3cm}
\begin{minipage}{13cm}%
\centering INSTITUTO FEDERAL DE EDUCAÇÃO, CIÊNCIA E TECNOLOGIA DO CEARÁ- IFCE\\
CAMPUS JUAZEIRO DO NORTE\\
CURSO SUPERIOR EM AUTOMAÇÃO INDUSTRIAL\\
PROGRAMA DE UNIDADE DIDÁTICA – PUD\\
\end{minipage}%
\end{Spacing}

\begin{longtable}{|p{14cm}|}
%primeiro cabeçalho
\hline
\rowcolor{lightgray}
\multicolumn{1}{p{14cm}}{\textbf{Disciplina: \imprimirdisciplina}}\\
\hline
\endfirsthead

%cabeçalho
\hline
continuação PUD \imprimirdisciplina\\
\hline
\endhead

\hline
continua...\\
\hline
\endfoot

\hline
\rowcolor{lightgray}

\begin{tabular}{p{5.5 cm}| l}
coordenação & departamento pedagogico\\[16 ex]
\end{tabular}\\

\hline

\endlastfoot

%elementos
\textbf{Código:} \imprimircodigo\\


\textbf{Carga Horária } Teórica: \imprimircargaHorariaTeorica, Prática \imprimircargaHorariaPratica, Total: \imprimircargaHorariaTotal\\


\textbf{Número de créditos:} \imprimircreditos\\


\textbf{Código pré-requisitos:} \imprimircodigoPrerequisitos\\


\textbf{Semestre:} \imprimirsemestre\\


\textbf{Nível:} \imprimirnivel\\
\hline

\rowcolor{lightgray}
\multicolumn{1}{|p{14cm}|}{\textbf{Ementa}}\\
\hline
\multicolumn{1}{|p{14cm}|}{\imprimirementa}\\
\hline

\rowcolor{lightgray}
\multicolumn{1}{|p{14cm}|}{\textbf{Objetivo}}\\
\hline
\imprimirobjetivo\\
\hline

\rowcolor{lightgray}
\multicolumn{1}{|p{14cm}|}{\textbf{Programa}}\\
\hline
\imprimirprograma\\
\hline

\rowcolor{lightgray}
\multicolumn{1}{|p{14cm}|}{\textbf{Metodologia de ensino}}\\
\hline
\imprimirmetodologiaEnsino\\
\hline

\rowcolor{lightgray}
\multicolumn{1}{|p{14cm}|}{\textbf{Recursos}}\\
\hline
\imprimirrecursos\\
\hline

\rowcolor{lightgray}
\multicolumn{1}{|p{14cm}|}{\textbf{Avaliação}}\\
\hline
\imprimiravaliacao\\
\hline

\rowcolor{lightgray}
\multicolumn{1}{|p{14cm}|}{\textbf{Bibliografia básica}}\\
\hline
\imprimirbibliografiaBasica\\
\hline

\rowcolor{lightgray}
\multicolumn{1}{|p{14cm}|}{\textbf{Bibliografia complementar}}\\
\hline
\imprimirbibliografiaComplementar\\
\hline
\end{longtable}
\pagebreak
}
\begin{document}

\disciplina{Instrumentação industrial}
\codigo{AUT2436}
\cargaHorariaTotal{40}
\cargaHorariaPratica{20}
\cargaHorariaTeorica{20}
\creditos{2}
\codigoPrerequisitos{AUT2407}
\semestre{6º}
\nivel{Superior}

\ementa{
Conceitos básicos sobre medição de pressão, conceitos básicos sobre medição de nível, conceitos básicos sobre medição de vazão, conceitos básicos sobre medição de temperatura, conceitos sobre instrumentação analítica.
}

\objetivo{
• Apresentar os conceitos básicos sobre medição de pressão; \\
• Conhecer os conceitos básicos sobre medição de nível; \\
• Estudar os conceitos básicos sobre medição de vazão; \\
• Descrever os conceitos básicos sobre medição de temperatura;\\
• Descrever os elementos finais de controle.\\
}

\programa{
• Conceitos gerais sobre instrumentação industrial\\
• SPAN;\\
• RANGE;\\
• Erro;\\
• Precisão;\\
• Zona morta;\\
• Repetibilidade\\
• Alibração;\\
• Aferição;\\
• Instrumentos para medição de pressão\\
• Manômetro (Bourdon);\\
• Medição de pressão diferencial;\\
• Instrumentos para medição de nível\\
• Medidores capacitivos;\\
• Ultra-som;\\
• Por bóia;\\
• Instrumentos para medição de fluxo de fluidos\\
• Medidores magnéticos;\\
• Rotâmetros;\\
• Placas de orifício;\\
• Instrumentos para medição de temperatura\\
• Termômetros de bulbo de vidro;\\
• Termopares;\\
• Termoresistências de platina;\\
• Resistores variáveis (PTC e NTC).\\
• Instrumentação analítica\\
• Medidores de Ph;\\
• Analisadores de condutividade;\\
• Cromatógrafos;\\
• Analisadores de densidade.\\
}

\metodologiaEnsino{
Aulas expositivas;\\
Aulas práticas em laboratório; \\
Lista de exercícios;\\
}

\recursos{
Livros contidos na bibliografia; \\
Equipamentos instrumentais de laboratório\\
Protobords, componentes disponíveis no laboratório, placas de circuitos impressos, etc. \\
Quadro e pincel.\\
Data-show\\
Simulação computacional utilizando software dedicado.\\
}

\avaliacao{
Avaliação de aprendizagem escrita;\\
Práticas individuais e em grupo no laboratório; Relatório de prática;\\
Listas de exercícios;\\
Poderão ser inseridas outras avaliações durante o semestre\\
}

\bibliografiaBasica{
• BEGA, Egídio A (Org). Instrumentação Industrial. Rio de Janeiro: Interciência, Instituto Brasileiro de Petróleo e Gás, 2006.\\

• THOMAZINI, Daniel;   ALBUQUERQUE,   Pedro   Urbano   Braga   de.   Sensores   industriais: fundamentos e aplicações. São Paulo: Érica, 2009.\\

• ALVES, José Luiz Loureiro. Instrumentação, controle e automação de processos. Rio de Janeiro: LTC, 2005.\\
}

\bibliografiaComplementar{
• Luis Antonio Aguirre. Fundamentos de instrumentação31. São Paulo: Pearson, 2013.\\

• FIALHO, Arivelto Bustamente. Instrumentação industrial: conceitos aplicações e análise. São Paulo: Erica, 2006.\\

• BALBINOT, Alexandre; BRUSAMARELLO, Valner João. Instrumentação e fundamentos de medidas: Volume 1. Rio de Janeiro: LTC, 2006.\\
    
• SIGHIERI, L.; NISHINARI, A. Controle automático de processos industriais: Instrumentação, Edgard Blücher, 1973.\\

• OGATA, K. Teoria de controle moderno. Prentice Hall, 1998  \\
}


\imprimirPUD

\end{document}