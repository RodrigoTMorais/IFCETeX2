\input{preambulo}
%arquivo de template para os PUDS

%definição das variaveis das seções
\newcommand{\disciplina}{\def \disciplina}
\newcommand{\imprimirdisciplina}{\disciplina}

\newcommand{\codigo}{\def \codigo}
\newcommand{\imprimircodigo}{\codigo}

\newcommand{\cargaHorariaTotal}{\def \cargaHorariaTotal}
\newcommand{\imprimircargaHorariaTotal}{\cargaHorariaTotal}

\newcommand{\cargaHorariaPratica}{\def \cargaHorariaPratica}
\newcommand{\imprimircargaHorariaPratica}{\cargaHorariaPratica}

\newcommand{\cargaHorariaTeorica}{\def \cargaHorariaTeorica}
\newcommand{\imprimircargaHorariaTeorica}{\cargaHorariaTeorica}

\newcommand{\creditos}{\def \creditos}
\newcommand{\imprimircreditos}{\creditos}

\newcommand{\codigoPrerequisitos}{\def \codigoPrerequisitos}
\newcommand{\imprimircodigoPrerequisitos}{\codigoPrerequisitos}

\newcommand{\semestre}{\def \semestre}
\newcommand{\imprimirsemestre}{\semestre}

\newcommand{\nivel}{\def \nivel}
\newcommand{\imprimirnivel}{\nivel}

\newcommand{\codigoEquivalencias}{\def \codigoEquivalencias}
\newcommand{\imprimircodigoEquivalencias}{\codigoEquivalencias}

\newcommand{\ementa}{\def \ementa}
\newcommand{\imprimirementa}{\ementa}

\newcommand{\objetivo}{\def \objetivo}
\newcommand{\imprimirobjetivo}{\objetivo}

\newcommand{\programa}{\def \programa}
\newcommand{\imprimirprograma}{\programa}

\newcommand{\metodologiaEnsino}{\def \metodologiaEnsino}
\newcommand{\imprimirmetodologiaEnsino}{\metodologiaEnsino}

\newcommand{\recursos}{\def \recursos}
\newcommand{\imprimirrecursos}{\recursos}

\newcommand{\avaliacao}{\def \avaliacao}
\newcommand{\imprimiravaliacao}{\avaliacao}

\newcommand{\bibliografiaBasica}{\def \bibliografiaBasica}
\newcommand{\imprimirbibliografiaBasica}{\bibliografiaBasica}

\newcommand{\bibliografiaComplementar}{\def \bibliografiaComplementar}
\newcommand{\imprimirbibliografiaComplementar}{\bibliografiaComplementar}

\newcommand{\versao}{\def \versao}
\newcommand{\imprimirversao}{\versao}


%comando de impressão da estrutura
\newcommand{\imprimirPUD}{
%Cabeçalho do PUD
\begin{Spacing}{1}

\noindent \begin{minipage}{2.5cm}%
\includegraphics[scale=0.12]{logo-ifce}
\end{minipage}
\hspace{0.3cm}
\begin{minipage}{13cm}%
\centering INSTITUTO FEDERAL DE EDUCAÇÃO, CIÊNCIA E TECNOLOGIA DO CEARÁ- IFCE\\
CAMPUS JUAZEIRO DO NORTE\\
CURSO SUPERIOR EM AUTOMAÇÃO INDUSTRIAL\\
PROGRAMA DE UNIDADE DIDÁTICA – PUD\\
\end{minipage}%
\end{Spacing}

\begin{longtable}{|p{14cm}|}
%primeiro cabeçalho
\hline
\rowcolor{lightgray}
\multicolumn{1}{p{14cm}}{\textbf{Disciplina: \imprimirdisciplina}}\\
\hline
\endfirsthead

%cabeçalho
\hline
continuação PUD \imprimirdisciplina\\
\hline
\endhead

\hline
continua...\\
\hline
\endfoot

\hline
\rowcolor{lightgray}

\begin{tabular}{p{5.5 cm}| l}
coordenação & departamento pedagogico\\[16 ex]
\end{tabular}\\

\hline

\endlastfoot

%elementos
\textbf{Código:} \imprimircodigo\\


\textbf{Carga Horária } Teórica: \imprimircargaHorariaTeorica, Prática \imprimircargaHorariaPratica, Total: \imprimircargaHorariaTotal\\


\textbf{Número de créditos:} \imprimircreditos\\


\textbf{Código pré-requisitos:} \imprimircodigoPrerequisitos\\


\textbf{Semestre:} \imprimirsemestre\\


\textbf{Nível:} \imprimirnivel\\
\hline

\rowcolor{lightgray}
\multicolumn{1}{|p{14cm}|}{\textbf{Ementa}}\\
\hline
\multicolumn{1}{|p{14cm}|}{\imprimirementa}\\
\hline

\rowcolor{lightgray}
\multicolumn{1}{|p{14cm}|}{\textbf{Objetivo}}\\
\hline
\imprimirobjetivo\\
\hline

\rowcolor{lightgray}
\multicolumn{1}{|p{14cm}|}{\textbf{Programa}}\\
\hline
\imprimirprograma\\
\hline

\rowcolor{lightgray}
\multicolumn{1}{|p{14cm}|}{\textbf{Metodologia de ensino}}\\
\hline
\imprimirmetodologiaEnsino\\
\hline

\rowcolor{lightgray}
\multicolumn{1}{|p{14cm}|}{\textbf{Recursos}}\\
\hline
\imprimirrecursos\\
\hline

\rowcolor{lightgray}
\multicolumn{1}{|p{14cm}|}{\textbf{Avaliação}}\\
\hline
\imprimiravaliacao\\
\hline

\rowcolor{lightgray}
\multicolumn{1}{|p{14cm}|}{\textbf{Bibliografia básica}}\\
\hline
\imprimirbibliografiaBasica\\
\hline

\rowcolor{lightgray}
\multicolumn{1}{|p{14cm}|}{\textbf{Bibliografia complementar}}\\
\hline
\imprimirbibliografiaComplementar\\
\hline
\end{longtable}
\pagebreak
}
\begin{document}

\disciplina{Desenho assistido por computador}
\codigo{AUT2406}
\cargaHorariaTotal{80}
\cargaHorariaPratica{75}
\cargaHorariaTeorica{5}
\creditos{4}
\codigoPrerequisitos{-}
\semestre{3º}
\nivel{Superior}

\ementa{
Noções e interpretação de desenho técnico mecânico. Introdução aos Sistemas de Desenho Assistido por computador. Noções, conceitos e técnicas fundamentais dos sistemas CAD. Coordenadas. Elementos geométricos básicos. CAD paramétrico, criação de sólidos geométricos, operações com sólidos, cotas, elementos padronizados, criação de desenhos técnicos, interpretação,vistas e cortes, detalhes e anotações, montagens virtuais, vistas explodidas de conjuntos.
}

\objetivo{
• Conhecer um software de desenho.\\
• Ler e interpretar desenho técnico mecânico.\\
• Desenhar usando software de desenho paramétrico.\\
• Criar pranchetas de desenho técnico com as principais vistas e detalhes.\\
• Montagens virtuais\\
• Desenho de vista explodida de conjuntos.\\
}

\programa{
• INTRODUÇÃO A UM SOFTWARE CAD: tipos de CAD, vistas, planos e eixos de desenho, introdução a modelagem paramétrica. Interface do software de desenho.\\
• RECURSOS BASICO DE ESBOLSOS; Retângulo, circulo, linha, corte, referencias, restrições e cotas.\\
• RECURSO BASICO DE CRIAÇÃO DE SOLIDOS. Extrusão, revolução, arredondamento, chanfro, casca;\\
• RECURSOS DE PADRÃO; Padrão retangular, padrão circular, plano, espelho.\\
• RECURSO DE COMPONENTES PADRONIZADOS; Furos padronizados, tipos de parafusos, folgas, roscas e rebaixos de parafusos; engrenagens cremalheiras e polias.\\
• CRIAÇÃO DE PRANCHAS DE DESENHO; Definição da folha e norma de bordas e legendas, adição de vistas, anotação de cotas e detalhes, vistas de detalhes.\\
• MONTAGEM VIRTUAL DE CONJUNTOS; Adição de peças, importação de peças, edição de peça na montagem, restrições, vista explodida, lista de materiais.\\
}

\metodologiaEnsino{
Aulas práticas onde será introduzido o software de modelagem mostrando a cada ferramenta e em seguida realizando atividades práticas para fixação do entendimento;\\
Atividades;\\
Vídeo aulas de reforço;\\
Criação de protótipo através da impressão 3D, possibilitando que o aluno tenha noção de sua própria criação virtual no mundo real;\\
}

\recursos{
Laboratório de informática com software específico;\\
Equipamento de apresentação (data-show, Tv ou equivalente)\\
Laboratório de prototipagem com impressoras 3D;\\
Consumíveis de impressora 3D;\\
Vídeo aulas\\
}

\avaliacao{
A avaliação será realizada através da aplicação de atividades práticas desenvolvidas no software utilizado. Será avaliada a correta utilização dos recursos, dimensões e forma final.\\
Também será considerada a capacidade de interpretação do desenho técnico através da observação do desenho final.\\
Os alunos também poderão ser avaliados pela assiduidade, entende-se, que por esta disciplina demandar habilidade, esta é obtida pela pratica o que é executado no decorrer das aulas.\\
}

\bibliografiaBasica{
• FreeCAD. Manual: Introdução. Versão 0.18. Disponível em: FreeCAD Documentation.
• DEHMLOW, Martin; KIEL, E. Desenho mecânico. São Paulo: EPU : EDUSP, 1974. v. 1.\\
• JONES, Franklin D. Manual técnico para desenhistas e projetistas de máquinas. 14. ed.São Paulo: Hemus, 1975. v.1.\\
}

\bibliografiaComplementar{
• JONES, Franklin D. Manual técnico para desenhistas e projetistas de máquinas. 14. ed. São Paulo: Hemus, 1975. v. 2.\\
• MANFÉ, Giovanni. Desenho técnico mecânico: curso completo para as escolas técnicas e ciclo básico das faculdades de engenharia: v. 1. São Paulo: Hemus, 2004. v.1.\\
• AUTODESK. Support \& Learning. Disponível em \url{https://www.autodesk.com.br/support/technical/product/inventor} .Acesso em 23/02/2024.\\
• SOLIDWORKS. Community. Disponível em \url{https://www.solidworks.com/pt-br/support/student} . Acesso em 23/02/2024.\\
• SKETCHUP. Centro de Aprendizagem. Disponível em < \url{https://www.sketchup.com/pt-BR/learn}. Acesso em 23/03/2016.\\
}


\imprimirPUD

\end{document}