\input{preambulo}
%arquivo de template para os PUDS

%definição das variaveis das seções
\newcommand{\disciplina}{\def \disciplina}
\newcommand{\imprimirdisciplina}{\disciplina}

\newcommand{\codigo}{\def \codigo}
\newcommand{\imprimircodigo}{\codigo}

\newcommand{\cargaHorariaTotal}{\def \cargaHorariaTotal}
\newcommand{\imprimircargaHorariaTotal}{\cargaHorariaTotal}

\newcommand{\cargaHorariaPratica}{\def \cargaHorariaPratica}
\newcommand{\imprimircargaHorariaPratica}{\cargaHorariaPratica}

\newcommand{\cargaHorariaTeorica}{\def \cargaHorariaTeorica}
\newcommand{\imprimircargaHorariaTeorica}{\cargaHorariaTeorica}

\newcommand{\creditos}{\def \creditos}
\newcommand{\imprimircreditos}{\creditos}

\newcommand{\codigoPrerequisitos}{\def \codigoPrerequisitos}
\newcommand{\imprimircodigoPrerequisitos}{\codigoPrerequisitos}

\newcommand{\semestre}{\def \semestre}
\newcommand{\imprimirsemestre}{\semestre}

\newcommand{\nivel}{\def \nivel}
\newcommand{\imprimirnivel}{\nivel}

\newcommand{\codigoEquivalencias}{\def \codigoEquivalencias}
\newcommand{\imprimircodigoEquivalencias}{\codigoEquivalencias}

\newcommand{\ementa}{\def \ementa}
\newcommand{\imprimirementa}{\ementa}

\newcommand{\objetivo}{\def \objetivo}
\newcommand{\imprimirobjetivo}{\objetivo}

\newcommand{\programa}{\def \programa}
\newcommand{\imprimirprograma}{\programa}

\newcommand{\metodologiaEnsino}{\def \metodologiaEnsino}
\newcommand{\imprimirmetodologiaEnsino}{\metodologiaEnsino}

\newcommand{\recursos}{\def \recursos}
\newcommand{\imprimirrecursos}{\recursos}

\newcommand{\avaliacao}{\def \avaliacao}
\newcommand{\imprimiravaliacao}{\avaliacao}

\newcommand{\bibliografiaBasica}{\def \bibliografiaBasica}
\newcommand{\imprimirbibliografiaBasica}{\bibliografiaBasica}

\newcommand{\bibliografiaComplementar}{\def \bibliografiaComplementar}
\newcommand{\imprimirbibliografiaComplementar}{\bibliografiaComplementar}

\newcommand{\versao}{\def \versao}
\newcommand{\imprimirversao}{\versao}


%comando de impressão da estrutura
\newcommand{\imprimirPUD}{
%Cabeçalho do PUD
\begin{Spacing}{1}

\noindent \begin{minipage}{2.5cm}%
\includegraphics[scale=0.12]{logo-ifce}
\end{minipage}
\hspace{0.3cm}
\begin{minipage}{13cm}%
\centering INSTITUTO FEDERAL DE EDUCAÇÃO, CIÊNCIA E TECNOLOGIA DO CEARÁ- IFCE\\
CAMPUS JUAZEIRO DO NORTE\\
CURSO SUPERIOR EM AUTOMAÇÃO INDUSTRIAL\\
PROGRAMA DE UNIDADE DIDÁTICA – PUD\\
\end{minipage}%
\end{Spacing}

\begin{longtable}{|p{14cm}|}
%primeiro cabeçalho
\hline
\rowcolor{lightgray}
\multicolumn{1}{p{14cm}}{\textbf{Disciplina: \imprimirdisciplina}}\\
\hline
\endfirsthead

%cabeçalho
\hline
continuação PUD \imprimirdisciplina\\
\hline
\endhead

\hline
continua...\\
\hline
\endfoot

\hline
\rowcolor{lightgray}

\begin{tabular}{p{5.5 cm}| l}
coordenação & departamento pedagogico\\[16 ex]
\end{tabular}\\

\hline

\endlastfoot

%elementos
\textbf{Código:} \imprimircodigo\\


\textbf{Carga Horária } Teórica: \imprimircargaHorariaTeorica, Prática \imprimircargaHorariaPratica, Total: \imprimircargaHorariaTotal\\


\textbf{Número de créditos:} \imprimircreditos\\


\textbf{Código pré-requisitos:} \imprimircodigoPrerequisitos\\


\textbf{Semestre:} \imprimirsemestre\\


\textbf{Nível:} \imprimirnivel\\
\hline

\rowcolor{lightgray}
\multicolumn{1}{|p{14cm}|}{\textbf{Ementa}}\\
\hline
\multicolumn{1}{|p{14cm}|}{\imprimirementa}\\
\hline

\rowcolor{lightgray}
\multicolumn{1}{|p{14cm}|}{\textbf{Objetivo}}\\
\hline
\imprimirobjetivo\\
\hline

\rowcolor{lightgray}
\multicolumn{1}{|p{14cm}|}{\textbf{Programa}}\\
\hline
\imprimirprograma\\
\hline

\rowcolor{lightgray}
\multicolumn{1}{|p{14cm}|}{\textbf{Metodologia de ensino}}\\
\hline
\imprimirmetodologiaEnsino\\
\hline

\rowcolor{lightgray}
\multicolumn{1}{|p{14cm}|}{\textbf{Recursos}}\\
\hline
\imprimirrecursos\\
\hline

\rowcolor{lightgray}
\multicolumn{1}{|p{14cm}|}{\textbf{Avaliação}}\\
\hline
\imprimiravaliacao\\
\hline

\rowcolor{lightgray}
\multicolumn{1}{|p{14cm}|}{\textbf{Bibliografia básica}}\\
\hline
\imprimirbibliografiaBasica\\
\hline

\rowcolor{lightgray}
\multicolumn{1}{|p{14cm}|}{\textbf{Bibliografia complementar}}\\
\hline
\imprimirbibliografiaComplementar\\
\hline
\end{longtable}
\pagebreak
}
\begin{document}

\disciplina{Comandos Elétricos}
\codigo{AUT2421}
\cargaHorariaTotal{80}
\cargaHorariaPratica{40}
\cargaHorariaTeorica{40}
\creditos{4}
\codigoPrerequisitos{AUT2424}
\semestre{5º}
\nivel{Superior}

\ementa{
Características e especificações dos dispositivos de proteção e comandos. Esquemas e Simbologias de comandos e suas normas. Sistemas de partidas de Motores. Simulação de comandos no computador.
}

\objetivo{
• Identificar e especificar componentes utilizados nas chaves de comando.\\
• Analisar esquemas de comando e proteção em baixa tensão.\\
• Dimensionar dispositivos de comandos elétricos para partida de motores.\\
• Identificar e resolver problemas de comandos elétricos.\\
• Projetar quadros de comandos para equipamentos industriais.\\
}

\programa{
• Características e especificações dos dispositivos de proteção e comandos\\
• Fusíveis e disjuntores\\
• Contactores e relés térmicos\\
• Botões de comando e sinaleiros\\
• Relés eletrônicos de comando e proteção\\
• Auto transformador de partida\\
• Esquemas e Simbologias de comandos e suas normas\\
• Normas\\
• Simbologia\\
• Esquemas de ligação\\
• Esquema de força e comando\\
• Simulação e Técnicas de partida de motores\\
• Partida direta\\
• Partida direta com Reversão\\
• Partida estrela triângulo\\
• Partida estrela triângulo com reversão\\
• Partida compensada\\
}

\metodologiaEnsino{
Aulas expositivas.\\
Aulas práticas em laboratório.\\
Resolução de lista de exercícios.\\
Realização de seminários.\\
Visitas técnicas.\\
}

\recursos{
Recursos audiovisuais.\\
Laboratório de comandos elétricos\\
Livros contidos na bibliografia.\\
Artigos.\\
Quadro e pincel.\\
Data-show.\\
Leitura e pesquisa.\\
Transporte para visitas técnicas.\\
}

\avaliacao{
Avaliação escrita.\\
Práticas individuais e em grupo no laboratório.\\
Apresentação de seminários.\\
Apresentação de relatório.\\
Avaliação de exercícios resolvidos.\\
}

\bibliografiaBasica{
• FRANCHI, Clainton Moro. Acionamentos Elétricos. São Paulo: Érica, 2008.\\

• NASCIMENTO, Geraldo. Comandos Elétricos Teoria e Atividades. São Paulo: Érica,
2018.\\

• PAPENKORT, Franz. Esquemas elétricos comandos de proteção. São Paulo: EPU,
2010.\\
}

\bibliografiaComplementar{
• PERAIRE, J. M. P. Manual do Montador de Quadros Elétricos. 2ª ed. São Paulo: Hemus, 2004. \\

• FRANCHI, C. M. Inversores de Freqüência - Teoria e Aplicações. 1ª ed. São Paulo: Érica, 2008.\\

• FRANCHI, C. M. Acionamentos Elétricos. 4ª ed. São Paulo: Érica, 2007. \\

• ALBUQUERQUE, P. U. B. de. Sensores Industriais: Fundamentos e Aplicações. 6ª ed. São Paulo: Érica, 2008.\\

• ROLDAN, J. Manual de Bobinagem. 1ª ed. São Paulo: Hemus, 2002. \\
}


\imprimirPUD

\end{document}