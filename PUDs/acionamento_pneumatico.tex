\input{preambulo}
%arquivo de template para os PUDS

%definição das variaveis das seções
\newcommand{\disciplina}{\def \disciplina}
\newcommand{\imprimirdisciplina}{\disciplina}

\newcommand{\codigo}{\def \codigo}
\newcommand{\imprimircodigo}{\codigo}

\newcommand{\cargaHorariaTotal}{\def \cargaHorariaTotal}
\newcommand{\imprimircargaHorariaTotal}{\cargaHorariaTotal}

\newcommand{\cargaHorariaPratica}{\def \cargaHorariaPratica}
\newcommand{\imprimircargaHorariaPratica}{\cargaHorariaPratica}

\newcommand{\cargaHorariaTeorica}{\def \cargaHorariaTeorica}
\newcommand{\imprimircargaHorariaTeorica}{\cargaHorariaTeorica}

\newcommand{\creditos}{\def \creditos}
\newcommand{\imprimircreditos}{\creditos}

\newcommand{\codigoPrerequisitos}{\def \codigoPrerequisitos}
\newcommand{\imprimircodigoPrerequisitos}{\codigoPrerequisitos}

\newcommand{\semestre}{\def \semestre}
\newcommand{\imprimirsemestre}{\semestre}

\newcommand{\nivel}{\def \nivel}
\newcommand{\imprimirnivel}{\nivel}

\newcommand{\codigoEquivalencias}{\def \codigoEquivalencias}
\newcommand{\imprimircodigoEquivalencias}{\codigoEquivalencias}

\newcommand{\ementa}{\def \ementa}
\newcommand{\imprimirementa}{\ementa}

\newcommand{\objetivo}{\def \objetivo}
\newcommand{\imprimirobjetivo}{\objetivo}

\newcommand{\programa}{\def \programa}
\newcommand{\imprimirprograma}{\programa}

\newcommand{\metodologiaEnsino}{\def \metodologiaEnsino}
\newcommand{\imprimirmetodologiaEnsino}{\metodologiaEnsino}

\newcommand{\recursos}{\def \recursos}
\newcommand{\imprimirrecursos}{\recursos}

\newcommand{\avaliacao}{\def \avaliacao}
\newcommand{\imprimiravaliacao}{\avaliacao}

\newcommand{\bibliografiaBasica}{\def \bibliografiaBasica}
\newcommand{\imprimirbibliografiaBasica}{\bibliografiaBasica}

\newcommand{\bibliografiaComplementar}{\def \bibliografiaComplementar}
\newcommand{\imprimirbibliografiaComplementar}{\bibliografiaComplementar}

\newcommand{\versao}{\def \versao}
\newcommand{\imprimirversao}{\versao}


%comando de impressão da estrutura
\newcommand{\imprimirPUD}{
%Cabeçalho do PUD
\begin{Spacing}{1}

\noindent \begin{minipage}{2.5cm}%
\includegraphics[scale=0.12]{logo-ifce}
\end{minipage}
\hspace{0.3cm}
\begin{minipage}{13cm}%
\centering INSTITUTO FEDERAL DE EDUCAÇÃO, CIÊNCIA E TECNOLOGIA DO CEARÁ- IFCE\\
CAMPUS JUAZEIRO DO NORTE\\
CURSO SUPERIOR EM AUTOMAÇÃO INDUSTRIAL\\
PROGRAMA DE UNIDADE DIDÁTICA – PUD\\
\end{minipage}%
\end{Spacing}

\begin{longtable}{|p{14cm}|}
%primeiro cabeçalho
\hline
\rowcolor{lightgray}
\multicolumn{1}{p{14cm}}{\textbf{Disciplina: \imprimirdisciplina}}\\
\hline
\endfirsthead

%cabeçalho
\hline
continuação PUD \imprimirdisciplina\\
\hline
\endhead

\hline
continua...\\
\hline
\endfoot

\hline
\rowcolor{lightgray}

\begin{tabular}{p{5.5 cm}| l}
coordenação & departamento pedagogico\\[16 ex]
\end{tabular}\\

\hline

\endlastfoot

%elementos
\textbf{Código:} \imprimircodigo\\


\textbf{Carga Horária } Teórica: \imprimircargaHorariaTeorica, Prática \imprimircargaHorariaPratica, Total: \imprimircargaHorariaTotal\\


\textbf{Número de créditos:} \imprimircreditos\\


\textbf{Código pré-requisitos:} \imprimircodigoPrerequisitos\\


\textbf{Semestre:} \imprimirsemestre\\


\textbf{Nível:} \imprimirnivel\\
\hline

\rowcolor{lightgray}
\multicolumn{1}{|p{14cm}|}{\textbf{Ementa}}\\
\hline
\multicolumn{1}{|p{14cm}|}{\imprimirementa}\\
\hline

\rowcolor{lightgray}
\multicolumn{1}{|p{14cm}|}{\textbf{Objetivo}}\\
\hline
\imprimirobjetivo\\
\hline

\rowcolor{lightgray}
\multicolumn{1}{|p{14cm}|}{\textbf{Programa}}\\
\hline
\imprimirprograma\\
\hline

\rowcolor{lightgray}
\multicolumn{1}{|p{14cm}|}{\textbf{Metodologia de ensino}}\\
\hline
\imprimirmetodologiaEnsino\\
\hline

\rowcolor{lightgray}
\multicolumn{1}{|p{14cm}|}{\textbf{Recursos}}\\
\hline
\imprimirrecursos\\
\hline

\rowcolor{lightgray}
\multicolumn{1}{|p{14cm}|}{\textbf{Avaliação}}\\
\hline
\imprimiravaliacao\\
\hline

\rowcolor{lightgray}
\multicolumn{1}{|p{14cm}|}{\textbf{Bibliografia básica}}\\
\hline
\imprimirbibliografiaBasica\\
\hline

\rowcolor{lightgray}
\multicolumn{1}{|p{14cm}|}{\textbf{Bibliografia complementar}}\\
\hline
\imprimirbibliografiaComplementar\\
\hline
\end{longtable}
\pagebreak
}
\begin{document}

\disciplina{Acionamento pneumático e eletropneumático}
\codigo{AUT2428}
\cargaHorariaTotal{80}
\cargaHorariaPratica{40}
\cargaHorariaTeorica{40}
\creditos{4}
\codigoPrerequisitos{-}
\semestre{5º}
\nivel{Superior}

\ementa{
Componentes de circuitos pneumáticos. Circuitos pneumáticos. Dimensionar componentes Eletropneumáticos. Projetar circuitos Eletropneumáticos. Executar manutenção preventiva em circuitos Eletropneumáticos. Realizar manutenção corretiva em circuitos Eletropneumáticos.
}

\objetivo{
• Dimensionar componentes pneumáticos Projetar circuitos hidráulicos e pneumáticos\\
• Executar manutenção preventiva em circuitos pneumáticos Realizar manutenção corretiva em circuitos pneumáticos\\
• Identificar os componentes utilizados nos circuitos Eletropneumáticos Analisar e desenvolver circuitos Eletropneumáticos\\
• Justificar a utilização de circuitos Eletropneumáticos\\
}

\programa{
• Pneumática Considerações gerais\\
• Características do ar comprimido Vantagens\\
• Desvantagens Compressores Classificação Tipos\\
• Regulagem da capacidade Manutenção\\
• Sistemas de refrigeração Ar comprimido Reservatório\\
• Dimensionamento da rede condutora Escolha do diâmetro da tubulação Cálculo da tubulação\\
• Distribuição Tubulações Preparação\\
• Unidades de Conservação Manutenção\\
• Elementos Pneumáticos de Trabalho\\
• Cilindros pneumáticos de simples e dupla ação Cálculos dos cilindros\\
• Força do embolo Consumo de ar Motores pneumáticos Válvulas\\
• Válvulas direcionais Meios de acionamentos\\
• Características de construção Valores de vazão\\
• Válvulas de bloqueio Válvulas de pressão Válvulas de fluxo Válvulas de fechamento\\
• Emissão de Sinais por Detecção Tipos de sinais por detecção Barreira de ar\\
• Sensores de reflexão Tubo sensor\\
• Comutação por detecção magnética Amplificadores\
• Comandos Básicos Comando direto de cilindros\\
• Comando de duas diferentes posições\\
• Comando com velocidade do embolo controlada no avanço e no retorno Comando com velocidade do embolo acelerada\\
• Comando com acionamento simultâneo de duas válvulas direcionais Comando indireto de um cilindro de ação simples\\
• Eletropneumática Considerações gerais Vantagens Desvantagens\\
• Comandos eletropneumáticos básicos Construção do esquema de comandos Construção de esquemas de comando Elementos eletropneumáticos de trabalho\\
• Cilindros pneumáticos de simples e dupla ação Cálculo de Força do embolo\\
• Motores pneumáticos Válvulas\\
• Válvulas direcionais Meios de acionamentos\\
• Características de construção Valores de vazão\\
• Válvulas de bloqueio Válvulas de pressão Válvulas de fluxo Válvulas de fechamento\\
• Emissão de Sinais por Detecção Tipos de sinais por detecção Barreira de ar\\
• Sensores de reflexão Tubo sensor\\
• Comutação por detecção magnética Amplificadores\\
}

\metodologiaEnsino{
Aulas expositivas;\\
Aulas práticas em laboratório; Exercícios e projetos.\\
Lista de exercícios;\\
Simulação computacional utilizando software dedicado.\\
}

\recursos{
Livros contidos na bibliografia;\\
Quadro Branco e pincel;\\
Data-show;\\
Bancada Didática.\\
}

\avaliacao{
Avaliação escrita;\\
Práticas individuais e em grupo no laboratório; Relatório de prática;\\
Listas de exercícios;\\
Poderão ser inseridas outras avaliações durante o semestre.\\
}

\bibliografiaBasica{
• FIALHO, Arivelto Bustamante. Automação Pneumática: projetos, dimensionamento e análise de circuitos. São Paulo: Érica, 2007.\\

• STEWART, Harry L. Pneumática e hidráulicas. Curitiba: Hemus, s.d.
BONACORSO, Nelso Gauze. NOLL Valdir. Automação eletropneumática. São Paulo: Erica, 2006.\\

• COSTA, Ennio Cruz da. Compressores. São Paulo: Edgar Blücher, 1988.\\

}

\bibliografiaComplementar{
• FESTO DIDATIC, Automação Pneumática. 10 ª edição. São Paulo: Festo Didatic, 2002;\\
• FESTO DIDATIC, Introdução a Pneumática. São Paulo: Festo Didatic, 2004; \\
• FESTO DIDATIC, Introdução a Hidráulica. São Paulo: Festo Didatic, 2004; \\
• FESTO DIDATIC, Introdução a Sistemas Eletropneumáticos. São Paulo: Festo Didatic, 2004; \\
• FESTO DIDATIC, Introdução a Sistemas Eletro-Hidráulicos. São Paulo: Festo Didatic, 2004;\
}


\imprimirPUD

\end{document}