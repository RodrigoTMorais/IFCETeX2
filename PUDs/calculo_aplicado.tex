\input{preambulo}
%arquivo de template para os PUDS

%definição das variaveis das seções
\newcommand{\disciplina}{\def \disciplina}
\newcommand{\imprimirdisciplina}{\disciplina}

\newcommand{\codigo}{\def \codigo}
\newcommand{\imprimircodigo}{\codigo}

\newcommand{\cargaHorariaTotal}{\def \cargaHorariaTotal}
\newcommand{\imprimircargaHorariaTotal}{\cargaHorariaTotal}

\newcommand{\cargaHorariaPratica}{\def \cargaHorariaPratica}
\newcommand{\imprimircargaHorariaPratica}{\cargaHorariaPratica}

\newcommand{\cargaHorariaTeorica}{\def \cargaHorariaTeorica}
\newcommand{\imprimircargaHorariaTeorica}{\cargaHorariaTeorica}

\newcommand{\creditos}{\def \creditos}
\newcommand{\imprimircreditos}{\creditos}

\newcommand{\codigoPrerequisitos}{\def \codigoPrerequisitos}
\newcommand{\imprimircodigoPrerequisitos}{\codigoPrerequisitos}

\newcommand{\semestre}{\def \semestre}
\newcommand{\imprimirsemestre}{\semestre}

\newcommand{\nivel}{\def \nivel}
\newcommand{\imprimirnivel}{\nivel}

\newcommand{\codigoEquivalencias}{\def \codigoEquivalencias}
\newcommand{\imprimircodigoEquivalencias}{\codigoEquivalencias}

\newcommand{\ementa}{\def \ementa}
\newcommand{\imprimirementa}{\ementa}

\newcommand{\objetivo}{\def \objetivo}
\newcommand{\imprimirobjetivo}{\objetivo}

\newcommand{\programa}{\def \programa}
\newcommand{\imprimirprograma}{\programa}

\newcommand{\metodologiaEnsino}{\def \metodologiaEnsino}
\newcommand{\imprimirmetodologiaEnsino}{\metodologiaEnsino}

\newcommand{\recursos}{\def \recursos}
\newcommand{\imprimirrecursos}{\recursos}

\newcommand{\avaliacao}{\def \avaliacao}
\newcommand{\imprimiravaliacao}{\avaliacao}

\newcommand{\bibliografiaBasica}{\def \bibliografiaBasica}
\newcommand{\imprimirbibliografiaBasica}{\bibliografiaBasica}

\newcommand{\bibliografiaComplementar}{\def \bibliografiaComplementar}
\newcommand{\imprimirbibliografiaComplementar}{\bibliografiaComplementar}

\newcommand{\versao}{\def \versao}
\newcommand{\imprimirversao}{\versao}


%comando de impressão da estrutura
\newcommand{\imprimirPUD}{
%Cabeçalho do PUD
\begin{Spacing}{1}

\noindent \begin{minipage}{2.5cm}%
\includegraphics[scale=0.12]{logo-ifce}
\end{minipage}
\hspace{0.3cm}
\begin{minipage}{13cm}%
\centering INSTITUTO FEDERAL DE EDUCAÇÃO, CIÊNCIA E TECNOLOGIA DO CEARÁ- IFCE\\
CAMPUS JUAZEIRO DO NORTE\\
CURSO SUPERIOR EM AUTOMAÇÃO INDUSTRIAL\\
PROGRAMA DE UNIDADE DIDÁTICA – PUD\\
\end{minipage}%
\end{Spacing}

\begin{longtable}{|p{14cm}|}
%primeiro cabeçalho
\hline
\rowcolor{lightgray}
\multicolumn{1}{p{14cm}}{\textbf{Disciplina: \imprimirdisciplina}}\\
\hline
\endfirsthead

%cabeçalho
\hline
continuação PUD \imprimirdisciplina\\
\hline
\endhead

\hline
continua...\\
\hline
\endfoot

\hline
\rowcolor{lightgray}

\begin{tabular}{p{5.5 cm}| l}
coordenação & departamento pedagogico\\[16 ex]
\end{tabular}\\

\hline

\endlastfoot

%elementos
\textbf{Código:} \imprimircodigo\\


\textbf{Carga Horária } Teórica: \imprimircargaHorariaTeorica, Prática \imprimircargaHorariaPratica, Total: \imprimircargaHorariaTotal\\


\textbf{Número de créditos:} \imprimircreditos\\


\textbf{Código pré-requisitos:} \imprimircodigoPrerequisitos\\


\textbf{Semestre:} \imprimirsemestre\\


\textbf{Nível:} \imprimirnivel\\
\hline

\rowcolor{lightgray}
\multicolumn{1}{|p{14cm}|}{\textbf{Ementa}}\\
\hline
\multicolumn{1}{|p{14cm}|}{\imprimirementa}\\
\hline

\rowcolor{lightgray}
\multicolumn{1}{|p{14cm}|}{\textbf{Objetivo}}\\
\hline
\imprimirobjetivo\\
\hline

\rowcolor{lightgray}
\multicolumn{1}{|p{14cm}|}{\textbf{Programa}}\\
\hline
\imprimirprograma\\
\hline

\rowcolor{lightgray}
\multicolumn{1}{|p{14cm}|}{\textbf{Metodologia de ensino}}\\
\hline
\imprimirmetodologiaEnsino\\
\hline

\rowcolor{lightgray}
\multicolumn{1}{|p{14cm}|}{\textbf{Recursos}}\\
\hline
\imprimirrecursos\\
\hline

\rowcolor{lightgray}
\multicolumn{1}{|p{14cm}|}{\textbf{Avaliação}}\\
\hline
\imprimiravaliacao\\
\hline

\rowcolor{lightgray}
\multicolumn{1}{|p{14cm}|}{\textbf{Bibliografia básica}}\\
\hline
\imprimirbibliografiaBasica\\
\hline

\rowcolor{lightgray}
\multicolumn{1}{|p{14cm}|}{\textbf{Bibliografia complementar}}\\
\hline
\imprimirbibliografiaComplementar\\
\hline
\end{longtable}
\pagebreak
}
\begin{document}

\disciplina{Cálculo aplicado}
\codigo{AUT2407}
\cargaHorariaTotal{80}
\cargaHorariaPratica{0}
\cargaHorariaTeorica{80}
\creditos{4}
\codigoPrerequisitos{AUT2405}
\semestre{3º}
\nivel{Superior}

\ementa{
Limites, derivadas e integrais.
}

\objetivo{
• Localizar, acessar e utilizar informações necessárias, usando-as na resolução de problemas.\\
• Elaborar situações problemas que envolvam conceitos de cálculos (limite, derivada e integral) resolvendo-as.\\
• Aplicar os conceitos do cálculo na resolução de problemas, sobretudo a outras áreas do conhecimento.\\
• Utilizar adequadamente as tecnologias da informação na aprendizagem da matemática e do cálculo, observando seus limites e possibilidades.\\
• Utilizar o cálculo para determinar o comportamento de funções.\\
}

\programa{
• Limites
• Definição de limites; Propriedades de limites; Continuidade de funções;\\
• Limites de funções descontínua em a quando x tende a a; Limites de funções\\
compostas; Limites e continuidades laterais; Limites envolvendo infinito;\\
• Limites de funções trigonométricas;\\
• Limites de funções exponenciais e logarítmicas. Derivadas\\
• Definição de derivadas;\\
• Derivada de uma função em um ponto; Taxa de variação;\\
• Coeficiente angular, retas tangentes e retas normais; Aplicações das
derivadas; Regras básicas de derivação; Regra da cadeia;\\
• Teorema do valor intermediário e teorema do valor médio; Derivadas de funções inversas e derivadas implícitas;\\
• Derivadas de funções trigonométricas, logarítmicas e exponenciais; Derivadas de ordem superior; Máximos e mínimos; Integral\\
• Antiderivadas (Primitivas); Conceito de integral; Técnicas de integração; Integração por substituição; Integração por partes; Integral definida; Teorema fundamental do cálculo;\\
• Área sob uma curva\\
}

\metodologiaEnsino{
Aulas expositivas.\\
Resolução de lista de exercícios envolvendo situações-problema.\\
Leitura e pesquisa.\\
}

\recursos{
Livros contidos na bibliografia.\\
Quadro e pincel.\\
Lista de exercícios\\
}

\avaliacao{
Avaliação escrita.\\
Avaliação de exercícios resolvidos.\\
Poderão ser inseridas outras avaliações durante o semestre.\\
}

\bibliografiaBasica{
• LEITHOLD, Louis. O cálculo com geometria analítica, São Paulo: Harbra,1994. v 1.\\
• STEWART, James. Cálculo. São Paulo: Cengage Learning, 2006. v. 1.\\
• THOMAS, George B. Cálculo. 11a ed. São Paulo: Addison Wesley, 2009. v.1.\\
}

\bibliografiaComplementar{
• WEIR, Maurice D.; HASS, Joel; GIORDANO, Frank R. Cálculo. São Paulo: Pearson,
2009.\\
• WEIR, Maurice D.; Hass, Joel; Giordano, Frank R. Cálculo 10, São Paulo: Pearson, 2012. v. 1.\\
• GUIDORIZZI, Hamilton Luz. Um curso de cálculo. Rio de Janeiro. LTC, 2001. v. 1.\\
• HOFFMANN, L. D. Cálculo: um curso moderno e suas aplicações. Rio de Janeiro: LTC, 2002.\\
• MUNEM, Mustafa A. ; FOULIS, David J. Cálculo. Rio de Janeiro: LTC, 1982.\\
• SIMONS, George F. Cálculo com geometria analítica 1. São Paulo: Makron books do
Brasil, 1987.\\
}


\imprimirPUD

\end{document}