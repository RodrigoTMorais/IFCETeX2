\input{preambulo}
%arquivo de template para os PUDS

%definição das variaveis das seções
\newcommand{\disciplina}{\def \disciplina}
\newcommand{\imprimirdisciplina}{\disciplina}

\newcommand{\codigo}{\def \codigo}
\newcommand{\imprimircodigo}{\codigo}

\newcommand{\cargaHorariaTotal}{\def \cargaHorariaTotal}
\newcommand{\imprimircargaHorariaTotal}{\cargaHorariaTotal}

\newcommand{\cargaHorariaPratica}{\def \cargaHorariaPratica}
\newcommand{\imprimircargaHorariaPratica}{\cargaHorariaPratica}

\newcommand{\cargaHorariaTeorica}{\def \cargaHorariaTeorica}
\newcommand{\imprimircargaHorariaTeorica}{\cargaHorariaTeorica}

\newcommand{\creditos}{\def \creditos}
\newcommand{\imprimircreditos}{\creditos}

\newcommand{\codigoPrerequisitos}{\def \codigoPrerequisitos}
\newcommand{\imprimircodigoPrerequisitos}{\codigoPrerequisitos}

\newcommand{\semestre}{\def \semestre}
\newcommand{\imprimirsemestre}{\semestre}

\newcommand{\nivel}{\def \nivel}
\newcommand{\imprimirnivel}{\nivel}

\newcommand{\codigoEquivalencias}{\def \codigoEquivalencias}
\newcommand{\imprimircodigoEquivalencias}{\codigoEquivalencias}

\newcommand{\ementa}{\def \ementa}
\newcommand{\imprimirementa}{\ementa}

\newcommand{\objetivo}{\def \objetivo}
\newcommand{\imprimirobjetivo}{\objetivo}

\newcommand{\programa}{\def \programa}
\newcommand{\imprimirprograma}{\programa}

\newcommand{\metodologiaEnsino}{\def \metodologiaEnsino}
\newcommand{\imprimirmetodologiaEnsino}{\metodologiaEnsino}

\newcommand{\recursos}{\def \recursos}
\newcommand{\imprimirrecursos}{\recursos}

\newcommand{\avaliacao}{\def \avaliacao}
\newcommand{\imprimiravaliacao}{\avaliacao}

\newcommand{\bibliografiaBasica}{\def \bibliografiaBasica}
\newcommand{\imprimirbibliografiaBasica}{\bibliografiaBasica}

\newcommand{\bibliografiaComplementar}{\def \bibliografiaComplementar}
\newcommand{\imprimirbibliografiaComplementar}{\bibliografiaComplementar}

\newcommand{\versao}{\def \versao}
\newcommand{\imprimirversao}{\versao}


%comando de impressão da estrutura
\newcommand{\imprimirPUD}{
%Cabeçalho do PUD
\begin{Spacing}{1}

\noindent \begin{minipage}{2.5cm}%
\includegraphics[scale=0.12]{logo-ifce}
\end{minipage}
\hspace{0.3cm}
\begin{minipage}{13cm}%
\centering INSTITUTO FEDERAL DE EDUCAÇÃO, CIÊNCIA E TECNOLOGIA DO CEARÁ- IFCE\\
CAMPUS JUAZEIRO DO NORTE\\
CURSO SUPERIOR EM AUTOMAÇÃO INDUSTRIAL\\
PROGRAMA DE UNIDADE DIDÁTICA – PUD\\
\end{minipage}%
\end{Spacing}

\begin{longtable}{|p{14cm}|}
%primeiro cabeçalho
\hline
\rowcolor{lightgray}
\multicolumn{1}{p{14cm}}{\textbf{Disciplina: \imprimirdisciplina}}\\
\hline
\endfirsthead

%cabeçalho
\hline
continuação PUD \imprimirdisciplina\\
\hline
\endhead

\hline
continua...\\
\hline
\endfoot

\hline
\rowcolor{lightgray}

\begin{tabular}{p{5.5 cm}| l}
coordenação & departamento pedagogico\\[16 ex]
\end{tabular}\\

\hline

\endlastfoot

%elementos
\textbf{Código:} \imprimircodigo\\


\textbf{Carga Horária } Teórica: \imprimircargaHorariaTeorica, Prática \imprimircargaHorariaPratica, Total: \imprimircargaHorariaTotal\\


\textbf{Número de créditos:} \imprimircreditos\\


\textbf{Código pré-requisitos:} \imprimircodigoPrerequisitos\\


\textbf{Semestre:} \imprimirsemestre\\


\textbf{Nível:} \imprimirnivel\\
\hline

\rowcolor{lightgray}
\multicolumn{1}{|p{14cm}|}{\textbf{Ementa}}\\
\hline
\multicolumn{1}{|p{14cm}|}{\imprimirementa}\\
\hline

\rowcolor{lightgray}
\multicolumn{1}{|p{14cm}|}{\textbf{Objetivo}}\\
\hline
\imprimirobjetivo\\
\hline

\rowcolor{lightgray}
\multicolumn{1}{|p{14cm}|}{\textbf{Programa}}\\
\hline
\imprimirprograma\\
\hline

\rowcolor{lightgray}
\multicolumn{1}{|p{14cm}|}{\textbf{Metodologia de ensino}}\\
\hline
\imprimirmetodologiaEnsino\\
\hline

\rowcolor{lightgray}
\multicolumn{1}{|p{14cm}|}{\textbf{Recursos}}\\
\hline
\imprimirrecursos\\
\hline

\rowcolor{lightgray}
\multicolumn{1}{|p{14cm}|}{\textbf{Avaliação}}\\
\hline
\imprimiravaliacao\\
\hline

\rowcolor{lightgray}
\multicolumn{1}{|p{14cm}|}{\textbf{Bibliografia básica}}\\
\hline
\imprimirbibliografiaBasica\\
\hline

\rowcolor{lightgray}
\multicolumn{1}{|p{14cm}|}{\textbf{Bibliografia complementar}}\\
\hline
\imprimirbibliografiaComplementar\\
\hline
\end{longtable}
\pagebreak
}
\begin{document}

\disciplina{Máquinas elétricas}
\codigo{AUT2421}
\cargaHorariaTotal{80}
\cargaHorariaPratica{20}
\cargaHorariaTeorica{60}
\creditos{4}
\codigoPrerequisitos{AUT2424}
\semestre{5º}
\nivel{Superior}

\ementa{
Introdução aos circuitos magnéticos. Operação, conexões e ensaios de transformadores e máquinas rotativas. Conceitos e princípios de funcionamento de transformadores e máquinas rotativas. Aspectos construtivos de transformadores e máquinas rotativas.
}

\objetivo{
• Analisar circuitos magnéticos aplicados nos diversos tipos de máquinas e transformadores.\\
• Compreender o funcionamento das máquinas elétricas rotativas e transformadores.\\
• Realizar os ensaios aplicados nas máquinas rotativas e transformadores.\\
• Realizar as conexões das máquinas necessárias para o seu funcionamento.\\
}

\programa{
• Circuitos Magnéticos Introdução e conceitos básicos Permeabilidade e saturação Leis dos circuitos magnéticos\\
• Propriedade das materiais magnéticos Operação em C.A. e perdas\\
• Transformador\\
• Circuitos acoplados magneticamente Transformador ideal\\
• Transformador de Potência\\
• Operação do transformador e lei de Faraday Equação de FEM de um transformador Perdas do transformador\\
• Circuitos equivalentes de transformadores reais Ensaios em transformadores\\
• Conexões de transformadores Transformadores trifásicos Auto-transformadores\\
• Máquinas Rotativas\\
• Conceitos básicos\\
• Definições de armadura, campo, rotor e estator Relação entre ornastic elétrica e ornastic mecânica\\
• Tensão gerada e fmm de enrolamentos distribuídos Campos magnéticos girantes\\
• Máquinas de Corrente Contínua\\
• Princípios de operação Ação do comutador\\
• Enrolamento da armadura e características físicas Equação da FEM\\
• Equação do conjugado Equação da velocidade Classificação das máquinas Perdas de rendimento\\
• Características de motores e geradores\\
• 6. Máquinas Síncronas\\
• Tipos e aspectos construtivos\\
• Operação motora e geradora (Equação da FEM)\\
• Características do gerador a vazio, em curto-circuito e regulação de tensão Características potência x ângulo de uma máquina de rotor cilíndrico Desempenho do motor de rotor cilíndrico\\
• Máquinas síncronas de pólos salientes\\
• 7. Motores de Indução Polifásicos\\
• Aspectos gerais\\
• FMM dos enrolamentos da armadura Produção de campos magnéticos girantes\\
• Escorregamento, circuitos equivalentes da máquina Cálculos a partir dos circuitos equivalentes Testes para obtenção dos parâmetros do circuito equivalente aproximado\\
• Motores de Indução Monofásicos\\
• Pequenos motores de C.A.\\
• Análise de motores de indução monofásicos\\
}

\metodologiaEnsino{
Aulas expositivas/dialogadas.\\
Aulas práticas em laboratórios.\\
Elaboração e apresentação de seminários.\\
Debates e intervenções sobre os seminários apresentados.\\
Visitas técnicas.\\
Utilização de lista de exercícios.\\
Simulação computacional utilizando software dedicado.\\
Leitura e pesquisa.\\
}

\recursos{
Livros contidos na bibliografia.\\
Artigos.\\
Datashow.\\
Quadro e pincel.\\
Laboratório de máquinas elétricas\\
Transporte para visitas técnicas\\
Computadores.\\
Lista de exercícios\\
}

\avaliacao{
Avaliação escrita.\\
Práticas individuais e em grupo no laboratório.\\
Relatório de prática.\\
Avaliação de exercícios resolvidos.\\
Poderão ser inseridas outras avaliações durante o semestre.\\
}

\bibliografiaBasica{
• BIM, Edson. Máquinas Elétricas e Acionamento. Rio de Janeiro: Editora Campus, 2009.\\

• CHAPMAN, Stephen J. Fundamentos de máquinas elétricas. Porto Alegre: AMGH, 2013.\\

• UMANS, Stephen D. Máquinas Elétricas de Fitzgerald e Kingsley. 7 ed. Porto Alegre: AMGH, 2017.\\
}

\bibliografiaComplementar{
• BARBI, Ivo. Teoria fundamental do motor de indução. Florianópolis: Editora da UFSC, 1985.\\

• FITZGERALD, A. E; KINGSLEY Jr, Charles. Máquinas elétricas. Porto Alegre: Bookman, 2006.\\

• KOSOW, Irving L. Máquinas elétricas e transformadores. São Paulo: Globo, 2005.\\

• MARTIGNONI, Alfonso. Transformadores. São Paulo: Globo, 1991.\\

• NASAR, Syed A. Máquinas elétricas. São Paulo: Makron Books, 1984.\\

• NASCIMENTO JUNIOR, Geraldo Carvalho do. Máquinas elétricas: teoria e ensaios. São Paulo: • Erica, 2007.\\

• OLIVEIRA, José Carlos de. Transformadores: teoria e ensaios. São Paulo: Edgar • Blücher, 2006.\\

• SIMONE, Gilio Aluisio. Transformadores teoria e exercícios. São Paulo: Erica, 1998.\\
}


\imprimirPUD

\end{document}