\input{preambulo}
%arquivo de template para os PUDS

%definição das variaveis das seções
\newcommand{\disciplina}{\def \disciplina}
\newcommand{\imprimirdisciplina}{\disciplina}

\newcommand{\codigo}{\def \codigo}
\newcommand{\imprimircodigo}{\codigo}

\newcommand{\cargaHorariaTotal}{\def \cargaHorariaTotal}
\newcommand{\imprimircargaHorariaTotal}{\cargaHorariaTotal}

\newcommand{\cargaHorariaPratica}{\def \cargaHorariaPratica}
\newcommand{\imprimircargaHorariaPratica}{\cargaHorariaPratica}

\newcommand{\cargaHorariaTeorica}{\def \cargaHorariaTeorica}
\newcommand{\imprimircargaHorariaTeorica}{\cargaHorariaTeorica}

\newcommand{\creditos}{\def \creditos}
\newcommand{\imprimircreditos}{\creditos}

\newcommand{\codigoPrerequisitos}{\def \codigoPrerequisitos}
\newcommand{\imprimircodigoPrerequisitos}{\codigoPrerequisitos}

\newcommand{\semestre}{\def \semestre}
\newcommand{\imprimirsemestre}{\semestre}

\newcommand{\nivel}{\def \nivel}
\newcommand{\imprimirnivel}{\nivel}

\newcommand{\codigoEquivalencias}{\def \codigoEquivalencias}
\newcommand{\imprimircodigoEquivalencias}{\codigoEquivalencias}

\newcommand{\ementa}{\def \ementa}
\newcommand{\imprimirementa}{\ementa}

\newcommand{\objetivo}{\def \objetivo}
\newcommand{\imprimirobjetivo}{\objetivo}

\newcommand{\programa}{\def \programa}
\newcommand{\imprimirprograma}{\programa}

\newcommand{\metodologiaEnsino}{\def \metodologiaEnsino}
\newcommand{\imprimirmetodologiaEnsino}{\metodologiaEnsino}

\newcommand{\recursos}{\def \recursos}
\newcommand{\imprimirrecursos}{\recursos}

\newcommand{\avaliacao}{\def \avaliacao}
\newcommand{\imprimiravaliacao}{\avaliacao}

\newcommand{\bibliografiaBasica}{\def \bibliografiaBasica}
\newcommand{\imprimirbibliografiaBasica}{\bibliografiaBasica}

\newcommand{\bibliografiaComplementar}{\def \bibliografiaComplementar}
\newcommand{\imprimirbibliografiaComplementar}{\bibliografiaComplementar}

\newcommand{\versao}{\def \versao}
\newcommand{\imprimirversao}{\versao}


%comando de impressão da estrutura
\newcommand{\imprimirPUD}{
%Cabeçalho do PUD
\begin{Spacing}{1}

\noindent \begin{minipage}{2.5cm}%
\includegraphics[scale=0.12]{logo-ifce}
\end{minipage}
\hspace{0.3cm}
\begin{minipage}{13cm}%
\centering INSTITUTO FEDERAL DE EDUCAÇÃO, CIÊNCIA E TECNOLOGIA DO CEARÁ- IFCE\\
CAMPUS JUAZEIRO DO NORTE\\
CURSO SUPERIOR EM AUTOMAÇÃO INDUSTRIAL\\
PROGRAMA DE UNIDADE DIDÁTICA – PUD\\
\end{minipage}%
\end{Spacing}

\begin{longtable}{|p{14cm}|}
%primeiro cabeçalho
\hline
\rowcolor{lightgray}
\multicolumn{1}{p{14cm}}{\textbf{Disciplina: \imprimirdisciplina}}\\
\hline
\endfirsthead

%cabeçalho
\hline
continuação PUD \imprimirdisciplina\\
\hline
\endhead

\hline
continua...\\
\hline
\endfoot

\hline
\rowcolor{lightgray}

\begin{tabular}{p{5.5 cm}| l}
coordenação & departamento pedagogico\\[16 ex]
\end{tabular}\\

\hline

\endlastfoot

%elementos
\textbf{Código:} \imprimircodigo\\


\textbf{Carga Horária } Teórica: \imprimircargaHorariaTeorica, Prática \imprimircargaHorariaPratica, Total: \imprimircargaHorariaTotal\\


\textbf{Número de créditos:} \imprimircreditos\\


\textbf{Código pré-requisitos:} \imprimircodigoPrerequisitos\\


\textbf{Semestre:} \imprimirsemestre\\


\textbf{Nível:} \imprimirnivel\\
\hline

\rowcolor{lightgray}
\multicolumn{1}{|p{14cm}|}{\textbf{Ementa}}\\
\hline
\multicolumn{1}{|p{14cm}|}{\imprimirementa}\\
\hline

\rowcolor{lightgray}
\multicolumn{1}{|p{14cm}|}{\textbf{Objetivo}}\\
\hline
\imprimirobjetivo\\
\hline

\rowcolor{lightgray}
\multicolumn{1}{|p{14cm}|}{\textbf{Programa}}\\
\hline
\imprimirprograma\\
\hline

\rowcolor{lightgray}
\multicolumn{1}{|p{14cm}|}{\textbf{Metodologia de ensino}}\\
\hline
\imprimirmetodologiaEnsino\\
\hline

\rowcolor{lightgray}
\multicolumn{1}{|p{14cm}|}{\textbf{Recursos}}\\
\hline
\imprimirrecursos\\
\hline

\rowcolor{lightgray}
\multicolumn{1}{|p{14cm}|}{\textbf{Avaliação}}\\
\hline
\imprimiravaliacao\\
\hline

\rowcolor{lightgray}
\multicolumn{1}{|p{14cm}|}{\textbf{Bibliografia básica}}\\
\hline
\imprimirbibliografiaBasica\\
\hline

\rowcolor{lightgray}
\multicolumn{1}{|p{14cm}|}{\textbf{Bibliografia complementar}}\\
\hline
\imprimirbibliografiaComplementar\\
\hline
\end{longtable}
\pagebreak
}
\begin{document}

\disciplina{Metrologia}
\codigo{AUT2408}
\cargaHorariaTotal{40}
\cargaHorariaPratica{20}
\cargaHorariaTeorica{20}
\creditos{4}
\codigoPrerequisitos{-}
\semestre{2º}
\nivel{Superior}

\ementa{
Instrumentos de medições, fontes de erro e conversão de sistemas de unidades.
}

\objetivo{
• Conhecer instrumentos de medições mecânicas.\\
• Identificar os fenômenos que interferem na precisão de medidas.\\
• Conhecer sistemas de unidades de medidas mecânicas.\\
}

\programa{
• Classificação dos instrumentos de medição.\\
• Principais instrumentos de medição usados em Metrologia Mecânica.\\
• Principais fontes de erros na medição.\\
• As diversas influências e os possíveis erros causados pelos seguintes fatores: variação com a temperatura, força de medição, forma da peça, forma de contato, erro de paralaxe, estado de conservação do instrumento e habilidade do operador.\\
• Conversão entre os sistemas de medição (Sistema Internacional e Sistema Inglês).\\
• O sistema internacional e suas subdivisões e o sistema inglês com a polegada milésima e a polegada fracionária.\\
• Trânsito entre os dois sistemas através de conversões matemáticas para uso na metrologia dimensional.\\
• Instrumentos de Medição – Leitura\\
• Paquímetro (Teórico e Prática)\\
• Micrômetro (Teórico e Prática)\\
• Relógio Comparador (Teórico e Prática)\\
}

\metodologiaEnsino{
Aulas expositivas.\\
Lista de exercícios envolvendo situações-problema.\\
Leitura e pesquisa.\\
Pratica com instrumentos de medidas.\\
}

\recursos{
Livros contidos na bibliografia.\\
Quadro e pincel.\\
Instrumentos e peças para medidas\\
}

\avaliacao{
Avaliação escrita.\\
Avaliação de exercícios resolvidos.\\
Poderão ser inseridas outras avaliações durante o semestre.\\
}

\bibliografiaBasica{
• LIRA, Francisco Adval de. Metrologia na indústria. São Paulo: Erica, 2006.\\
• ALVES, José Luiz Loureiro. Instrumentação controle e automação de processos. Rio de Janeiro: LTC, 2005.\\
• INSTITUTO NACIONAL DE METROLOGIA, NORMALIZAÇÃO E QUALIDADE INDUSTRIAL. Guia para a Expressão da Incerteza de Medição terceira edição brasileira - Rio de Janeiro: ABNT, INETRO 2003. 120 p.\\
}

\bibliografiaComplementar{
• MONTEIRO, Elisabeth Costa; LESSA Marcelo Lúcio. A Metrologia na Área de Saúde: Garantia da Segurança e da Qualidade dos Equipamentos Eletromédicos. ENGEVISTA, v. 7, n. 2, p. 51-60, dezembro 2005. Artigo.\\
• COSTA MONTEIRO, Elisabeth. Confiabilidade nas Biomedições e suas repercussões éticas. Revista Metrologia e Instrumentação. Pag. 6 a 11, ago/Nov 2007.\\
• ORAGNIZAÇÂO INTERNACIONAL DE METROLOGIA LEGAL. OIMLR 16-1:2002 - Non-invasive mechanical sphygmomanometers. Disponível em <http://www.oiml.org/publications/R/R016-1-e02.pdf>. Acesso em 20/01/2007.\\
• ORGANIZAÇÂO INTERNACIONAL DE METROLOGIA LEGAL. OIMLR 7:1979 - Clinical thermometers (mercury-in-glass, with maximum device). Disponível em: <http://www.oiml.org/publications/R/R007-e79.pdf>. Acesso em 20/01/2007.\\
• INSTITUTO NACIONAL DE METROLOGIA, NORMALIZAÇÃO E QUALIDADE INDUSTRIAL. Em 8/05/97, o Inmetro concluiu a análise em esfigmomanômetros através da verificação do estado de calibração dos mesmos. Disponível em: <http://www.inmetro.gov.br/consumidor/produtos/esfigmo.asp>. Acesso em 13/01/2007.\\
}


\imprimirPUD

\end{document}