\input{preambulo}
%arquivo de template para os PUDS

%definição das variaveis das seções
\newcommand{\disciplina}{\def \disciplina}
\newcommand{\imprimirdisciplina}{\disciplina}

\newcommand{\codigo}{\def \codigo}
\newcommand{\imprimircodigo}{\codigo}

\newcommand{\cargaHorariaTotal}{\def \cargaHorariaTotal}
\newcommand{\imprimircargaHorariaTotal}{\cargaHorariaTotal}

\newcommand{\cargaHorariaPratica}{\def \cargaHorariaPratica}
\newcommand{\imprimircargaHorariaPratica}{\cargaHorariaPratica}

\newcommand{\cargaHorariaTeorica}{\def \cargaHorariaTeorica}
\newcommand{\imprimircargaHorariaTeorica}{\cargaHorariaTeorica}

\newcommand{\creditos}{\def \creditos}
\newcommand{\imprimircreditos}{\creditos}

\newcommand{\codigoPrerequisitos}{\def \codigoPrerequisitos}
\newcommand{\imprimircodigoPrerequisitos}{\codigoPrerequisitos}

\newcommand{\semestre}{\def \semestre}
\newcommand{\imprimirsemestre}{\semestre}

\newcommand{\nivel}{\def \nivel}
\newcommand{\imprimirnivel}{\nivel}

\newcommand{\codigoEquivalencias}{\def \codigoEquivalencias}
\newcommand{\imprimircodigoEquivalencias}{\codigoEquivalencias}

\newcommand{\ementa}{\def \ementa}
\newcommand{\imprimirementa}{\ementa}

\newcommand{\objetivo}{\def \objetivo}
\newcommand{\imprimirobjetivo}{\objetivo}

\newcommand{\programa}{\def \programa}
\newcommand{\imprimirprograma}{\programa}

\newcommand{\metodologiaEnsino}{\def \metodologiaEnsino}
\newcommand{\imprimirmetodologiaEnsino}{\metodologiaEnsino}

\newcommand{\recursos}{\def \recursos}
\newcommand{\imprimirrecursos}{\recursos}

\newcommand{\avaliacao}{\def \avaliacao}
\newcommand{\imprimiravaliacao}{\avaliacao}

\newcommand{\bibliografiaBasica}{\def \bibliografiaBasica}
\newcommand{\imprimirbibliografiaBasica}{\bibliografiaBasica}

\newcommand{\bibliografiaComplementar}{\def \bibliografiaComplementar}
\newcommand{\imprimirbibliografiaComplementar}{\bibliografiaComplementar}

\newcommand{\versao}{\def \versao}
\newcommand{\imprimirversao}{\versao}


%comando de impressão da estrutura
\newcommand{\imprimirPUD}{
%Cabeçalho do PUD
\begin{Spacing}{1}

\noindent \begin{minipage}{2.5cm}%
\includegraphics[scale=0.12]{logo-ifce}
\end{minipage}
\hspace{0.3cm}
\begin{minipage}{13cm}%
\centering INSTITUTO FEDERAL DE EDUCAÇÃO, CIÊNCIA E TECNOLOGIA DO CEARÁ- IFCE\\
CAMPUS JUAZEIRO DO NORTE\\
CURSO SUPERIOR EM AUTOMAÇÃO INDUSTRIAL\\
PROGRAMA DE UNIDADE DIDÁTICA – PUD\\
\end{minipage}%
\end{Spacing}

\begin{longtable}{|p{14cm}|}
%primeiro cabeçalho
\hline
\rowcolor{lightgray}
\multicolumn{1}{p{14cm}}{\textbf{Disciplina: \imprimirdisciplina}}\\
\hline
\endfirsthead

%cabeçalho
\hline
continuação PUD \imprimirdisciplina\\
\hline
\endhead

\hline
continua...\\
\hline
\endfoot

\hline
\rowcolor{lightgray}

\begin{tabular}{p{5.5 cm}| l}
coordenação & departamento pedagogico\\[16 ex]
\end{tabular}\\

\hline

\endlastfoot

%elementos
\textbf{Código:} \imprimircodigo\\


\textbf{Carga Horária } Teórica: \imprimircargaHorariaTeorica, Prática \imprimircargaHorariaPratica, Total: \imprimircargaHorariaTotal\\


\textbf{Número de créditos:} \imprimircreditos\\


\textbf{Código pré-requisitos:} \imprimircodigoPrerequisitos\\


\textbf{Semestre:} \imprimirsemestre\\


\textbf{Nível:} \imprimirnivel\\
\hline

\rowcolor{lightgray}
\multicolumn{1}{|p{14cm}|}{\textbf{Ementa}}\\
\hline
\multicolumn{1}{|p{14cm}|}{\imprimirementa}\\
\hline

\rowcolor{lightgray}
\multicolumn{1}{|p{14cm}|}{\textbf{Objetivo}}\\
\hline
\imprimirobjetivo\\
\hline

\rowcolor{lightgray}
\multicolumn{1}{|p{14cm}|}{\textbf{Programa}}\\
\hline
\imprimirprograma\\
\hline

\rowcolor{lightgray}
\multicolumn{1}{|p{14cm}|}{\textbf{Metodologia de ensino}}\\
\hline
\imprimirmetodologiaEnsino\\
\hline

\rowcolor{lightgray}
\multicolumn{1}{|p{14cm}|}{\textbf{Recursos}}\\
\hline
\imprimirrecursos\\
\hline

\rowcolor{lightgray}
\multicolumn{1}{|p{14cm}|}{\textbf{Avaliação}}\\
\hline
\imprimiravaliacao\\
\hline

\rowcolor{lightgray}
\multicolumn{1}{|p{14cm}|}{\textbf{Bibliografia básica}}\\
\hline
\imprimirbibliografiaBasica\\
\hline

\rowcolor{lightgray}
\multicolumn{1}{|p{14cm}|}{\textbf{Bibliografia complementar}}\\
\hline
\imprimirbibliografiaComplementar\\
\hline
\end{longtable}
\pagebreak
}
\begin{document}

\disciplina{Gestão Empresarial}
\codigo{AUT2440}
\cargaHorariaTotal{40}
\cargaHorariaPratica{20}
\cargaHorariaTeorica{20}
\creditos{2}
\codigoPrerequisitos{-}
\semestre{5º}
\nivel{Superior}

\ementa{
Estimular a atuação profissional em organizações, desenvolvendo habilidades gerenciais, compreendendo a necessidade do contínuo desenvolvimento humano, profissional e da organização.
}

\objetivo{
• Compreender os processos da moderna gestão empresarial.\\
• Discutir a relação entre Direitos Humanos e Gestão Empresarial\\
• Desenvolver as estratégias emergentes de gestão.\\
• Elaborar um projeto empreendedor.\\
}

\programa{
• Introdução à administração – conceitos gerais em administração (Administração, eficiência, eficácia, concorrência, competitividade, economia, capital de giro, organização);\\
• Fundamentos da Administração: o processo administrativo; evolução do pensamento administrativo (principais escolas/teorias);\\
• Níveis da administração e habilidades gerenciais;\\
• As áreas básicas da administração/da organização: marketing, produção/operações, finanças, gestão de pessoas, tecnologia de informação – seu papel na estrutura administrativa/organizacional e instrumentos/técnicas aplicadas a área de indústria;\\
• Estratégias emergentes de gestão. O processo empreendedor.\\
• Identificando oportunidades. O plano de negócios.\\
• Questões legais de constituição da empresa.\\
• Liderança.\\
}

\metodologiaEnsino{
Aulas expositivas;\\
Lista de exercícios;\\
Simulação computacional utilizando software dedicado.\\
}

\recursos{
Livros contidos na bibliografia;\\
Quadro e pincel.\\
Data-show\\
}

\avaliacao{
Avaliação de aprendizagem escrita;\\
Leitura, Estudo e Debates em Sala de Aula; Listas de exercícios;\\
Poderão ser inseridas outras avaliações durante o semestre. Seminários e/ou Mesas Redondas;\\
Exposição oral dialogada.\\
}

\bibliografiaBasica{
• CHIAVENATO, I. Empreendedorismo: dando asas ao espírito empreendedor. São Paulo: Saraiva, 2008.\\

• DOLABELA, F. O segredo de luísa. Rio de Janeiro: Sextante, 2008.\\

• GAUTHIER. F. A. O.; MACEDO, M.; LABIAK Jr., S. Empreendedorismo. Curitiba: Editora do Livro Técnico, 2010.\\
}

\bibliografiaComplementar{
• DEGEN, R. J.; MELLO, A. A. A. O empreendedor: fundamentos da iniciativa empresarial. São Paulo: Makron Books, 2005.\\

• DRUCKER. Inovação e espírito empreendedor: prática e princípios (entrepreneurship): prática e princípios. São Paulo: Pioneira Thomson, 2003.\\

• JALOWITZKI, M. Jogos e técnicas vivenciais nas empresas: guia prático de dinâmicas de grupo. 3 ed. São Paulo: Madras, 2007.\\

• MAXIMINIANO, A. Teoria geral da administração: da revolução urbana à revolução digital. 6 ed. São Paulo: Atlas, 2008.\\

• MONTIBELLER F., G. Empresas, desenvolvimento e ambiente: Diagnósticos e diretrizes de sustentabilidade. São Paulo: Manoel, 2007.\\
}


\imprimirPUD

\end{document}