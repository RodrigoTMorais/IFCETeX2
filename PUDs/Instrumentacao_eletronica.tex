\input{preambulo}
%arquivo de template para os PUDS

%definição das variaveis das seções
\newcommand{\disciplina}{\def \disciplina}
\newcommand{\imprimirdisciplina}{\disciplina}

\newcommand{\codigo}{\def \codigo}
\newcommand{\imprimircodigo}{\codigo}

\newcommand{\cargaHorariaTotal}{\def \cargaHorariaTotal}
\newcommand{\imprimircargaHorariaTotal}{\cargaHorariaTotal}

\newcommand{\cargaHorariaPratica}{\def \cargaHorariaPratica}
\newcommand{\imprimircargaHorariaPratica}{\cargaHorariaPratica}

\newcommand{\cargaHorariaTeorica}{\def \cargaHorariaTeorica}
\newcommand{\imprimircargaHorariaTeorica}{\cargaHorariaTeorica}

\newcommand{\creditos}{\def \creditos}
\newcommand{\imprimircreditos}{\creditos}

\newcommand{\codigoPrerequisitos}{\def \codigoPrerequisitos}
\newcommand{\imprimircodigoPrerequisitos}{\codigoPrerequisitos}

\newcommand{\semestre}{\def \semestre}
\newcommand{\imprimirsemestre}{\semestre}

\newcommand{\nivel}{\def \nivel}
\newcommand{\imprimirnivel}{\nivel}

\newcommand{\codigoEquivalencias}{\def \codigoEquivalencias}
\newcommand{\imprimircodigoEquivalencias}{\codigoEquivalencias}

\newcommand{\ementa}{\def \ementa}
\newcommand{\imprimirementa}{\ementa}

\newcommand{\objetivo}{\def \objetivo}
\newcommand{\imprimirobjetivo}{\objetivo}

\newcommand{\programa}{\def \programa}
\newcommand{\imprimirprograma}{\programa}

\newcommand{\metodologiaEnsino}{\def \metodologiaEnsino}
\newcommand{\imprimirmetodologiaEnsino}{\metodologiaEnsino}

\newcommand{\recursos}{\def \recursos}
\newcommand{\imprimirrecursos}{\recursos}

\newcommand{\avaliacao}{\def \avaliacao}
\newcommand{\imprimiravaliacao}{\avaliacao}

\newcommand{\bibliografiaBasica}{\def \bibliografiaBasica}
\newcommand{\imprimirbibliografiaBasica}{\bibliografiaBasica}

\newcommand{\bibliografiaComplementar}{\def \bibliografiaComplementar}
\newcommand{\imprimirbibliografiaComplementar}{\bibliografiaComplementar}

\newcommand{\versao}{\def \versao}
\newcommand{\imprimirversao}{\versao}


%comando de impressão da estrutura
\newcommand{\imprimirPUD}{
%Cabeçalho do PUD
\begin{Spacing}{1}

\noindent \begin{minipage}{2.5cm}%
\includegraphics[scale=0.12]{logo-ifce}
\end{minipage}
\hspace{0.3cm}
\begin{minipage}{13cm}%
\centering INSTITUTO FEDERAL DE EDUCAÇÃO, CIÊNCIA E TECNOLOGIA DO CEARÁ- IFCE\\
CAMPUS JUAZEIRO DO NORTE\\
CURSO SUPERIOR EM AUTOMAÇÃO INDUSTRIAL\\
PROGRAMA DE UNIDADE DIDÁTICA – PUD\\
\end{minipage}%
\end{Spacing}

\begin{longtable}{|p{14cm}|}
%primeiro cabeçalho
\hline
\rowcolor{lightgray}
\multicolumn{1}{p{14cm}}{\textbf{Disciplina: \imprimirdisciplina}}\\
\hline
\endfirsthead

%cabeçalho
\hline
continuação PUD \imprimirdisciplina\\
\hline
\endhead

\hline
continua...\\
\hline
\endfoot

\hline
\rowcolor{lightgray}

\begin{tabular}{p{5.5 cm}| l}
coordenação & departamento pedagogico\\[16 ex]
\end{tabular}\\

\hline

\endlastfoot

%elementos
\textbf{Código:} \imprimircodigo\\


\textbf{Carga Horária } Teórica: \imprimircargaHorariaTeorica, Prática \imprimircargaHorariaPratica, Total: \imprimircargaHorariaTotal\\


\textbf{Número de créditos:} \imprimircreditos\\


\textbf{Código pré-requisitos:} \imprimircodigoPrerequisitos\\


\textbf{Semestre:} \imprimirsemestre\\


\textbf{Nível:} \imprimirnivel\\
\hline

\rowcolor{lightgray}
\multicolumn{1}{|p{14cm}|}{\textbf{Ementa}}\\
\hline
\multicolumn{1}{|p{14cm}|}{\imprimirementa}\\
\hline

\rowcolor{lightgray}
\multicolumn{1}{|p{14cm}|}{\textbf{Objetivo}}\\
\hline
\imprimirobjetivo\\
\hline

\rowcolor{lightgray}
\multicolumn{1}{|p{14cm}|}{\textbf{Programa}}\\
\hline
\imprimirprograma\\
\hline

\rowcolor{lightgray}
\multicolumn{1}{|p{14cm}|}{\textbf{Metodologia de ensino}}\\
\hline
\imprimirmetodologiaEnsino\\
\hline

\rowcolor{lightgray}
\multicolumn{1}{|p{14cm}|}{\textbf{Recursos}}\\
\hline
\imprimirrecursos\\
\hline

\rowcolor{lightgray}
\multicolumn{1}{|p{14cm}|}{\textbf{Avaliação}}\\
\hline
\imprimiravaliacao\\
\hline

\rowcolor{lightgray}
\multicolumn{1}{|p{14cm}|}{\textbf{Bibliografia básica}}\\
\hline
\imprimirbibliografiaBasica\\
\hline

\rowcolor{lightgray}
\multicolumn{1}{|p{14cm}|}{\textbf{Bibliografia complementar}}\\
\hline
\imprimirbibliografiaComplementar\\
\hline
\end{longtable}
\pagebreak
}
\begin{document}

\disciplina{Instrumentação Eletrônica}
\codigo{AUT2412}
\cargaHorariaTotal{40}
\cargaHorariaPratica{15}
\cargaHorariaTeorica{25}
\creditos{2}
\codigoPrerequisitos{AUT2401}
\semestre{2º}
\nivel{Superior}

\ementa{
 Sistema internacional de unidades. Principais instrumentos elétricos
de medição. Métodos aplicados na medição das grandezas elétricas. Especificar
Instrumentos para medição das grandezas elétricas. Métodos e/ou instrumentos
empregados na medição das grandezas elétricas.
}

\objetivo{
• Utilizar os instrumentos na medição das principais grandezas elétricas.\\
• Executar ensaios de medição de grandezas elétricas analisando os resultados obtidos.\\
• Descrever os principais instrumentos empregados na medição das grandezas elétricas.\\
}

\programa{
• ODOLOGIASistema Internacional de Unidades – SI Unidades base e unidades\\ derivadas Múltiplos e submúltiplos do SI Revisão da Teoria dos Erros\\
• Definição e classificação dos erros Calculo do erro\\
• Exatidão e precisão\\
• Generalidades dos Instrumentos Elétricos de Medição Do processo de construção\\
• Dados característicos dos instrumentos elétricos de medição\\
• Símbolos encontrados nos instrumentos elétricos de medição Galvanômetro de bobina móvel\\
• Construção e Funcionamento\\
• Ação dos conjugados motor, antagonista e de amortecimento Estudo da sensibilidade do galvanômetro\\
• Amperímetro DC Construção e funcionamento Medições de corrente DC Voltímetro DC\\
• Construção e funcionamento Medições de tensão DC Voltímetro CA\\
• Retificador de meia onda e de onda completa Construção da escala do voltímetro\\
• Medições de tensão CA\\
• Ohmímetro a pilha Circuito do ohmímetro Construção da escala Ajuste de zero\\
• Medição de resistência com o ohmímetro Ponte de Wheatstone/ Ponte de Kelvin\\ Circuito da ponte de Wheatstone Medição de resistência de valor médio Circuito da ponte de Kelvin\\
• Medição de resistência de valor baixo Estudo do multímetro analógico\\
• Especificação dos multímetros Multímetro como amperímetro, como voltímetro cc/ca e como ohmímetro\\
• Teste de continuidade e teste de semicondutores com o multímetro Megaohmímetro\\
• Circuito do megaohmímetro\\
• Medição de resistência de valor elevado (resistência de isolamento). Osciloscópio de Raios catódicos\\
• Construção e funcionamento\\
• Medições de tensão e corrente com o osciloscópio\\
• Geração de figuras de Lissajous\\
}

\metodologiaEnsino{
Aulas expositivas.\\
Aulas práticas em laboratório.\\
Resolução de lista de exercícios.\\
Realização de visitas técnicas.\\
Leitura e pesquisa.\\
}

\recursos{
Manuais Técnicos.\\
Quadro e pincel.\\
Laboratório de eletrônica.\\
Data-show.\\
Visitas técnicas.\\
Lista de exercícios.\
}

\avaliacao{
Avaliação escrita.\\
Práticas individuais e em grupo no laboratório.\\
Relatório de prática.\\
Avaliação de exercícios resolvidos.\\
}

\bibliografiaBasica{
• CAPUANO, Francisco Gabriel. Laboratório de eletricidade e eletrônica. São Paulo: Érica,2010.\\
• URBANETZ JUNIOR, Jair. Eletrônica aplicada. Curitiba: Base Editorial, 2010.\\
• VISACRO FILHO, Silvério. Aterramentos elétricos: conceitos básicos, técnicas de medição e instrumentação, filosofias de aterramento. São Paulo: Artliber, 2002.\\
}

\bibliografiaComplementar{
• BALBINOT, Alexandre; BRUSAMARELLO, Valner J. Instrumentação e fundamentos de
medidas. Rio de Janeiro: LTC, 2006. v 1.\\
• FILHO, Solon de Medeiros. Fundamentos de medidas elétricas. Rio de Janeiro: LTC, 1981.\\ 
• TURNER, L. W. et al. Eletrônica aplicada. Curitiba: Hemus, 2004.\\
• SIGHIERI, Luciano. Controle automatico de processos industriais: instrumentação. São Paulo: E. Blucher, 1966, 240 p.\\
• BEGA, Egídio Alberto (Org.). Instrumentação industrial. 2. ed. Rio de Janeiro: Interciência, 2006. 583 p. ISBN 9788571931374.\\
• FIALHO, Arivelto Bustamante. Instrumentação industrial: conceitos, aplicações e análises . 7. ed. Juiz de Fora: Érica, 2011. 280 p. ISBN 9788571949225.\\
}


\imprimirPUD

\end{document}