\input{preambulo}
%arquivo de template para os PUDS

%definição das variaveis das seções
\newcommand{\disciplina}{\def \disciplina}
\newcommand{\imprimirdisciplina}{\disciplina}

\newcommand{\codigo}{\def \codigo}
\newcommand{\imprimircodigo}{\codigo}

\newcommand{\cargaHorariaTotal}{\def \cargaHorariaTotal}
\newcommand{\imprimircargaHorariaTotal}{\cargaHorariaTotal}

\newcommand{\cargaHorariaPratica}{\def \cargaHorariaPratica}
\newcommand{\imprimircargaHorariaPratica}{\cargaHorariaPratica}

\newcommand{\cargaHorariaTeorica}{\def \cargaHorariaTeorica}
\newcommand{\imprimircargaHorariaTeorica}{\cargaHorariaTeorica}

\newcommand{\creditos}{\def \creditos}
\newcommand{\imprimircreditos}{\creditos}

\newcommand{\codigoPrerequisitos}{\def \codigoPrerequisitos}
\newcommand{\imprimircodigoPrerequisitos}{\codigoPrerequisitos}

\newcommand{\semestre}{\def \semestre}
\newcommand{\imprimirsemestre}{\semestre}

\newcommand{\nivel}{\def \nivel}
\newcommand{\imprimirnivel}{\nivel}

\newcommand{\codigoEquivalencias}{\def \codigoEquivalencias}
\newcommand{\imprimircodigoEquivalencias}{\codigoEquivalencias}

\newcommand{\ementa}{\def \ementa}
\newcommand{\imprimirementa}{\ementa}

\newcommand{\objetivo}{\def \objetivo}
\newcommand{\imprimirobjetivo}{\objetivo}

\newcommand{\programa}{\def \programa}
\newcommand{\imprimirprograma}{\programa}

\newcommand{\metodologiaEnsino}{\def \metodologiaEnsino}
\newcommand{\imprimirmetodologiaEnsino}{\metodologiaEnsino}

\newcommand{\recursos}{\def \recursos}
\newcommand{\imprimirrecursos}{\recursos}

\newcommand{\avaliacao}{\def \avaliacao}
\newcommand{\imprimiravaliacao}{\avaliacao}

\newcommand{\bibliografiaBasica}{\def \bibliografiaBasica}
\newcommand{\imprimirbibliografiaBasica}{\bibliografiaBasica}

\newcommand{\bibliografiaComplementar}{\def \bibliografiaComplementar}
\newcommand{\imprimirbibliografiaComplementar}{\bibliografiaComplementar}

\newcommand{\versao}{\def \versao}
\newcommand{\imprimirversao}{\versao}


%comando de impressão da estrutura
\newcommand{\imprimirPUD}{
%Cabeçalho do PUD
\begin{Spacing}{1}

\noindent \begin{minipage}{2.5cm}%
\includegraphics[scale=0.12]{logo-ifce}
\end{minipage}
\hspace{0.3cm}
\begin{minipage}{13cm}%
\centering INSTITUTO FEDERAL DE EDUCAÇÃO, CIÊNCIA E TECNOLOGIA DO CEARÁ- IFCE\\
CAMPUS JUAZEIRO DO NORTE\\
CURSO SUPERIOR EM AUTOMAÇÃO INDUSTRIAL\\
PROGRAMA DE UNIDADE DIDÁTICA – PUD\\
\end{minipage}%
\end{Spacing}

\begin{longtable}{|p{14cm}|}
%primeiro cabeçalho
\hline
\rowcolor{lightgray}
\multicolumn{1}{p{14cm}}{\textbf{Disciplina: \imprimirdisciplina}}\\
\hline
\endfirsthead

%cabeçalho
\hline
continuação PUD \imprimirdisciplina\\
\hline
\endhead

\hline
continua...\\
\hline
\endfoot

\hline
\rowcolor{lightgray}

\begin{tabular}{p{5.5 cm}| l}
coordenação & departamento pedagogico\\[16 ex]
\end{tabular}\\

\hline

\endlastfoot

%elementos
\textbf{Código:} \imprimircodigo\\


\textbf{Carga Horária } Teórica: \imprimircargaHorariaTeorica, Prática \imprimircargaHorariaPratica, Total: \imprimircargaHorariaTotal\\


\textbf{Número de créditos:} \imprimircreditos\\


\textbf{Código pré-requisitos:} \imprimircodigoPrerequisitos\\


\textbf{Semestre:} \imprimirsemestre\\


\textbf{Nível:} \imprimirnivel\\
\hline

\rowcolor{lightgray}
\multicolumn{1}{|p{14cm}|}{\textbf{Ementa}}\\
\hline
\multicolumn{1}{|p{14cm}|}{\imprimirementa}\\
\hline

\rowcolor{lightgray}
\multicolumn{1}{|p{14cm}|}{\textbf{Objetivo}}\\
\hline
\imprimirobjetivo\\
\hline

\rowcolor{lightgray}
\multicolumn{1}{|p{14cm}|}{\textbf{Programa}}\\
\hline
\imprimirprograma\\
\hline

\rowcolor{lightgray}
\multicolumn{1}{|p{14cm}|}{\textbf{Metodologia de ensino}}\\
\hline
\imprimirmetodologiaEnsino\\
\hline

\rowcolor{lightgray}
\multicolumn{1}{|p{14cm}|}{\textbf{Recursos}}\\
\hline
\imprimirrecursos\\
\hline

\rowcolor{lightgray}
\multicolumn{1}{|p{14cm}|}{\textbf{Avaliação}}\\
\hline
\imprimiravaliacao\\
\hline

\rowcolor{lightgray}
\multicolumn{1}{|p{14cm}|}{\textbf{Bibliografia básica}}\\
\hline
\imprimirbibliografiaBasica\\
\hline

\rowcolor{lightgray}
\multicolumn{1}{|p{14cm}|}{\textbf{Bibliografia complementar}}\\
\hline
\imprimirbibliografiaComplementar\\
\hline
\end{longtable}
\pagebreak
}
\begin{document}

\disciplina{Microprocessadores 1}
\codigo{AUT2418}
\cargaHorariaTotal{80}
\cargaHorariaPratica{40}
\cargaHorariaTeorica{40}
\creditos{4}
\codigoPrerequisitos{AUT2409, AUT2410}
\semestre{4º}
\nivel{Superior}

\ementa{
Arquitetura de microprocessadores e microcontroladores. Conjunto de instruções de um microcontrolador. Noções de linguagem assembly. Programação de microcontroladores. Entradas e saídas digitais. Conversor AD. Interrupções. Memórias não voláteis. USART. Projeto de sistemas microcontrolados.
}

\objetivo{
• Compreender o funcionamento de microprocessadores e microcontroladores, bem como seus principais módulos internos.
}

\programa{
• Arquitetura e Organização de Computadores:\\
• Breve histórico da evolução dos computadores; Elementos de um computador;\\
• Unidade central de processamento; Memórias;\\
• Arquiteturas de Processadores.\\
• Introdução aos Microcontroladores:\\
• Microcontrolador versus microprocessador;\\
• Estrutura interna de um microcontrolador;\\
• Conjunto de Instruções de um microcontrolador; \\
• Programação em Linguagem assembly; Programação em Linguagem C;\\
• Entradas e saídas digitais:\\
• Acionamento de Leds; Leitura de Botões;\\
• Displays de Segmentos e matriz de led; Displays LCD;\\
• Conversor AD\\
• Sistema de Interrupções Memórias não voláteis USART\\
• Projeto de Sistemas Microcontrolados: Unidade métrica e imperial;\\
• Encapsulametos;\\
• Pads, vias, e trilhas; Projeto de PCI para microcontroladores.\\
}

\metodologiaEnsino{
Aulas expositivas.\\
Aulas práticas em laboratório.\\
Resolução de exercícios e projetos.\\
Leitura e pesquisa bibliográfica.\\
}

\recursos{
Livros contidos na bibliografia.\\
Computador.\\
Projetor.\\
Softwares de simulação de microcontroladores.\\
Softwares de programação de microcontroladores.\\
Componentes Eletrônicos diversos.\\
}

\avaliacao{
Avaliação Teórica.\\
Avaliação Prática.\\
Trabalhos realizados.\\
Projetos elaborados.\\
}

\bibliografiaBasica{
• PEREIRA, Fábio. Microcontroladores PIC: técnicas avançadas. São Paulo: Érica, 2007.\\

• SOUZA, David José de. Desbravando o PIC: ampliado e atualizado para PIC16F628A. São Paulo: Érica, 2007.\\

• ZANCO, Wagner da Silva. Microcontroladores PIC 16F628A/648a: uma abordagem prática e objetiva. São Paulo:Erica, 2005.\\
}

\bibliografiaComplementar{
• GIMENEZ, Salvador P. Microcontroladores 8051. São Paulo: Pearson, 2002.\\

• NICOLOSI, Denys Emílio Campion. Laboratório de microcontroladores família 8051:
treino de instruções, hardware e software. 3 ed. São Paulo: Érica, 2004.\\

• PEREIRA, Fábio. Microcontroladores PIC 18 detalhado: hardware e software. São Paulo: Érica, 2010.\\

• PEREIRA, Fábio. Microcontroladores PIC: programação em C. São Paulo: Érica, 2007.\\

• TOCCI, Ronaldo J.; LASKOWSKI, Lester P. Microprocessadores e Microcomputadores:
hardware e software. Rio de Janeiro: Prentice Hall, 1990.\\
}


\imprimirPUD

\end{document}