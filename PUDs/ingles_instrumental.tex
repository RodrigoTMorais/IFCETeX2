\input{preambulo}
%arquivo de template para os PUDS

%definição das variaveis das seções
\newcommand{\disciplina}{\def \disciplina}
\newcommand{\imprimirdisciplina}{\disciplina}

\newcommand{\codigo}{\def \codigo}
\newcommand{\imprimircodigo}{\codigo}

\newcommand{\cargaHorariaTotal}{\def \cargaHorariaTotal}
\newcommand{\imprimircargaHorariaTotal}{\cargaHorariaTotal}

\newcommand{\cargaHorariaPratica}{\def \cargaHorariaPratica}
\newcommand{\imprimircargaHorariaPratica}{\cargaHorariaPratica}

\newcommand{\cargaHorariaTeorica}{\def \cargaHorariaTeorica}
\newcommand{\imprimircargaHorariaTeorica}{\cargaHorariaTeorica}

\newcommand{\creditos}{\def \creditos}
\newcommand{\imprimircreditos}{\creditos}

\newcommand{\codigoPrerequisitos}{\def \codigoPrerequisitos}
\newcommand{\imprimircodigoPrerequisitos}{\codigoPrerequisitos}

\newcommand{\semestre}{\def \semestre}
\newcommand{\imprimirsemestre}{\semestre}

\newcommand{\nivel}{\def \nivel}
\newcommand{\imprimirnivel}{\nivel}

\newcommand{\codigoEquivalencias}{\def \codigoEquivalencias}
\newcommand{\imprimircodigoEquivalencias}{\codigoEquivalencias}

\newcommand{\ementa}{\def \ementa}
\newcommand{\imprimirementa}{\ementa}

\newcommand{\objetivo}{\def \objetivo}
\newcommand{\imprimirobjetivo}{\objetivo}

\newcommand{\programa}{\def \programa}
\newcommand{\imprimirprograma}{\programa}

\newcommand{\metodologiaEnsino}{\def \metodologiaEnsino}
\newcommand{\imprimirmetodologiaEnsino}{\metodologiaEnsino}

\newcommand{\recursos}{\def \recursos}
\newcommand{\imprimirrecursos}{\recursos}

\newcommand{\avaliacao}{\def \avaliacao}
\newcommand{\imprimiravaliacao}{\avaliacao}

\newcommand{\bibliografiaBasica}{\def \bibliografiaBasica}
\newcommand{\imprimirbibliografiaBasica}{\bibliografiaBasica}

\newcommand{\bibliografiaComplementar}{\def \bibliografiaComplementar}
\newcommand{\imprimirbibliografiaComplementar}{\bibliografiaComplementar}

\newcommand{\versao}{\def \versao}
\newcommand{\imprimirversao}{\versao}


%comando de impressão da estrutura
\newcommand{\imprimirPUD}{
%Cabeçalho do PUD
\begin{Spacing}{1}

\noindent \begin{minipage}{2.5cm}%
\includegraphics[scale=0.12]{logo-ifce}
\end{minipage}
\hspace{0.3cm}
\begin{minipage}{13cm}%
\centering INSTITUTO FEDERAL DE EDUCAÇÃO, CIÊNCIA E TECNOLOGIA DO CEARÁ- IFCE\\
CAMPUS JUAZEIRO DO NORTE\\
CURSO SUPERIOR EM AUTOMAÇÃO INDUSTRIAL\\
PROGRAMA DE UNIDADE DIDÁTICA – PUD\\
\end{minipage}%
\end{Spacing}

\begin{longtable}{|p{14cm}|}
%primeiro cabeçalho
\hline
\rowcolor{lightgray}
\multicolumn{1}{p{14cm}}{\textbf{Disciplina: \imprimirdisciplina}}\\
\hline
\endfirsthead

%cabeçalho
\hline
continuação PUD \imprimirdisciplina\\
\hline
\endhead

\hline
continua...\\
\hline
\endfoot

\hline
\rowcolor{lightgray}

\begin{tabular}{p{5.5 cm}| l}
coordenação & departamento pedagogico\\[16 ex]
\end{tabular}\\

\hline

\endlastfoot

%elementos
\textbf{Código:} \imprimircodigo\\


\textbf{Carga Horária } Teórica: \imprimircargaHorariaTeorica, Prática \imprimircargaHorariaPratica, Total: \imprimircargaHorariaTotal\\


\textbf{Número de créditos:} \imprimircreditos\\


\textbf{Código pré-requisitos:} \imprimircodigoPrerequisitos\\


\textbf{Semestre:} \imprimirsemestre\\


\textbf{Nível:} \imprimirnivel\\
\hline

\rowcolor{lightgray}
\multicolumn{1}{|p{14cm}|}{\textbf{Ementa}}\\
\hline
\multicolumn{1}{|p{14cm}|}{\imprimirementa}\\
\hline

\rowcolor{lightgray}
\multicolumn{1}{|p{14cm}|}{\textbf{Objetivo}}\\
\hline
\imprimirobjetivo\\
\hline

\rowcolor{lightgray}
\multicolumn{1}{|p{14cm}|}{\textbf{Programa}}\\
\hline
\imprimirprograma\\
\hline

\rowcolor{lightgray}
\multicolumn{1}{|p{14cm}|}{\textbf{Metodologia de ensino}}\\
\hline
\imprimirmetodologiaEnsino\\
\hline

\rowcolor{lightgray}
\multicolumn{1}{|p{14cm}|}{\textbf{Recursos}}\\
\hline
\imprimirrecursos\\
\hline

\rowcolor{lightgray}
\multicolumn{1}{|p{14cm}|}{\textbf{Avaliação}}\\
\hline
\imprimiravaliacao\\
\hline

\rowcolor{lightgray}
\multicolumn{1}{|p{14cm}|}{\textbf{Bibliografia básica}}\\
\hline
\imprimirbibliografiaBasica\\
\hline

\rowcolor{lightgray}
\multicolumn{1}{|p{14cm}|}{\textbf{Bibliografia complementar}}\\
\hline
\imprimirbibliografiaComplementar\\
\hline
\end{longtable}
\pagebreak
}
\begin{document}

\disciplina{Inglês instrumental}
\codigo{AUT2445}
\cargaHorariaTotal{40}
\cargaHorariaPratica{0}
\cargaHorariaTeorica{40}
\creditos{2}
\codigoPrerequisitos{-}
\semestre{opcional}
\nivel{Superior}

\ementa{
Discussão acerca da relevância do estudo da língua inglesa no contexto de Automação Industrial, justificando a leitura de textos nas diversas áreas que compõem este campo de estudos. Estudo das estruturas básicas da gramática da Língua Inglesa bem como do vocabulário pertinente à Automação Industrial, visando à compreensão de textos de diversos gêneros como artigo científico, manual de instrução, etc., e em diversos níveis de compreensão, de modo a atender às necessidades 149orna149stica149 dos aprendizes durante e
depois de seus estudos formais.
}

\objetivo{
• Elaborar, através de pistas textuais, a idéia principal do texto e as secundárias; Utilizar, de forma autônoma e eficiente, o dicionário;\\
• Traduzir, sem maiores esforços cognitivos, os sintagmas nominais e os verbais;\\ Escolher e usar a estratégia de leitura adequada aos diferentes gêneros textuais;\\
• Usar o conhecimento enciclopédico, junto com outros tipos de conhecimento, para construir o significado dos textos;\\
• Familiarizar-se com a estrutura dos variados gêneros textuais tais como o texto acadêmico, o manual de instrução etc.;\\
• Identificar e operacionalizar os elementos de coesão e coerência do texto;\\
• Identificar e operacionalizar os cognatos e o vocabulário técnico pertinente a cada gênero textual relevante para a Automação Industrial.\\
}

\programa{
• Considerações gerais sobre o processo de leitura:\\
• Conceituação e contextualização da Língua Inglesa no universo da Automação Industrial; Razões para se ler em Língua Inglesa na Automação Industrial;\\
• Leitura intensiva e leitura extensiva; Níveis de compreensão leitora.\\
• Introdução às estratégias de leitura: Lay-out do texto;\\
• Skimming-scanning;\\
• Convenções gráficas;\\
• Palavras-chave; Palavras repetidas; Cognatos; Predição; Seletividade;\\
• Aspectos morfo-lexico-semânticos da Língua Inglesa: Formação de palavras\\
• Prefixação;\\
• Sufixação;\\
• Composição;\\
• Vocabulário técnico de Automação Industrial. Coesão Textual- Palavras de ligação e de referência: Conjunções;\\
• Advérbios; Sequenciadores; Pronomes;\\
• Marcadores de discurso.\\
• Grupo Nominal:\\
• Substantivos;\\
• Adjetivos;\\
• Quantificadores;\\
• Artigos;\\
• Particípios. Grupo Verbal:\\
• Voz ativa e passiva; Verbos no presente; Verbos no passado;\\
• Formas futuras do verbo em inglês; Tempos compostos.\\

}

\metodologiaEnsino{
Aulas expositivas;\\
Seminários de textos pertinentes à Automação Industrial; \\
Exercícios e trabalhos em grupo.\\
}

\recursos{
Computador;\\
Quadro branco e pincel;\\
Data show.\\
}

\avaliacao{
Avaliação dos pontos gramaticais e do vocabulário relacionado à Automação Industrial; \\
Avaliação das apresentações de seminários de texto de Automação Industrial;\\
Trabalhos em grupo e individuais\\
}

\bibliografiaBasica{
• MURPHY, Raymond. English grammar in use: a self-study reference and practice book for intermediate students. Nova York: Cambridge University Press, 1997.\\

• MUNHOZ, R. Inglês instrumental: estratégias de leitura: módulo I, 2000.\\

• MUNHOZ, R. Inglês instrumental: estratégias de leitura: módulo II, 2004.\\
}

\bibliografiaComplementar{
• MURPHY, Raymond. Essential grammar in use: gramática básica da língua inglesa. São Paulo: Cambridge University Press, Martins Fontes, 2004.\\

• LONGMAN: gramática escolar da língua inglesa. São Paulo: Pearson, 2004.\\

• AUN, Eliana; MORAES, M. D. de.; SANSANOVICZ, N. B. Get to the point 1. São Paulo: Saraiva, 1995.\\

• LOPES, Carolina. Inglês Instrumental: leitura e compreensão de textos. Recife: Imprima, 2012.\
 
• MASCHERPE, Mário e ZAMARIN, Laura. Os Falsos Cognatos . 4 ed. São Paulo: Difel , 1984. \\
}


\imprimirPUD

\end{document}