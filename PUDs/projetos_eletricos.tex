\input{preambulo}
%arquivo de template para os PUDS

%definição das variaveis das seções
\newcommand{\disciplina}{\def \disciplina}
\newcommand{\imprimirdisciplina}{\disciplina}

\newcommand{\codigo}{\def \codigo}
\newcommand{\imprimircodigo}{\codigo}

\newcommand{\cargaHorariaTotal}{\def \cargaHorariaTotal}
\newcommand{\imprimircargaHorariaTotal}{\cargaHorariaTotal}

\newcommand{\cargaHorariaPratica}{\def \cargaHorariaPratica}
\newcommand{\imprimircargaHorariaPratica}{\cargaHorariaPratica}

\newcommand{\cargaHorariaTeorica}{\def \cargaHorariaTeorica}
\newcommand{\imprimircargaHorariaTeorica}{\cargaHorariaTeorica}

\newcommand{\creditos}{\def \creditos}
\newcommand{\imprimircreditos}{\creditos}

\newcommand{\codigoPrerequisitos}{\def \codigoPrerequisitos}
\newcommand{\imprimircodigoPrerequisitos}{\codigoPrerequisitos}

\newcommand{\semestre}{\def \semestre}
\newcommand{\imprimirsemestre}{\semestre}

\newcommand{\nivel}{\def \nivel}
\newcommand{\imprimirnivel}{\nivel}

\newcommand{\codigoEquivalencias}{\def \codigoEquivalencias}
\newcommand{\imprimircodigoEquivalencias}{\codigoEquivalencias}

\newcommand{\ementa}{\def \ementa}
\newcommand{\imprimirementa}{\ementa}

\newcommand{\objetivo}{\def \objetivo}
\newcommand{\imprimirobjetivo}{\objetivo}

\newcommand{\programa}{\def \programa}
\newcommand{\imprimirprograma}{\programa}

\newcommand{\metodologiaEnsino}{\def \metodologiaEnsino}
\newcommand{\imprimirmetodologiaEnsino}{\metodologiaEnsino}

\newcommand{\recursos}{\def \recursos}
\newcommand{\imprimirrecursos}{\recursos}

\newcommand{\avaliacao}{\def \avaliacao}
\newcommand{\imprimiravaliacao}{\avaliacao}

\newcommand{\bibliografiaBasica}{\def \bibliografiaBasica}
\newcommand{\imprimirbibliografiaBasica}{\bibliografiaBasica}

\newcommand{\bibliografiaComplementar}{\def \bibliografiaComplementar}
\newcommand{\imprimirbibliografiaComplementar}{\bibliografiaComplementar}

\newcommand{\versao}{\def \versao}
\newcommand{\imprimirversao}{\versao}


%comando de impressão da estrutura
\newcommand{\imprimirPUD}{
%Cabeçalho do PUD
\begin{Spacing}{1}

\noindent \begin{minipage}{2.5cm}%
\includegraphics[scale=0.12]{logo-ifce}
\end{minipage}
\hspace{0.3cm}
\begin{minipage}{13cm}%
\centering INSTITUTO FEDERAL DE EDUCAÇÃO, CIÊNCIA E TECNOLOGIA DO CEARÁ- IFCE\\
CAMPUS JUAZEIRO DO NORTE\\
CURSO SUPERIOR EM AUTOMAÇÃO INDUSTRIAL\\
PROGRAMA DE UNIDADE DIDÁTICA – PUD\\
\end{minipage}%
\end{Spacing}

\begin{longtable}{|p{14cm}|}
%primeiro cabeçalho
\hline
\rowcolor{lightgray}
\multicolumn{1}{p{14cm}}{\textbf{Disciplina: \imprimirdisciplina}}\\
\hline
\endfirsthead

%cabeçalho
\hline
continuação PUD \imprimirdisciplina\\
\hline
\endhead

\hline
continua...\\
\hline
\endfoot

\hline
\rowcolor{lightgray}

\begin{tabular}{p{5.5 cm}| l}
coordenação & departamento pedagogico\\[16 ex]
\end{tabular}\\

\hline

\endlastfoot

%elementos
\textbf{Código:} \imprimircodigo\\


\textbf{Carga Horária } Teórica: \imprimircargaHorariaTeorica, Prática \imprimircargaHorariaPratica, Total: \imprimircargaHorariaTotal\\


\textbf{Número de créditos:} \imprimircreditos\\


\textbf{Código pré-requisitos:} \imprimircodigoPrerequisitos\\


\textbf{Semestre:} \imprimirsemestre\\


\textbf{Nível:} \imprimirnivel\\
\hline

\rowcolor{lightgray}
\multicolumn{1}{|p{14cm}|}{\textbf{Ementa}}\\
\hline
\multicolumn{1}{|p{14cm}|}{\imprimirementa}\\
\hline

\rowcolor{lightgray}
\multicolumn{1}{|p{14cm}|}{\textbf{Objetivo}}\\
\hline
\imprimirobjetivo\\
\hline

\rowcolor{lightgray}
\multicolumn{1}{|p{14cm}|}{\textbf{Programa}}\\
\hline
\imprimirprograma\\
\hline

\rowcolor{lightgray}
\multicolumn{1}{|p{14cm}|}{\textbf{Metodologia de ensino}}\\
\hline
\imprimirmetodologiaEnsino\\
\hline

\rowcolor{lightgray}
\multicolumn{1}{|p{14cm}|}{\textbf{Recursos}}\\
\hline
\imprimirrecursos\\
\hline

\rowcolor{lightgray}
\multicolumn{1}{|p{14cm}|}{\textbf{Avaliação}}\\
\hline
\imprimiravaliacao\\
\hline

\rowcolor{lightgray}
\multicolumn{1}{|p{14cm}|}{\textbf{Bibliografia básica}}\\
\hline
\imprimirbibliografiaBasica\\
\hline

\rowcolor{lightgray}
\multicolumn{1}{|p{14cm}|}{\textbf{Bibliografia complementar}}\\
\hline
\imprimirbibliografiaComplementar\\
\hline
\end{longtable}
\pagebreak
}
\begin{document}

\disciplina{Projetos elétricos}
\codigo{AUT2450}
\cargaHorariaTotal{40}
\cargaHorariaPratica{20}
\cargaHorariaTeorica{20}
\creditos{2}
\codigoPrerequisitos{-}
\semestre{opcional}
\nivel{Superior}

\ementa{
Conceitos elétricos básicos. Condutores elétricos. Elementos de circuitos elétricos. Dispositivos de proteção e eletrodutos. Projetos de circuitos elétricos prediais.
}

\objetivo{
• Dimensionar componentes Elétricos Prediais \\
• Projetar circuitos Elétricos Prediais.\\
• Executar manutenção preventiva em circuitos Elétricos Prediais.\\
• Realizar manutenção corretiva em circuitos Elétricos Prediais.\\
}

\programa{
• Projeto Elétrico\\
• Considerações gerais\\
• Elaboração\\
• Normas Regulamentadoras\\
• Conceitos Elétricos Básicos\\
• Eletricidade\\
• Geração\\
• Tipos de alimentação Elétrica e tensões\\
• Corrente Elétrica\\
• Potência Elétrica\\
• Condutores Elétricos\\
• Tipos de condutores\\
• Tipos de revestimentos\\
• Tipos de instalação\\
• Dimensionamento\\
• Variáveis do dimensionamento\\
• Tipos de emendas\\
• Seleção do condutor\\
• Seleção do condutor neutro e terra\\
• Elementos do Circuito Elétrico\\
• Caixas de Passagem\\
• Quadros Medidores\\
• Quadros de distribuição\\
• Interruptores\\
• Tomadas de uso geral\\
• Tomadas de uso específico\\
• Iluminação fluorescente e incandescente\\
• Elementos de Proteção\\
• Disjuntores\\
• Fusíveis\\
• Relés\\
• Dimensionamento\\
• Variáveis do dimensionamento\\
• Seleção\\
• Eletrodutos\\
• Conceitos básicos\\
• Tipos\\
• Dimensionamento\\
• Instalação\\
• Diagramas\\
• Diagrama Unifilar\\
• Diagrama Multifilar\\
• Desenho e Interpretação\\
• Desenho de circuitos elétricos\\
• Simbologia\\
• Interpretação de circuitos elétricos\\
}

\metodologiaEnsino{
Aulas expositivas.\\
Aulas práticas em laboratório.\\
Resolução de lista de exercícios.\\
Visitas técnicas.\\
Leitura e pesquisa bibliográfica.\\
}

\recursos{
Livros contidos na bibliografia.\\
Artigos.\\
Quadro e pincel.\\
Data-show.\\
Lista de exercícios.\\
Transporte para visitas técnicas.\\
Laboratório de instalações elétricas\\
}

\avaliacao{
Avaliação escrita.\\
Práticas individuais e em grupo no laboratório.\\
Relatório de prática.\\
Avaliação de exercícios resolvidos.\\
}

\bibliografiaBasica{
• ADEMARO, A. M. B. Cotrim. Instalações elétricas. São Paulo: Prentice Hall, 2010.\\

• CAVALIN, Geraldo; CERVELIN, Severino. Instalações elétricas prediais. São Paulo: Érica, 2010.\\

• CREDER , Hélio. Instalações elétricas. Rio de Janeiro: LTC, 2004.\\
}

\bibliografiaComplementar{
• NISKIER, Julio; MACINTYRE, Archibald Joseph. Instalações elétricas. Rio de Janeiro: LTC, 2000.\\

• LIMA FILHO, Domingos Leite. Projeto de instalações elétricas prediais. São Paulo: Erica, 2005.\\

• MAMEDE FILHO, João. Instalações elétricas industriais. Rio de Janeiro: LTC: 2010.\\

• VASQUEZ, José Ramirez. Instalações elétricas 1. Lisboa: Platamo Edições Técnicas, 1998.\\

• PROCEL. Conservação de Energia: Ed. Clássica. Rio de Janeiro: EFEI, 2001.

}


\imprimirPUD

\end{document}