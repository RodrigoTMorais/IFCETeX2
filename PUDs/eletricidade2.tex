\input{preambulo}
%arquivo de template para os PUDS

%definição das variaveis das seções
\newcommand{\disciplina}{\def \disciplina}
\newcommand{\imprimirdisciplina}{\disciplina}

\newcommand{\codigo}{\def \codigo}
\newcommand{\imprimircodigo}{\codigo}

\newcommand{\cargaHorariaTotal}{\def \cargaHorariaTotal}
\newcommand{\imprimircargaHorariaTotal}{\cargaHorariaTotal}

\newcommand{\cargaHorariaPratica}{\def \cargaHorariaPratica}
\newcommand{\imprimircargaHorariaPratica}{\cargaHorariaPratica}

\newcommand{\cargaHorariaTeorica}{\def \cargaHorariaTeorica}
\newcommand{\imprimircargaHorariaTeorica}{\cargaHorariaTeorica}

\newcommand{\creditos}{\def \creditos}
\newcommand{\imprimircreditos}{\creditos}

\newcommand{\codigoPrerequisitos}{\def \codigoPrerequisitos}
\newcommand{\imprimircodigoPrerequisitos}{\codigoPrerequisitos}

\newcommand{\semestre}{\def \semestre}
\newcommand{\imprimirsemestre}{\semestre}

\newcommand{\nivel}{\def \nivel}
\newcommand{\imprimirnivel}{\nivel}

\newcommand{\codigoEquivalencias}{\def \codigoEquivalencias}
\newcommand{\imprimircodigoEquivalencias}{\codigoEquivalencias}

\newcommand{\ementa}{\def \ementa}
\newcommand{\imprimirementa}{\ementa}

\newcommand{\objetivo}{\def \objetivo}
\newcommand{\imprimirobjetivo}{\objetivo}

\newcommand{\programa}{\def \programa}
\newcommand{\imprimirprograma}{\programa}

\newcommand{\metodologiaEnsino}{\def \metodologiaEnsino}
\newcommand{\imprimirmetodologiaEnsino}{\metodologiaEnsino}

\newcommand{\recursos}{\def \recursos}
\newcommand{\imprimirrecursos}{\recursos}

\newcommand{\avaliacao}{\def \avaliacao}
\newcommand{\imprimiravaliacao}{\avaliacao}

\newcommand{\bibliografiaBasica}{\def \bibliografiaBasica}
\newcommand{\imprimirbibliografiaBasica}{\bibliografiaBasica}

\newcommand{\bibliografiaComplementar}{\def \bibliografiaComplementar}
\newcommand{\imprimirbibliografiaComplementar}{\bibliografiaComplementar}

\newcommand{\versao}{\def \versao}
\newcommand{\imprimirversao}{\versao}


%comando de impressão da estrutura
\newcommand{\imprimirPUD}{
%Cabeçalho do PUD
\begin{Spacing}{1}

\noindent \begin{minipage}{2.5cm}%
\includegraphics[scale=0.12]{logo-ifce}
\end{minipage}
\hspace{0.3cm}
\begin{minipage}{13cm}%
\centering INSTITUTO FEDERAL DE EDUCAÇÃO, CIÊNCIA E TECNOLOGIA DO CEARÁ- IFCE\\
CAMPUS JUAZEIRO DO NORTE\\
CURSO SUPERIOR EM AUTOMAÇÃO INDUSTRIAL\\
PROGRAMA DE UNIDADE DIDÁTICA – PUD\\
\end{minipage}%
\end{Spacing}

\begin{longtable}{|p{14cm}|}
%primeiro cabeçalho
\hline
\rowcolor{lightgray}
\multicolumn{1}{p{14cm}}{\textbf{Disciplina: \imprimirdisciplina}}\\
\hline
\endfirsthead

%cabeçalho
\hline
continuação PUD \imprimirdisciplina\\
\hline
\endhead

\hline
continua...\\
\hline
\endfoot

\hline
\rowcolor{lightgray}

\begin{tabular}{p{5.5 cm}| l}
coordenação & departamento pedagogico\\[16 ex]
\end{tabular}\\

\hline

\endlastfoot

%elementos
\textbf{Código:} \imprimircodigo\\


\textbf{Carga Horária } Teórica: \imprimircargaHorariaTeorica, Prática \imprimircargaHorariaPratica, Total: \imprimircargaHorariaTotal\\


\textbf{Número de créditos:} \imprimircreditos\\


\textbf{Código pré-requisitos:} \imprimircodigoPrerequisitos\\


\textbf{Semestre:} \imprimirsemestre\\


\textbf{Nível:} \imprimirnivel\\
\hline

\rowcolor{lightgray}
\multicolumn{1}{|p{14cm}|}{\textbf{Ementa}}\\
\hline
\multicolumn{1}{|p{14cm}|}{\imprimirementa}\\
\hline

\rowcolor{lightgray}
\multicolumn{1}{|p{14cm}|}{\textbf{Objetivo}}\\
\hline
\imprimirobjetivo\\
\hline

\rowcolor{lightgray}
\multicolumn{1}{|p{14cm}|}{\textbf{Programa}}\\
\hline
\imprimirprograma\\
\hline

\rowcolor{lightgray}
\multicolumn{1}{|p{14cm}|}{\textbf{Metodologia de ensino}}\\
\hline
\imprimirmetodologiaEnsino\\
\hline

\rowcolor{lightgray}
\multicolumn{1}{|p{14cm}|}{\textbf{Recursos}}\\
\hline
\imprimirrecursos\\
\hline

\rowcolor{lightgray}
\multicolumn{1}{|p{14cm}|}{\textbf{Avaliação}}\\
\hline
\imprimiravaliacao\\
\hline

\rowcolor{lightgray}
\multicolumn{1}{|p{14cm}|}{\textbf{Bibliografia básica}}\\
\hline
\imprimirbibliografiaBasica\\
\hline

\rowcolor{lightgray}
\multicolumn{1}{|p{14cm}|}{\textbf{Bibliografia complementar}}\\
\hline
\imprimirbibliografiaComplementar\\
\hline
\end{longtable}
\pagebreak
}
\begin{document}

\disciplina{Eletricidade 2}
\codigo{AUT2411}
\cargaHorariaTotal{80}
\cargaHorariaPratica{20}
\cargaHorariaTeorica{60}
\creditos{4}
\codigoPrerequisitos{AUT2401, AUT2405}
\semestre{3º}
\nivel{Superior}

\ementa{
Fontes de tensão senoidal, o valor médio e o valor eficaz de uma forma de onda. Estudo do vetor rotativo e a notação tensão, corrente e fluxo de potência em corrente alternada. Conhecer e entender os Elementos capacitivos e indutivos. Especificar Elementos capacitivo e indutivo. Analisar Circuitos de corrente alternada em regime permanente.
}

\objetivo{
• Realizar conexões série e paralela de fontes de tensão senoidal, capacitores e indutores.\\
• Calcular constante de tempo, corrente e traçar as curvas nos circuitos de carga e descarga de capacitores e indutores.\\
• Aplicar as leis de análise de circuitos CA no estudo dos circuitos RC, RL e RLC.\\
}

\programa{
• Estudo das principais formas de Ondas. Parâmetros de forma de onda\\
• Valor médio\\
• Valor eficaz (RMS) Potência\\
• Estudo da senóide\\
• Estudo do vetor rotativo (Fasores) Notação de Tensão e Corrente Notação de tensão\\
• Notação de corrente\\
• Notação em análise de potência Potência no circuito resistivo puro Resistência com excitação senoidal\\
• Formas de onda da tensão, corrente e Potência no circuito resistivo puro.\\
• Potência média no circuito resistivo e lei de Ohm para circuitos CA Capacitância\\
• Carga e descarga de capacitor Energia armazenada pelo capacitor Geometria do capacitor\\
• Tensão de trabalho do capacitor Capacitores em série Capacitores em paralelo\\
• Corrente no capacitor\\
• Capacitor com excitação senoidal Reatância capacitiva\\
• Potência no circuito capacitivo puro. Indutância Tensão induzida – Lei de Faraday Corrente induzida Armazenamento de energia no indutor Geometria do indutor Indutores em série\
• Indutores em paralelo\\
• Indutores com excitação senoidal Reatância indutiva\\
• Potência em circuitos puramente indutivos Análise de Circuitos RLC\\
• Lei de Ohm para circuitos C.A. O conceito de impedância Circuito RLC série\\
• Admitância e circuito RLC paralelo Potência no circuito RLC Máxima transferência de energia\\
}

\metodologiaEnsino{
Aulas expositivas.\\
Aulas práticas em laboratório.\\
Resolução de lista de exercícios.\\
Visitas técnicas.\\
Leitura e pesquisa.\\
}

\recursos{
Livros contidos na bibliografia.\\
Artigos.\\
Quadro e pincel.\\
Data-show.\\
Lista de exercícios.\\
Transporte para visitas técnicas.\\
Aulas práticas em laboratório\\
}

\avaliacao{
Avaliação escrita.\\
Avaliação de exercícios resolvidos.\\
}

\bibliografiaBasica{
• BOYLESTAD, Robert L. Introdução à Análise de Circuitos. 10 ed. São Paulo: Pearson, 2004.\\

• CUTLER, Phillip. Análise de Circuitos CA. 1 ed. São Paulo: McGraw-Hill, 1976.\\

• O'MALLEY, John. Análise de Circuitos. 2 ed. São Paulo: McGraw-Hill, 1993.\\

}

\bibliografiaComplementar{
• CAPUANO, F. G.; MARINO, M. A. M. Laboratório de Eletricidade e Eletrônica. 24ª ed. São Paulo: Érica, 2008.\\

• ROBBINS, A. H.; MILLER, W. C. Análise de Circuitos – Teoria e Prática: Tradução da 4ª edição norte-americana. Vol. 1 e 2. São Paulo: Cengage Learning, 2010.\\ 

• MARKUS, O. Circuitos Elétricos: Corrente Contínua e Corrente Alternada - Teoria e Exercícios. 8ª ed. São Paulo: Érica, 2008. \\

• ALEXANDER, C. K.; SADIKU, M. N. Fundamentos de Circuitos Elétricos. 1ª ed. São Paulo: Bookman, 2003. \\

• ALBUQUERQUE, R. O. Análise de Circuitos em Corrente Alternada. 2ª ed. São Paulo: Érica, 2006. \\
}


\imprimirPUD

\end{document}