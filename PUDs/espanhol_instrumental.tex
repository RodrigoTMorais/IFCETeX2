\input{preambulo}
%arquivo de template para os PUDS

%definição das variaveis das seções
\newcommand{\disciplina}{\def \disciplina}
\newcommand{\imprimirdisciplina}{\disciplina}

\newcommand{\codigo}{\def \codigo}
\newcommand{\imprimircodigo}{\codigo}

\newcommand{\cargaHorariaTotal}{\def \cargaHorariaTotal}
\newcommand{\imprimircargaHorariaTotal}{\cargaHorariaTotal}

\newcommand{\cargaHorariaPratica}{\def \cargaHorariaPratica}
\newcommand{\imprimircargaHorariaPratica}{\cargaHorariaPratica}

\newcommand{\cargaHorariaTeorica}{\def \cargaHorariaTeorica}
\newcommand{\imprimircargaHorariaTeorica}{\cargaHorariaTeorica}

\newcommand{\creditos}{\def \creditos}
\newcommand{\imprimircreditos}{\creditos}

\newcommand{\codigoPrerequisitos}{\def \codigoPrerequisitos}
\newcommand{\imprimircodigoPrerequisitos}{\codigoPrerequisitos}

\newcommand{\semestre}{\def \semestre}
\newcommand{\imprimirsemestre}{\semestre}

\newcommand{\nivel}{\def \nivel}
\newcommand{\imprimirnivel}{\nivel}

\newcommand{\codigoEquivalencias}{\def \codigoEquivalencias}
\newcommand{\imprimircodigoEquivalencias}{\codigoEquivalencias}

\newcommand{\ementa}{\def \ementa}
\newcommand{\imprimirementa}{\ementa}

\newcommand{\objetivo}{\def \objetivo}
\newcommand{\imprimirobjetivo}{\objetivo}

\newcommand{\programa}{\def \programa}
\newcommand{\imprimirprograma}{\programa}

\newcommand{\metodologiaEnsino}{\def \metodologiaEnsino}
\newcommand{\imprimirmetodologiaEnsino}{\metodologiaEnsino}

\newcommand{\recursos}{\def \recursos}
\newcommand{\imprimirrecursos}{\recursos}

\newcommand{\avaliacao}{\def \avaliacao}
\newcommand{\imprimiravaliacao}{\avaliacao}

\newcommand{\bibliografiaBasica}{\def \bibliografiaBasica}
\newcommand{\imprimirbibliografiaBasica}{\bibliografiaBasica}

\newcommand{\bibliografiaComplementar}{\def \bibliografiaComplementar}
\newcommand{\imprimirbibliografiaComplementar}{\bibliografiaComplementar}

\newcommand{\versao}{\def \versao}
\newcommand{\imprimirversao}{\versao}


%comando de impressão da estrutura
\newcommand{\imprimirPUD}{
%Cabeçalho do PUD
\begin{Spacing}{1}

\noindent \begin{minipage}{2.5cm}%
\includegraphics[scale=0.12]{logo-ifce}
\end{minipage}
\hspace{0.3cm}
\begin{minipage}{13cm}%
\centering INSTITUTO FEDERAL DE EDUCAÇÃO, CIÊNCIA E TECNOLOGIA DO CEARÁ- IFCE\\
CAMPUS JUAZEIRO DO NORTE\\
CURSO SUPERIOR EM AUTOMAÇÃO INDUSTRIAL\\
PROGRAMA DE UNIDADE DIDÁTICA – PUD\\
\end{minipage}%
\end{Spacing}

\begin{longtable}{|p{14cm}|}
%primeiro cabeçalho
\hline
\rowcolor{lightgray}
\multicolumn{1}{p{14cm}}{\textbf{Disciplina: \imprimirdisciplina}}\\
\hline
\endfirsthead

%cabeçalho
\hline
continuação PUD \imprimirdisciplina\\
\hline
\endhead

\hline
continua...\\
\hline
\endfoot

\hline
\rowcolor{lightgray}

\begin{tabular}{p{5.5 cm}| l}
coordenação & departamento pedagogico\\[16 ex]
\end{tabular}\\

\hline

\endlastfoot

%elementos
\textbf{Código:} \imprimircodigo\\


\textbf{Carga Horária } Teórica: \imprimircargaHorariaTeorica, Prática \imprimircargaHorariaPratica, Total: \imprimircargaHorariaTotal\\


\textbf{Número de créditos:} \imprimircreditos\\


\textbf{Código pré-requisitos:} \imprimircodigoPrerequisitos\\


\textbf{Semestre:} \imprimirsemestre\\


\textbf{Nível:} \imprimirnivel\\
\hline

\rowcolor{lightgray}
\multicolumn{1}{|p{14cm}|}{\textbf{Ementa}}\\
\hline
\multicolumn{1}{|p{14cm}|}{\imprimirementa}\\
\hline

\rowcolor{lightgray}
\multicolumn{1}{|p{14cm}|}{\textbf{Objetivo}}\\
\hline
\imprimirobjetivo\\
\hline

\rowcolor{lightgray}
\multicolumn{1}{|p{14cm}|}{\textbf{Programa}}\\
\hline
\imprimirprograma\\
\hline

\rowcolor{lightgray}
\multicolumn{1}{|p{14cm}|}{\textbf{Metodologia de ensino}}\\
\hline
\imprimirmetodologiaEnsino\\
\hline

\rowcolor{lightgray}
\multicolumn{1}{|p{14cm}|}{\textbf{Recursos}}\\
\hline
\imprimirrecursos\\
\hline

\rowcolor{lightgray}
\multicolumn{1}{|p{14cm}|}{\textbf{Avaliação}}\\
\hline
\imprimiravaliacao\\
\hline

\rowcolor{lightgray}
\multicolumn{1}{|p{14cm}|}{\textbf{Bibliografia básica}}\\
\hline
\imprimirbibliografiaBasica\\
\hline

\rowcolor{lightgray}
\multicolumn{1}{|p{14cm}|}{\textbf{Bibliografia complementar}}\\
\hline
\imprimirbibliografiaComplementar\\
\hline
\end{longtable}
\pagebreak
}
\begin{document}

\disciplina{Espanhol instrumental}
\codigo{AUT2446}
\cargaHorariaTotal{40}
\cargaHorariaPratica{0}
\cargaHorariaTeorica{40}
\creditos{2}
\codigoPrerequisitos{-}
\semestre{opcional}
\nivel{Superior}

\ementa{
Introdução ao estudo da língua espanhola. Desenvolvimento da competência 
comunicativa, em nível instrumental, através do estudo de estruturas linguísticas e funções elementares da comunicação em língua espanhola, de atividades de prática de comunicação oral, de leitura e de produção textual e de aquisição de vocabulário básico específico da área.
}

\objetivo{
• Capacitar o aluno para o uso da língua espanhola em funções comunicativas básicas; \\
• Desenvolver, em nível instrumental, a habilidade auditiva, oral e escrita;\\ 
• Conceber, ao discente, estratégias de leitura que promovam a compreensão de diferentes gêneros textuais vinculados a área;\\ 
• Desenvolver, no aluno, habilidades linguísticas e socioculturais, em língua espanhola, no âmbito do turismo.\\
}

\programa{
•  El alfabeto; \\
•  Los artículos y apócope; \\
•  Numerales cardinales y ordinales; \\
•  La fecha y las horas; \\
•  Pronombres personales; \\
•  Presente de Indicativo y verbos para expresar gustos y preferencias; \\
•  Adverbios y preposiciones; \\
•  Pretérito Perfecto y Pretérito Indefinido; \\
•  Imperativo; \\
•  Estratégias de leitura.\\
•  Situaciones en el aeropuerto, en el hotel, en la agencia de viajes y en el restaurante;\\
•  Saludar y despedirse formal e informalmente; \\
•  Solicitar y dar informaciones; \\
•  Expresar sugerencias y peticiones; \\
•  Dar y pedir direcciones.\\
 Números cardinales y ordinales; \\
•  El aeropuerto, el avión; \\
•  Los colores; \\
•  Tipos de hoteles, estancias, habitaciones; \\
•  Mobiliario y objetos de una habitación del hotel; \\
•  Informaciones turística; \\
•  Vocabulario relacionado con la carta de un restaurante; \\
•  Comidas típicas españolas; \\
•  Expresión de la preferencia; \\
•  Profesiones relacionadas al aeropuerto, hotel y restaurante. \\
}

\metodologiaEnsino{
Exposição oral, diálogos; Leitura individual e participativa; Audição de CDs e de 
fitas cassetes; Projeção de filmes; Debates; Práticas de conversação. 
}

\recursos{
Sala de aula\\
bibliografia\\
data-show\\
}

\avaliacao{
Provas escritas e orais, com análise, interpretação e síntese; 
Exposição de trabalhos; Discussão em grupo; Exercícios. 
}

\bibliografiaBasica{
• PALOMINO, María Ángeles. Primer Plano 1. Gramática de español lengua 
extrajera. Madrid: Edelsa. 2001. \\

• HERMOSO, A. González; CUENOT, J. R. ALFARO, M. Sánchez. Español sin 
fronteras. SGEL. Madrid: Edelsa, 1996.\\
 
• LOBATO, Jesús Sánchez; MORENO, Concha; GARGALLO, Isabel Santos. 
Técnico Niveles 1,2,3. sl: Editora ao Livro, 1997.\\
 
}

\bibliografiaComplementar{
• PALOMINO, María Ángeles. Dual – pretextos para hablar. Madrid: Edelsa, 
2001.\\

• CERROLAZA, Matilde et al. Planeta ELE – Libro de referencia gramatical: 
fichas y ejercicios 1. Madrid: Edelsa, 1998.\\

• CASSANY, D. et al. Enseñar lengua. Barcelona: Grao, 1994. \\

• SEDYCIAS, J. O que é espanhol instrumental? 2002. Disponível em: 
<http://www.sedycias.com/espinst.htm>, Acesso em: 13 de set 2015.\\

• SOLE. I. Estrategias de Lectura. Barcelona: Grao, 1994.
}


\imprimirPUD

\end{document}