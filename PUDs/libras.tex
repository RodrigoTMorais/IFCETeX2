\input{preambulo}
%arquivo de template para os PUDS

%definição das variaveis das seções
\newcommand{\disciplina}{\def \disciplina}
\newcommand{\imprimirdisciplina}{\disciplina}

\newcommand{\codigo}{\def \codigo}
\newcommand{\imprimircodigo}{\codigo}

\newcommand{\cargaHorariaTotal}{\def \cargaHorariaTotal}
\newcommand{\imprimircargaHorariaTotal}{\cargaHorariaTotal}

\newcommand{\cargaHorariaPratica}{\def \cargaHorariaPratica}
\newcommand{\imprimircargaHorariaPratica}{\cargaHorariaPratica}

\newcommand{\cargaHorariaTeorica}{\def \cargaHorariaTeorica}
\newcommand{\imprimircargaHorariaTeorica}{\cargaHorariaTeorica}

\newcommand{\creditos}{\def \creditos}
\newcommand{\imprimircreditos}{\creditos}

\newcommand{\codigoPrerequisitos}{\def \codigoPrerequisitos}
\newcommand{\imprimircodigoPrerequisitos}{\codigoPrerequisitos}

\newcommand{\semestre}{\def \semestre}
\newcommand{\imprimirsemestre}{\semestre}

\newcommand{\nivel}{\def \nivel}
\newcommand{\imprimirnivel}{\nivel}

\newcommand{\codigoEquivalencias}{\def \codigoEquivalencias}
\newcommand{\imprimircodigoEquivalencias}{\codigoEquivalencias}

\newcommand{\ementa}{\def \ementa}
\newcommand{\imprimirementa}{\ementa}

\newcommand{\objetivo}{\def \objetivo}
\newcommand{\imprimirobjetivo}{\objetivo}

\newcommand{\programa}{\def \programa}
\newcommand{\imprimirprograma}{\programa}

\newcommand{\metodologiaEnsino}{\def \metodologiaEnsino}
\newcommand{\imprimirmetodologiaEnsino}{\metodologiaEnsino}

\newcommand{\recursos}{\def \recursos}
\newcommand{\imprimirrecursos}{\recursos}

\newcommand{\avaliacao}{\def \avaliacao}
\newcommand{\imprimiravaliacao}{\avaliacao}

\newcommand{\bibliografiaBasica}{\def \bibliografiaBasica}
\newcommand{\imprimirbibliografiaBasica}{\bibliografiaBasica}

\newcommand{\bibliografiaComplementar}{\def \bibliografiaComplementar}
\newcommand{\imprimirbibliografiaComplementar}{\bibliografiaComplementar}

\newcommand{\versao}{\def \versao}
\newcommand{\imprimirversao}{\versao}


%comando de impressão da estrutura
\newcommand{\imprimirPUD}{
%Cabeçalho do PUD
\begin{Spacing}{1}

\noindent \begin{minipage}{2.5cm}%
\includegraphics[scale=0.12]{logo-ifce}
\end{minipage}
\hspace{0.3cm}
\begin{minipage}{13cm}%
\centering INSTITUTO FEDERAL DE EDUCAÇÃO, CIÊNCIA E TECNOLOGIA DO CEARÁ- IFCE\\
CAMPUS JUAZEIRO DO NORTE\\
CURSO SUPERIOR EM AUTOMAÇÃO INDUSTRIAL\\
PROGRAMA DE UNIDADE DIDÁTICA – PUD\\
\end{minipage}%
\end{Spacing}

\begin{longtable}{|p{14cm}|}
%primeiro cabeçalho
\hline
\rowcolor{lightgray}
\multicolumn{1}{p{14cm}}{\textbf{Disciplina: \imprimirdisciplina}}\\
\hline
\endfirsthead

%cabeçalho
\hline
continuação PUD \imprimirdisciplina\\
\hline
\endhead

\hline
continua...\\
\hline
\endfoot

\hline
\rowcolor{lightgray}

\begin{tabular}{p{5.5 cm}| l}
coordenação & departamento pedagogico\\[16 ex]
\end{tabular}\\

\hline

\endlastfoot

%elementos
\textbf{Código:} \imprimircodigo\\


\textbf{Carga Horária } Teórica: \imprimircargaHorariaTeorica, Prática \imprimircargaHorariaPratica, Total: \imprimircargaHorariaTotal\\


\textbf{Número de créditos:} \imprimircreditos\\


\textbf{Código pré-requisitos:} \imprimircodigoPrerequisitos\\


\textbf{Semestre:} \imprimirsemestre\\


\textbf{Nível:} \imprimirnivel\\
\hline

\rowcolor{lightgray}
\multicolumn{1}{|p{14cm}|}{\textbf{Ementa}}\\
\hline
\multicolumn{1}{|p{14cm}|}{\imprimirementa}\\
\hline

\rowcolor{lightgray}
\multicolumn{1}{|p{14cm}|}{\textbf{Objetivo}}\\
\hline
\imprimirobjetivo\\
\hline

\rowcolor{lightgray}
\multicolumn{1}{|p{14cm}|}{\textbf{Programa}}\\
\hline
\imprimirprograma\\
\hline

\rowcolor{lightgray}
\multicolumn{1}{|p{14cm}|}{\textbf{Metodologia de ensino}}\\
\hline
\imprimirmetodologiaEnsino\\
\hline

\rowcolor{lightgray}
\multicolumn{1}{|p{14cm}|}{\textbf{Recursos}}\\
\hline
\imprimirrecursos\\
\hline

\rowcolor{lightgray}
\multicolumn{1}{|p{14cm}|}{\textbf{Avaliação}}\\
\hline
\imprimiravaliacao\\
\hline

\rowcolor{lightgray}
\multicolumn{1}{|p{14cm}|}{\textbf{Bibliografia básica}}\\
\hline
\imprimirbibliografiaBasica\\
\hline

\rowcolor{lightgray}
\multicolumn{1}{|p{14cm}|}{\textbf{Bibliografia complementar}}\\
\hline
\imprimirbibliografiaComplementar\\
\hline
\end{longtable}
\pagebreak
}
\begin{document}

\disciplina{Libras}
\codigo{AUT2444}
\cargaHorariaTotal{40}
\cargaHorariaPratica{0}
\cargaHorariaTeorica{40}
\creditos{2}
\codigoPrerequisitos{-}
\semestre{Opcional}
\nivel{Superior}

\ementa{
Ter conhecimento sobre a Língua Brasileira de Sinais – LIBRAS; Ler, interpretar textos e conversar em LIBRAS; Sistematizar informações; Identificar as ações facilitadoras da inclusão; Compreender a dinâmica dos serviços de apoio especializado no contexto escolar; Entender como ocorre a aquisição da Língua Portuguesa por ouvintes e surdos; Compreender os critérios de avaliação diferenciados dos alunos surdos conforme o Aviso
Circular 277/94 do MEC, garantindo-lhe a escolarização da Educação Básica à Superior e executar o papel que a mesma tem na constituição e educação da pessoa surda;
}

\objetivo{
• Conhecer as especificidades 164orna164stica164 e culturais das pessoas surdas;\\
• Conhecer os aspectos 164orna164stica164 da Língua Brasileira de Sinais;\\
• Conhecer características culturais das comunidades surdas;\\
• Refletir sobre o papel da Língua de Sinais na constituição da identidade da pessoa surda;\\
• Refletir sobre o papel da Língua de Sinais na educação dos alunos surdos;\\
• Aprender a estabelecer uma conversação básica em LIBRAS;\\
• Ter noção básica do que é a surdez do ponto de vista orgânico;\\
• Conhecer os principais documentos que tratam dos direitos do cidadão Surdo;\\
• Conhecer os recursos que propiciam a acessibilidade da pessoa Surda ao mundo ouvinte.\\
}

\programa{
• Surdez, Cultura e Identidade. LIBRAS: A língua natural dos surdos. O bilinguismo na educação de surdos.\\
• Unidade IV – Ações facilitadoras da inclusão. Módulo 2\\
• Ações facilitadoras da inclusão.\\
• Características do Português como segunda língua. Critérios diferenciados na avaliação da escrita do surdo.\\
• Leitura e produção de textos na perspectiva do português como segunda língua.\\ 
• Inicialização da LIBRAS – Alfabeto e Numerais. Parâmetros principais da LIBRAS.\\
• Sinais da LIBRAS.\\
}

\metodologiaEnsino{
Leitura, estudo e debates em sala de aula.\\
Apresentação e interação com alunos surdos.\\
Seminários.\\
Observação em campo.\\
Socialização de informações em sala de aula.\\
Atividades ligada a pessoa surda.\\
}

\recursos{
Sala de aula\\
}

\avaliacao{
Participação dos alunos nas atividades propostas.\\
Trabalhos individuais e/ou em grupo.\\
Avaliação do material estudado fora e em sala de aula.\\
Relatório e apresentação das aulas de campo.\\

OBS: A primeira nota corresponderá à participação do(a) aluno(a) nas atividades propostas (estudos e debates do material estudado em sala); a segunda nota será atribuída pelos trabalhos realizados (seminário, trabalhos em grupo etc.); e a terceira decorrerá do relatório e apresentação das aulas de campo.
}

\bibliografiaBasica{
• MOREIRA LIMA, Heloisa Maria. Ensino de língua portuguesa para surdos: caminhos para a prática pedagógica Volume 1 – 2. Ed. Brasília: MEC, SEESP, 2007.\\

• MOREIRA LIMA, Heloisa Maria. Ensino de língua portuguesa para surdos: caminhos para a prática pedagógica Volume 2 – 2 ed. Brasília: MEC, SEESP, 2007.\\

• SEESP, Secretaria de Educação Especial. O tradutor e intérprete de língua brasileira de sinais e língua portuguesa – 2 ed. Brasília: MEC, SEESP, 2007.\\
}

\bibliografiaComplementar{
• SEESP, Secretaria de Educação Especial. Diretrizes Nacionais para a Educação Especial na Educação Básica – 1 ed. Brasília: MEC, SEESP, 2001.\\

• CAPOVILLA, Fernando César. Dicionário Enciclopédico Ilustrado Trilíngüe da Língua de Sinais Brasileira – 1 ed. São Paulo: Editora da Universidade de São Pau
lo, 2001.\\

• FELIPE, Tanya A. Libras em Contexto: curso básico – 1 ed. Brasília: MEC, SEESP, 2001.\\

• QUADROS, Ronice Müller de. Língua de Sinais Brasileira: estudos 165orna165stica – 1 ed. Porto Alegre: Editora Artmed, 2004.\\

• FELIPE, T A; MONTEIRO, M S. Libras em Contexto: curso básico, livro do professor instrutor. Brasília: Programa Nacional de Apoio à Educação dos Surdos, MEC: SEESP, 2001.
}


\imprimirPUD

\end{document}