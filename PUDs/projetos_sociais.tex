\input{preambulo}
%arquivo de template para os PUDS

%definição das variaveis das seções
\newcommand{\disciplina}{\def \disciplina}
\newcommand{\imprimirdisciplina}{\disciplina}

\newcommand{\codigo}{\def \codigo}
\newcommand{\imprimircodigo}{\codigo}

\newcommand{\cargaHorariaTotal}{\def \cargaHorariaTotal}
\newcommand{\imprimircargaHorariaTotal}{\cargaHorariaTotal}

\newcommand{\cargaHorariaPratica}{\def \cargaHorariaPratica}
\newcommand{\imprimircargaHorariaPratica}{\cargaHorariaPratica}

\newcommand{\cargaHorariaTeorica}{\def \cargaHorariaTeorica}
\newcommand{\imprimircargaHorariaTeorica}{\cargaHorariaTeorica}

\newcommand{\creditos}{\def \creditos}
\newcommand{\imprimircreditos}{\creditos}

\newcommand{\codigoPrerequisitos}{\def \codigoPrerequisitos}
\newcommand{\imprimircodigoPrerequisitos}{\codigoPrerequisitos}

\newcommand{\semestre}{\def \semestre}
\newcommand{\imprimirsemestre}{\semestre}

\newcommand{\nivel}{\def \nivel}
\newcommand{\imprimirnivel}{\nivel}

\newcommand{\codigoEquivalencias}{\def \codigoEquivalencias}
\newcommand{\imprimircodigoEquivalencias}{\codigoEquivalencias}

\newcommand{\ementa}{\def \ementa}
\newcommand{\imprimirementa}{\ementa}

\newcommand{\objetivo}{\def \objetivo}
\newcommand{\imprimirobjetivo}{\objetivo}

\newcommand{\programa}{\def \programa}
\newcommand{\imprimirprograma}{\programa}

\newcommand{\metodologiaEnsino}{\def \metodologiaEnsino}
\newcommand{\imprimirmetodologiaEnsino}{\metodologiaEnsino}

\newcommand{\recursos}{\def \recursos}
\newcommand{\imprimirrecursos}{\recursos}

\newcommand{\avaliacao}{\def \avaliacao}
\newcommand{\imprimiravaliacao}{\avaliacao}

\newcommand{\bibliografiaBasica}{\def \bibliografiaBasica}
\newcommand{\imprimirbibliografiaBasica}{\bibliografiaBasica}

\newcommand{\bibliografiaComplementar}{\def \bibliografiaComplementar}
\newcommand{\imprimirbibliografiaComplementar}{\bibliografiaComplementar}

\newcommand{\versao}{\def \versao}
\newcommand{\imprimirversao}{\versao}


%comando de impressão da estrutura
\newcommand{\imprimirPUD}{
%Cabeçalho do PUD
\begin{Spacing}{1}

\noindent \begin{minipage}{2.5cm}%
\includegraphics[scale=0.12]{logo-ifce}
\end{minipage}
\hspace{0.3cm}
\begin{minipage}{13cm}%
\centering INSTITUTO FEDERAL DE EDUCAÇÃO, CIÊNCIA E TECNOLOGIA DO CEARÁ- IFCE\\
CAMPUS JUAZEIRO DO NORTE\\
CURSO SUPERIOR EM AUTOMAÇÃO INDUSTRIAL\\
PROGRAMA DE UNIDADE DIDÁTICA – PUD\\
\end{minipage}%
\end{Spacing}

\begin{longtable}{|p{14cm}|}
%primeiro cabeçalho
\hline
\rowcolor{lightgray}
\multicolumn{1}{p{14cm}}{\textbf{Disciplina: \imprimirdisciplina}}\\
\hline
\endfirsthead

%cabeçalho
\hline
continuação PUD \imprimirdisciplina\\
\hline
\endhead

\hline
continua...\\
\hline
\endfoot

\hline
\rowcolor{lightgray}

\begin{tabular}{p{5.5 cm}| l}
coordenação & departamento pedagogico\\[16 ex]
\end{tabular}\\

\hline

\endlastfoot

%elementos
\textbf{Código:} \imprimircodigo\\


\textbf{Carga Horária } Teórica: \imprimircargaHorariaTeorica, Prática \imprimircargaHorariaPratica, Total: \imprimircargaHorariaTotal\\


\textbf{Número de créditos:} \imprimircreditos\\


\textbf{Código pré-requisitos:} \imprimircodigoPrerequisitos\\


\textbf{Semestre:} \imprimirsemestre\\


\textbf{Nível:} \imprimirnivel\\
\hline

\rowcolor{lightgray}
\multicolumn{1}{|p{14cm}|}{\textbf{Ementa}}\\
\hline
\multicolumn{1}{|p{14cm}|}{\imprimirementa}\\
\hline

\rowcolor{lightgray}
\multicolumn{1}{|p{14cm}|}{\textbf{Objetivo}}\\
\hline
\imprimirobjetivo\\
\hline

\rowcolor{lightgray}
\multicolumn{1}{|p{14cm}|}{\textbf{Programa}}\\
\hline
\imprimirprograma\\
\hline

\rowcolor{lightgray}
\multicolumn{1}{|p{14cm}|}{\textbf{Metodologia de ensino}}\\
\hline
\imprimirmetodologiaEnsino\\
\hline

\rowcolor{lightgray}
\multicolumn{1}{|p{14cm}|}{\textbf{Recursos}}\\
\hline
\imprimirrecursos\\
\hline

\rowcolor{lightgray}
\multicolumn{1}{|p{14cm}|}{\textbf{Avaliação}}\\
\hline
\imprimiravaliacao\\
\hline

\rowcolor{lightgray}
\multicolumn{1}{|p{14cm}|}{\textbf{Bibliografia básica}}\\
\hline
\imprimirbibliografiaBasica\\
\hline

\rowcolor{lightgray}
\multicolumn{1}{|p{14cm}|}{\textbf{Bibliografia complementar}}\\
\hline
\imprimirbibliografiaComplementar\\
\hline
\end{longtable}
\pagebreak
}
\begin{document}

\disciplina{Projetos sociais}
\codigo{AUT2443}
\cargaHorariaTotal{40}
\cargaHorariaPratica{30}
\cargaHorariaTeorica{10}
\creditos{2}
\codigoPrerequisitos{-}
\semestre{2º}
\nivel{Superior}

\ementa{
Análise do contexto sócio-político-econômico da sociedade brasileira e global. Formação de valores éticos e de autonomia. Participação social. Relações étnico-raciais, direitos humanos, educação ambiental. Relações com Movimentos Sociais e o terceiro setor. Formas de organização e participação em trabalhos sociais. Métodos e Técnicas de elaboração de projetos sociais. Pressupostos teóricos e práticos a serem considerados na construção de projetos sociais.
}

\objetivo{
• Conhecer a realidade brasileira e global e os desafios dos diferentes contextos profissionais.\\
• Discutir participação social e cidadania na busca de uma formação engajada profissional.\\
• Intervir técnica e pedagogicamente na realidade social.\\
• Resolver situações-problema utilizando-se dos diversos tipos de linguagem.\\
• Organizar o trabalho de forma que possa Pensar sobre a organização do trabalho, como desenvolvê-lo de maneira competente e com isto ser para que possa ser valorizado como sujeito histórico, crítico e participativo.\\
}

\programa{
• Formação de valores éticos e de autonomia e formas de participação social.\\
• Análise do contexto sócio-político-econômico da sociedade brasileira e global.\\
• Relações com Movimentos Sociais e o terceiro setor.\\
• Formas de organização e participação em trabalhos sociais.\\
• Discussão sobre as Relações étnico-raciais e formação social brasileira.\\
• Apresentação dos contextos sobre direitos humanos e seu histórico.\\
• Desafios globais da atualidade e educação ambiental;\\
• Métodos e Técnicas de elaboração de projetos sociais.\\
• Execução de projetos e avaliação de intervenção voltada para sustentabilidade.\\
}

\metodologiaEnsino{
Discussão dialogada através de textos de apoio. Encontros expositivos para introdução às ferramentas de projetos. Utilização de visitas técnicas para montagem de diagnósticos e reconhecimento de campo da realidade a ser trabalhada. Elaboração de projeto e intervenção na comunidade. Avaliação realizada através da produção de textos e impactos gerados nas intervenções propostas nos projetos
}

\recursos{
Material didático-pedagógico.\\
Recursos audiovisuais Data-show.\\
Transporte para visitas técnicas.\\
}

\avaliacao{
Participação dos alunos nas atividades propostas.\\
Trabalhos individuais ou em grupo.\\
Seminários e/ou mesas redondas.\\
Provas que envolvam respostas livres de análise crítica sobre o conteúdo programático da disciplina em foco.\\
Avaliação realizada através da produção de textos e impactos gerados nas intervenções propostas nos projetos.\\
}

\bibliografiaBasica{
• BELLO, Enzo org. Ensaios críticos sobre direitos humanos e constitucionalismo. Caxias do Sul: Educs, 2012. BVU.\\
• FREIRE, P. Pedagogia da autonomia: saberes necessários à prática educativa. São Paulo: Paz e Terra, 2007.\\
• GIANEZINI, Miguelangelo; RAMOS, Ieda Cristina Alves. Elaboração de projetos sociais. Série Por Dentro das Ciências Sociais. Editora Intersaberes, 2015.\\
}

\bibliografiaComplementar{
• COHEN, E. \& FRANCO, R. Avaliação de projetos sociais. 6.ed. Petrópolis: Vozes, 1993.\\
• DEMO, Pedro. Participação é conquista: noções de política social participativa. São Paulo: Cortez, 2001.\\
• DIMENSTEIN, Gilberto. O cidadão de papel: a infância, a adolescência e os direitos humanos no Brasil. São Paulo: Ática, 2003.\\
• MARTINS, Carlos Benedito. O que é sociologia. 61 ed. São Paulo, Brasiliense, 2006.\\
• HOLANDA, Nilson. Elaboração e avaliação de projetos. Rio de Janeiro: APEC, 1969\\
}


\imprimirPUD

\end{document}