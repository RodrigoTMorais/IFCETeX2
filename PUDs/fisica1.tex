\input{preambulo}
%arquivo de template para os PUDS

%definição das variaveis das seções
\newcommand{\disciplina}{\def \disciplina}
\newcommand{\imprimirdisciplina}{\disciplina}

\newcommand{\codigo}{\def \codigo}
\newcommand{\imprimircodigo}{\codigo}

\newcommand{\cargaHorariaTotal}{\def \cargaHorariaTotal}
\newcommand{\imprimircargaHorariaTotal}{\cargaHorariaTotal}

\newcommand{\cargaHorariaPratica}{\def \cargaHorariaPratica}
\newcommand{\imprimircargaHorariaPratica}{\cargaHorariaPratica}

\newcommand{\cargaHorariaTeorica}{\def \cargaHorariaTeorica}
\newcommand{\imprimircargaHorariaTeorica}{\cargaHorariaTeorica}

\newcommand{\creditos}{\def \creditos}
\newcommand{\imprimircreditos}{\creditos}

\newcommand{\codigoPrerequisitos}{\def \codigoPrerequisitos}
\newcommand{\imprimircodigoPrerequisitos}{\codigoPrerequisitos}

\newcommand{\semestre}{\def \semestre}
\newcommand{\imprimirsemestre}{\semestre}

\newcommand{\nivel}{\def \nivel}
\newcommand{\imprimirnivel}{\nivel}

\newcommand{\codigoEquivalencias}{\def \codigoEquivalencias}
\newcommand{\imprimircodigoEquivalencias}{\codigoEquivalencias}

\newcommand{\ementa}{\def \ementa}
\newcommand{\imprimirementa}{\ementa}

\newcommand{\objetivo}{\def \objetivo}
\newcommand{\imprimirobjetivo}{\objetivo}

\newcommand{\programa}{\def \programa}
\newcommand{\imprimirprograma}{\programa}

\newcommand{\metodologiaEnsino}{\def \metodologiaEnsino}
\newcommand{\imprimirmetodologiaEnsino}{\metodologiaEnsino}

\newcommand{\recursos}{\def \recursos}
\newcommand{\imprimirrecursos}{\recursos}

\newcommand{\avaliacao}{\def \avaliacao}
\newcommand{\imprimiravaliacao}{\avaliacao}

\newcommand{\bibliografiaBasica}{\def \bibliografiaBasica}
\newcommand{\imprimirbibliografiaBasica}{\bibliografiaBasica}

\newcommand{\bibliografiaComplementar}{\def \bibliografiaComplementar}
\newcommand{\imprimirbibliografiaComplementar}{\bibliografiaComplementar}

\newcommand{\versao}{\def \versao}
\newcommand{\imprimirversao}{\versao}


%comando de impressão da estrutura
\newcommand{\imprimirPUD}{
%Cabeçalho do PUD
\begin{Spacing}{1}

\noindent \begin{minipage}{2.5cm}%
\includegraphics[scale=0.12]{logo-ifce}
\end{minipage}
\hspace{0.3cm}
\begin{minipage}{13cm}%
\centering INSTITUTO FEDERAL DE EDUCAÇÃO, CIÊNCIA E TECNOLOGIA DO CEARÁ- IFCE\\
CAMPUS JUAZEIRO DO NORTE\\
CURSO SUPERIOR EM AUTOMAÇÃO INDUSTRIAL\\
PROGRAMA DE UNIDADE DIDÁTICA – PUD\\
\end{minipage}%
\end{Spacing}

\begin{longtable}{|p{14cm}|}
%primeiro cabeçalho
\hline
\rowcolor{lightgray}
\multicolumn{1}{p{14cm}}{\textbf{Disciplina: \imprimirdisciplina}}\\
\hline
\endfirsthead

%cabeçalho
\hline
continuação PUD \imprimirdisciplina\\
\hline
\endhead

\hline
continua...\\
\hline
\endfoot

\hline
\rowcolor{lightgray}

\begin{tabular}{p{5.5 cm}| l}
coordenação & departamento pedagogico\\[16 ex]
\end{tabular}\\

\hline

\endlastfoot

%elementos
\textbf{Código:} \imprimircodigo\\


\textbf{Carga Horária } Teórica: \imprimircargaHorariaTeorica, Prática \imprimircargaHorariaPratica, Total: \imprimircargaHorariaTotal\\


\textbf{Número de créditos:} \imprimircreditos\\


\textbf{Código pré-requisitos:} \imprimircodigoPrerequisitos\\


\textbf{Semestre:} \imprimirsemestre\\


\textbf{Nível:} \imprimirnivel\\
\hline

\rowcolor{lightgray}
\multicolumn{1}{|p{14cm}|}{\textbf{Ementa}}\\
\hline
\multicolumn{1}{|p{14cm}|}{\imprimirementa}\\
\hline

\rowcolor{lightgray}
\multicolumn{1}{|p{14cm}|}{\textbf{Objetivo}}\\
\hline
\imprimirobjetivo\\
\hline

\rowcolor{lightgray}
\multicolumn{1}{|p{14cm}|}{\textbf{Programa}}\\
\hline
\imprimirprograma\\
\hline

\rowcolor{lightgray}
\multicolumn{1}{|p{14cm}|}{\textbf{Metodologia de ensino}}\\
\hline
\imprimirmetodologiaEnsino\\
\hline

\rowcolor{lightgray}
\multicolumn{1}{|p{14cm}|}{\textbf{Recursos}}\\
\hline
\imprimirrecursos\\
\hline

\rowcolor{lightgray}
\multicolumn{1}{|p{14cm}|}{\textbf{Avaliação}}\\
\hline
\imprimiravaliacao\\
\hline

\rowcolor{lightgray}
\multicolumn{1}{|p{14cm}|}{\textbf{Bibliografia básica}}\\
\hline
\imprimirbibliografiaBasica\\
\hline

\rowcolor{lightgray}
\multicolumn{1}{|p{14cm}|}{\textbf{Bibliografia complementar}}\\
\hline
\imprimirbibliografiaComplementar\\
\hline
\end{longtable}
\pagebreak
}
\begin{document}

\disciplina{Física 1}
\codigo{AUT2413}
\cargaHorariaTotal{80}
\cargaHorariaPratica{20}
\cargaHorariaTeorica{60}
\creditos{4}
\codigoPrerequisitos{AUT2405}
\semestre{3º}
\nivel{Superior}

\ementa{
Medidas. Movimento unidimensional. Vetores. Movimento em duas e três dimensões. Dinâmica newtoniana. Trabalho e energia. Conservação da energia mecânica. Centro de massa. Momento linear: conservação e colisões. Cinemática e dinâmica da rotação. Rolamento, torque e momento angular. Equilíbrio e elasticidade.
}

\objetivo{
• Apreender conhecimentos e conceitos
introdutórios de mecânica clássica.
}

\programa{
• Medidas.\\
• Padrões e unidades.\\
• Incerteza e algarismos significativos.\\
• Movimento unidimensional.\\
• Deslocamento, tempo e velocidade média.\\
• Velocidade instantânea.\\
• Aceleração instantânea e aceleração média.\\
• Movimento com aceleração constante.\\
• Queda livre.\\
• Vetores.\\
• Soma de vetores.\\
• Decomposição de vetores.\\
• Vetores unitários.\\
• Produtos de vetores.\\
• Movimento em duas e três dimensões.\\
• Vetor posição e vetor velocidade.\\
• Vetor aceleração.\\
• Movimento de um projétil.\\
• Movimento circular.\\
• Velocidade relativa.\\
• Dinâmica newtoniana.\\
• Primeira lei de Newton.\\
• Segunda lei de Newton.\\
• Massa e peso.\\
• Terceira lei de Newton.\\
• Dinâmica das partículas.\\
• Forças de atrito.\\
• Dinâmica do movimento circular uniforme.\\
• Movimento de projéteis com resistência do ar.\\
• Trabalho e energia.\\
• Trabalho.\\
• Trabalho e energia cinética.\\
• Trabalho de forças variáveis.\\
• Potência.\\
• Conservação da energia mecânica.\\
• Forças conservativas.\\
• Energia potencial gravitacional.\\
• Energia potencial elástica.\\
• Conservação de energia em um sistema de partículas.\\
• Centro de massa.\\
• Sistemas de duas partículas.\\
• Sistemas de muitas partículas.\\
• Centro de massa de objetos sólidos.\\
• Momentum linear: conservação e colisões.\\
• Momento linear e impulso.\\
• Conservação do momento linear.\\
• Colisões elásticas e inelásticas.\\
• Cinemática e dinâmica da rotação.\\
• Velocidade angular e aceleração angular.\\
• Rotação com aceleração angular constante.\\
• Grandezas rotacionais como vetores.\\
• Relação entre variáveis lineares e angulares.\\
• Energia do movimento de rotação.\\
• Teorema dos eixos paralelos.\\
• Momento de inércia.\\
• Rolamento, torque e momentum angular.\\
• Torque.\\
• Dinâmica rotacional de um corpo rígido.\\
• Momento angular.\\
• Conservação do momento angular.\\
• Giroscópios e precessão.\\
• Equilíbrio e elasticidade.\\
• Condições de equilíbrio.\\
• Centro de gravidade.\\
• Equilíbrio estável, instável e neutro.\\
}

\metodologiaEnsino{
Aulas expositivas e aulas práticas de laboratório.
}

\recursos{
Quadro e pincel.\\
Data-show.\\
Laboratório de física\\
}

\avaliacao{
Prova escrita.\\
Relatórios de práticas de laboratório.\\
}

\bibliografiaBasica{
• HALLIDAY, David; RESNICK, J. W.; WALKER, J. Fundamentos de física: mecânica. Riode Janeiro:LTC, 2006. v. 1.\\

• TIPLER, Paul A./Mosca, Gene. Física: para cientistas e engenheiros: mecânica. Rio de Janeiro: Guanabara Koogan, 1994. v. 1.\\

• YOUNG, Hugh D.; FREEDMAN, Roger A. Sears e Zemansky Física. São Paulo: Pearson
Addison Wesley, 2008. v. 1.\\
}

\bibliografiaComplementar{
• GONÇALVES, Dalton. Física: mecânica. Rio de Janeiro: LTC, 1979. v. 1.\\

• NUSSENZWEIG, Moysés. Curso de Física básica I: mecânica. São Paulo: Blucher, 2008.\\

• SERWAY, Raymond A. / Jewett Jr., John W. Princípios de física: mecânica clássica.\\

• ALONSO, M. \& FINN, E. J. Física Um Curso Universitário, Mecânica, Vol. 1., Editora Edgard Blücher Ltda. 2009\\

• MCKELVEY, J. P. \& GROTCH, H. Física Geral, Vol. 1, Harbra. 1978.\\ 


}


\imprimirPUD

\end{document}