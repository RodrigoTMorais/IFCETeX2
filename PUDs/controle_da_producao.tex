\input{preambulo}
%arquivo de template para os PUDS

%definição das variaveis das seções
\newcommand{\disciplina}{\def \disciplina}
\newcommand{\imprimirdisciplina}{\disciplina}

\newcommand{\codigo}{\def \codigo}
\newcommand{\imprimircodigo}{\codigo}

\newcommand{\cargaHorariaTotal}{\def \cargaHorariaTotal}
\newcommand{\imprimircargaHorariaTotal}{\cargaHorariaTotal}

\newcommand{\cargaHorariaPratica}{\def \cargaHorariaPratica}
\newcommand{\imprimircargaHorariaPratica}{\cargaHorariaPratica}

\newcommand{\cargaHorariaTeorica}{\def \cargaHorariaTeorica}
\newcommand{\imprimircargaHorariaTeorica}{\cargaHorariaTeorica}

\newcommand{\creditos}{\def \creditos}
\newcommand{\imprimircreditos}{\creditos}

\newcommand{\codigoPrerequisitos}{\def \codigoPrerequisitos}
\newcommand{\imprimircodigoPrerequisitos}{\codigoPrerequisitos}

\newcommand{\semestre}{\def \semestre}
\newcommand{\imprimirsemestre}{\semestre}

\newcommand{\nivel}{\def \nivel}
\newcommand{\imprimirnivel}{\nivel}

\newcommand{\codigoEquivalencias}{\def \codigoEquivalencias}
\newcommand{\imprimircodigoEquivalencias}{\codigoEquivalencias}

\newcommand{\ementa}{\def \ementa}
\newcommand{\imprimirementa}{\ementa}

\newcommand{\objetivo}{\def \objetivo}
\newcommand{\imprimirobjetivo}{\objetivo}

\newcommand{\programa}{\def \programa}
\newcommand{\imprimirprograma}{\programa}

\newcommand{\metodologiaEnsino}{\def \metodologiaEnsino}
\newcommand{\imprimirmetodologiaEnsino}{\metodologiaEnsino}

\newcommand{\recursos}{\def \recursos}
\newcommand{\imprimirrecursos}{\recursos}

\newcommand{\avaliacao}{\def \avaliacao}
\newcommand{\imprimiravaliacao}{\avaliacao}

\newcommand{\bibliografiaBasica}{\def \bibliografiaBasica}
\newcommand{\imprimirbibliografiaBasica}{\bibliografiaBasica}

\newcommand{\bibliografiaComplementar}{\def \bibliografiaComplementar}
\newcommand{\imprimirbibliografiaComplementar}{\bibliografiaComplementar}

\newcommand{\versao}{\def \versao}
\newcommand{\imprimirversao}{\versao}


%comando de impressão da estrutura
\newcommand{\imprimirPUD}{
%Cabeçalho do PUD
\begin{Spacing}{1}

\noindent \begin{minipage}{2.5cm}%
\includegraphics[scale=0.12]{logo-ifce}
\end{minipage}
\hspace{0.3cm}
\begin{minipage}{13cm}%
\centering INSTITUTO FEDERAL DE EDUCAÇÃO, CIÊNCIA E TECNOLOGIA DO CEARÁ- IFCE\\
CAMPUS JUAZEIRO DO NORTE\\
CURSO SUPERIOR EM AUTOMAÇÃO INDUSTRIAL\\
PROGRAMA DE UNIDADE DIDÁTICA – PUD\\
\end{minipage}%
\end{Spacing}

\begin{longtable}{|p{14cm}|}
%primeiro cabeçalho
\hline
\rowcolor{lightgray}
\multicolumn{1}{p{14cm}}{\textbf{Disciplina: \imprimirdisciplina}}\\
\hline
\endfirsthead

%cabeçalho
\hline
continuação PUD \imprimirdisciplina\\
\hline
\endhead

\hline
continua...\\
\hline
\endfoot

\hline
\rowcolor{lightgray}

\begin{tabular}{p{5.5 cm}| l}
coordenação & departamento pedagogico\\[16 ex]
\end{tabular}\\

\hline

\endlastfoot

%elementos
\textbf{Código:} \imprimircodigo\\


\textbf{Carga Horária } Teórica: \imprimircargaHorariaTeorica, Prática \imprimircargaHorariaPratica, Total: \imprimircargaHorariaTotal\\


\textbf{Número de créditos:} \imprimircreditos\\


\textbf{Código pré-requisitos:} \imprimircodigoPrerequisitos\\


\textbf{Semestre:} \imprimirsemestre\\


\textbf{Nível:} \imprimirnivel\\
\hline

\rowcolor{lightgray}
\multicolumn{1}{|p{14cm}|}{\textbf{Ementa}}\\
\hline
\multicolumn{1}{|p{14cm}|}{\imprimirementa}\\
\hline

\rowcolor{lightgray}
\multicolumn{1}{|p{14cm}|}{\textbf{Objetivo}}\\
\hline
\imprimirobjetivo\\
\hline

\rowcolor{lightgray}
\multicolumn{1}{|p{14cm}|}{\textbf{Programa}}\\
\hline
\imprimirprograma\\
\hline

\rowcolor{lightgray}
\multicolumn{1}{|p{14cm}|}{\textbf{Metodologia de ensino}}\\
\hline
\imprimirmetodologiaEnsino\\
\hline

\rowcolor{lightgray}
\multicolumn{1}{|p{14cm}|}{\textbf{Recursos}}\\
\hline
\imprimirrecursos\\
\hline

\rowcolor{lightgray}
\multicolumn{1}{|p{14cm}|}{\textbf{Avaliação}}\\
\hline
\imprimiravaliacao\\
\hline

\rowcolor{lightgray}
\multicolumn{1}{|p{14cm}|}{\textbf{Bibliografia básica}}\\
\hline
\imprimirbibliografiaBasica\\
\hline

\rowcolor{lightgray}
\multicolumn{1}{|p{14cm}|}{\textbf{Bibliografia complementar}}\\
\hline
\imprimirbibliografiaComplementar\\
\hline
\end{longtable}
\pagebreak
}
\begin{document}

\disciplina{Controle da produção}
\codigo{AUT2442}
\cargaHorariaTotal{40}
\cargaHorariaPratica{0}
\cargaHorariaTeorica{40}
\creditos{2}
\codigoPrerequisitos{-}
\semestre{7º}
\nivel{Superior}

\ementa{
Sistemas de Produção e trabalho. Sistemas de produção em massa. Sistema de produção Flexíveis. Capacidade de Produção e produtividade. Sistemas de controle da produção. Gestão de Processos. PERT/CPM.
Novas formas de Organização da produção e a Intensificação tecnológica. Inteligencia Artificial na produção. Aprendizagem de máquinas para o trabalho.
}

\objetivo{
• Apresentar os principais métodos e sistemas de produção;\\
• Introduzir os conceitos básicos de gerenciamento de projetos.\\
}

\programa{
• Introdução aos sistemas de produção;\\
• Sistemas de Produção em massa, produção flexível e novas formas de fabricação;\\
• Capacidade de produção e medição do trabalho;\\
• Fluxogramas e gestão de Processos;\\
• Processos de padronização do trabalho;\\
• Diagramas de Rede e gestão de projetos de trabalho;\\
• PERT/CPM\\
• Ferrametnas de Gestão de Projetos e sistemas produtivos;\\
• Ferramentas de qualidade;\\
• Organização da produção e a Intensificação tecnológica.\\
}

\metodologiaEnsino{
Discussão dialogada através de textos teóricos, estudos de casos. Mapeamento e elaboração de projetos de melhoria de produtividade com introdução às ferramentas de projetos. Realização de visitas técnicas para aproximação com a realidade prática dos conteúdos. Avaliação realizada através de provas, seminários e
exercícios práticos.
}

\recursos{
Material didático-pedagógico.\\
Recursos audiovisuais.\\
Transporte para visitas técnicas\\
}

\avaliacao{
Participação dos alunos nas atividades propostas; trabalhos individuais ou em grupo; estudos de caso sobre o conteúdo programático da disciplina em foco.
}

\bibliografiaBasica{
• MARTINS, Petrônio G.; LAUGENI, Fernando P. Administração da produção. São Paulo: Saraiva,2005.\\

• STEVENSON, William J. Administração das operações de produção. Rio de Janeiro: LTC, 2001 \\

• MOREIRA, D. Administração da produção e operações. Pioneira, 2004. · CORRÊA,
}

\bibliografiaComplementar{
• CONTADOR, J. C. (Coord.). Gestão de operações: a engenharia de produção a serviço da modernização da empresa. São Paulo: Edgard Blücher, 2004.\\

• HENRIQUE L.; GIANESI, IRINEU G. N; CAON, M. Planejamento, programação e controle da produção: MRP II/ERP: conceito, uso e implantação. São Paulo: Atlas, 2001.\\

• ROBSON SELEME E HUMBERTO STADLER. Controle da qualidade: as ferramentas essenciais. InterSaberes.	E-book.	(186p.).	ISBN	9788565704861.\\

• SELEME, Robson. Métodos e Tempos: racionalizando a produção de bens e serviços. InterSaberes. E-book.	(164p.).	ISBN	9788582122587.\\

• TUBINO, D. F. Manual de Planejamento e controle da produção. 2ª ed. São Paulo: 
Atlas, 2000.\\
}


\imprimirPUD

\end{document}