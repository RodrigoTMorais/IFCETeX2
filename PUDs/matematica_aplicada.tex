\input{preambulo}
%arquivo de template para os PUDS

%definição das variaveis das seções
\newcommand{\disciplina}{\def \disciplina}
\newcommand{\imprimirdisciplina}{\disciplina}

\newcommand{\codigo}{\def \codigo}
\newcommand{\imprimircodigo}{\codigo}

\newcommand{\cargaHorariaTotal}{\def \cargaHorariaTotal}
\newcommand{\imprimircargaHorariaTotal}{\cargaHorariaTotal}

\newcommand{\cargaHorariaPratica}{\def \cargaHorariaPratica}
\newcommand{\imprimircargaHorariaPratica}{\cargaHorariaPratica}

\newcommand{\cargaHorariaTeorica}{\def \cargaHorariaTeorica}
\newcommand{\imprimircargaHorariaTeorica}{\cargaHorariaTeorica}

\newcommand{\creditos}{\def \creditos}
\newcommand{\imprimircreditos}{\creditos}

\newcommand{\codigoPrerequisitos}{\def \codigoPrerequisitos}
\newcommand{\imprimircodigoPrerequisitos}{\codigoPrerequisitos}

\newcommand{\semestre}{\def \semestre}
\newcommand{\imprimirsemestre}{\semestre}

\newcommand{\nivel}{\def \nivel}
\newcommand{\imprimirnivel}{\nivel}

\newcommand{\codigoEquivalencias}{\def \codigoEquivalencias}
\newcommand{\imprimircodigoEquivalencias}{\codigoEquivalencias}

\newcommand{\ementa}{\def \ementa}
\newcommand{\imprimirementa}{\ementa}

\newcommand{\objetivo}{\def \objetivo}
\newcommand{\imprimirobjetivo}{\objetivo}

\newcommand{\programa}{\def \programa}
\newcommand{\imprimirprograma}{\programa}

\newcommand{\metodologiaEnsino}{\def \metodologiaEnsino}
\newcommand{\imprimirmetodologiaEnsino}{\metodologiaEnsino}

\newcommand{\recursos}{\def \recursos}
\newcommand{\imprimirrecursos}{\recursos}

\newcommand{\avaliacao}{\def \avaliacao}
\newcommand{\imprimiravaliacao}{\avaliacao}

\newcommand{\bibliografiaBasica}{\def \bibliografiaBasica}
\newcommand{\imprimirbibliografiaBasica}{\bibliografiaBasica}

\newcommand{\bibliografiaComplementar}{\def \bibliografiaComplementar}
\newcommand{\imprimirbibliografiaComplementar}{\bibliografiaComplementar}

\newcommand{\versao}{\def \versao}
\newcommand{\imprimirversao}{\versao}


%comando de impressão da estrutura
\newcommand{\imprimirPUD}{
%Cabeçalho do PUD
\begin{Spacing}{1}

\noindent \begin{minipage}{2.5cm}%
\includegraphics[scale=0.12]{logo-ifce}
\end{minipage}
\hspace{0.3cm}
\begin{minipage}{13cm}%
\centering INSTITUTO FEDERAL DE EDUCAÇÃO, CIÊNCIA E TECNOLOGIA DO CEARÁ- IFCE\\
CAMPUS JUAZEIRO DO NORTE\\
CURSO SUPERIOR EM AUTOMAÇÃO INDUSTRIAL\\
PROGRAMA DE UNIDADE DIDÁTICA – PUD\\
\end{minipage}%
\end{Spacing}

\begin{longtable}{|p{14cm}|}
%primeiro cabeçalho
\hline
\rowcolor{lightgray}
\multicolumn{1}{p{14cm}}{\textbf{Disciplina: \imprimirdisciplina}}\\
\hline
\endfirsthead

%cabeçalho
\hline
continuação PUD \imprimirdisciplina\\
\hline
\endhead

\hline
continua...\\
\hline
\endfoot

\hline
\rowcolor{lightgray}

\begin{tabular}{p{5.5 cm}| l}
coordenação & departamento pedagogico\\[16 ex]
\end{tabular}\\

\hline

\endlastfoot

%elementos
\textbf{Código:} \imprimircodigo\\


\textbf{Carga Horária } Teórica: \imprimircargaHorariaTeorica, Prática \imprimircargaHorariaPratica, Total: \imprimircargaHorariaTotal\\


\textbf{Número de créditos:} \imprimircreditos\\


\textbf{Código pré-requisitos:} \imprimircodigoPrerequisitos\\


\textbf{Semestre:} \imprimirsemestre\\


\textbf{Nível:} \imprimirnivel\\
\hline

\rowcolor{lightgray}
\multicolumn{1}{|p{14cm}|}{\textbf{Ementa}}\\
\hline
\multicolumn{1}{|p{14cm}|}{\imprimirementa}\\
\hline

\rowcolor{lightgray}
\multicolumn{1}{|p{14cm}|}{\textbf{Objetivo}}\\
\hline
\imprimirobjetivo\\
\hline

\rowcolor{lightgray}
\multicolumn{1}{|p{14cm}|}{\textbf{Programa}}\\
\hline
\imprimirprograma\\
\hline

\rowcolor{lightgray}
\multicolumn{1}{|p{14cm}|}{\textbf{Metodologia de ensino}}\\
\hline
\imprimirmetodologiaEnsino\\
\hline

\rowcolor{lightgray}
\multicolumn{1}{|p{14cm}|}{\textbf{Recursos}}\\
\hline
\imprimirrecursos\\
\hline

\rowcolor{lightgray}
\multicolumn{1}{|p{14cm}|}{\textbf{Avaliação}}\\
\hline
\imprimiravaliacao\\
\hline

\rowcolor{lightgray}
\multicolumn{1}{|p{14cm}|}{\textbf{Bibliografia básica}}\\
\hline
\imprimirbibliografiaBasica\\
\hline

\rowcolor{lightgray}
\multicolumn{1}{|p{14cm}|}{\textbf{Bibliografia complementar}}\\
\hline
\imprimirbibliografiaComplementar\\
\hline
\end{longtable}
\pagebreak
}
\begin{document}

\disciplina{Matemática aplicada}
\codigo{AUT2405}
\cargaHorariaTotal{80}
\cargaHorariaPratica{0}
\cargaHorariaTeorica{80}
\creditos{4}
\codigoPrerequisitos{-}
\semestre{1º}
\nivel{Superior}
\ementa{
Funções (afim, quadrática, exponencial, logarítmica, seno e cosseno). Números complexos. Limites.
}

\objetivo{
• Ler, identificar e utilizar dados matemáticos representados em tabelas, gráficos, diagramas e fórmulas.\\
• Utilizar as diferentes linguagens matemáticas (algébrica, geométrica, gráfica, ...) aplicando- as na resolução de problemas.\\
• Explicar oralmente ou por escrito os procedimentos utilizados na resolução de situações problemas.\\
• Aplicar os conhecimentos matemáticos no diagnóstico e equacionamento de questões cotidianas.\\
• Relacionar conhecimentos e métodos matemáticos em situações concretas, sobretudo a outras áreas de conhecimento.\\
}

\programa{
• Função Afim\\
• Definição de função e tipos de funções; Definição de função afim;\\
• Gráficos, raiz e estudo do sinal; Inequações: produto e quociente. Função quadrática\\
• Definição e gráficos;\\
• Raízes da função quadrática;\\
• Intersecção com os eixos (vertical e horizontal); Vértice da parábola;\\
• Máximos e mínimos da função quadrática; Estudo do sinal da função quadrática;\\ • Inequações: produto e quociente.\\
• Função exponencial Revisão de potenciação;\\
• Definição, gráficos e propriedades; Equação exponencial; Inequação exponencial. Função logarítmica\\
• Logaritmo: definição e propriedades; Definição da função logarítmica; Gráficos e propriedades da função logarítmica;Equação logarítmica; Inequação logarítmica.\\
• Função seno e função cosseno\\
• Definição: domínio, imagem, amplitude, frequência e período;\\
• Gráficos; Relações Trigonométricas. Números complexos\\
• Definição, número complexo real, imaginário e imaginário puro; Igualdade e conjugado de números complexos;\\
• Adição, subtração, multiplicação, divisão e potenciação de números complexos;\\ • Módulo e argumento de um número complexo;\\
• Forma trigonométrica de um número complexo;\\
• Multiplicação, divisão, potenciação e radiciação de números complexos na forma trigonométrica. Limites\\
• Definição, gráficos e propriedades; Continuidade de funções;\\
• Limites de funções descontínua no ponto a quando x tende a a; Limites de funções compostas; Limites e continuidades laterais;\\
• Limites envolvendo o infinito\\
}

\metodologiaEnsino{
Aulas expositivas.\\
Leitura e pesquisa.\\
Aulas práticas em laboratório de informática.\\
Resolução de exercícios utilizando software apropriado.\\
}

\recursos{
Livros contidos na bibliografia.\\
Quadro e pincel.\\
Data-show.\\
Lista de exercícios.\\
Laboratório.\\
Computadores.\\
}

\avaliacao{
Avaliação escrita.\\
Resolução individual ou em grupo de algoritmos no software apropriado.\\
Avaliação dos exercícios resolvidos.\\
Poderão ser inseridas outras avaliações durante o semestre.\\
}

\bibliografiaBasica{
• DEMANA, Franklin D. et al. Pré-cálculo 8. São Paulo: Pearson, 2009.
• DEMANA, Franklin D. Pré-cálculo7. São Paulo: Pearson, 2013.
• IEZZI, Gelson; MURAKAMI, Carlos; DOLCE, Osvaldo. Fundamentos de matemática elementar 2. São Paulo: Atual, 1993.\\
}

\bibliografiaComplementar{
• AVILA, Geraldo. Introdução ao cálculo. Rio de Janeiro: LTC, 1998.\\
• MEDEIROS, Veleiria Zuma (Cood). Pré-calculo. São Paulo: Cengage Leaening, 2010.\\
• PAIVA, Manoel Rodrigues. Matemática 3. São Paulo: Moderna, 2002.\\
• IEZZI, Gelson; MURAKAMI, Carlos; MACHADO, Nilson José. Fundamentos de matemática elementar 8. São Paulo: Atual, 1993.\\
• MUSATAFA, A. Munem; DAVID, J. Foulis. Cálculo 1. Rio de Janeiro: LTC, 1982.\\
}


\imprimirPUD

\end{document}