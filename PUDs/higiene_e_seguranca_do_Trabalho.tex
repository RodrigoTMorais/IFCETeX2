\input{preambulo}
%arquivo de template para os PUDS

%definição das variaveis das seções
\newcommand{\disciplina}{\def \disciplina}
\newcommand{\imprimirdisciplina}{\disciplina}

\newcommand{\codigo}{\def \codigo}
\newcommand{\imprimircodigo}{\codigo}

\newcommand{\cargaHorariaTotal}{\def \cargaHorariaTotal}
\newcommand{\imprimircargaHorariaTotal}{\cargaHorariaTotal}

\newcommand{\cargaHorariaPratica}{\def \cargaHorariaPratica}
\newcommand{\imprimircargaHorariaPratica}{\cargaHorariaPratica}

\newcommand{\cargaHorariaTeorica}{\def \cargaHorariaTeorica}
\newcommand{\imprimircargaHorariaTeorica}{\cargaHorariaTeorica}

\newcommand{\creditos}{\def \creditos}
\newcommand{\imprimircreditos}{\creditos}

\newcommand{\codigoPrerequisitos}{\def \codigoPrerequisitos}
\newcommand{\imprimircodigoPrerequisitos}{\codigoPrerequisitos}

\newcommand{\semestre}{\def \semestre}
\newcommand{\imprimirsemestre}{\semestre}

\newcommand{\nivel}{\def \nivel}
\newcommand{\imprimirnivel}{\nivel}

\newcommand{\codigoEquivalencias}{\def \codigoEquivalencias}
\newcommand{\imprimircodigoEquivalencias}{\codigoEquivalencias}

\newcommand{\ementa}{\def \ementa}
\newcommand{\imprimirementa}{\ementa}

\newcommand{\objetivo}{\def \objetivo}
\newcommand{\imprimirobjetivo}{\objetivo}

\newcommand{\programa}{\def \programa}
\newcommand{\imprimirprograma}{\programa}

\newcommand{\metodologiaEnsino}{\def \metodologiaEnsino}
\newcommand{\imprimirmetodologiaEnsino}{\metodologiaEnsino}

\newcommand{\recursos}{\def \recursos}
\newcommand{\imprimirrecursos}{\recursos}

\newcommand{\avaliacao}{\def \avaliacao}
\newcommand{\imprimiravaliacao}{\avaliacao}

\newcommand{\bibliografiaBasica}{\def \bibliografiaBasica}
\newcommand{\imprimirbibliografiaBasica}{\bibliografiaBasica}

\newcommand{\bibliografiaComplementar}{\def \bibliografiaComplementar}
\newcommand{\imprimirbibliografiaComplementar}{\bibliografiaComplementar}

\newcommand{\versao}{\def \versao}
\newcommand{\imprimirversao}{\versao}


%comando de impressão da estrutura
\newcommand{\imprimirPUD}{
%Cabeçalho do PUD
\begin{Spacing}{1}

\noindent \begin{minipage}{2.5cm}%
\includegraphics[scale=0.12]{logo-ifce}
\end{minipage}
\hspace{0.3cm}
\begin{minipage}{13cm}%
\centering INSTITUTO FEDERAL DE EDUCAÇÃO, CIÊNCIA E TECNOLOGIA DO CEARÁ- IFCE\\
CAMPUS JUAZEIRO DO NORTE\\
CURSO SUPERIOR EM AUTOMAÇÃO INDUSTRIAL\\
PROGRAMA DE UNIDADE DIDÁTICA – PUD\\
\end{minipage}%
\end{Spacing}

\begin{longtable}{|p{14cm}|}
%primeiro cabeçalho
\hline
\rowcolor{lightgray}
\multicolumn{1}{p{14cm}}{\textbf{Disciplina: \imprimirdisciplina}}\\
\hline
\endfirsthead

%cabeçalho
\hline
continuação PUD \imprimirdisciplina\\
\hline
\endhead

\hline
continua...\\
\hline
\endfoot

\hline
\rowcolor{lightgray}

\begin{tabular}{p{5.5 cm}| l}
coordenação & departamento pedagogico\\[16 ex]
\end{tabular}\\

\hline

\endlastfoot

%elementos
\textbf{Código:} \imprimircodigo\\


\textbf{Carga Horária } Teórica: \imprimircargaHorariaTeorica, Prática \imprimircargaHorariaPratica, Total: \imprimircargaHorariaTotal\\


\textbf{Número de créditos:} \imprimircreditos\\


\textbf{Código pré-requisitos:} \imprimircodigoPrerequisitos\\


\textbf{Semestre:} \imprimirsemestre\\


\textbf{Nível:} \imprimirnivel\\
\hline

\rowcolor{lightgray}
\multicolumn{1}{|p{14cm}|}{\textbf{Ementa}}\\
\hline
\multicolumn{1}{|p{14cm}|}{\imprimirementa}\\
\hline

\rowcolor{lightgray}
\multicolumn{1}{|p{14cm}|}{\textbf{Objetivo}}\\
\hline
\imprimirobjetivo\\
\hline

\rowcolor{lightgray}
\multicolumn{1}{|p{14cm}|}{\textbf{Programa}}\\
\hline
\imprimirprograma\\
\hline

\rowcolor{lightgray}
\multicolumn{1}{|p{14cm}|}{\textbf{Metodologia de ensino}}\\
\hline
\imprimirmetodologiaEnsino\\
\hline

\rowcolor{lightgray}
\multicolumn{1}{|p{14cm}|}{\textbf{Recursos}}\\
\hline
\imprimirrecursos\\
\hline

\rowcolor{lightgray}
\multicolumn{1}{|p{14cm}|}{\textbf{Avaliação}}\\
\hline
\imprimiravaliacao\\
\hline

\rowcolor{lightgray}
\multicolumn{1}{|p{14cm}|}{\textbf{Bibliografia básica}}\\
\hline
\imprimirbibliografiaBasica\\
\hline

\rowcolor{lightgray}
\multicolumn{1}{|p{14cm}|}{\textbf{Bibliografia complementar}}\\
\hline
\imprimirbibliografiaComplementar\\
\hline
\end{longtable}
\pagebreak
}
\begin{document}

\disciplina{Higiene e segurança do trabalho}
\codigo{AUT2415}
\cargaHorariaTotal{40}
\cargaHorariaPratica{10}
\cargaHorariaTeorica{30}
\creditos{2}
\codigoPrerequisitos{-}
\semestre{2º}
\nivel{Superior}

\ementa{
Definição de acidente de trabalho. Tipos de acidentes de trabalho. Causas de acidente de trabalho. Riscos de acidentes. EPI e EPC. NR 4, NR 5, NR 10, NR 23. Organização de programas e serviços de segurança e saúde ocupacional. Metodologia da ação prevencionista. Mapa de risco.
}

\objetivo{
• Identificar os tipos, causas e riscos de acidentes de trabalho.\\
• Analisar o funcionamento dos dispositivos de proteção de segurança coletiva e individual.\\
• Interpretar as NRs 4, 5, 6, 7, 9, 10, 12.\\
• Avaliar as condições de segurança e higiene de trabalho em ambientes industriais.\\
• Conhecer os procedimentos de primeiros socorros.\\
}

\programa{
• Conceito de acidente de trabalho segundo a CLT e pelo aspecto técnico. Reflexo do acidente de trabalho na empresa, sociedade e na família.\\
• Obrigações das empresas quanto a prevenção e responsabilidades. Tipos de acidentes de trabalho; Acidente típico, trajeto e doenças ocupacionais.\\
• Importância da classificação quanto as formas de prevenção Causas de acidente de trabalho;\\
• Condição insegura e atos inseguros Riscos de acidentes;\\
• Grupos de riscos físicos; Grupo de riscos químicos; Grupo de riscos biológicos;\\
• Grupo de riscos ergonômicos; Grupo de riscos mecânicos EPI e EPC; Uso e obrigações, tipos e classificações, medidas e cuidados. NR 4, NR 5, NR 10, NR 23;\\
• Entendimento das normas técnicas, aplicação no ambiente de trabalho e aspectos legais.\\
• Organização de programas e serviços de segurança e saúde ocupacional;\\
• Apresentação da NR 9 (PPRA) programa de prevenção de riscos ambientais.\\ 
• Metodologia da ação prevencionista;\\
• Linhas de defesa: 1o linha, 2o linha e 3o linha de controle e eliminação dos riscos ambientais. Mapa de risco;\\
• Elaboração, normatização e aplicação da técnica de rastreamento e identificação dos riscos ambientais\\
}

\metodologiaEnsino{
Aulas expositivas.\\
Resolução de lista de exercícios.\\
Visitas técnicas.\\
Leitura e pesquisa bibliográfica.\\
}

\recursos{
Data-show.\\
Computador.\\
Quadro Branco e Pincel.\\
Transporte.\\
Lista de exercícios.\\
}

\avaliacao{
Avaliação escrita.\\
Práticas individuais e em grupo.\\
Relatório de visita técnica.\\
Apresentação de Seminários.\\
Poderão ser inseridas outras avaliações durante o semestre.\\
}

\bibliografiaBasica{
• PEPPLOW, Luiz Amilton. Segurança do trabalho. Curitiba: Base Editorial, 2010.\\
• SEGURANÇA e medicina do trabalho: NR-1 a 33... Acompanhados de dispositivos da Constituição Federal e CLT, bem como...São Paulo: Saraiva, 2010. (Manuais de legislação Atlas).\\
• SEGURANÇA e medicina do trabalho. São Paulo: Atlas, 2004. (Manuais de legislação Atlas).\\
}

\bibliografiaComplementar{
• CAMPOS, Armando; TAVARES, José da Cunha. Prevenção e controle de risco em máquinas equipamentos e instalações. São Paulo: SENAC, 2007.\\
• ZOCCHIO, Álvaro. Prática da Prevenção de Acidentes: ABC da segurança do trabalho. São Paulo: Atlas, 2002.\\
• ZOCCHIO, Álvaro. Como entender e cumprir as obrigações pertinentes a segurança e saúde no trabalho. São Paulo: LTR, 2008.\\
• BRASIL, Governo Federal — Manual de orientação do eSocial, versão 2.4, de setembro de 2017.\\
• BRASIL, Governo Federal — Leiautes do eSocial. Versão 2.4.1, de dezembro de 2017.\\
• BRASIL, Governo Federal — Perguntas e Respostas do eSocial. Versão 2.0, agosto de 2014.\\
}


\imprimirPUD

\end{document}