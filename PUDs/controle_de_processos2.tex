\input{preambulo}
%arquivo de template para os PUDS

%definição das variaveis das seções
\newcommand{\disciplina}{\def \disciplina}
\newcommand{\imprimirdisciplina}{\disciplina}

\newcommand{\codigo}{\def \codigo}
\newcommand{\imprimircodigo}{\codigo}

\newcommand{\cargaHorariaTotal}{\def \cargaHorariaTotal}
\newcommand{\imprimircargaHorariaTotal}{\cargaHorariaTotal}

\newcommand{\cargaHorariaPratica}{\def \cargaHorariaPratica}
\newcommand{\imprimircargaHorariaPratica}{\cargaHorariaPratica}

\newcommand{\cargaHorariaTeorica}{\def \cargaHorariaTeorica}
\newcommand{\imprimircargaHorariaTeorica}{\cargaHorariaTeorica}

\newcommand{\creditos}{\def \creditos}
\newcommand{\imprimircreditos}{\creditos}

\newcommand{\codigoPrerequisitos}{\def \codigoPrerequisitos}
\newcommand{\imprimircodigoPrerequisitos}{\codigoPrerequisitos}

\newcommand{\semestre}{\def \semestre}
\newcommand{\imprimirsemestre}{\semestre}

\newcommand{\nivel}{\def \nivel}
\newcommand{\imprimirnivel}{\nivel}

\newcommand{\codigoEquivalencias}{\def \codigoEquivalencias}
\newcommand{\imprimircodigoEquivalencias}{\codigoEquivalencias}

\newcommand{\ementa}{\def \ementa}
\newcommand{\imprimirementa}{\ementa}

\newcommand{\objetivo}{\def \objetivo}
\newcommand{\imprimirobjetivo}{\objetivo}

\newcommand{\programa}{\def \programa}
\newcommand{\imprimirprograma}{\programa}

\newcommand{\metodologiaEnsino}{\def \metodologiaEnsino}
\newcommand{\imprimirmetodologiaEnsino}{\metodologiaEnsino}

\newcommand{\recursos}{\def \recursos}
\newcommand{\imprimirrecursos}{\recursos}

\newcommand{\avaliacao}{\def \avaliacao}
\newcommand{\imprimiravaliacao}{\avaliacao}

\newcommand{\bibliografiaBasica}{\def \bibliografiaBasica}
\newcommand{\imprimirbibliografiaBasica}{\bibliografiaBasica}

\newcommand{\bibliografiaComplementar}{\def \bibliografiaComplementar}
\newcommand{\imprimirbibliografiaComplementar}{\bibliografiaComplementar}

\newcommand{\versao}{\def \versao}
\newcommand{\imprimirversao}{\versao}


%comando de impressão da estrutura
\newcommand{\imprimirPUD}{
%Cabeçalho do PUD
\begin{Spacing}{1}

\noindent \begin{minipage}{2.5cm}%
\includegraphics[scale=0.12]{logo-ifce}
\end{minipage}
\hspace{0.3cm}
\begin{minipage}{13cm}%
\centering INSTITUTO FEDERAL DE EDUCAÇÃO, CIÊNCIA E TECNOLOGIA DO CEARÁ- IFCE\\
CAMPUS JUAZEIRO DO NORTE\\
CURSO SUPERIOR EM AUTOMAÇÃO INDUSTRIAL\\
PROGRAMA DE UNIDADE DIDÁTICA – PUD\\
\end{minipage}%
\end{Spacing}

\begin{longtable}{|p{14cm}|}
%primeiro cabeçalho
\hline
\rowcolor{lightgray}
\multicolumn{1}{p{14cm}}{\textbf{Disciplina: \imprimirdisciplina}}\\
\hline
\endfirsthead

%cabeçalho
\hline
continuação PUD \imprimirdisciplina\\
\hline
\endhead

\hline
continua...\\
\hline
\endfoot

\hline
\rowcolor{lightgray}

\begin{tabular}{p{5.5 cm}| l}
coordenação & departamento pedagogico\\[16 ex]
\end{tabular}\\

\hline

\endlastfoot

%elementos
\textbf{Código:} \imprimircodigo\\


\textbf{Carga Horária } Teórica: \imprimircargaHorariaTeorica, Prática \imprimircargaHorariaPratica, Total: \imprimircargaHorariaTotal\\


\textbf{Número de créditos:} \imprimircreditos\\


\textbf{Código pré-requisitos:} \imprimircodigoPrerequisitos\\


\textbf{Semestre:} \imprimirsemestre\\


\textbf{Nível:} \imprimirnivel\\
\hline

\rowcolor{lightgray}
\multicolumn{1}{|p{14cm}|}{\textbf{Ementa}}\\
\hline
\multicolumn{1}{|p{14cm}|}{\imprimirementa}\\
\hline

\rowcolor{lightgray}
\multicolumn{1}{|p{14cm}|}{\textbf{Objetivo}}\\
\hline
\imprimirobjetivo\\
\hline

\rowcolor{lightgray}
\multicolumn{1}{|p{14cm}|}{\textbf{Programa}}\\
\hline
\imprimirprograma\\
\hline

\rowcolor{lightgray}
\multicolumn{1}{|p{14cm}|}{\textbf{Metodologia de ensino}}\\
\hline
\imprimirmetodologiaEnsino\\
\hline

\rowcolor{lightgray}
\multicolumn{1}{|p{14cm}|}{\textbf{Recursos}}\\
\hline
\imprimirrecursos\\
\hline

\rowcolor{lightgray}
\multicolumn{1}{|p{14cm}|}{\textbf{Avaliação}}\\
\hline
\imprimiravaliacao\\
\hline

\rowcolor{lightgray}
\multicolumn{1}{|p{14cm}|}{\textbf{Bibliografia básica}}\\
\hline
\imprimirbibliografiaBasica\\
\hline

\rowcolor{lightgray}
\multicolumn{1}{|p{14cm}|}{\textbf{Bibliografia complementar}}\\
\hline
\imprimirbibliografiaComplementar\\
\hline
\end{longtable}
\pagebreak
}
\begin{document}

\disciplina{Controle de processos 2}
\codigo{AUT2432}
\cargaHorariaTotal{80}
\cargaHorariaPratica{40}
\cargaHorariaTeorica{40}
\creditos{4}
\codigoPrerequisitos{AUT2426}
\semestre{7º}
\nivel{Superior}

\ementa{
Representação de sistemas por diagramas de blocos, redução de digramas de blocos, Análise de resposta transitória e de regime estacionário, Controladores PID.
}

\objetivo{
• Representar sistemas por diagramas de blocos;\\
• Aplicar técnicas de redução de diagramas de blocos de sistemas físicos;\\ 
• Empregar gráficos de fluxos de sinais na análise de sistemas de controle;\\ 
• Analisar a resposta de sistemas no domínio do tempo;\\
• Determinar parâmetros de desempenho de sistemas de 1° e 2° ordem; \\
• Estudar a estabilidade de sistemas controlados;\\
• Projetar sistemas de controle com ações PID.\\
}

\programa{
• Diagrama de blocos Definição Componentes\\
• Diagrama de blocos de um sistema de malha fechada Função de transferência de malha aberta\\
• Função de transferência de alimentação direta Função de transferência de malha fechada Sistema de malha fechada sujeito a perturbação\\
• Procedimentos para construção de um diagrama de blocos Redução de diagrama de blocos\\
• Gráfico de fluxo de sinal Definição\\
• Componentes Propriedades\\
• Álgebra do gráfico de fluxo de sinal\\

• Representação de sistemas lineares pelo gráfico de fluxo de sinal Gráfico de fluxo de sinal para sistemas de controle\\
• Fórmula do ganho de Mason para gráficos de fluxo de sinal Análise de resposta transitória e de regime estacionário Resposta de sistemas de primeira ordem\\
• Resposta de sistemas de segunda ordem Estabilidade\\
• Critério de Estabilidade de Routh\\

• Erro estacionário em sistemas de controle com realimentação unitária Princípios básicos de projeto de Sistemas de Controle.\\
• Ações de controle básicas e controladores automáticos industriais Controladores ON-OFF, PD, PI e PID.\\
• Regras de sintonia de Ziegler-Nichols para controladores PID.\\
}

\metodologiaEnsino{
Aulas expositivas; \\
Lista de exercícios;\\
Simulação computacional utilizando software dedicado.\\
}

\recursos{
Livros contidos na bibliografia; \\
Quadro e pincel.\\
Data-show\\
}

\avaliacao{
Avaliação escrita;\\
Práticas individuais e em grupo no laboratório; \\
Listas de exercícios;\\
Poderão ser inseridas outras avaliações durante o semestre.\\
}

\bibliografiaBasica{
• DORF, Richard C.; BISCHOP, Robert H. Sistemas de controle modernos. Rio de Janeiro, LTC, 2018.\\

• NISE, Norman S. Engenharia de sistemas de controle. Rio de Janeiro: LTC. 2018.\\
• OGATA, Katsuhiko. Engenharia de controle moderno. Rio de Janeiro: Prentice Hall do Brasil, 2010.\\
}

\bibliografiaComplementar{
• SILVEIRA, Paulo R. da; SANTOS, Winderson E. Automação e controle discreto. São Paulo: Érica, 2007.\\

• MAYA, Paulo Álvaro; LEONARDI, Fabrizio. Controle essencial 23. 2 ed. São Paulo: Pearson, 2014.\\

• CARVALHO, J. L. Martins de.	Sistemas de controle automático.	Rio	de Janeiro: LTC, 2000.\\

• PHILLIPS, Charles L.; HARBOR, Royce D. Sistemas de controle e realimentação. São Paulo: Makron Books, 1996.\\

• SPIEGEL, Murray R. Transformadas de ornas: 263   problemas resolvidos,	614 problemas propostos. São Paulo: Makon Books, 1971.\\

• CRUZ, José Jaime da. Controle robusto multivariável. São Paulo: Editora da Universidade de São, 1996.\\
}


\imprimirPUD

\end{document}