\input{preambulo}
%arquivo de template para os PUDS

%definição das variaveis das seções
\newcommand{\disciplina}{\def \disciplina}
\newcommand{\imprimirdisciplina}{\disciplina}

\newcommand{\codigo}{\def \codigo}
\newcommand{\imprimircodigo}{\codigo}

\newcommand{\cargaHorariaTotal}{\def \cargaHorariaTotal}
\newcommand{\imprimircargaHorariaTotal}{\cargaHorariaTotal}

\newcommand{\cargaHorariaPratica}{\def \cargaHorariaPratica}
\newcommand{\imprimircargaHorariaPratica}{\cargaHorariaPratica}

\newcommand{\cargaHorariaTeorica}{\def \cargaHorariaTeorica}
\newcommand{\imprimircargaHorariaTeorica}{\cargaHorariaTeorica}

\newcommand{\creditos}{\def \creditos}
\newcommand{\imprimircreditos}{\creditos}

\newcommand{\codigoPrerequisitos}{\def \codigoPrerequisitos}
\newcommand{\imprimircodigoPrerequisitos}{\codigoPrerequisitos}

\newcommand{\semestre}{\def \semestre}
\newcommand{\imprimirsemestre}{\semestre}

\newcommand{\nivel}{\def \nivel}
\newcommand{\imprimirnivel}{\nivel}

\newcommand{\codigoEquivalencias}{\def \codigoEquivalencias}
\newcommand{\imprimircodigoEquivalencias}{\codigoEquivalencias}

\newcommand{\ementa}{\def \ementa}
\newcommand{\imprimirementa}{\ementa}

\newcommand{\objetivo}{\def \objetivo}
\newcommand{\imprimirobjetivo}{\objetivo}

\newcommand{\programa}{\def \programa}
\newcommand{\imprimirprograma}{\programa}

\newcommand{\metodologiaEnsino}{\def \metodologiaEnsino}
\newcommand{\imprimirmetodologiaEnsino}{\metodologiaEnsino}

\newcommand{\recursos}{\def \recursos}
\newcommand{\imprimirrecursos}{\recursos}

\newcommand{\avaliacao}{\def \avaliacao}
\newcommand{\imprimiravaliacao}{\avaliacao}

\newcommand{\bibliografiaBasica}{\def \bibliografiaBasica}
\newcommand{\imprimirbibliografiaBasica}{\bibliografiaBasica}

\newcommand{\bibliografiaComplementar}{\def \bibliografiaComplementar}
\newcommand{\imprimirbibliografiaComplementar}{\bibliografiaComplementar}

\newcommand{\versao}{\def \versao}
\newcommand{\imprimirversao}{\versao}


%comando de impressão da estrutura
\newcommand{\imprimirPUD}{
%Cabeçalho do PUD
\begin{Spacing}{1}

\noindent \begin{minipage}{2.5cm}%
\includegraphics[scale=0.12]{logo-ifce}
\end{minipage}
\hspace{0.3cm}
\begin{minipage}{13cm}%
\centering INSTITUTO FEDERAL DE EDUCAÇÃO, CIÊNCIA E TECNOLOGIA DO CEARÁ- IFCE\\
CAMPUS JUAZEIRO DO NORTE\\
CURSO SUPERIOR EM AUTOMAÇÃO INDUSTRIAL\\
PROGRAMA DE UNIDADE DIDÁTICA – PUD\\
\end{minipage}%
\end{Spacing}

\begin{longtable}{|p{14cm}|}
%primeiro cabeçalho
\hline
\rowcolor{lightgray}
\multicolumn{1}{p{14cm}}{\textbf{Disciplina: \imprimirdisciplina}}\\
\hline
\endfirsthead

%cabeçalho
\hline
continuação PUD \imprimirdisciplina\\
\hline
\endhead

\hline
continua...\\
\hline
\endfoot

\hline
\rowcolor{lightgray}

\begin{tabular}{p{5.5 cm}| l}
coordenação & departamento pedagogico\\[16 ex]
\end{tabular}\\

\hline

\endlastfoot

%elementos
\textbf{Código:} \imprimircodigo\\


\textbf{Carga Horária } Teórica: \imprimircargaHorariaTeorica, Prática \imprimircargaHorariaPratica, Total: \imprimircargaHorariaTotal\\


\textbf{Número de créditos:} \imprimircreditos\\


\textbf{Código pré-requisitos:} \imprimircodigoPrerequisitos\\


\textbf{Semestre:} \imprimirsemestre\\


\textbf{Nível:} \imprimirnivel\\
\hline

\rowcolor{lightgray}
\multicolumn{1}{|p{14cm}|}{\textbf{Ementa}}\\
\hline
\multicolumn{1}{|p{14cm}|}{\imprimirementa}\\
\hline

\rowcolor{lightgray}
\multicolumn{1}{|p{14cm}|}{\textbf{Objetivo}}\\
\hline
\imprimirobjetivo\\
\hline

\rowcolor{lightgray}
\multicolumn{1}{|p{14cm}|}{\textbf{Programa}}\\
\hline
\imprimirprograma\\
\hline

\rowcolor{lightgray}
\multicolumn{1}{|p{14cm}|}{\textbf{Metodologia de ensino}}\\
\hline
\imprimirmetodologiaEnsino\\
\hline

\rowcolor{lightgray}
\multicolumn{1}{|p{14cm}|}{\textbf{Recursos}}\\
\hline
\imprimirrecursos\\
\hline

\rowcolor{lightgray}
\multicolumn{1}{|p{14cm}|}{\textbf{Avaliação}}\\
\hline
\imprimiravaliacao\\
\hline

\rowcolor{lightgray}
\multicolumn{1}{|p{14cm}|}{\textbf{Bibliografia básica}}\\
\hline
\imprimirbibliografiaBasica\\
\hline

\rowcolor{lightgray}
\multicolumn{1}{|p{14cm}|}{\textbf{Bibliografia complementar}}\\
\hline
\imprimirbibliografiaComplementar\\
\hline
\end{longtable}
\pagebreak
}
\begin{document}

\disciplina{Estatística}
\codigo{AUT2415}
\cargaHorariaTotal{40}
\cargaHorariaPratica{0}
\cargaHorariaTeorica{40}
\creditos{2}
\codigoPrerequisitos{-}
\semestre{2º}
\nivel{Superior}

\ementa{
Amostragem, medidas de tendência central, medidas de dispersão, probabilidade e intervalos
de confiança.
}

\objetivo{
• Aprender diferentes formas de coleta e apresentação de dados.\\
• Conhecer algumas técnicas estatísticas para o uso na interpretação e análise de dados.\\
• Realizar aplicação prática da estatística no contexto do curso.\\
}

\programa{
• Métodos Estatísticos.\\
• Características: elementos de amostragem e estrutura de pesquisa.\\
• Revisão dos conceitos necessários para estudar estatística: razão, proporção,\\ porcentagem e critérios de arredondamento, somatório.\\
• Apresentação de dados: tabela de distribuição de frequência, gráfico de barras, colunas setor, histograma, polígono de frequência e ogiva.\\
• Medidas de tendência central: média, moda, mediana.\\
• Medidas de dispersão: variância, desvio padrão, coeficiente de variação,\\ critério de homogeneidade.\\
• Probabilidade.\\
• Distribuição Normal.\\
• Interpretação do desvio padrão – curva normal.\\
• Intervalo de confiança.\\
• Ao final do curso, os alunos deverão fazer uma pesquisa voltada para o controle de qualidade, apresentando dados e relatório de conclusão.\\
}

\metodologiaEnsino{
Aulas expositivas.\\
Realização de Seminários.\\
Leitura e pesquisa bibliográfica.\\
}

\recursos{
Livros contidos na bibliografia.\\
Quadro e pincel.\\
Data-show.\\
}

\avaliacao{
Realização de Provas e Trabalhos.\\
Avaliação de exercícios resolvidos.\\
}

\bibliografiaBasica{
• CRESPO, Antônio Arnot. Estatística fácil. São Paulo: Saraiva, 2002.\\
• FONSECA, Jairo Sinon da; MARTINS, Gilberto de Andrade. Curso de estatística. São Paulo: Atlas, 1996.\\
• LEVINE, David M. et. Al. Estatística: teoria e aplicações usando o 106orna106sti excel em português. Rio de Janeiro: LTC, 2005.\\
}

\bibliografiaComplementar{
• MUCELIN, C. A. Estatística. Curitiba: Editora do Livro Técnico, 2010.\\
• TOLEDO, Geraldo L.; OVALLE, I. I. Estatística básica. São Paulo: Atlas, 2008.\\
• WHITE, R. S.; WHITE, J. S. Estatística. Rio de Janeiro: LTC, 2005.\\
• HOEL, Paul G. Estatística elementar. São Paulo: Atlas, 1989.\\
• JOHN E. Freund. Estatística aplicada: Economia, Administração e Contabilidade. Porto Alegre: Bookman, 2006.\\
• MORETTIN, Luiz Gonzaga. Estatística básica: probabilidade e inferência. São Paulo: Pearson, 2010.\\
• TOLEDO, Geraldo L.; OVALLE, I. I. Estatística básica. São Paulo: Atlas, 2008.\\
}


\imprimirPUD

\end{document}