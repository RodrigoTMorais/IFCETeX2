\input{preambulo}
%arquivo de template para os PUDS

%definição das variaveis das seções
\newcommand{\disciplina}{\def \disciplina}
\newcommand{\imprimirdisciplina}{\disciplina}

\newcommand{\codigo}{\def \codigo}
\newcommand{\imprimircodigo}{\codigo}

\newcommand{\cargaHorariaTotal}{\def \cargaHorariaTotal}
\newcommand{\imprimircargaHorariaTotal}{\cargaHorariaTotal}

\newcommand{\cargaHorariaPratica}{\def \cargaHorariaPratica}
\newcommand{\imprimircargaHorariaPratica}{\cargaHorariaPratica}

\newcommand{\cargaHorariaTeorica}{\def \cargaHorariaTeorica}
\newcommand{\imprimircargaHorariaTeorica}{\cargaHorariaTeorica}

\newcommand{\creditos}{\def \creditos}
\newcommand{\imprimircreditos}{\creditos}

\newcommand{\codigoPrerequisitos}{\def \codigoPrerequisitos}
\newcommand{\imprimircodigoPrerequisitos}{\codigoPrerequisitos}

\newcommand{\semestre}{\def \semestre}
\newcommand{\imprimirsemestre}{\semestre}

\newcommand{\nivel}{\def \nivel}
\newcommand{\imprimirnivel}{\nivel}

\newcommand{\codigoEquivalencias}{\def \codigoEquivalencias}
\newcommand{\imprimircodigoEquivalencias}{\codigoEquivalencias}

\newcommand{\ementa}{\def \ementa}
\newcommand{\imprimirementa}{\ementa}

\newcommand{\objetivo}{\def \objetivo}
\newcommand{\imprimirobjetivo}{\objetivo}

\newcommand{\programa}{\def \programa}
\newcommand{\imprimirprograma}{\programa}

\newcommand{\metodologiaEnsino}{\def \metodologiaEnsino}
\newcommand{\imprimirmetodologiaEnsino}{\metodologiaEnsino}

\newcommand{\recursos}{\def \recursos}
\newcommand{\imprimirrecursos}{\recursos}

\newcommand{\avaliacao}{\def \avaliacao}
\newcommand{\imprimiravaliacao}{\avaliacao}

\newcommand{\bibliografiaBasica}{\def \bibliografiaBasica}
\newcommand{\imprimirbibliografiaBasica}{\bibliografiaBasica}

\newcommand{\bibliografiaComplementar}{\def \bibliografiaComplementar}
\newcommand{\imprimirbibliografiaComplementar}{\bibliografiaComplementar}

\newcommand{\versao}{\def \versao}
\newcommand{\imprimirversao}{\versao}


%comando de impressão da estrutura
\newcommand{\imprimirPUD}{
%Cabeçalho do PUD
\begin{Spacing}{1}

\noindent \begin{minipage}{2.5cm}%
\includegraphics[scale=0.12]{logo-ifce}
\end{minipage}
\hspace{0.3cm}
\begin{minipage}{13cm}%
\centering INSTITUTO FEDERAL DE EDUCAÇÃO, CIÊNCIA E TECNOLOGIA DO CEARÁ- IFCE\\
CAMPUS JUAZEIRO DO NORTE\\
CURSO SUPERIOR EM AUTOMAÇÃO INDUSTRIAL\\
PROGRAMA DE UNIDADE DIDÁTICA – PUD\\
\end{minipage}%
\end{Spacing}

\begin{longtable}{|p{14cm}|}
%primeiro cabeçalho
\hline
\rowcolor{lightgray}
\multicolumn{1}{p{14cm}}{\textbf{Disciplina: \imprimirdisciplina}}\\
\hline
\endfirsthead

%cabeçalho
\hline
continuação PUD \imprimirdisciplina\\
\hline
\endhead

\hline
continua...\\
\hline
\endfoot

\hline
\rowcolor{lightgray}

\begin{tabular}{p{5.5 cm}| l}
coordenação & departamento pedagogico\\[16 ex]
\end{tabular}\\

\hline

\endlastfoot

%elementos
\textbf{Código:} \imprimircodigo\\


\textbf{Carga Horária } Teórica: \imprimircargaHorariaTeorica, Prática \imprimircargaHorariaPratica, Total: \imprimircargaHorariaTotal\\


\textbf{Número de créditos:} \imprimircreditos\\


\textbf{Código pré-requisitos:} \imprimircodigoPrerequisitos\\


\textbf{Semestre:} \imprimirsemestre\\


\textbf{Nível:} \imprimirnivel\\
\hline

\rowcolor{lightgray}
\multicolumn{1}{|p{14cm}|}{\textbf{Ementa}}\\
\hline
\multicolumn{1}{|p{14cm}|}{\imprimirementa}\\
\hline

\rowcolor{lightgray}
\multicolumn{1}{|p{14cm}|}{\textbf{Objetivo}}\\
\hline
\imprimirobjetivo\\
\hline

\rowcolor{lightgray}
\multicolumn{1}{|p{14cm}|}{\textbf{Programa}}\\
\hline
\imprimirprograma\\
\hline

\rowcolor{lightgray}
\multicolumn{1}{|p{14cm}|}{\textbf{Metodologia de ensino}}\\
\hline
\imprimirmetodologiaEnsino\\
\hline

\rowcolor{lightgray}
\multicolumn{1}{|p{14cm}|}{\textbf{Recursos}}\\
\hline
\imprimirrecursos\\
\hline

\rowcolor{lightgray}
\multicolumn{1}{|p{14cm}|}{\textbf{Avaliação}}\\
\hline
\imprimiravaliacao\\
\hline

\rowcolor{lightgray}
\multicolumn{1}{|p{14cm}|}{\textbf{Bibliografia básica}}\\
\hline
\imprimirbibliografiaBasica\\
\hline

\rowcolor{lightgray}
\multicolumn{1}{|p{14cm}|}{\textbf{Bibliografia complementar}}\\
\hline
\imprimirbibliografiaComplementar\\
\hline
\end{longtable}
\pagebreak
}
\begin{document}

\disciplina{Linguagem de programação 1}
\codigo{AUT2410}
\cargaHorariaTotal{80}
\cargaHorariaPratica{60}
\cargaHorariaTeorica{20}
\creditos{4}
\codigoPrerequisitos{AUT2004}
\semestre{2º}
\nivel{Superior}

\ementa{
Programas em linguagem estruturada. Aplicar Estruturas de dados, decisão e
repetição em linguagem estruturada. Utilizar Técnicas de modularização, como funções e procedimentos para construção de programas. Aplicar Técnicas para a criação de novos tipos de dados.
}

\objetivo{
• Conhecer técnicas de programação em linguagem estruturada.\\
• Desenvolver programas em linguagem estruturada.\\
• Identificar técnicas de programação orientada a objetos.\\
}

\programa{
• Construção Switch case\\
• Estruturas aninhadas\\
• Estruturas de Repetição\\
• Laços de Repetição com teste no início ( While)\\
• Laços de Repetição com teste no final ( Do-While)\\
• Laços de Repetição com variável de controle(For)\\
• Laços Aninhados\\
• Modularização\\
• Funções\\
• Protótipo de funções\\
• Chamada por valor e por referência\\
• Tipos de funções\\
• Sobrecarga de funções\\
• Estrutura de Dados\\
• Vetores\\
• Matrizes\\
• Estruturas\\
• Programação Orientada a Objetos\\
• Objetos\\
• Classes\\
}

\metodologiaEnsino{
• Aulas expositivas.\\
• Aulas práticas em laboratório de informática.\\
• Resolução de exercícios utilizando software apropriado.\\
• Leitura e pesquisa.\\
}

\recursos{
• Utilização de Laboratório de Informática.\\
• Livros contidos na bibliografia.\\
• Quadro e pincel.\\
• Data-show.\\
• Lista de exercícios.\\
}

\avaliacao{
• Avaliação escrita.\\
• Resolução individual ou em grupo de programas no software apropriado.\\
• Avaliação de exercícios resolvidos.\\
• Poderão ser inseridas outras avaliações durante o semestre.\\
}

\bibliografiaBasica{
• ASCENCIO, Ana Fernanda Gomes; CAMPUS, Edilene Aparecida Veneruchi de. Fundamentos da Programação de Computadores algoritmos Pascal e C11. São Paulo: Pearson,2012.\\
• DEITEL, Harvey M. et al. C como programar12. São Paulo: Pearson, 2011.\\
• MIZRAHI, Victorine Viviane. Treinamento em Linguagem C++ . São Paulo: Makron Books, 2006. Módulo 1.\\

}

\bibliografiaComplementar{
• MIZRAHI, Victorine Viviane. Treinamento em Linguagem C++ .. São Paulo: Makron Books, 2006. Módulo 2.\\
• SCHILDT, Herbert; GUNTLE, Greg. Borland C++ Builder: referência ornast. Rio de Janeiro: Campus, 2001.\\
• ECKEL, Bruce. C++, Guia do Usuário. Makron Books, 1991. \\
• ELLIS, Margaret A.; STROUSTRUP, Bjarne. C++, Manual de Referência Comentado. Editora Campus, 1993. \\
• GRAHAM, Neill. Learning C++. McGraw-Hill, 1991.\\
}


\imprimirPUD

\end{document}