\input{preambulo}

% Opções disponíveis

%\trabalhoacademico{tese}
%\trabalhoacademico{dissertacao}
%\trabalhoacademico{tccespecializacao}
%\trabalhoacademico{tccgraduacao}
\trabalhoacademico{ppc}

% Define se o trabalho é uma qualificação, coloque 'nao' para versão final do trabalho

\ehqualificacao{nao}

% Remove as bordas vermelhas e verdes do PDF gerado, coloque 'sim' para remover

\removerbordasdohyperlink{sim} 

% Adiciona a cor azul a todos os hyperlinks

\cordohyperlink{nao}

% Informação relacionadas ao trabalho

\autor{Nome Completo do Autor}
\titulo{Título do Trabalho: Subtítulo (se houver)}
\local{Juazeiro do Norte}
\data{2024}

% Informação sobre a IES

\ies{Instituto Federal de Educação, Ciência e Tecnologia do Ceará}
\iessigla{IFCE}
\centro{\textit{Campus} Juazeiro do Norte}

% Informação para Tese

\programadoutorado{Programa de Pós-Graduação em Saúde Coletiva}
\nomedodoutorado{Doutorado em Saúde Coletiva}
\doutorem{Saúde Coletiva}
\areadeconcentracaodoutorado{Saúde Coletiva}

% Informação para Dissertacao

\programamestrado{Programa de Pós-Graduação em Ciência da Computação}
\nomedomestrado{Mestrado Acadêmico em Ciência da Computação}
\mestreem{Ciência da Computação}
\areadeconcentracaomestrado{Ciência da Computação}

% Informação para TCC de Especialização

\especializacaoem{Ensino de Ciências da Natureza e Matemática}
\habilitacaoesp{Especialista}
\areadeconcentracaoespecializacao{Ensino de Matemática}

% Informação para TCC de Graduação

\graduacaoem{Licenciatura em Matemática} 
\habilitacao{Licenciado} % Licenciado ou Bacharel
\areadeconcentracaograduacao{Matemática}

% Data de Aprovação

\dataaprovacao{\underline{\hspace*{1cm}}/\underline{\hspace*{1cm}}/\underline{\hspace*{1.2cm}}.}

% Abreviaturas dos títulos acadêmicos

% Especialista: Esp.
% Mestre: Me.
% Mestra: Ma.
% Doutor: Dr.
% Doutora: Dra.

% Informação sobre o Orientador

\orientador{Prof. Dr. XXXXX}
\orientadories{Instituto Federal de Educação, Ciência e Tecnologia do Ceará}
\orientadoriessigla{IFCE}
\orientadorcentro{\textit{Campus} Caucaia}
\orientadorfeminino{nao} % Coloque 'sim' se for do sexo feminino

% Informação sobre o Coorientador

\coorientador{Profa. Ma. XXXXX} % Deixe o nome do coorientador em branco para remover do documento
\coorientadories{Instituto Federal de Educação, Ciência e Tecnologia do Ceará}
\coorientadoriessigla{IFCE}
\coorientadorcentro{\textit{Campus} Caucaia}
\coorientadorfeminino{sim} % Coloque 'sim' se for do sexo feminino

% Informação sobre a banca

% Membro da Banca Dois

\membrodabancadois{Prof. Dra. XXXXX}
\membrodabancadoisies{Instituto Federal de Educação, Ciência e Tecnologia do Ceará}
\membrodabancadoisiessigla{IFCE}
\membrodabancadoiscentro{\textit{Campus} Caucaia}

% Membro da Banca Três

\membrodabancatres{Prof. Dra. XXXXX}
\membrodabancatresies{Instituto Federal de Educação, Ciência e Tecnologia do Ceará}
\membrodabancatresiessigla{IFCE}
\membrodabancatrescentro{\textit{Campus} Caucaia}

% Informação Complementar sobre a banca

\membrodabancaquatro{Membro da Banca Quatro}
\membrodabancaquatrocentro{Centro de Ciências e Tecnologia - CCT}
\membrodabancaquatroies{Universidade do Membro da Banca Quatro - SIGLA}
\membrodabancacinco{Membro da Banca Cinco}
\membrodabancacincocentro{Teste}
\membrodabancacincoies{Universidade do Membro da Banca Cinco - SIGLA}
\membrodabancaseis{Membro da Banca Seis}
\membrodabancaseiscentro{}
\membrodabancaseisies{Universidade do Membro da Banca Seis - SIGLA}

% Informações sobre PPC
\curso{Tecnologia em automação Industrial}
\reitor{José Wally Mendonça Menezes}
\proreitorensino{Cristiane Borges Braga}
\proreitorpesquisa{Joélia Marques de Carvalho}
\proreitorextensao{Ana Cláudia Uchôa Araújo}
\proreitorgestaopessoas{Marcel Ribeiro Mendonça}
\proreitoradministracao{Reuber Saraiva de Santiago}
\diretorcampus{Alex Jussileno Viana Bezerra}
\diretorensino{Maria Regilene Gonçalves de Alcântara}
\coordenadorpedagogico{Luiza Maria Vieira de Lima}
\chefedepartamentopesquisa{Carlos Régis Torquato Rocha}
\chefedepartamentoextensao{Narcélio Pinheiro Victor}
\coordenadorbiblioteca{João Paulo Correia Ferreira}
\coordenadordecurso{Rodrigo Tavares de Morais}

\nucleodocenteestruturante{Rodrigo Tavares de Morais\\
Régia Talina Silva Araújo\\
Alexandre Magno Fereira Diniz\\
Flavio César Brito Nunes\\
Manuel Edervaldo Souto Araújo\\}

\colegiado{Rodrigo Tavares de Morais – Presidente\\
Josemeire Medeiros Silveira de melo - Pedagoga\\
Derig Almeida Vidal – Representante Docente\\
Flávio Cesar Brito Nunes – Representante Docente\\
Fábio Lavor Bezerra – Representante Docente\\
Ágio Gonçalves de Moraes Felipe - Representante docente\\
Francisco Erlânio Teles Pereira – Representante Discente\\
Italo Jean Silva Sousa - Representante discente}


\begin{document}	

% Elementos pré-textuais

\imprimircapa
\imprimirfolhaderostoppc
%\imprimirfolhaderosto{}
%\imprimirfichacatalografica{elementos-pre-textuais/ficha-catalografica}
%\imprimirerrata{elementos-pre-textuais/errata}
%\imprimirfolhadeaprovacao
%\imprimirdedicatoria{elementos-pre-textuais/dedicatoria}
%\imprimiragradecimentos{elementos-pre-textuais/agradecimentos}
%\imprimirepigrafe{elementos-pre-textuais/epigrafe}
%\imprimirresumo{elementos-pre-textuais/resumo}
%\imprimirabstract{elementos-pre-textuais/abstract}
%\imprimirlistadefiguras
%\imprimirlistadetabelas
%\imprimirlistadequadros
%\imprimirlistadealgoritmos
%\imprimirlistadecodigosfonte
%\imprimirlistadesiglas{elementos-pre-textuais/lista-de-siglas}
%\imprimirlistadesimbolos{elementos-pre-textuais/lista-de-simbolos}
\imprimirsumario
	
% Elementos textuais

\textual
\justifying
\textbf{Dados do curso}\\
\vspace{1cm}\\
Identificação da instituição de ensino\\

%identificação da instituição
\begin{tabular}{|p{5cm}|p{1.5cm}|p{7cm}|}
\hline 
\multicolumn{3}{|p{14cm}|}{Nome: Instituto Federal de Educação, Ciência e Tecnologia do Ceará - campus Juazeiro do Norte} \\ 
\hline 
\multicolumn{3}{|p{14cm}|}{CNPJ/MF: 10.744.098/0005-79} \\ 
\hline 
\multicolumn{3}{|p{14cm}|}{Endereço: Av. Plácido Aderaldo Castelo, 1646 – Bairro Planalto, cidade
Juazeiro do Norte – CE, CEP. 63.040-540.} \\ 
\hline 
Cidade: Juazeiro do Norte & UF:
CE & Fone: (88) 2101-5300 \\ 
\hline 
\multicolumn{2}{|p{7cm}|}{E-mail: gabinetejn@ifce.edu.br} & Página institucional da internet:
http://ifce.edu.br/juazeirodonorte \\ 
\hline 
\end{tabular} \\

\vspace{1cm}
Informações Gerais do Curso

\begin{tabular}{|p{7cm}|p{7cm}|}
\hline 
Denominação & Curso Superior de
Tecnologia em Automação
Industrial \\ 
\hline 
Titulação/certificação & Tecnólogo em Automação
Industrial \\ 
\hline 
Nível & Superior \\ 
\hline 
Modalidade & Presencial (até 20\% EAD) \\ 
\hline 
Duração & Mínimo: 07 (sete) semestres \newline
Máximo: 11 (onze) semestres \\ 
\hline 
Periodicidade & Semestral \\ 
\hline 
Forma de Ingresso & SISU/Transferência/Diplomados \\ 
\hline 
Número de vagas Anuais & 80 vagas \\ 
\hline 
Turno de funcionamento & Matutino/Noturno \\ 
\hline 
Ano e semestre do início de funcionamento & 2000.1 \\ 
\hline 
Carga horária dos
componentes curriculares
(Disciplinas) & 2760 h/a de
disciplinas
obrigatórias 
\newline
40 h/a de disciplinas opcionais \\ 
\hline 
Carga Horária do Estágio & não obrigatório \\ 
\hline 
Carga horária do Trabalho de Conclusão do Curso (inclusa como disciplina obrigatória)& 40 h/a \\ 
\hline
Carga horária de extensão& 280h\\
\hline 
Carga horária Total & 3080 h/a \\ 
\hline 
Sistema de Carga Horária & 01 crédito = 20h/a \\ 
\hline 
Duração da hora-aula diurna & 60 minutos \\ 
\hline 
Duração da hora-aula noturna & 50 minutos \\
\hline 
\end{tabular}

\pagebreak

\chapter{Apresentação}
%\DoubleSpacing
\OnehalfSpacing
%\SingleSpacing

O presente documento trata da alteração e atualização do Projeto Pedagógico do Curso Superior de Tecnologia em Automação Industrial, demanda ocorrida a partir de 2022, para adequação de normas de adição da extenção a grade curricular, bem como ajustes de disciplinas e adição de disciplinas optativas para atender demandas das indústrias locais.\\

O curso Superior de Tecnologia em Automação Industrial, objeto deste Projeto Pedagógico, vinculado ao Eixo de Controle e Processos Industriais, conforme Catálogo Nacional de Cursos Tecnológicos do MEC, e ofertado pelo Instituto Federal de Educação, Ciência e Tecnologia do Ceará, campus Juazeiro do Norte foi criado com base em um estudo sistemático das  potencialidades da microrregião do Cariri, no qual foi identificado a área da indústria como uma potência mercadológica regional.\\

Apoiados na constatação, os docentes do Curso Técnico em eletrônica, que naquele período, era ofertado pelo campus Juazeiro do Norte, realizaram várias discussões para a construção do projeto do Curso superior de Tecnologia em Automática (nome anterior do curso). A partir daí constituiu-se uma comissão interna para a elaboração do documento.\\

Para a elaboração do referido projeto, observou-se as diretrizes curriculares nacionais para os Cursos Superiores de Tecnologia, tais como: Decreto no 5.154/2004, Parecer CNE/CES no 436/2001, Parecer CNE/CP no 29/2002, Resolução CNE/CP no 3/2002, Parecer CNE/CES no 277/2006, Parecer CNE/CES no 19/2008, e a Lei de Diretrizes e Bases da Educação Nacional – LDB 9.394/96.\\

O Curso Superior de Tecnologia em Automática teve o projeto aprovado pelo Conselho Diretor do CEFETCE através da portaria no 436/GDG de 10 de dezembro de 1999,constante nos anexos deste PPC. O curso Iniciou com carga horária de 2.490h para disciplinas e 400h para estágio supervisionado, totalizando 2.890h e a primeira turma ingressou no semestre letivo de 2000.1.\\

O reconhecimento do curso superior de Tecnologia em Automática deu-se através da Portaria no 161, de 24 de novembro de 2006 (constante nos anexos), e nos termos da portaria citada, o curso passou a se chamar Curso Superior de Tecnologia em Automação Industrial.\\

Dessa forma, o curso superior de Tecnologia em Automação Industrial foi estruturado com uma matriz curricular que contempla uma base sólida de conhecimentos científicos e tecnológicos, com carga horária de 2800 horas/aula para disciplinas e 400h para estágio supervisionado, totalizando 3200 horas/aula, distribuídas ao longo de sete semestres. Este curso está em conformidade com as diretrizes curriculares nacionais segundo o parecer CNE/CES No 436/2001. Em anexo.\\

Entretanto Dado que a matriz curricular atual data de 2006, a evolução tecnológica demandou que ao longo do tempo os componentes curriculares tivessem seus conteúdos atualizados, porem, em idos de 2019, percebeu-se a necessidade de uma atualização mais profunda envolvendo a matriz curricular inteira. também mudanças na legislação demandaram mudanças significativas, de modo que o Núcleo Docente Estruturante (NDE), juntamente com os docentes do curso, iniciaram o trabalho de atualização deste projeto pedagógico de curso, no qual apresentamos neste documento.

\chapter{Contextualização da Instituição}
 
O Instituto Federal de Educação, Ciência e Tecnologia do Ceará (IFCE) é uma autarquia federal vinculada ao Ministério da Educação (MEC), gozando de autonomia pedagógica, administrativa e financeira.\\

O IFCE foi criado a partir da fusão entre o Centro Federal de Educação Tecnológica do Ceará (CEFET-CE) e as Escolas Agrotécnicas Federais (EAF) localizadas nas cidades de Crato e Iguatu, sendo regulamentado através da lei no 11.892/2008. O instituto tem como missão produzir, disseminar e aplicar conhecimentos técnicos, tecnológicos e acadêmicos visando à formação cidadã, por meio do Ensino, da Pesquisa e da Extensão, contribuindo para o progresso socioeconômico local, regional e nacional. Oferece cursos regulares de formação técnica, assim como, cursos superiores tecnológicos, licenciaturas, bacharelados e, ainda, pós-graduação (especialização e mestrado).\\

Atualmente a instituição dispõe de 35 campi localizados em diversos municípios do Ceará, caracterizando-se pela ampla capilaridade, com oferta de cursos sintonizados com as demandas regionais. Assim, a implantação do IFCE no interior do estado atende a meta do programa de expansão da rede federal de educação profissional e tecnológica e a própria natureza dos institutos federais de educação tecnológica, no que diz respeito à descentralização da oferta de qualificação profissional, levando em conta as necessidades socioeconômicas de cada região e ainda o propósito de evitar o êxodo de jovens estudantes para a capital.\\

O IFCE/Campus Juazeiro do Norte localiza-se na região do Cariri, sul do estado do Ceará. Foi inaugurado em dezembro de 1994 como Unidade de Ensino Descentralizada de Juazeiro do Norte (UNED) do Centro Federal de Educação Tecnológica do Ceará – CEFET CE, conforme Lei 8.948 de 08 de dezembro de 1994, tendo iniciado seu funcionamento, efetivamente, em setembro de 1995, com a oferta de cursos técnicos de nível médio. Atualmente, o IFCE/Campus de Juazeiro do Norte possui cinco cursos de graduação (Licenciatura em Matemática, Educação Física - ABI, Tecnologia em Automação Industrial, Bacharelado em Engenharia Civil e Bacharelado em Engenharia Ambiental e Sanitária), quatro cursos técnicos integrados (Técnico em Edificações, Técnico em Eletrotécnica, Técnico em Brinquedoteca e Técnico em Controle Ambiental), dois cursos Técnicos Subsequentes (Técnico em Geoprocessamento e Técnico em Sistemas de Energia Renovável) e um curso técnico integrado ao ensino médio na modalidade de Educação de Jovens e Adultos (Técnico em Mecânica Industrial Integrado ao Ensino Médio) que, com esta nova proposta curricular, para implementação em 2024.1 receberá a denominação de Técnico em Mecânica Integrado ao Ensino Médio. O campus oferta ainda, um curso de Licenciatura em Matemática na modalidade de Ensino à Distância (EaD), realizado através do Programa da Universidade Aberta do Brasil (UAB) e financiado pela Coordenação de Aperfeiçoamento de Pessoal Docente (CAPES), e dois cursos de Pós-Graduação Lato Sensu: Especialização em Ensino de Matemática com Ênfase na Formação de Professores da Educação Básica; e Especialização em Educação Física, Saúde e Lazer. A partir de 2024, o Campus Juazeiro do Norte, também estará implantando o curso de Pós-Graduação Stricto Sensu - Programa de Mestrado em Meio Ambiente.

\chapter{Justificativa para criação do Curso}

Profundas transformações no mercado de trabalho no século XXI têm operado mudanças significativas na produção e na prestação de serviços com aumento da necessidade de especialização e diversificação da força de trabalho. Uma das consequências deste momento histórico é a reestruturação do mercado e dos perfis profissionais, demandando cada vez mais investimento na formação e capacitação profissional de mão-de-obra qualificada.\\

Como parte do processo de globalização há a irreversível utilização, cada vez mais intensa, da tecnologia da informação, dos processos de automação do trabalho e melhoria da competitividade das organizações. Atualmente, o setor da indústria é influenciado por esses mecanismos, presentes tanto em empresas de pequeno porte, quanto em organizações de grande porte. A automação industrial  consiste na aplicação de técnicas, programas e/ou equipamentos específicos em uma determinada máquina ou processo industrial, objetivando o aumento de sua eficiência, maximização da produção com o menor consumo de insumos, diminuição da emissão de resíduos de qualquer espécie, melhores condições de segurança, seja material, humana ou das informações decorrentes dessa dinâmica, ou ainda, redução do esforço ou da interferência humana sobre esse trabalho ou máquina.\\

Esses avanços têm sido influenciados pelo desenvolvimento da robótica e da inteligência artificial e, acabam provocando uma nova configuração no mundo do trabalho. A automação industrial se apresenta não apenas como uma tendência, mas como um desafio para os próximos anos, haja visto o grande número a de trabalhadores atualmente alocadao em empregos de utilização do processo de automação.\\

O IPEA (2019) aponta que “em média, 54,45\% dos atuais empregos no Brasil correm risco elevado ou muito elevado de automação até 2046, um valor consistente com estudos similares que analisaram demais países da Europa, América do Norte e América Latina”. Diante deste cenário, o desafio é construir políticas públicas de educação que promovam oportunidades de práticas que preparem os aos trabalhadores para que atuem atuarem em várias atividades da área de Automação.\\

Para além disso, o cenário para o Brasil, não se mostra tão favorável. Pesquisas do Instituto de Estudos para o Desenvolvimento Industrial (IEDI) mostram que nos últimos anos o crescimento da produtividade na indústria brasileira foi de 0,8\%. No contexto de recesso econômico mundial enfrentado por diversos países que resultou em baixas de produtividade em economias como a China e os Estados Unidos, ainda é um cenário de difícil recuperação. Tendo em vista que a produtividade agregada dos Estados Unidos é cerca de 6 vezes maior que a do Brasil, o que evidencia a grande distância do Brasil em relação à fronteira tecnológica \cite{IEDI}.\\

No entanto, especialistas defendem que um dos fatores que poderiam elevar esse número no Brasil seria aumentar a capacidade inovadora dos processos produtivos industriais. Para tanto, faz-se necessário oferecer oportunidade de formação qualificada, para tornar a indústria nacional competitiva e de alta produtividade.\\

No contexto local, a região do Cariri no Ceará destaca-se como um importante polo industrial. Essa região é composta pelos municípios de Juazeiro do Norte, Crato, Barbalha, Caririaçu, Farias Brito, Jardim, Missão Velha, Nova Olinda e Santana do Cariri. Seus principais setores industriais se concentram na fabricação de calçados, bebidas não alcoólicas, produtos de limpeza, joias, medicamentos alopáticos, cimento, artefatos de cerâmica, extração e beneficiamento de gesso, argila, pedra e outros materiais para construção, cultivo de frutas e cana-de-açúcar. Segundo dados do Sindicato das Indústrias de Calçados e Vestuário de Juazeiro do Norte e região (SINDINDÚSTRIA). Só neste ramo de atividade, existem 81 empresas filiadas. Além desses importantes setores produtivos, merecem destaque também o comércio local e o forte turismo religioso no município de Juazeiro do Norte, relacionado principalmente ao Padre Cícero. O Sul do Cariri representa 7,89\% do PIB do Estado do Ceará (IPECE, 2017) tendo sido uma Região atrativa para novos investimentos em virtude da atratividade ocasionada pela consolidação como polo universitário cearense, abrigando mais de 14 IES no território. \\

Dessa forma, a proposta de um Curso Superior de Tecnologia em Automação Industrial surgiu com o objetivo de formar profissionais que possam atender à demanda gerencial e técnica identificada na região do Cariri, que tem a cidade de Juazeiro do Norte, como epicentro. Os conhecimentos da automação destacam-se principalmente nas áreas de eletricidade, mecânica, eletropneumática, eletrônica geral, eletrônica embarcada e informática e atendem a necessidade, cada vez mais presente, nas indústrias da região.\\

Desde 1995, o IFCE campus de Juazeiro do Norte tem colaborado para elevar o grau de aperfeiçoamento da mão-de-obra destinada à indústria e serviços da região do Cariri. Inicialmente, em nível técnico (com o curso Técnico em Eletrônica), logo se constatou a necessidade de formar profissionais com graduação superior que possuísse formação especializada em automação.\\

Três fatores foram determinantes para a proposição do curso: a demanda regional por profissionais qualificados para atuarem no setor industrial, a infraestrutura e a qualificação docente existente no IFCE campus de Juazeiro do Norte. Importante ressaltar que a universalização das ferramentas e plataformas nesta área, proporciona ao profissional uma grande mobilidade, não apenas para o mercado local, mas para o Brasil.

\chapter{Fundamentação Legal}

O Curso Superior de Tecnologia em Automação Industrial está legalmente embasado nas diretrizes educacionais referentes à Educação Profissional Tecnológica, conforme legislação abaixo relacionada:
\begin{itemize}
\item \textbf{Lei de Diretrizes e Bases da Educação Nacional}. LDB (Lei 9.394/96). Estabelece as Diretrizes e Bases da Educação Nacional;

\item \textbf{Lei no 11.741/2008}. Altera dispositivos da Lei no 9.394, de 20 de dezembro de 1996, que estabelece as diretrizes e bases da educação nacional, para redimensionar, institucionalizar e integrar as ações da educação profissional técnica de nível médio, da educação de jovens e adultos e da educação profissional e tecnológica.

\item \textbf{Resolução CNE/CES no 3, de 2 de julho de 2007}. Dispõe sobre procedimentos a serem adotados quanto ao conceito de hora-aula, e dá outras providências.

\item \textbf{Portaria MEC no 40, de 12 de dezembro de 2007}, reeditada em 29 de dezembro de 2011. Institui o e-MEC – sistema eletrônico de fluxo de trabalho e gerenciamento de informações relativas aos processos de regulação, avaliação e supervisão da educação superior no sistema federal de educação –, o Cadastro e-MEC de Instituições e Cursos Superiores e consolida disposições sobre indicadores de qualidade, banco de avaliadores (Basis) e o Exame Nacional de Desempenho de Estudantes (Enade), entre outras disposições.

\item \textbf{Lei no 10.861, de 14 de abril de 2004}. Institui o Sistema Nacional de Avaliação da Educação Superior (SINAES) e dá outras providências.

\item \textbf{Parecer CES no 277/2006}. Versa sobre nova forma de organização da Educação Profissional e Tecnológica de graduação.

\item \textbf{Resolução CNE/CP no 3/2002, de 18 de dezembro de 2002}. Institui as Diretrizes Curriculares Nacionais Gerais para a organização e o funcionamento dos cursos superiores de tecnologia.

\item \textbf{Parecer CNE/CES no 583, de 4 de abril de 2001}, que dispõe sobre a orientação para as Diretrizes Curriculares dos Cursos de Graduação.

\item \textbf{Decreto no 6.303, de 12 de dezembro de 2007}. Altera dispositivos dos Decretos no 5.622, de 19 de dezembro de 2005, que estabelece as diretrizes e bases da educação nacional, e no 5.773, de 9 de maio de 2006, que dispõe sobre o exercício das funções de regulação, supervisão e avaliação de instituições de educação superior e cursos superiores de graduação e sequenciais no sistema federal de ensino.

\item \textbf{Lei no 11.788, de 25 de setembro de 2008}. Dispõe sobre o estágio de estudantes; altera a redação do art. 428 da Consolidação das Leis do Trabalho – CLT, aprovada pelo Decreto-Lei no 5.452, de 1o de maio de 1943, e a Lei no 9.394, de 20 de dezembro de 1996; revoga as Leis no 6.494, de 7 de dezembro de 1977, e 8.859, de 23 de março de 1994, o parágrafo único do art. 82 da Lei no 9.394, de 20 de dezembro de 1996, e o art. 6o da Medida Provisória no 2.164-41, de 24 de agosto de 2001; e dá outras providências.

\item \textbf{Resolução no 2, de 4 de abril de 2005}. Modifica a redação do § 3o do artigo 5o da Resolução CNE/CEB no 1/2004, até nova manifestação sobre estágio supervisionado pelo Conselho Nacional de Educação.

\item \textbf{Parecer CNE/CEB no 40, de 08 de dezembro de 2004}. Trata das normas para execução de avaliação, reconhecimento e certificação de estudos previstos no Artigo 41 da Lei no 9.394/96 (LDB).

\item \textbf{Decreto no 5.154, de 23 de julho de 2004}. Regulamenta o § 2o do art. 36 e os arts. 39 a 41da Lei no 9.394/96.

\item \textbf{Parecer CNE/CES no 436, de 2 de abril de 2001}. Orienta sobre os Cursos Superiores de Tecnologia - Formação de Tecnólogo.

\item \textbf{Parecer CNE/CP no 29, de 3 de dezembro de 2002}. Institui as Diretrizes Curriculares Nacionais Gerais para a organização e o funcionamento dos cursos superiores de tecnologia.

\item \textbf{Parecer CNE/CES no 277, de 7 de dezembro de 2006}. Define nova forma de organização da Educação Profissional e Tecnológica de graduação.

\item \textbf{Parecer CNE/CES no 19, de 31 de janeiro de 2008}. Consulta sobre o aproveitamento de competência de que trata o art. 9o da Resolução CNE/CP no3/2002, que institui as Diretrizes Curriculares Nacionais Gerais para a organização e o funcionamento dos cursos superiores de tecnologia.

\item \textbf{Catálogo Nacional de Cursos Superiores de Tecnologia do MEC}. Manual que organiza e orienta a oferta de cursos superiores de tecnologia, inspirado nas diretrizes curriculares nacionais e em sintonia com a dinâmica do setor produtivo e as expectativas da sociedade.

\item \textbf{Resolução CNE/CP no 1, de 17 de junho de 2004}. Diretrizes Curriculares Nacionais para a Educação das Relações Étnico- Raciais e para o Ensino de História e Cultura Afro-Brasileira e Africana. Orienta ementas de disciplinas específicas, mas também uma compreensão curricular de valorização dos povos originários do Brasil, bem como do seu legado cultural presente em nossa vida e educação.

\item \textbf{Decreto no 6.872, de 4 de junho de 2009}. Aprova o Plano Nacional de Promoção da Igualdade Racial – PLANAPIR e institui o seu Comitê de Articulação e Monitoramento.

\item \textbf{Plano Nacional de Educação em Direitos Humanos (PNEDH)}. Constitui política pública para um projeto de sociedade baseado nos princípios da democracia, da cidadania e da justiça social, por meio de um instrumento de construção de uma cultura de direitos humanos, visando ao exercício da solidariedade e do respeito às diversidades.

\item \textbf{Decreto no 7.037, de 21 de dezembro de 2009}. Institui o Programa Nacional de Direitos Humanos.

\item \textbf{Resolução CNE/CP no1, de 30 de maio de 2012}. Diretrizes Nacionais para a Educação em Direitos Humanos. Estabelece fundamentos para a discussão das temáticas da inclusão, da tolerância e do direito como princípio educativo.

\item \textbf{Lei no 9.795, de 27 de abril de 1999}. Dispõe sobre a educação ambiental e institui a Política Nacional de Educação Ambiental.

\item \textbf{Resolução CNE/CP no 2, de 15 de junho de 2012}. Diretrizes Curriculares Nacionais para a Educação Ambiental. Apresenta as orientações sobre a Educação Ambiental, que perpassa diversas disciplinas como princípio curricular e forma de ser e estar no mundo.

\item \textbf{Decreto no 5.296, de 2 de dezembro de 2004}. Regulamenta as Leis no 10.048, de 8 de novembro de 2000, que dá prioridade de atendimento às pessoas com necessidades específicas, e no 10.098, de 19 de dezembro de 2000, que estabelece normas gerais e critérios básicos para a promoção da acessibilidade das pessoas portadoras de deficiência ou com mobilidade reduzida, e dá outras providências.

\item \textbf{Decreto no 5.626, de 22 de dezembro de 2005}. Regulamenta a Lei no 10.436, de 24 de abril de 2002, que dispõe sobre a Língua Brasileira de Sinais – Libras, e o art. 18 da Lei no 10.098, de 19 de dezembro de 2000.

\item \textbf{Decreto no 6.571, de 17 de setembro de 2008}. Dispõe sobre o atendimento
educacional especializado, regulamenta o parágrafo único do art. 60 da Lei no 9.394, de 20 de dezembro de 1996, e acrescenta dispositivo ao Decreto no 6.253, de 13 de novembro de 2007. (Revogado pelo Decreto no 7.611/ 2011, mas citado no Parecer CNE/CEB no 11/2012).

\item \textbf{Decreto no 6.949, de 25 de agosto de 2009}. Promulga a Convenção Internacional sobre os Direitos das Pessoas com Deficiência e seu Protocolo Facultativo, assinados em Nova York, em 30 de março de 2007.

\item \textbf{Decreto no 7.611, de 17 de novembro de 2011}. Dispõe sobre a educação especial, o atendimento educacional especializado e dá outras providências.

\end{itemize}

\chapter{Objetivos do Curso}

\section{Objetivo Geral}
Formar profissionais de nível superior na área de tecnologia em automação industrial para atuar a serviço da modernização das técnicas de produção utilizadas no setor industrial, atuando no planejamento, instalação e supervisão de sistemas de automação, com compromisso ético e responsabilidade social e ambiental.

\section{Objetivos específicos}

\begin{itemize}

\item preparar profissionais para atuarem na execução de processos do âmbito industrial, instalação e supervisão de sistemas de automação;

\item Formar profissionais com capacidade de pensar, planejar e agir na execução e manutenção dos sistemas automatizados;

\item Formar cidadãos com postura ética e responsabilidade social;

\item Ofertar um curso de pós-graduação na área do curso;

\item Capacitar e recapacitar pessoal docente, para manter o curso atualizado com as novas tecnologias do setor industrial;

\item Manter relações com as indústrias a fim de receber informações sobre necessidades da indústria, bem como encaminhar os discentes ao mercado de trabalho;

\item  Desenvolver Projetos Sociais fortalecendo a formação cidadã e a inclusão social e tecnológica;

\item Incentivar a formação inovadora e empreendedora;

\item Promover a produção, o desenvolvimento tecnológico e a transferência de tecnologias, observando os aspectos pertinentes à preservação do meio ambiente e ganho de produtividade;

\item Promover as atividades de pesquisa e iniciação científica na área de interesse do curso e/ou áreas equivalentes.

\end{itemize}

\chapter{Formas de Ingresso}

O ingresso no curso de Automação Industrial ocorre através do Processo Seletivo do Sistema de Seleção Unificada – SiSU ou através de edital específico para candidatos graduados e transferidos.

\chapter{Áreas de Atuação}

O tecnólogo em automação industrial projeta e gerencia a instalação e o uso de sistemas automatizados de controle e supervisão de processos industriais. Supervisiona a implantação e operação de redes industriais, sistemas supervisórios, controladores lógicos programáveis, sensores e atuadores presentes nos processos. Além disso, faz vistoria, realiza perícia, avalia, emite laudo e parecer técnico em sua área de formação.

\chapter{Perfil Esperado do Futuro Profissional}

O processo de formação do aluno está focado na preparação para o mercado de trabalho, na apropriação do saber tecnológico, na mobilização dos valores necessários à tomada de decisões com autonomia, na formação de uma postura empreendedora, sem abrir mão da cultura regional e dos valores sociais, de forma que os egressos atuem no mercado de trabalho como agentes de mudanças, contribuindo para o progresso social do país, em especial da Região Nordeste.\\

O aluno egresso do Curso Superior de Tecnologia em Automação Industrial é um profissional de nível superior que está a serviço da modernização das técnicas de produção utilizadas no setor industrial, atuando no planejamento, instalação e supervisão de sistemas de integração e automação. Deverá possuir um conjunto de características capazes de prover as habilidades e competências para cumprir suas atribuições básicas.\\

Dentro das atribuições, o tecnólogo em Automação Industrial estará apto a exercer as seguintes atividades:

\begin{itemize}
\item Desenvolver, implementar e integrar sistemas de automação industrial, integrando sensores, atuadores, dispositivos programáveis e sistemas de supervisão;

\item Coordenar, implementar e realizar manutenção em sistemas elétricos, eletrônicos, pneumáticos e hidráulicos;

\item Projetar, instalar e administrar redes Industriais;

\item Realizar ajuste e calibração de instrumentos e equipamentos utilizados nos sistemas industriais;

\item Programar controladores lógico-programáveis, microprocessadores,
 microcontroladores e demais dispositivos aplicados à automação industrial;
 
\item Projetar e implementar sistema de manufatura automatizada;

\item Implementar e realizar manutenção em sistemas eletrônicos analógicos e digitais industriais;

\item Operar máquinas, equipamentos e instrumentos comandados por sistemas convencionais ou automatizados;

\item coordenar implantação de sistemas automatizados;

\item liderar equipes de trabalho na área da automação Industrial;

\item Pesquisar novas tecnologias e aplicações na área de automação;

\item Treinar pessoal para trabalho em ambiente automatizado.

\end{itemize}

Para o exercício destas atividades, o egresso terá desenvolvido as seguintes competências e habilidades:

\begin{itemize}
\item Compreensão da necessidade de constante e contínuo aperfeiçoamento profissional;

\item Capacidade de empreender, colocando-se em condições de desenvolver seu próprio negócio ou participar da estruturação de micro e pequenas empresas.

\item Liderança;

\item Atuação participativa em equipes multidisciplinares;

\item Capacidade de aplicação de método científico para pesquisa e desenvolvimento de novas tecnologias;

\item Raciocínio Lógico crítico e analítico.

\end{itemize}

\chapter{Metodologia}

O curso de tecnologia em Automação Industrial utiliza, predominantemente, metodologia interacionista, porque defende a relação dialética entre teoria e prática, entendendo que são dimensões distintas e interdependentes. Assim, há valorização de diferentes áreas do conhecimento: técnicas, científicas, humanas e sociais. O desenvolvimento das práticas pedagógicas, previstas nos respectivos Planos de Unidade Didática (PUD) devem ser efetuadas através de atividades curriculares aliando ensino, pesquisa e extensão. Nessa perspectiva o processo de ensino e aprendizagem apresenta caráter inovador, visto que possibilitará a criação de tecnologia a utilização de recursos tecnológicos no desenvolvimento de  atividades, destacando-se: práticas laboratoriais, seminários, visitas técnicas, sistemas multimídias, estágios, projetos sociais, realização e participação em eventos científicos e culturais.\\

Neste processo educacional serão abordados, os conhecimentos referentes a Educação Ambiental, Direitos Humanos e relações Étnico-raciais, objetivando atender as determinações de legislação específica tais como Diretrizes Curriculares Nacionais para a Educação Ambiental ( Resolução CNE/CP No 2, de 15 de junho de 2012). Diretrizes Curriculares Nacionais para a Educação em Direitos Humanos (Resolução CNE/CP No 1, de 30 de maio de 2012), Diretrizes Curriculares Nacionais para a Educação das Relações Étnico-raciais e para o Ensino de História e Cultura Afro-Brasileira e Africana (Resolução CNE/CP No 1, de 17 de junho de 2004).\\

Nessa perspetiva, todas as disciplinas poderão abordar, de maneira transversal, os assuntos referentes a essas questões no decorrer do curso e de maneira específica essas temáticas serão trabalhadas nas disciplinas de  Projetos Sociais e Gestão Empresarial.\\

No intuito de promover o fortalecimento das ações de ensino e aprendizagem o curso incentiva a participação dos discentes em atividades de monitoria (voluntária e remunerada), projetos de iniciação científica, projetos de extensão e estágios supervisionados.\\

Em cumprimento ao Decreto no 5.626, de 22/12/2005 No âmbito da educação inclusiva, será ofertada a disciplina de Libras, como componente curricular optativo. No que tange ao processo de educação inclusiva, serão desenvolvidos projetos em parceria com o Núcleo de Apoio às Pessoas com Necessidades Educacionais Específicas (NAPNE), objetivando atender os alunos cuja condição requeira atendimento especializado. Para tanto, os docentes deverão realizar adaptação das atividades e conteúdos a serem desenvolvidos, com o apoio da equipe pedagógica e dos profissionais que compõem o NAPNE.\\

No curso de Tecnologia em Automação Industrial, diversos recursos tecnológicos são aplicados em conjunto com as diferentes disciplinas da matriz curricular de modo a produzir um clima propício ao desenvolvimento de projetos de aprendizagem através das Tecnologias de Informação e Comunicação – TIC’s. Dentre as TIC’s mais usadas destacam-se: softwares de simulação, sistemas de gerenciamento e supervisão, sistemas de aquisição de dados, sistema virtual de aprendizagem (AVA) baseado em ferramentas da WEB.

Coerente com o exposto, a estrutura curricular apresenta três quatro áreas específicas e interligadas, fundamentais para atingir os objetivos do curso: Formação Básica, Formação Profissionalizante, e Formação Específica e Núcleo de Disciplinas Optativas. O núcleo de Formação Básica diz respeito às disciplinas com conhecimentos necessários para embasar as de caráter profissionalizante e específico. A Formação Profissionalizante é constituída de disciplinas referentes aos fundamentos, aos sistemas e aos processos da especialização. A Formação Específica refere-se ao aprofundamento dos conhecimentos na área de automação industrial.\\

Coerente com o exposto, a estrutura curricular apresenta três quatro áreas específicas e interligadas, fundamentais para atingir os objetivos do curso: Formação Básica, Formação Profissionalizante, e Formação Específica e Núcleo de Disciplinas Optativas. O núcleo de Formação Básica diz respeito às disciplinas com conhecimentos necessários para embasar as de caráter profissionalizante e específico. A Formação Profissionalizante é constituída de disciplinas referentes aos fundamentos, aos sistemas e aos processos da especialização. A Formação Específica refere-se ao aprofundamento dos conhecimentos na área de automação industrial.\\

\chapter{Estrutura Curricular}

\section{Organização Curricular}

O Curso Superior de Tecnologia em Automação Industrial do Instituto Federal de Ciência e Tecnologia do Ceará – campus Juazeiro do Norte foi estruturado em sete semestres letivos com componentes curriculares e estágio supervisionado. Os componentes curriculares estão organizados em três quatro núcleos distintos e articulados: Formação Básica (Geral), Formação Profissionalizante, e Formação Específica e Núcleo de Disciplinas Optativas. Eles estão presentes nas diretrizes curriculares nacionais do nível tecnológico, para serem desenvolvidos de forma integrada no decorrer do curso.\\

No curso são ofertados nove componentes curriculares para o núcleo básico com uma carga horária de 520 horas. Os componentes curriculares desse núcleo são apresentados no Quadro \ref{qua:dis-nucleo-basico}.\\

\begin{quadro}[!h]	
\centering
\Caption{\label{qua:dis-nucleo-basico} Componentes curriculares do núcleo básico}		
\IFCEqua{}{
\begin{tabular}{|c|c|l|}
\hline
\multicolumn{3}{|c|}{ Disciplinas do núcleo de conteúdos básicos } \\
\hline
\textbf{DISCIPLINA} & \textbf{C.H.} & \textbf{Créditos} \\
\hline
Eletricidade I & 120 & 6 \\
\hline
Matemática Aplicada & 80 & 4 \\
\hline
Cálculo Aplicado & 80 & 4 \\
\hline
Física 1 & 80 & 4 \\
\hline
Estatistica & 40 & 2 \\
\hline
Física 2 & 40 & 2 \\
\hline
Metodologia Científica & 40 & 2 \\
\hline
Projetos Sociais & 40 & 2 \\
\hline
\textbf{Total} & \textbf{520} & \textbf{26} \\
\hline
\end{tabular}
}{
\Fonte{Elaborado pelo autor}
}
\end{quadro}

Para o núcleo profissionalizante, que tem por objetivo conferir conhecimento e habilidades referentes aos fundamentos, aos sistemas e aos processos da especialização, são ofertados dezenove componentes curriculares com carga horária de 1.320h/aulas. Os componentes curriculares desse núcleo são apresentados no quadro \ref{qua:dis-nucleo-profissionalizante}\\

\begin{quadro}[!h]	
\centering
\Caption{\label{qua:dis-nucleo-profissionalizante} Componentes curriculares do Núcleo Profissionalizante}		
\IFCEqua{}{
\begin{tabular}{|c|c|l|}
\hline
\multicolumn{3}{|c|}{ Disciplinas do núcleo de conteúdos Profissionalizantes } \\
\hline
\textbf{DISCIPLINA} & \textbf{C.H.} & \textbf{Créditos} \\
\hline
Eletrônica Digital 1 & 80 & 4 \\
\hline
Fundamentos da Programação & 80 & 4 \\
\hline
Desenho Assistido por Computador & 80 & 4 \\
\hline
Metrologia & 80 & 2 \\
\hline
Eletrônica Digital 2 & 40 & 2 \\
\hline
Linguagem de Programação 1 & 80 & 4 \\
\hline
Eletricidade 2 & 80 & 4 \\
\hline
Instrumentação Eletrônica & 40 & 2 \\
\hline
Higiene e segurança no trabalho & 40 & 2 \\
\hline
Eletrônica Geral  & 80 & 4\\
\hline
Microprocessadores 1 & 80 & 4\\
\hline
Projetos em Eletrônica & 80 & 4\\
\hline
Eletrônica de potência & 80 & 4\\
\hline
Comandos Elétricos & 80 & 2\\
\hline
Linguagem de Programação 2 & 80 & 2\\
\hline
Máquinas Elétricas & 80 & 4\\
\hline
Microprocessadores 2 & 80 & 2\\
\hline
Rede de Computadores & 40 & 2\\
\hline
Gestão Empresarial & 40 & 2\\
\hline
 
\textbf{Total} & \textbf{1320} & \textbf{66} \\
\hline
\end{tabular}
}{
\Fonte{Elaborado pelo autor}
}
\end{quadro}

Para o núcleo específico são ofertados treze componentes curriculares com carga horária de 840h/aulas. Os componentes curriculares desse núcleo são apresentados no Quadro \ref{qua:dis-nucleo-especifico}\\

\begin{quadro}[!h]	
\centering
\Caption{\label{qua:dis-nucleo-especifico} Componentes curriculares do Núcleo Específico}		
\IFCEqua{}{
\begin{tabular}{|c|c|l|}
\hline
\multicolumn{3}{|c|}{ Disciplinas do núcleo de conteúdos Específicos } \\
\hline
\textbf{DISCIPLINA} & \textbf{C.H.} & \textbf{Créditos} \\
\hline
 Elementos de máquinas & 40 & 2\\
\hline
 Controle de Processos 1 & 80 & 4\\
\hline
Acionamento Pneumático e Eletropneumático & 80 & 4\\
\hline
Processos de fabricação & 120 & 6\\
\hline
Acionamentos de Máquinas & 80 & 4\\
\hline
Controle de Processos 2 & 80 & 4\\
\hline
Acionamentos Hidráulicos e Eletro-hidráulico & 80 & 4\\
\hline
Instrumentação Industrial & 40 & 2\\
\hline
Redes Industriais & 80 & 4\\
\hline
Controlador Lógico Programável & 80 & 4\\
\hline
Engenharia Assistida por computador & 80 & 4\\
\hline
Trabalho de Conclusão de Curso & 40 & 2\\
\hline
Controle da Produção & 40 & 2\\
\hline
 
\textbf{Total} & \textbf{840} & \textbf{42} \\
\hline
\end{tabular}
}{
\Fonte{Elaborado pelo autor}
}
\end{quadro}

As disciplinas do núcleo profissionalizante e específico são realizadas mediante abordagem de conteúdos teóricos e práticos, procurando superar a dicotomia entre o pensar e o agir. Os exemplos, a seguir, ratificam essa afirmação. Na disciplina Gestão Empresarial os alunos desenvolvem exercícios práticos, elaborando planos de negócios para abertura de sua própria empresa e aprendem como assumir uma gerência, vivenciando situações reais do cotidiano. Na disciplina Projetos Sociais, os alunos desenvolvem em instituições, assim como em comunidades carentes, atividades que contribuem para melhoria da qualidade de vida e exercício da cidadania.\\

%Vale mencionar que, visando atender às demandas da comunidade em que o curso está inserido, houve necessidade de alterar a estrutura curricular do curso, incorporando novas disciplinas aos núcleos básico, específico e profissionalizante, a saber: Desenho Assistido por Computador, Metrologia, Estatística, Física I (Mecânica Clássica), Tecnologia Mecânica I, Tecnologia Mecânica II, Laboratório de Tecnologia Mecânica, Instrumentação Industrial, Controle da Produção e Engenharia assistida por Computador.\\

Cabe citar que além dos núcleos apresentados, há um outro Núcleo com componentes curriculares de caráter optativos demonstrados no quadro \ref{qua:dis-nucleo-optativo}, Essas disciplinas são adições de conhecimentos que poderão ser de grande valia para vida profissional do aluno. O curso deverá ofertar no mínimo uma das optativas por semestre mesmo que seja compartilhada com outros cursos do mesmo nível ofertado pelo campus, o horário reservado no 4 semestre tem por objetivo possibilitar aos alunos cursarem a disciplina em horário regular. A coordenação poderá, dependendo da disponibilidade de pessoal e demanda, ofertar mais de uma disciplina optativa por semestre.\\

\begin{quadro}[!h]	
\centering
\Caption{\label{qua:dis-nucleo-optativo} Disciplinas do Núcleo Optativo}		
\IFCEqua{}{
\begin{tabular}{|c|c|l|}
\hline
\multicolumn{3}{|c|}{ Disciplinas do núcleo Optativo } \\
\hline
\textbf{DISCIPLINA} & \textbf{C.H.} & \textbf{Créditos} \\
\hline
 Libras & 40 & 2\\
\hline
 Inglês Instrumenta & 40 & 2\\
\hline
 Espanhol Instrumental & 40 & 2\\
\hline
 Robótica Industrial & 40 & 2\\
\hline
 Álgebra Linear & 40 & 2\\
\hline
 Fundamentos de Energias Renováveis & 40 & 2\\
\hline
 
\textbf{Total} & \textbf{240} & \textbf{12} \\
\hline
\end{tabular}
}{
\Fonte{Elaborado pelo autor}
}
\end{quadro}

Objetivando assegurar atendimento e tratamento adequado aos portadores de deficiência auditiva e em consonância com a Lei N° 10.436/2002, o Curso Superior de Tecnologia em Automação Industrial oferta como componente curricular optativo no Semestre VII a disciplina de Libras com 40h/aula, conforme apresentada no Quadro \ref{qua:dis-nucleo-optativo}. As demais disciplinas constam neste núcleo em atendimento às solicitações dos discentes e demandas apresentadas pelos professores, decorrentes da necessidade de aprofundamento e atualização de conhecimentos e poderão ser ofertada para os alunos que cumpram os pre-requisitos de cada disciplina.\\

De acordo com a portaria MEC n°1134/2016, no seu artigo primeiro,  regularizando as atividade remotas para cursos presenciais, algumas disciplinas, a critério do docente com anuência da coordenação do curso, poderão se ofertadas apresentando parcialmente ou integralmente o seu conteúdo na modalidade a distância, desde que a carga horária em EaD não ultrapasse o limite máximo de 20\% da carga horária total do curso.\\

Para controle de registro das atividades de em EaD, o docente deverá entregar o planejamento das destas atividades por disciplina, no semestre anterior à sua oferta. O coordenador do curso, por sua vez, verificará a carga horária, registrará e encaminhará para ciência da direção de ensino.\\

\section{Matriz Curricular}

A matriz curricular proposta considera a inter-relação existente entre ensino, pesquisa e extensão, articulando as dimensões teórica e prática, de maneira dialética. Os componentes curriculares encontram-se distribuídos de maneira a possibilitar a interdisciplinaridade entre os conhecimentos de âmbito pessoal, profissional, empreendedorismo, educação ambiental, direitos humanos e relações étnico-raciais.

\begin{quadro}[!h]	
\centering
\Caption{\label{qua:matriz-sem1} Disciplinas 1º Semestre}		
\IFCEqua{}{
\begin{tabular}{|c|c|c|c|c|c|}
\hline
\multicolumn{6}{|c|}{ Disciplinas do 1º Semestre} \\
\hline
\textbf{Código} & \textbf{DISCIPLINA} & \textbf{C.H. teórica} & \textbf{C.H. prática} & Cred. & PR. \\
\hline
01 & Eletricidade 1 & 80 & 0 & 4 & -\\
\hline
03 & Eletrônica Digital 1& 60 & 20 & 4 & -\\
\hline
04 & Fundamentos da Programação & 30 & 50 & 4 & -\\
\hline
05 & Matemática Aplicada & 80 & 0 & 4 & -\\
\hline
35 & Metodologia Científica & 20 & 20 & 2 & -\\
\hline


\multicolumn{2}{|l|}{ \textbf{Subtotal}} & \multicolumn{2}{|c|}{\textbf{400}} &  \textbf{20} &  \\
\hline
\end{tabular}
}{
\Fonte{Elaborado pelo autor}
}
\end{quadro}

\begin{quadro}[!h]	
\centering
\Caption{\label{qua:matriz-sem2} Disciplinas 2º Semestre}		
\IFCEqua{}{
\begin{tabular}{|c|c|c|c|c|c|}
\hline
\multicolumn{6}{|c|}{ Disciplinas do 2º Semestre} \\
\hline
\textbf{Código} & \textbf{DISCIPLINA} & \textbf{C.H. teórica} & \textbf{C.H. prática} & Cred. & PR. \\

\hline
8 & Metrologia & 20 & 60 & 4 & -\\
\hline
9 & Eletrônica Digital 2 & 30 & 10 & 2 & 03\\
\hline
10 & Linguagem de Programação 1 & 20 & 60 & 4 & 04\\
\hline
12 & Instrumentação Eletrônica & 26 & 14 & 2 & 01\\
\hline
14 & Higiene e Segurança do Trabalho & 20 & 20 & 2 & -\\
\hline
15 & Estatística & 40 & 0 & 2 & -\\
\hline
33 & Rede de Computadores  & 20 & 20 & 2 & - \\
\hline
43 & Projetos Sociais & 30 & 10 & 2 & -\\
\hline

\multicolumn{2}{|l|}{ \textbf{Subtotal}} & \multicolumn{2}{|c|}{\textbf{400}} &  \textbf{20} &  \\
\hline
\end{tabular}
}{
\Fonte{Elaborado pelo autor}
}
\end{quadro}

\begin{quadro}[!h]	
\centering
\Caption{\label{qua:matriz-sem3} Disciplinas 3º Semestre}		
\IFCEqua{}{
\begin{tabular}{|c|c|c|c|c|c|}
\hline
\multicolumn{6}{|c|}{ Disciplinas do 3º Semestre} \\
\hline
\textbf{Código} & \textbf{DISCIPLINA} & \textbf{C.H. teórica} & \textbf{C.H. prática} & Cred. & PR. \\
\hline
06 & Desenho Assistido por Computador & 0 & 80 & 4 & -\\
\hline
07 & Cálculo Aplicado & 80 & 0 & 4 & 05\\
\hline
11 & Eletricidade 2 & 60 & 20 & 4 & 01,05\\
\hline
13 & Física 1 & 60 & 20 & 4 & 05\\
\hline
16 & Eletrônica Geral & 40 & 40 & 4 & 01\\
\hline


\multicolumn{2}{|l|}{ \textbf{Subtotal}} & \multicolumn{2}{|c|}{\textbf{400}} &  \textbf{20} &  \\
\hline
\end{tabular}
}{
\Fonte{Elaborado pelo autor}
}
\end{quadro}

\begin{quadro}[!h]	
\centering
\Caption{\label{qua:matriz-sem4} Disciplinas 4º Semestre}		
\IFCEqua{}{
\begin{tabular}{|c|c|c|c|c|c|}
\hline
\multicolumn{6}{|c|}{ Disciplinas do 4º Semestre} \\
\hline
\textbf{Código} & \textbf{DISCIPLINA} & \textbf{C.H. teórica} & \textbf{C.H. prática} & Cred. & PR. \\
\hline
17 & (optativa) & - & - & 2 & -\\
\hline
18 & Microprocessadores 1 & 40 & 40 & 4 & 09,10\\
\hline
19 & Projetos em Eletrônica  & 40 & 40 & 4 & 16\\
\hline
20 & Eletrônica de potência & 60 & 20 & 4 & 11,16\\
\hline
22 & Linguagem de Programação 2 & 40 & 40 & 4 & 10\\
\hline
24 & Física 2 & 40 & 0 & 2 & 13\\
\hline


\multicolumn{2}{|l|}{ \textbf{Subtotal}} & \multicolumn{2}{|c|}{\textbf{380}} &  \textbf{18} &  \\
\hline
\end{tabular}
}{
\Fonte{Elaborado pelo autor}
}
\end{quadro}

\begin{quadro}[!h]	
\centering
\Caption{\label{qua:matriz-sem5} Disciplinas 5º Semestre}		
\IFCEqua{}{
\begin{tabular}{|c|c|c|c|c|c|}
\hline
\multicolumn{6}{|c|}{ Disciplinas do 5º Semestre} \\
\hline
\textbf{Código} & \textbf{DISCIPLINA} & \textbf{C.H. teórica} & \textbf{C.H. prática} & Cred. & PR. \\
\hline
21 & Comandos Elétricos & 40 & 40 & 4 & 24\\
\hline
23 & Elementos de máquinas & 20 & 20 & 2 & 08\\
\hline
25 & Máquinas Elétricas & 60 & 20 & 4 & 24\\
\hline
26 & Controle de Processos 1 & 80 & 0 & 4 & 07\\
\hline
28 & Acionamento Pneumático e Eletropneumático & 40 & 40 & 4 & -\\
\hline
40 & Gestão Empresarial & 20 & 20 & 2 & -\\
\hline



\multicolumn{2}{|l|}{ \textbf{Subtotal}} & \multicolumn{2}{|c|}{\textbf{400}} &  \textbf{20} &  \\
\hline
\end{tabular}
}{
\Fonte{Elaborado pelo autor}
}
\end{quadro}

\begin{quadro}[!h]	
\centering
\Caption{\label{qua:matriz-sem6} Disciplinas 6º Semestre}		
\IFCEqua{}{
\begin{tabular}{|c|c|c|c|c|c|}
\hline
\multicolumn{6}{|c|}{ Disciplinas do 6º Semestre} \\
\hline
\textbf{Código} & \textbf{DISCIPLINA} & \textbf{C.H. teórica} & \textbf{C.H. prática} & Cred. & PR. \\
\hline
27 & Microprocessadores 2 & 40 & 40 & 4 & 18\\
\hline
29 & Processos de fabricação & 40 & 80  & 6 & 23\\
\hline
31 & Acionamentos de Máquinas & 60 & 20 & 4 & 25\\
\hline
34 & Acionamentos Eletro hidráulicos e Eletropneumáticos & 40 & 40 & 4 & 28\\
\hline
36 & Instrumentação Industrial & 20 & 20 & 2 & 07\\
\hline



\multicolumn{2}{|l|}{ \textbf{Subtotal}} & \multicolumn{2}{|c|}{\textbf{400}} &  \textbf{20} &  \\
\hline
\end{tabular}
}{
\Fonte{Elaborado pelo autor}
}
\end{quadro}


\begin{quadro}[!h]	
\centering
\Caption{\label{qua:matriz-sem7} Disciplinas 7º Semestre}		
\IFCEqua{}{
\begin{tabular}{|c|c|c|c|c|c|}
\hline
\multicolumn{6}{|c|}{ Disciplinas do 7º Semestre} \\
\hline
\textbf{Código} & \textbf{DISCIPLINA} & \textbf{C.H. teórica} & \textbf{C.H. prática} & Cred. & PR. \\
\hline
32 & Controle de Processos 2 & 40 & 40 & 4 & 26\\
\hline
37 & Redes Industriais & 40 & 40 & 4 & 36\\
\hline
38 & Controlador Lógico Programável & 20 & 60 & 4 & 21\\
\hline
39 & Engenharia Assistida por Computador & 75 & 5 & 4 & 06,23\\
\hline
41 & Trabalho de Conclusão de Curso & 40 & 0 & 2 & 35\\
\hline
42 & Controle da Produção & 30 & 10 & 2 & -\\
\hline


\multicolumn{2}{|l|}{ \textbf{Subtotal}} & \multicolumn{2}{|c|}{\textbf{400}} &  \textbf{20} &  \\
\hline
\end{tabular}
}{
\Fonte{Elaborado pelo autor}
}
\end{quadro}

\begin{quadro}[!h]	
\centering
\Caption{\label{qua:matriz-optativas} Disciplinas optativas}		
\IFCEqua{}{
\begin{tabular}{|c|c|c|c|c|c|}
\hline
\multicolumn{6}{|c|}{ Disciplinas Optativas} \\
\hline
\textbf{Código} & \textbf{DISCIPLINA} & \textbf{C.H. teórica} & \textbf{C.H. prática} & Cred. & PR. \\
\hline
44 & Libras & 0 & 40 & 2 & -\\
\hline
45 & Inglês Instrumental & 40 & 0 & 2 & -\\
\hline
46 & Espanhol Instrumental & 40 & 0 & 2 & -\\
\hline
47 & Robótica Industrial & 20 & 20 & 2 & -\\
\hline
48 & Álgebra Linear equações diferenciais & 40 & 0 & 2 & -\\
\hline
49 & Fundamentos de Energias Renováveis & 30 & 10 & 2 & -\\
\hline
50 & Projetos elétricos & 30 & 10 & 2 & -\\
\hline
51 & Domótica & 20 & 20 & 2 & -\\
\hline
52 & programação WEB & 30 & 10 & 2 & -\\
\hline


\end{tabular}
}{
\Fonte{Elaborado pelo autor}
}
\end{quadro}

\pagebreak

\begin{landscape}
\chapter{Fluxograma}
% Inserir fluxograma %%%%%%%%%
\usetikzlibrary{shapes.geometric, arrows}

\tikzstyle{bas} = [rectangle, rounded corners, minimum width=2.5cm, minimum height=1cm,text centered, text width = 3cm, draw=black, fill=green!10]

\tikzstyle{prof} = [rectangle, rounded corners, minimum width=2.5cm, minimum height=1cm,text centered, text width = 3cm, draw=black, fill=green!30]

\tikzstyle{esp} = [rectangle, rounded corners, minimum width=2.5cm, minimum height=1cm,text centered, text width = 3cm, draw=black, fill=green!50]

\tikzstyle{opta} = [rectangle, rounded corners, minimum width=2.5cm, minimum height=1cm,text centered, text width = 3cm, draw=black, fill=green!70]

\tikzstyle{texto} = [ text, minimum width=10cm, minimum height=1cm,text centered, text width = 3cm, draw=black, fill=white]



\tikzstyle{arrow} = [thick,->,>=stealth]

\begin{tikzpicture}[node distance = 1.7cm]
% 1 semestre %%%%%%%%%%%%
\draw node at (11,1.5) {\huge Fluxograma do curso};
\node (ele1) [bas] {\tiny 01-Eletricidade 1 \\ 120h};
\node (dig1) [prof, below of=ele1] {\tiny 03-Eletrônica Digital 1 \\ 80h};
\node (fundprog) [prof, below of=dig1] {\tiny 04-Fundamentos da programação \\ 80h};
\node (matap) [bas, below of=fundprog] {\tiny 05-Matemática aplicada \\ 80h};
\node (met) [bas, below of=matap] {\tiny 35-Metodologia da pesquisa Científica \\ 40h};

% 2 semestre %%%%%%%%%%%%%
\node (inst) [prof, right of=ele1, xshift=2cm] {\tiny 12-Instrumentação eletrônica \\ 40h (01)};
\node (dig2) [prof, below of= inst] {\tiny 09-Eletronica Digital 2 \\ 40h (03)};
\node (prog1) [prof, below of= dig2] {\tiny 10-Linguagem de programação 1 \\ 80h (04)};
\node (metr) [prof, below of= prog1] {\tiny 08-Metrologia \\ 80h};
\node (hst) [prof, below of= metr] {\tiny 14-Higiene e segurança no trabalho \\ 40h};
\node (est) [bas, below of= hst] {\tiny 15-Estatística \\ 40h};
\node (red) [prof, below of= est] {\tiny 33-Redes de computadores \\ 40h};
\node (pros) [bas, below of= red] {\tiny 43-Projetos sociais \\ 40h};

% 3 semestre %%%%%%%%%%%%%
\node (ele2) [prof, right of=inst, xshift=2cm] {\tiny 11-Eletricidade 2 \\ 80h (05,01)};
\node (calc) [bas, below of= ele2] {\tiny 07-Cálculo aplicado \\ 80h (05)};
\node (cad) [prof, below of = calc] {\tiny 06-Desenho assistido por computador \\ 80h};
\node (fis1) [bas, below of= cad] {\tiny 13-Física 1 \\ 80h (05)};
\node (eleg) [prof, below of= fis1] {\tiny 16-Eletronica Geral \\ 80h (01)};

% 4 semestre %%%%%%%%%%%%%
\node (opt) [opta, right of=ele2, xshift=2cm] {\tiny 17-Disciplina Optativa \\ 40h};
\node (micro1) [prof, below of= opt] {\tiny 18-Microprocrssadores 1 \\ 80h (09,10)};
\node (proe) [prof, below of= micro1] {\tiny 19-Projetos em eletrônica \\ 80h (16)};
\node (elind) [prof, below of= proe] {\tiny 20-Eletroônica de potência \\ 80h (16,11)};
\node (fis2) [bas, below of= elind] {\tiny 24-Física 2 \\ 40h (13)};
\node (prog2) [prof, below of= fis2] {\tiny 22-Linguagem de programação 2 \\ 80h (10)};

% 5 semestre %%%%%%%%%%%%%
\node (com) [prof, right of=opt, xshift=2cm] {\tiny 21-Comandos elétricos \\ 80h (01)};
\node (cont1) [esp, below of= com] {\tiny 26-Controle de processos 1 \\ 80h (07)};
\node (maqe) [prof, below of= cont1] {\tiny 25-Máquinas elétricas \\ 80h (24)};
\node (elem) [esp, below of= maqe] {\tiny 23-Elementos de máquinas \\ 40h (08)};
\node (gest) [prof, below of= elem] {\tiny 40-Gestão Empresarial \\ 40h};
\node (ac1) [esp, below of= gest, yshift=-0.3cm] {\tiny 28-Acionamento Pnemáticos e eletropneumáticos \\ 80h};


% 6 semestre %%%%%%%%%%%%%
\node (micro2) [prof, right of=com, xshift=2cm] {\tiny 27-Microprocessadores 2 \\ 80h (18)};
\node (prof) [esp, below of= micro2] {\tiny 29-Processos de fabricação \\ 80h (23)};
\node (acm) [esp, below of= prof, yshift=-0.2cm] {\tiny 31-Acionamento de máquinas \\ 80h (25)};
\node (ac2) [esp, below of= acm, yshift=-0.5cm] {\tiny 34-Acionamento hidraulicos e eletro hidraulicos \\ 80h (28)};
\node (Insti) [esp, below of= ac2, yshift=-0.5cm] {\tiny 36-Instrumentação Industrial \\ 40h (07)};

% 7 semestre %%%%%%%%%%%%%
\node (contp2) [esp, right of=micro2, xshift=2cm] {\tiny 32-Controle de processos 2 \\ 80h (26)};
\node (redi) [esp, below of= contp2] {\tiny 37-Redes Industriais \\ 80h (33)};
\node (clp) [esp, below of= redi, yshift=-0.2cm] {\tiny 38-Controladores lógicos programáveis \\ 80h (21)};
\node (cae) [esp, below of= clp, yshift=-0.3cm] {\tiny 39-Engenharia assistida por computador \\ 80h (06,23)};
\node (tcc) [esp, below of= cae, yshift=-0.4cm] {\tiny 41-Trabalho de conclusão de curso \\ 40h (35)};
\node (conpro) [esp, below of= tcc, yshift=-0.2cm] {\tiny 42-Controle da produção \\ 40h};

%legenda
\node (l1)[rectangle, rounded corners, minimum width=1cm, minimum height=1cm,text centered, draw=black, fill=green!10, right of =pros, xshift=2cm ]{};
\draw node at (l1)[xshift=2.3cm] {Disciplinas básicas};

\node (l2)[rectangle, rounded corners, minimum width=1cm, minimum height=1cm,text centered, draw=black, fill=green!30, right of =l1, xshift=3cm ]{};
\draw node at (l2)[xshift=3.3cm] {Disciplinas Profissionalizantes};

\node (l3)[rectangle, rounded corners, minimum width=1cm, minimum height=1cm,text centered, draw=black, fill=green!50, right of =l2, xshift=4.9cm ]{};
\draw node at (l3)[xshift=2.7cm] {Disciplinas Específicas};

% setas de pre requisito
\draw [arrow] (dig1) -- (dig2);
\draw [arrow] (fundprog) -- (prog1);
\draw [arrow] (ele1) -- (inst);
\draw [arrow] (matap) -| node[pos=0]{} ++(1.85,2.65) -| (calc);
\draw [arrow] (matap) -| node[pos=0]{} ++(1.85,4.25) -| (ele2);
\draw [arrow] (ele1) |- node[pos=0]{} ++(0,0.9) -| (ele2);
\draw [arrow] (matap) |- node[pos=0]{} ++(0,0.8) -| (fis1);
\draw [arrow] (ele1) |- node[pos=0]{} ++(0,0.9) -| node[pos=0]{} ++(5.5,0) |- (eleg);
\draw [arrow] (dig2) |- node[pos=0]{} ++(0,0.8) -| (micro1);
\draw [arrow] (prog1) |- node[pos=0]{} ++(0,0.9) -| (micro1);
\draw [arrow] (eleg) -| node[pos=0]{} ++(1.85,2.65) |- (proe);
\draw [arrow] (eleg) -| node[pos=0]{} ++(1.85,2.65) |- (elind);
\draw [arrow] (ele2) -| node[pos=0]{} ++(1.8,0) |- node[pos=0]{} ++(1.8,-4.3) -|(elind);
\draw [arrow] (fis1.south) |- node[pos=0]{} ++(0,-0.3) -| (fis2);
\draw [arrow] (prog1) -| node[pos=0]{} ++(1.75,0) |- (prog2);
\draw [arrow] (ele1) |- node[pos=0]{} ++(0,0.9) -| (com);
\draw [arrow] (fis2) -| node[pos=0]{} ++(1.85,2.65) |- (maqe);
\draw [arrow] (metr) |- node[pos=0]{} ++(0,0.75) -| (elem);
\draw [arrow] (calc) |- node[pos=0]{} ++(0,0.75) -| (cont1);
\draw [arrow] (micro1) -| node[pos=0]{} ++(1.8,0) |- node[pos=0]{} ++(1.8,0.85) -|(micro2);
\draw [arrow] (elem) -| node[pos=0]{} ++(1.85,2.60) |- (prof);
\draw [arrow] (maqe) -| node[pos=0]{} ++(1.9,0) |- (acm);
\draw [arrow] (ac1) -| node[pos=0]{} ++(1.9,0) |- (ac2);
\draw [arrow] (calc) -| node[pos=0]{} ++(1.9,0) -| node[pos=0]{} ++(0,0) |- node[pos=0]{} ++(0,-8.3) -|(Insti);
\draw [arrow] (cont1) -| node[pos=0]{} ++(1.7,0) |- node[pos=0]{} ++(1.7,0.80) -|(contp2);
\draw [arrow] (red) -| node[pos=0]{} ++(16.6,0) |- (redi);
\draw [arrow] (com) -| node[pos=0]{} ++(1.7,0) |- node[pos=0]{} ++(3.9,0.80) |-(clp);
\draw [arrow] (met) -| node[pos=0]{} ++(1.8,0) |- node[pos=0]{} ++(0,-1.2) -| node[pos=0]{} ++(0,-3) -- node[pos=0]{} ++(18.6,0) |-(tcc);


\end{tikzpicture}

%%%%%%%%%%%%%%%%%%%%%%%%%%%%%%

\end{landscape}

\chapter{Avaliação de Aprendizagem}

De acordo com artigo 90 do Regulamento de Organização Didática – ROD, \begin{quote}
O processo  de avaliação dá significado ao trabalho escolar e tem como objetivo acompanhar o desenvolvimento da aprendizagem do estudante nas suas diversas dimensões assegurando a progressão dos seus estudos, a fim de propiciar um diagnóstico do processo de ensino e aprendizagem que possibilite ao professor analisar sua prática e, ao estudante, desenvolver a autonomia no seu processo de aprendizagem para superar possíveis dificuldades.
\end{quote}

De acordo com artigo 91 do ROD, no IFCE a avaliação deve ter caráter diagnóstico, formativo, processual e contínuo, com a predominância dos aspectos qualitativos sobre os quantitativos e dos resultados parciais sobre os obtidos em provas finais.Portanto, o processo de avaliação do curso de Automação Industrial do IFCE, campus Juazeiro do Norte segue os princípios estabelecidos no ROD.\\

A avaliação do desempenho escolar é feita por disciplina, incidindo sobre a frequência e o aproveitamento.\\

Atendida a exigência do percentual mínimo de 75\% de frequência às aulas e demais atividades acadêmicas, prevista no art. 99 do ROD, será aprovado o aluno que obtiver nota de aproveitamento igual ou superior a 7,0 (sete), resultado da média ponderada das quatro avaliações parciais realizadas no semestre letivo, na forma do plano de ensino de cada disciplina, conforme mostrado nas Equações \eqref{eq1}.

\begin{equation}\label{eq1}
Mp=\dfrac{(N1 * 2)+(N2 * 3)}{5}\linebreak
\end{equation}
onde:\\
Mp - média parcial\\
N1 - Média da primeira etapa\\
N2 - Média da Segunda etapa\\

Caso o Aluno não atinja a média parcial superior a 7,0 , e tenha obtido média parcial superior a 3,0 , o aluno poderá fazer uma prova final na qual se aplicará a equação \eqref{eq2}. Neste caso será considerado aprovado caso consiga nota igual ou maior que 5,0.

\begin{equation}\label{eq2}
Mf=\dfrac{Mp + Af}{2}\\
\end{equation}  
onde:\\
Mf - Média final\\
Af - Nota da avalição final\\


É considerado reprovado na disciplina, o aluno que não obtiver a média mínima de aproveitamento semestral ou correspondente frequência mínima (75\%) do total de aulas e demais atividades programadas no semestre letivo.\\

O professor dispõe de autonomia para promover trabalhos de pesquisa e/ou de campo, exercícios e outras atividades em classe e extra classe, tais como apresentação de seminários; projetos interdisciplinares; resolução de situações- problema; provas objetivas e subjetivas que podem ser validados no processo somativo cumulativo e/ou qualitativo de notas ou conceitos das verificações parciais, nos limites definidos pela instituição.\\

A avaliação deverá ser feita de forma contínua e processual, com adoção de metodologias que estimulem a iniciativa, participação e interação dos alunos, prevalecendo os aspectos qualitativos, tendo como critérios a(o):

\begin{itemize}
\item Capacidade de síntese, de interpretação e de análise crítica.
\item Habilidade na leitura de códigos e linguagens.
\item Agilidade na tomada de decisões.
\item Postura cooperativa e ética.
\item Raciocínio lógico-matemático.
\item Raciocínio multi-relacional e interativo.
\end{itemize}

Ao final do processo de aprendizagem o professor avaliará se as competências e habilidades foram desenvolvidas pelo aluno em cada disciplina, correlacionando os critérios acima citados, com o sistema de registro do IFCE (notas).\\

\chapter{Atividade de extensão}
Este projeto propõe a integração das atividades de ensino, pesquisa e extensão com foco na interação dialógica da comunidade acadêmica com a sociedade, fortalecendo essa aproximação por meio de apropriação de saberes e novos conhecimentos inerentes ao campo de formação. Busca também atuar na solução de problemas emergentes de relevância social do entorno, considerando o interesse e necessidade da comunidade atendida, envolvendo ações de formação e difusão da informação, da ciência e tecnologia. Neste sentido, e em consonância com o art.8o da Resolução CNE/CES no. 7 de 18 de dezembro de 2018 (que estabelece as diretrizes para a Extensão na Educação Superior Brasileira) o Colegiado do Curso, a Coordenadoria do Curso e o Núcleo Docente Estruturante, em parceria com o Departamento de Extensão, Pesquisa, Pós-Graduação e Inovação (DEPPI) do campus Juazeiro do Norte promoverá projetos, cursos e oficinas, eventos e prestação de serviços à comunidade interna e externa possibilitando o desenvolvimento de atividades e ações empreendedoras e inovadoras, tendo como foco as vivências da aprendizagem para capacitação e para a inserção do público atendido ao mundo do trabalho. Essas atividades devem computar dez por cento da carga horária mínima exigida do curso, que atualmente é de 2800h. Assim, o estudante deve cumprir 280 horas comprovadas em atividades de extensão.\\

Dada as especificidades do curso de Automação Industrial, as atividades de extensão serão executadas nas modalidade III de acordo com a Resolução No 41, de 26 de maio de 2022.


%texto suprimido do original para evitar obrigação de realização da SETAI

%Semana de Tecnologia em Automação Industrial (SETAI). Esta semana ocorrerá anualmente e tem como objetivos principais: aproximar profissionais, pesquisadores e estudantes, a fim de proporcionar o debate acerca de temas relevantes na área de engenharia alinhados à automação industrial. Esse evento pretende também fazer a divulgação do curso de Tecnologia em Automação Industrial ofertado no campus Juazeiro do Norte, bem como apresentar produções científicas dos alunos. O evento é voltado para estudantes dos cursos técnicos, ensino médio e superior de toda região do Cariri, bem como cidades e estados circunvizinhos. Os estudantes deverão participar da semana desde o seu planejamento à execução da mesma, que poderá ocorrer mediante a organização da semana, apresentação de trabalhos, realização de minicurso, palestras e oficinas, dentre outras atividades fins. E poderá computar até 80h, subdivididas por atividades em: planejamento e organização da semana (até 40h); apresentação de minicurso (até 20h); palestras (até 4h) e realização de oficinas (até 20h).

Enquadrado na modalidade III, o curso de Tecnologia em Automação Industrial, desenvolverá:\\

\begin{itemize}
\item Cursos de Formação Inicial e Continuada, ofertados em modalidades presencial ou remota. Ministrado pelos alunos, orientados pelo professor da área. (carga horária até 40h por curso);
\item Administração e produção de conteúdo de canais digitais com acesso público, para divulgação técnico-científico dos trabalhos e projetos desenvolvidos por discentes e docentes no âmbito do curso (até 80h por semestre).
\item Criação de conteúdo digital com publicação em plataformas públicas de conteúdo técnico relativo ao conteúdo do curso, como demonstração técnica, apresentação de equipamentos e métodos, tutoriais e informativos (até 80h por semestre).
\item Outras linhas de extensão relacionadas ao perfil do curso, listadas pelo sistema de gestão de extensão do IFCE.
\end{itemize}

Os professores e a coordenação deverão estimular e apoiar a realização dos trabalhos de extensão dos alunos conscientizando-os da importância destas ações junto à comunidade, possibilitando a troca de saberes. Além disso, estas atividades são pré-requisito para sua colação de grau.\\

Este modelo de execução de extensão, permite que o aluno implemente as atividades de acordo com sua disponibilidade de tempo, sem comprometer o aprendizado das disciplinas de formação do curso, e ainda assim, dialogar com a comunidade em que vive, agindo como sujeito promotor de conhecimento e tecnologia.\\

\chapter{Estágio}

Conforme nova lei de Estágio supervisionado (lei n° 11.788, de 25/09/2008), o estágio visa o aprendizado de competências próprias da atividade profissional e à contextualização curricular, objetivando o desenvolvimento do aluno para a vida cidadã e para o trabalho, fazendo parte do projeto pedagógico do curso de forma não obrigatória.\\

Caso o estudante opte por realizar o Estágio Supervisionado, será orientado por um docente através de visitas regulares a empresa onde os alunos estão estagiando, e reuniões periódicas no próprio campus.\\

A entidade conveniada, tendo celebrado um Termo de Convênio ou Cadastro diretamente com esta instituição de ensino, ou indiretamente, através de agentes de integração Empresa-Escola, recebe o estagiário e desenvolve seu programa de estágio próprio. Serão aceitos programas com o mínimo de 160h de estágio.\\

O processo de realização do estágio deverá seguir o protocolo vigente Manual de Estágio do IFCE.\\

Após a conclusão do estágio, o aluno deverá confecionar um relatório de estágio submeter como material na disciplina TCC, o qual poderá ser validado como trabalho de conclusão realizado.\\

\chapter{ CRITÉRIOS DE APROVEITAMENTO DE CONHECIMENTOS E EXPERIÊNCIAS
ANTERIORES}

No que diz respeito ao aproveitamento de conhecimentos e experiências anteriores, os critérios deverão seguir o que está estabelecido no Regulamento de Organização Didática (ROD) vigente no IFCE.\\

\chapter{Trabalho de Conclusão de Curso}

O TCC trata da elaboração de um trabalho científico escrito e apresentado a uma banca examinadora, cujo projeto tenha sido executado pelo estudante sob a orientação e/ou supervisão de pelo menos um docente do quadro de professores do IFCE. Esta atividade é apresentada como uma disciplina onde o docente orientará os alunos para realização do trabalho. Vale ressaltar que a conclusão do TCC é item obrigatório para colação de grau.\\

O trabalho de conclusão de curso poderá ser apresentado nos formatos de monografia, relatório técnico, artigo científico, dentre outros gêneros acadêmicos que possam cumprir o objetivo de sistematização de conhecimentos obtidos ao longo do curso. A formatação dos trabalhos deverá estar de acordo com as orientações do Manual de Normalização de Trabalhos Acadêmicos do IFCE.\\

O TCC deverá ser elaborado individualmente e apresentado em defesa pública perante banca examinadora, com exceção de artigo publicado em periódico indexado na área de conhecimento do curso. Neste caso, o aluno não será obrigado a apresentar o trabalho diante da banca examinadora.\\

Caso o aluno tenha artigo aprovado em Congressos, Simpósios, Seminários, Encontros e similares que sejam devidamente qualificados pela Coordenação de Aperfeiçoamento de Pessoal de Nível Superior (CAPES), ou outros gêneros acadêmicos que apresentem sistematização de conhecimentos obtidos ao longo do curso, deverá anexar à documentação para efeito de aprovação na disciplina, o artigo trabalho aprovado e o comprovante de apresentação do artigo e enviar para o professor da disciplina de TCC.\\

É importante ressaltar que os artigos científicos apresentados em congressos estes só terão validade, para aprovação da disciplina de TCC, se tiverem sido publicados, no máximo um ano antes ou durante a referida disciplina. E para artigos submetidos a periódicos, o estudante terá a opção de apresentação do trabalho nos mesmos moldes no caso de monografia e relatório técnico. Além de comprovar a submissão do trabalho.\\

No caso, de monografia e relatório técnico, a banca examinadora deverá ser composta de no mínimo três avaliadores, a saber: um professor orientador (docente do IFCE), um professor (do curso de automação ou áreas afins) e um professor (interno ou externo à Instituição). Nesse caso, a nota da disciplina do TCC será a média das notas atribuída pelos três avaliadores. Será aprovado o aluno que obtiver nota de aproveitamento igual ou superior a 7,0 (sete).\\

O professor orientador terá como responsabilidade informar a nota do seu orientando para o professor da disciplina, bem como, uma cópia da ata de apresentação do trabalho. A defesa do trabalho deverá acontecer antes do término do semestre no qual o estudante está matriculado na disciplina de TCC.\\

O professor da disciplina de TCC poderá encaminhar os alunos para professores-orientadores, de acordo com a área de estudo de cada um. O número de discentes por professor-orientador será de, no máximo, 5 (cinco). Caso seja necessária a presença de um co-orientador, poderá ser convidado um profissional desta ou de outra instituição.\\


\chapter{EMISSÃO DE DIPLOMA}

Ao estudante que concluir, com êxito, todas as disciplinas da matriz curricular do curso, cumprir e comprovar a carga horária de atividades de extensão obrigatórias que correspondem ao total de 280 horas e cumprir com o Trabalho de conclusão de curso, todos com aproveitamento satisfatório, será conferido o Diploma de Tecnólogo em Automação Industrial.\\

\chapter{AVALIAÇÃO DO PROJETO DO CURSO}

O Instituto possui a Comissão Própria de Avaliação Institucional (CPA) prevista no art. 11 da Lei no. 10.861, de 14 de abril de 2004, regulada pela portaria no. 2.051, do Ministério da Educação – MEC. Esta comissão foi implantada no IFCE por instrução da Portaria número 228/GDG, de 21 de junho de 2004.\\
\section{CPA}
A CPA e as subcomissões têm como finalidade levar a efeito o processo de auto avaliação institucional do IFCE e seus campi, constituindo assim, um processo que se efetiva com a participação de todos os segmentos, com vistas a aprimorar o projeto institucional, a partir da reflexão sobre as práticas educativas que a instituição, por missão, vem desenvolvendo.\\

Os trabalhos de auto avaliação referentes a cada quadriênio pela subcomissão do campus Juazeiro do Norte são realizados respeitando as 3 (três) etapas: preparação, implementação e síntese. A preparação abrange as ações relacionadas à sensibilização da comunidade interna, por meio da divulgação de material informativo, de reuniões com os docentes, discentes e técnicos administrativos, da realização de seminários e jornadas, envolvendo as subcomissões, ocasiões em que são discutidas conjuntamente a proposta de avaliação.\\

A implementação compreende a definição de indicadores, a elaboração de instrumentais, a sistematização e a análise de dados obtidos em documentos e questionários, ações seguidas de apresentação e discussão de resultados com vários segmentos da instituição, com o objetivo precípuo de coligir dados necessários à elaboração dos relatórios parciais.\\

A síntese constitui a etapa de revisão do processo e ajustes: elaboração de relatório final e definição da forma de utilização dos resultados, divulgação do relatório conclusivo e envio do relatório aos órgãos competentes.\\

O IFCE utiliza, a cada final de período letivo, um instrumento – questionário – preenchido pelos alunos, em que atribuem valores de 1,0 (um) a 5,0 (cinco) em aspetos referentes a auto avaliação em cada disciplina (participação nas aulas, aproveitamento da disciplina, cumprimento ao horário das aulas e relação com os colegas) e avaliação dos professores (pontualidade, frequência, domínio de conteúdo, incentivo à participação do aluno, metodologia de ensino, relação professor e aluno, metodologia de ensino e de avaliação). Há ainda um espaço para que eles forneçam sugestões ou críticas visando à melhoria do trabalho educacional realizado.\\

Os integrantes da coordenação do curso de Automação Industrial conscientes da importância desta avaliação, realizam análise crítica dos dados, identificam aspectos positivos, problemas e buscam coletivamente soluções para as questões apresentadas.\\

\section{NDE}
O Núcleo Docente Estruturante (NDE) do curso de Tecnologia em Automação Industrial é composto por 5 professores pertencentes ao corpo docente do curso. Todos os membros possuem titulação acadêmica em programa de pós-graduação stricto sensu e regime de trabalho de tempo integral. O NDE estimula e desenvolve projetos de pesquisa e extensão, alinhadas com as áreas temáticas de atuação do curso e com as tendências do mercado de trabalho. O núcleo também contribui efetivamente com as reformulações do Projeto Pedagógico do Curso (PPC), visando consolidar o perfil profissional do egresso e assegurar a integração curricular interdisciplinar entre as diferentes atividades de ensino aplicadas no curso.\\

\section{Colegiado}
O colegiado do curso de Tecnologia em Automação Industrial é um órgão consultivo e deliberativo de planejamento acadêmico, para os assuntos de políticas de ensino, pesquisa e extensão em conformidade com as diretrizes da instituição para exercer as atribuições previstas no regulamento do colegiado de cursos do IFCE.\\

O colegiado é composto pelos seguintes membros: coordenador do curso; diretor de ensino; pedagogo; dois representantes docentes do núcleo básico; dois representantes docentes do núcleo específico e dois representantes do corpo discente.\\

Os representantes docentes têm mandatos de dois anos e são eleitos por seus pares, sendo permitida a reeleição. Os representantes discentes são eleitos por seus pares para mandato de um ano. O colegiado se reúne em sessão ordinária duas vezes por semestre. As reuniões extraordinárias podem ser convocadas pelo coordenador do curso ou por iniciativa ou requerimento de pelo menos 1/3 dos membros, com antecedência mínima de 48 horas mencionando o assunto a ser tratado. Após cada reunião é lavrada uma ata que é discutida e votada na reunião seguinte e, após aprovação, assinada pelos presentes. Os encaminhamentos e decisões ficam registrados em atas e os progressos das ações deliberadas são acompanhados nas reuniões ordinárias.\\

\chapter{POLÍTICAS INSTITUCIONAIS CONSTANTES DO PDI NO ÂMBITO DO CURSO}

O PDI (Plano de Desenvolvimento Institucional) é um planejamento realizado a cada 5 anos, que tem por objetivo traçar metas e definir estratégias para desenvolver a instituição. Tendo por objetivo primário, servir a sociedade através da educação e disseminação de saberes e competências.\\ 

No que se refere ao contexto inter-relacionado entre as políticas do PDI e o
curso de Tecnologia em Automação Industrial do IFCE campus Juazeiro do Norte, há uma ênfase ao destaca-se o compromisso do IFCE em cumprir o seu papel de produtor e disseminador do conhecimento, aprimorando continuamente as atividades do tripé ensino, pesquisa e extensão, por meio da oferta de uma infraestrutura adequada e de recursos humanos qualificados, fortalecendo, portanto, as ações desenvolvidas 
no curso.\\

\chapter{APOIO AO DISCENTE}

A política de assistência estudantil do IFCE (Resolução no 024/2015 do CONSUP) visa atender aos objetivos estabelecidos pelo Programa Nacional de Assistência Estudantil (Decreto no 7.234/2010) e, também, à redução das desigualdades sociais, e incentivar a participação da comunidade do IFCE em ações voltadas à sustentabilidade e à responsabilidade social, à ampliação das condições de participação democrática para formação e ao exercício de cidadania. Pretende-se ainda, a promoção do acesso universal à saúde, ancorado no princípio da integralidade, de modo a fortalecer a educação em saúde e a contribuição para a inserção do aluno no mundo do trabalho, enquanto ser social, político e técnico.\\

O público-alvo da Política de Assistência Estudantil são os estudantes que se encontram regularmente matriculados e, prioritariamente, em situação de vulnerabilidade.\\

O Departamento de Assuntos Estudantis (DAE) do Campus Juazeiro do Norte desenvolve um trabalho multidisciplinar através da prestação de serviços nas áreas de: serviço social; saúde; alimentação; psicologia; pedagogia; e execução de programas distribuídos por áreas temáticas:

\begin{enumerate}[label=(\roman*),ref=\roman*]
\item Trabalho, Educação e Cidadania: Programa de Incentivo à Participação Político-acadêmica; Programa de Orientação Profissional; Programa de Inclusão Social, Diversidade e Acessibilidade; e Programa de Promoção à Saúde Mental.
\item  Saúde: Programa de Assistência Integral à Saúde.
\item  Cultura, Arte, Desporto e Lazer: Programa de Incentivo à Arte e Cultura; e Programa de Incentivo ao Desporto e Lazer.
\item  Alimentação e Nutrição: Programa de Alimentação e Nutrição - Restaurante Acadêmico (RA) com oferta de lanches e refeição completa.
\item  Auxílios em Forma de Pecúnia: Moradia, Transporte, Óculos, Visitas e Viagens Técnicas, Acadêmico, Didático-pedagógico, Discentes Mães e Pais, Apoio a Desporto e Cultura, Formação e Pré-embarque internacional.

\end{enumerate}

Para o desenvolvimento e acompanhamento das atividades desses serviços e programas, o Campus Juazeiro do Norte conta com uma equipe formada por: 02 (dois) assistentes sociais, 01 (um) psicólogo, 01 (um) nutricionista, 01 (um) médico, 01 (um) enfermeiro, 01 (um) auxiliar em enfermagem, 02 (dois) odontólogos e 02 (dois) assistentes de alunos, que têm suas ações referenciadas tecnicamente, principalmente, pela Política de Assistência Estudantil do IFCE (Resolução no 024/2015); o Regulamento de Concessão de Auxílios Estudantis do IFCE (Resolução 052/2016); e os Referenciais de Atuação dos Profissionais de Assistência Estudantil (VOL. 1).\\

\chapter{Atuação do coordenador}

O Coordenador de Curso é o profissional que intermedia a relação com os estudantes, docentes, equipe gestora e equipe multidisciplinar objetivando o bom andamento das ações propostas no projeto do curso, o seu fortalecimento e, consequentemente, o da instituição. O MEC inclui alguns indicadores para o perfil do coordenador de curso superior, conforme o Instrumento de Avaliação de cursos de graduação (Presencial e a distância) – Reconhecimento e Renovação de Reconhecimento, destacando-se os seguintes: A participação do Coordenador do Curso nos órgãos colegiados acadêmicos da IES. Experiência profissional acadêmica. Experiência profissional não-acadêmica (relacionada ao curso). Área de Graduação (pertinência com o curso). Titulação - Dr/MS/Especialização (pertinência com a área do curso) Regime de trabalho na Instituição. No âmbito do IFCE as atribuições das coordenações de curso são definidas pela Nota Técnica n° 002/2015/PROEN/IFCE que ressalta como características primordiais do coordenador a liderança e a proatividade, a capacidade de promover e favorecer a implementação de mudanças que propiciem a melhoria do nível de 63 aprendizado, de estimular a crítica e a criatividade de todos os envolvidos no processo educacional. O coordenador é o servidor responsável por estimular a formação de uma equipe docente coesa propiciando um ambiente tranquilo, de confiança e respeito mútuo, de modo que os objetivos e metas constantes dos planos institucionais sejam conhecidos e executados. Nessa perspetiva, as atribuições do Coordenador de Curso foram distribuídas entre funções acadêmicas, gerenciais e institucionais, sendo as funções acadêmicas compreendidas como as atividades de cunho pedagógico que têm como principal objetivo desenvolver ações de caráter sistêmico relativas ao planejamento, acompanhamento e avaliação do processo de ensino e aprendizagem. Desta forma as atribuições do Coordenador de Curso nesse aspecto são assim definidas:

\begin{itemize}
\item Participar da elaboração e atualização do Projeto Pedagógico do Curso (PPC);
\item Elaborar junto com os professores e a Coordenação Técnico-Pedagógica os planos de curso com todos os quesitos e procedimentos que o compõem;
\item Responsabilizar-se pela qualidade e regularidade das avaliações desenvolvidas no curso;
\item Analisar, organizar, consolidar e avaliar juntamente com a equipe docente e a Coordenação Técnico-Pedagógica a execução do currículo do curso o qual coordena;
\item Dirimir com o apoio da Coordenação Técnico-Pedagógica problemas eventuais que possam ocorrer entre professores e alunos;
\item Orientar os alunos na participação de encontros de divulgação científica e nas disciplinas optativas do curso;
\item Realizar levantamento quanto à oferta de vagas de monitoria tomando por base a análise dos índices de retenção nos componentes curriculares do curso;
\item Realizar reuniões periódicas dos órgãos colegiados (Colegiado e NDE) do curso, atentando para o cumprimento das reuniões ordinárias e quando necessário, extraordinárias;
\item Monitorar e executar as ações do Plano de Permanência e Êxito do IFCE (PPE) no campus em conjunto com a comissão do PPE, Coordenação Técnico Pedagógica e Pró-Reitoria de Ensino.

\end{itemize}

As funções gerenciais são aquelas de caráter administrativo que buscam dar cumprimento às demandas advindas dos estudantes, docentes e gestão, dentre as quais:\\

\begin{itemize}
\item Emitir parecer em relação às solicitações de estudantes e professores;
\item Acompanhar a matrícula dos alunos do curso;
\item Acompanhar solicitações de trancamento e mudança de curso;
\item Estimular a frequência docente para o cumprimento da carga horária prevista para o curso;
\item Realizar Lotação dos professores para as disciplinas do curso;
\item Acompanhar o planejamento de visitas técnicas do curso;
\item Realizar controle das faltas dos docentes do curso organizando a programação de reposição/anteposição das aulas em formulário apropriado para tal fim;
\item Supervisionar as instalações físicas, laboratórios e equipamentos do curso;
\item Elaborar projetos para aquisição de materiais e equipamentos para o curso;
\item Organizar as aquisições de insumos gerais para manutenção do eixo Atividades Específicas do setor;
\item Apresentar ao Diretor/Chefe de Departamento de Ensino o relatório anual das atividades desenvolvidas;
\item Encaminhar ao Diretor/Chefe de Departamento de Ensino as especificações do perfil docente para a realização de concursos públicos ou seleção de professores;
\end{itemize}

Dentre suas atribuições, estão incluídas a representatividade no Núcleo Docente Estruturante (NDE) e a presidência no Colegiado do curso, esta última designada pela Resolução No 75, de 13 de agosto de 2018 do Consup/IFCE. O trabalho do coordenador será pautado por um plano de ação documentado e compartilhado, conforme orientação da Nota informativa da PROEN/IFCE (Processo SEI 0361564).\\

O Coordenador do curso deverá ser um docente do quadro efetivo em regime 40h e dedicação exclusiva. e de acordo com o RAD, o coordenador deverá ter o mínimo de 10 horas de aula semanais.\\

% Elementos pós-textuais
\bibliography{elementos-pos-textuais/referencias}
%\imprimirglossario	
%\imprimirapendices
%\input{elementos-pos-textuais/apendice-a}
%\imprimiranexos0
%\input{elementos-pos-textuais/anexo-a}		
\imprimirindice

\end{document}
